回顾常微分方程中的提到的$Picard$定理:

\paragraph*{$Picard$定理} \quad 对初值问题
\[\dv{}{t}x(t)=f(x(t),t) \qquad x(t_0)=x_0 \tag{1-1}\]
如果$f(x(t),t)$在区域$D$上满足$Lipschitz$条件($|f(x_1,t)-f(x_2,t)| \leq L|x_1-x_2|,L>0$对$\forall t \in D$都成立,其实只要在$t_0$的邻域成立即可),
则$\exists \, \varepsilon > 0$,初值问题在$t \in (t_0-\varepsilon,t_0+\varepsilon)$有唯一解$x(t)$。

回顾初等的证法,原初值问题等价于证明积分方程:
\[x(t)=x_0+\int_{t_0}^tf(x(s),s)\dd s \tag{1-2}\]

为证明上述积分方程的存在唯一性,我们可以构造如下函数列$\{x_n(t)\}$(又称$Picard$序列):
\[x_1(t)=x_0+\int_{t_0}^tf(x_0(s),s)\dd s\]
\[x_2(t)=x_0+\int_{t_0}^tf(x_1(s),s)\dd s\]
\[\vdots\]
\[x_{n+1}(t)=x_0+\int_{t_0}^tf(x_n(s),s)\dd s\]
若证得函数列$\{x_n(t)\}$收敛,则其一定会收敛于$x(t)$,即证明初值问题(1)。证明如下:
\begin{equation*}
    \begin{aligned}
        |x_{n+1}-x_n| & =\left|\int_{t_0}^tf(x_n(s),s)\dd s-\int_{t_0}^tf(x_{n-1}(s),s)\dd s\right| \leq \int_{t_0}^t \left| f(x_n(s),s)-f(x_{n-1}(s),s) \right| \dd s \\
        & \leq L\int_{t_0}^t \left| x_n(s)-x_{n-1}(s) \right| \dd s
    \end{aligned}
\end{equation*}
在$t_0$的邻域$(t_0-\varepsilon,t_0+\varepsilon)$上,我们容易得到(取上下确界不改变不等号):
\[|x_{n+1}-x_n| \leq {\mathop {\text{sup}}\limits_{t \in (t_0-\varepsilon,t_0+\varepsilon)}} (L\varepsilon)|x_{n}-x_{n-1}| \leq \cdots \leq {\mathop {\text{sup}}\limits_{t \in (t_0-\varepsilon,t_0+\varepsilon)}} (L\varepsilon)^n|x_1-x_0|\]
由于在$(t_0-\varepsilon,t_0+\varepsilon)$上$f(x(t),t)$满足$Lipschitz$条件,容易得到$f(x(t),t)$在该区间一致连续,故下式有限:
\[|x_1-x_0|=|\int_{t_0}^tf(x_0(s),s)\dd s|<+\infty\]
则:
\[|x_{n+1}-x_n| \leq (L\varepsilon)^nM \ , \ M:={\mathop {\text{sup}}\limits_{t \in (t_0-\varepsilon,t_0+\varepsilon)}} |x_1-x_0| <+\infty\]
进而我们可以得到:
\[|x_m(t)-x_n(t)| \leq M\sum_{i=n}^{m-1}(L\varepsilon)^i\]
取$\varepsilon \leq 1/2L$:
\[|x_m(t)-x_n(t)| \leq M\sum_{i=n}^{m-1}(L\varepsilon)^i \leq M\sum_{i=n}^{m-1}\frac{1}{2^i} < \frac{M}{2^{n-1}}\]
上式两边取极限即证明函数列$\{x_n(t)\}$为一致柯西列,且
\[x_n(t) \rightrightarrows x(t)\]
这时我们回到积分方程
\[x_{n+1}(t)=x_0+\int_{t_0}^tf(x_n(s),s)\dd s\]
对两边取极限,由于$f(x(t),t)$在$(t_0-\varepsilon,t_0+\varepsilon)$上一致连续,积分于极限可交换,(2)得证,进而原初值问题得证。
\[x(t)=\lim_{n \rightarrow +\infty}\left(x_0+\int_{t_0}^tf(x_n(s),s)\dd s\right)=x_0+\int_{t_0}^tf \left (\lim_{n \rightarrow +\infty}x_n(s),s \right )\dd s=x_0+\int_{t_0}^tf(x(s),s)\dd s\]

如果从泛函分析的观点出发,我们可做出如下不严格的说明:

原命题可以写成,取一个连续函数$x(t)$(这里连续意味着可导)满足:
\[x(t)=x_0+\int_{t_0}^tf(x(s),s)\dd s\]

我们记$t_0$的邻域$(t_0-\varepsilon,t_0+\varepsilon)$上全体连续函数的集合为$C(t_0-\varepsilon,t_0+\varepsilon)$,定义一个映射$T$:
\[T \ : \ C(t_0-\varepsilon,t_0+\varepsilon) \rightarrow C(t_0-\varepsilon,t_0+\varepsilon)\]
\[T[x(t)]=x_0+\int_{t_0}^tf(x(s),s)\dd s\]
这时我们的目标就变成了找一个$x(t)$满足$T[x(t)]=x(t)$,即不动点。

我们继续定义$C(t_0-\varepsilon,t_0+\varepsilon)$上的距离(函数空间上的距离):
\[d(x(t),y(t))={\mathop {\text{sup}}\limits_{t \in (t_0-\varepsilon,t_0+\varepsilon)}} |x(t)-y(t)|\]
则:
\[d(T[x(t)],T[y(t)])={\mathop {\text{sup}}\limits_{t \in (t_0-\varepsilon,t_0+\varepsilon)}} \left|\int_{t_0}^tf(x(s),s)\dd s-\int_{t_0}^tf(y(s),s)\dd s \right| \leq L\varepsilon d(x(t),y(t))\]
当取$L\varepsilon < 1$时,我们得到一个压缩映射,由$Banach$不动点定理即证明完毕:
\[d(T[x(t)],T[y(t)]) < d(x(t),y(t))\]

很显然上述只能叫说明,我们并没有定义一个$Banach$不动点定理适用的距离空间出来。不过也足以说明在泛函分析中我们比较感兴趣的两个对象:函数空间和函数空间上的映射。
在上述的说明中,我们通过距离这个几何概念诱导出了这个函数空间的拓扑结构,当然除此之外还有许多其他的拓扑结构也值得讨论,我们在泛函分析中一般默认讨论的函数空间的代数结构是无穷维线性空间。

还有一个值得思考的问题,我们在初等的证明中使用了柯西收敛定理,那么,在函数空间中柯西列一定收敛吗?
答案是否定的,函数空间中只有完备函数空间的柯西列才是收敛的,这个在后续的学习中会涉及。