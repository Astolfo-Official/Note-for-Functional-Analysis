\begin{introduction}
    \item 有界线性算子~\ref{BX}
    \item Banach-Steinhaus 定理~\ref{BS}
    \item 开映射定理与闭图像定理~\ref{kb}
    \item Hahn-Banach 定理~\ref{HB}
    \item Zorn 引理~\ref{Zorn}
\end{introduction}
\section{有界线性算子}\label{BX}
在正式介绍赋范线性空间的有界线性算子之前,我们可以回忆一下有限维的情形,设$X,Y$为有限维空间,根据上一小节的描述,我们不妨假设$X=\mathbb{R}^m,Y=\mathbb{R}^n$,考虑线性变换$L:\mathbb{R}^m \to \mathbb{R}^n$,选取基底$\{e_i\}_{i=1}^m \subset X , \ \{f_j\}_{j=1}^n \subset Y$,线性变换$L$在基底上有如下关系:
\[Le_i=\sum_j^na_{ij}f_j \ (a_{ij} \in \mathbb{R})\]
则$L$等价于矩阵$A=(a_{ij})_{1 \leq i \leq m , 1 \leq j \leq n}$,取:
\[x=\sum_{i=1}^m\xi_ie_i \in X \ , \ y=\sum_{j=1}^n\eta_jf_j \quad \Rightarrow \quad Lx=
\begin{pmatrix}
    a_{11} & a_{12} & \cdots & a_{1m} \\
    a_{21} & a_{22} & \cdots & a_{2m} \\
    \vdots & \vdots & \ddots & \vdots \\
    a_{n1} & a_{n2} & \cdots & a_{nm} \\
\end{pmatrix}
\begin{pmatrix}
    \xi_1 \\ \xi_2 \\ \vdots \\ \xi_m \\
\end{pmatrix}
=\begin{pmatrix}
    \eta_1 \\ \eta_2 \\ \vdots \\ \eta_n \\
\end{pmatrix}
=y
\]

\begin{definition}[线性算子]
设$X,Y$是赋范线性空间,$T$是从$X$的线性子空间$D(T) \subset X$到$Y$的映射,若其满足:
\[\forall x,y \in X \ , \ \alpha,\beta \in \mathbb{K} \ , \ T(\alpha x+\beta y)=\alpha Tx+\beta Ty\]
\end{definition}
\begin{remark}
线性算子的定义并不要求其定义域是出发空间的全空间,只要求是出发空间的一个线性子空间即可,当然可以是平凡子空间$(D(T)=X)$,同时无穷维的子空间也可以是稠密的$(C[0,1] \subset L[0,1])$。叫法上的区分:1、$T:D(T)=X \to Y$在$X$上到$Y$上;2、$T:D(T) \subset X \to Y$在$X$中到$Y$上。
\end{remark}
\begin{definition}[有界线性算子]
设$T:X \to Y$是线性算子,若$\forall x \in X \ , \ \exists \, M>0$使得$||Tx||_Y \leq M||x||_X$,则称其为有界线性算子。
\end{definition}
从$X \to Y$的全体有界线性算子记为$\mathscr{B}(X,Y)$。有界线性算子具有一些不错的性质,可以看下面一个定理。
\begin{theorem}\label{theorem:yjlx}
设$T:X \to Y$是线性算子,则:
\begin{itemize}
\item 1. $T$连续等价于$T$在一点处连续;
\item 2. $T$连续等价于$T$有界。
\end{itemize}
\end{theorem}
\begin{proof}
\begin{itemize}
\item 1. 充分性显然,下证必要性:设$T$在$x_0 \in X$连续,即$x_n \to x \ \Rightarrow \ Tx_n \to Tx$,则有:
\[Tx_n-Tx=T(x_n-x+y)-Ty \to 0 \quad (\forall y \in X)\]
\item 2. 充分性:由上一条知只需证明在一点处连续,不妨取为原点,取收敛于原点的点列$\{x_n\}$,由有界性:
\[||Tx_n|| \leq M||x_n|| \to 0 \quad \Rightarrow \quad Tx_n \to 0 \quad \Leftrightarrow \quad Tx_n-T(0) \to 0\]
必要性采用反证法:设$T$连续但是无界,$\forall n \in \mathbb{N} \ , \ \exists \, x_n \in X$满足$||Tx_n||>n||x_n||$,令:
\[y_n=\frac{x_n}{n||x_n||} \ , \ ||y_n||=\frac{||x_n||}{n||x_n||}=\frac{1}{n} \to 0 \quad \Rightarrow \quad y_n \to 0\]
由线性性可知$T(0)=0$,而
\[||Ty_n-T(0)||=||Ty_n||=\left\|\frac{Tx_n}{n||x_n||}\right\|>1\]
与连续性矛盾。
\end{itemize}
\end{proof}

\begin{definition}[算子范数]\label{definition:szfs}
设$T \in \mathscr{B}(X,Y)$,定义算子范数:
\[||T||=\mathop \text{sup}\limits_{x \in X/\{0\}}\frac{||Tx||}{||x||}=\mathop \text{sup}\limits_{||x||=1}||Tx||=\mathop \text{sup}\limits_{||x|| \leq 1}||Tx||\]
\end{definition}
既然叫范数说明$||T||$满足范数要求的条件,同时上述三个定义也是等价的。因此$(\mathscr{B}(X,Y),||\cdot||)$就构成了一个赋范线性空间。
\begin{example}
\quad $X=\mathbb{R}^m,Y=\mathbb{R}^n \ , \ L \in \mathscr{B}(X,Y)$,选定基底后$L$等价于矩阵$A=(a_{ij})_{m \times n}$,分别取$X,Y$空间中的两个元素,$x=(\xi_1,\xi_2,\cdots,\xi_m)^{\text{T}} \ , \ y=(\eta_1,\eta_2,\cdots,\eta_n)^{\text{T}}$满足$Lx=y$,则:
\[||Lx||=||y||=\sqrt{\sum_{i=1}^n|\eta_i|^2}=\sqrt{\sum_{i=1}^n\left|\sum_{j=1}^ma_{ij}\xi_j\right|^2} \leq \sqrt{\sum_{i=1}^n\sum_{j=1}^ma_{ij}^2 \cdot \sum_{j=1}^m\xi_j^2}=\text{tr}(A^{\text{T}}A)^{\frac{1}{2}}||x||:=||A||\cdot||X||\]
由上确界性质可知$||L|| \leq ||A||$,实际上是取等。
\end{example}

\begin{example}
\quad 微分算子,一维情形下,考虑$X=C[0,1] \ , \ T:X \to X \ , \ x(t) \mapsto Tx=x'(t)$,则该算子为无界算子,如取$x_n(t)=\sin nt \ , \ ||x_n||=1$,则$Tx_n=x_n'(t)=n \cos nt \ , \ ||Tx_n||=n \to +\infty$。高维情形下也类似,考虑$\Omega \subset \mathbb{R}^n$,记$C^k(\overline{\Omega})=\{f:\overline{\Omega} \to \mathbb{R}|\partial^{\alpha}f \in C(\overline{\Omega}) \ , \ |\alpha| \leq k\}$,其上范数为:
\[||f||_{C^k}=\mathop \text{max}\limits_{|\alpha| \leq k}||\partial^{\alpha}f||_{C^0}=\mathop \text{max}\limits_{|\alpha| \leq k}\mathop \text{max}\limits_{x \in \overline{\Omega}}|\partial^{\alpha}f(x)|\]
其中:
\[\alpha=(\alpha_1,\alpha_2,\cdots,\alpha_n) \ , \ \partial^{\alpha}f=\partial^{\alpha_1}_{x_1}\partial^{\alpha_2}_{x_2}\cdots\partial^{\alpha_n}_{x_n}f(x_1,x_2,\cdots,x_n) \ , \ \sum_{i=1}^n\alpha_i \leq k\]
因此,在高维版本中上述的一维微分算子可表示为:
\[X=C^{\infty}(\overline{\Omega})=\bigcap_{k=1}^{\infty}C^k(\overline{\Omega}) \ , \ T:X \to X \ , \ x(t) \mapsto Tx=\sum_{|\alpha| \leq k}a_{\alpha}(t)\cdot\partial^{\alpha}x(t)\]
其依然是无界算子,反例类似一维。
\end{example}

\begin{example}
\quad 积分算子,考虑 $L^2$上的 Lebesgue 积分:
\[X=L^2(\overline{\Omega}) \ , \ T:X \to \mathbb{R}^n \ , \ u(x) \mapsto Tu=\int_{\overline{\Omega}}u(x)\psi(x)\dd x\]
由柯西不等式:
\[||Tu||=|Tu| \leq ||u||_{L^2}||\psi||_{L^2} \quad \Rightarrow \quad ||T|| \leq ||\psi||_{L^2}\]
再由:
\[||T\psi||=|T\psi|=\int_{\overline{\Omega}}|\psi|^2=||\psi||_{L^2}^2 \quad \Rightarrow \quad ||T||=||\psi||_{L^2}\]
\end{example}

\section{有界线性算子空间}
上一小节中,我们提及有界线性算子全体构成一个赋范线性空间$(\mathscr{B}(X,Y),||\cdot||)$,自然而然我们也会想知道在这样一个赋范线性空间上,元素的收敛性该如何刻画,这对后面讨论空间完备性时至关重要。
\begin{theorem}
点列$\{T_n\} \subset \mathscr{B}(X,Y)$依范数收敛于$T$当且仅当$\{T_n\}$在单位球面$S=\{x \in X| \, ||x||=1\}$一致收敛于$T$。
\end{theorem} 
\begin{proof}
\[||T_n-T|| \to 0 \quad \Rightarrow \quad \forall x \in S \ , \ ||T_n(x)-T(x)|| \leq \mathop \text{sup}\limits_{x \in S}||(T_n-T)x||=||T_n-T||<\varepsilon\]
\[\forall x \in S \ , \ \exists \, N \in \mathbb{N} \quad \text{s.t.} \quad n>N \ , \ ||T_n(x)-T(x)||<\varepsilon \quad \Rightarrow \quad ||T_n-T||=\mathop \text{sup}\limits_{x \in S}||(T_n-T)x||<\varepsilon\]
\end{proof}

虽然这个定理一眼顶针,但既然上课讲了那我还是放进来了。下面这个定理就没那么显然了,如果我们想知道一个算子空间完备需要什么条件,那么按照定义自然而然地就需要其上的所有柯西列都收敛于其中。我们可能会猜想这需要出发空间和到达空间都完备,但如果考虑算子范数的定义,好像跟出发空间没什么太大关系,于是我们进一步猜测只跟到达空间有关系。那么到底是不是呢,答案是yes,下面我们来叙述这件事并证明它。
\begin{theorem}
如果$Y$是 Banach 空间,则$\mathscr{B}(X,Y)$也是 Banach 空间。
\end{theorem} 
\begin{proof}
设$\{T_n\}$为$\mathscr{B}(X,Y)$中的柯西列,即:
\[\forall \varepsilon>0 \ , \ \exists \, N \in \mathbb{N} \ , \ \forall n>N \ , \ p \in \mathbb{N} \quad \text{s.t.} \quad ||T_{n+p}-T_n||<\varepsilon\]
则:
\[\forall x \in X \ , \ ||T_{n+p}(x)-T_n(x)|| \leq ||T_{n+p}-T_n|| \cdot ||x||<\varepsilon||x||\]
即$\{T_n(x)\}$是$Y$中的柯西列,由完备性知其收敛,设$\forall x \in X \ , \ T_n(x) \to T(x)$,则只需验证$T \in \mathscr{B}(X,Y)$即可:
\begin{itemize}
\item 1. 线性性:
\[T(\alpha x+\beta y)=\lim_{n \to +\infty}T_n(\alpha x+\beta y)=\alpha\lim_{n \to +\infty}T_nx+\beta\lim_{n \to +\infty}T_ny=\alpha Tx+\beta Ty\]
\item 2. 有界性:
\[\forall x \in X \ , \ ||Tx||=\left\|\lim_{n \to +\infty}T_nx\right\|=\lim_{n \to +\infty}||T_nx||<+\infty \quad \Rightarrow \quad ||Tx|| \leq M||x||\]
\[||Tx|| \leq ||Tx-T_nx||+||T_nx|| \leq (\varepsilon+M)||X|| \quad \Rightarrow \quad ||T||<\varepsilon+M\]
\end{itemize}
\end{proof}

\begin{definition}[算子空间中的收敛]
设$T_n,T \in \mathscr{B}(X,Y)$:
\begin{itemize}
\item 1. 依范数收敛:$||T_n-T|| \to 0$;
\item 2. 一致收敛:$\forall \varepsilon>0 \ , \ x \in X \ , \ \exists \, N(\varepsilon) \in \mathbb{N} \quad \text{s.t.} \quad \forall n>N \ , \ ||T_n(x)-T(x)||<\varepsilon$;
\item 3. 逐点收敛:$\forall x \in X \ , \ \varepsilon>0 \ , \ \exists \, N(\varepsilon,x) \in \mathbb{N} \quad \text{s.t.} \quad \forall n>N \ , \ ||T_nx-Tx||<\varepsilon$。
\end{itemize}
\end{definition}
显然依范数收敛可以推出逐点收敛。
\begin{example}
\quad $X=l^p \ , \ \forall x=(\xi_1,\xi_2,\cdots,\xi_n,\cdots)$定义算子$T_n: \ X \to X \ , \ T_nx:=(\xi_{n+1},\xi_{n+2},\cdots)$,则对$\forall x \in X$:
\[||T_nx||=\left(\sum_{k=n}^{+\infty}|\xi_k|^p\right)^{\frac{1}{p}} \to 0\]
即$T_n$逐点收敛于$0$。但是$T_n$不依范数收敛于$0$,考虑$x_k=(0,0,\cdots,0,1,0,\cdots) \in S$ ($1$在第$k$位),则:
\[0 < ||T_n(x_n)||=(1,0,\cdots) \leq ||T_n||\]
\end{example}

\section{Banach-Steinhaus 定理}
\begin{definition}[算子有界性]
考虑算子空间中的一族算子$\{T_{\alpha}\}_{\alpha \in \Lambda} \subset \mathscr{B}(X,Y)$:
\begin{itemize}
\item 1. 一致有界:$\forall x \in X \ , \ \exists \, M>0 \quad \text{s.t.} \quad ||T_{\alpha}x|| \leq M||x||$,可以推出$\exists \, M>0 \quad \text{s.t.} \quad ||T_{\alpha}|| \leq M$;
\item 2. 逐点有界:$\forall x \in X \ , \ \exists \, M(x)>0 \quad \text{s.t.} \quad ||T_{\alpha}x|| \leq M(x)||x||$。
\end{itemize}
\end{definition}

在算子空间中一致有界还是强于逐点有界,那么我们自然会思考在什么条件下逐点有界能反推出一致有界。
\begin{theorem}[一致有界定理]
    $\{T_{\alpha}\}_{\alpha \in \Lambda} \subset \mathscr{B}(X,Y) \ , \ D(T_{\alpha})=X$ ,若$X$是 Banach 空间,则$\{T_{\alpha}\}$一致有界当且仅当$\{T_{\alpha}\}$逐点有界。
\end{theorem} 
\begin{proof}
证明分为两步,先证在某个开集(开球)上成立(Baire 纲定理[定理\ref{the:Baire}]),再证在全空间上成立(利用线性)。从一致有界到逐点有界显然,下证逐点有界到一致有界。定义映射:
\[p:X \to \mathbb{R} \ , \ p(x)=\mathop \text{sup}\limits_{\alpha \in \Lambda}||T_{\alpha}x||\]
显然有$\forall x \in X \ , \ p(x)<+\infty$。定义集合划分:
\[M_k=\{x \in X|p(x) \leq k, \ k \in \mathbb{N}\}=\bigcup_{\alpha \in \Lambda}\{x \in X| \, ||T_{\alpha}x|| \leq k, \ k \in \mathbb{N}\}\]
则显然有:
\[X=\bigcap_{k=1}^{\infty}M_k\]
由 Baire 纲定理,$X$是 Banach 空间,为第二纲集,从而存在$k_0 \in \mathbb{N}$使得$M_{k_0}$不是疏集[定义\ref{definition:sj}],即:
\[\exists \, B_r(x_0)=\{x \in X| \, ||x-x_0||<r\} \quad \text{s.t.} \quad B_r(x_0) \subset \overline{M}_k\]
由$T_{\alpha}$的连续性可知,$[0,k]$的原像$\{x \in X| \, ||T_{\alpha}x|| \leq k\}$是闭集,从而$M_k$也是闭集:
\[B_r(x_0) \subset \overline{M}_k=M_k \quad \Leftrightarrow \quad \forall x \in B_r(x_0) \ , \ p(x)=\mathop \text{sup}\limits_{\alpha \in \Lambda}||T_{\alpha}x|| \leq k\]
第一步即证,利用线性可以将该结论推广至全空间:
\[\forall y \in X \ , \ \alpha \in \Lambda \ , \ \exists \, M=\frac{2k}{r} \quad \text{s.t.} \quad ||T_{\alpha}y||=\left\|T_{\alpha}\left(x_0+\frac{r}{||y||}y\right)-T_{\alpha}x_0\right\| \cdot \frac{||y||}{r} \leq \frac{||y||}{r}(k+k)=M||y||\]
\end{proof}
一致有界定理可以扩展到出发空间是第二纲集,其逆否定理叫共鸣定理。
\begin{theorem}[共鸣定理]
    $\{T_{\alpha}\}_{\alpha \in \Lambda} \subset \mathscr{B}(X,Y) \ , \ D(T_{\alpha})=X$ , $X$是 Banach 空间,若$||T_{\alpha}||=+\infty$则,$\exists \, x \in X \quad \text{s.t.} \quad ||T_{\alpha}x||=+\infty$。
\end{theorem} 
共鸣定理指出了有界线性算子的点点有界等价于一致有界。相比于一般的常见赋范线性空间,算子空间的性质不是那么任意直观看出,如果能利用出发空间和到达空间的性质判断算子空间的性质自然是最好的。
\begin{theorem}[Banach-Steinhaus 定理]\label{BS}
$\{T_n\}_{n=1}^{+\infty} \subset \mathscr{B}(X,Y) \ , \ D(T_{\alpha})=X$ ,若$Y$是 Banach 空间,若:
\begin{itemize}
    \item 1. $\{T_n\}$一致有界;
    \item 2. $\{T_n\}$在$X$的一个稠密子集$G$上逐点收敛($\forall g \in G \ , \ \{T_ng\}$收敛)。
\end{itemize}
则$\exists \, T \in \mathscr{B}(X,Y)$满足$\{T_n\}$逐点收敛于$T$,且:
\[||T|| \leq \varliminf\limits_{n\to\infty}||T_n||\]
\end{theorem} 
\begin{proof}
由于$G$在$X$上稠密,即对$\forall x \in X \ , \ \varepsilon>0\ , \ \exists \, g \in G$满足$||x-g||<\varepsilon$。因此$\exists \, N \in \mathbb{N}$当$\forall m>n>N$时:
\[||T_mx-T_nx|| \leq ||T_mx-T_mg||+||T_mg-T_ng||+||T_ng-T_nx|| \leq \left(||T_m||+||T_n||\right)||x-g||+\varepsilon \leq (2M+1)\varepsilon\]
即$\{T_nx\}$是$Y$空间中的柯西列,由$Y$空间完备:
\[\exists \, y \in Y \quad \text{s.t.} \quad \lim_{n \to \infty}T_nx=y:=Tx\]
下面只需验证$T$为有界线性算子:
\begin{itemize}
\item 1. 线性:
\[T(\alpha x+\beta y)=\lim_{n \to +\infty}T_n(\alpha x+\beta y)=\alpha\lim_{n \to +\infty}T_nx+\beta\lim_{n \to +\infty}T_ny=\alpha Tx+\beta Ty\]
\item 2. 有界性:
\[||Tx||=\left\|\lim_{n \to +\infty}T_nx\right\|=\varliminf\limits_{n\to\infty}||T_nx|| \leq \varliminf\limits_{n\to\infty}||T_n|| \cdot ||x|| \quad \Rightarrow \quad ||T|| \leq \varliminf\limits_{n\to\infty}||T_n||\]
其中第一处等号使用了范数的连续性:由 $T_nx\to Tx$ 可得 $||T_nx||\to||Tx||$,因此:
\[||Tx||=\lim_{n\to\infty}||T_nx||=\varliminf_{n\to\infty}||T_nx||\]
若极限存在则上下极限存在且等于极限$\varliminf=\lim=\varlimsup$。
\end{itemize}
\end{proof}

Banach-Steinhaus 定理告诉我们利用完备到达空间的收敛怎么样推算算子空间的收敛,这无疑是一种极大的方便。同时,该定理也说明了$\mathscr{B}(X,Y)$在强收敛意义下完备。

\begin{example}
\quad 机械求积分公式:令$X=C[a,b] \ , \ \forall x \in X$,存在$[a,b]$的分割$a \leq t_1<t_2<\cdots<t_k\leq b$满足:
\[\int_a^bx(t)\dd t \approx \sum_{k=1}^nA_kx(t_k)\]
则下式(其中$A_k^{(n)}$只与参数$n$有关而与$x(t)$无关):
\[\int_a^bx(t)\dd t=\lim_{n \to \infty}\sum_{k=1}^nA_k^{(n)}x(t_k^{(n)})\]
成立的充要条件为:
\begin{itemize}
\item 1. 
\[\forall n \in \mathbb{N} \ , \ \exists \, M>0 \quad \text{s.t.} \quad \sum_{k=1}^n\left|A_k^{(n)}\right| \leq M\]
\item 2. 上式对$C[a,b]$的一个稠密子集$G$成立。
\end{itemize}
\end{example}
\begin{proof}
构造一列$X=C[a,b]$上的泛函$\{f_n\}$:
\[f_n:X \to \mathbb{R} \ , \ f_n(x)=\sum_{k=1}^nA_k^{(n)}x(t_k^{(n)})\]
为了保证$\{f_n\}$强收敛于原积分,必须保证其线性,至此我们只需要验证:
\[||f_n||=\sum_{k=1}^n\left|A_k^{(n)}\right|\]
即可利用 Banach-Steinhaus 定理证明原命题。一方面:
\[\forall x \in X \ , \ ||f_n(x)||=\left|\sum_{k=1}^nA_k^{(n)}x(t_k^{(n)})\right| \leq ||x||\sum_{k=1}^n\left|A_k^{(n)}\right| \quad \Rightarrow \quad ||f_n|| \leq \sum_{k=1}^n\left|A_k^{(n)}\right|\]
另一方面:
\[\exists \, x_0 \in X \ , \ s_0(t_k^{(n)})=\text{sgn}(A_k^{(n)}) \quad \Rightarrow \quad ||x_0||=1 \ , \ ||f_n(x_0)||=\left|\sum_{k=1}^nA_k^{(n)}\text{sgn}(A_k^{(n)})\right|=\sum_{k=1}^n\left|A_k^{(n)}\right| \leq ||f_n||\]
综上所述:
\[||f_n||=\sum_{k=1}^n\left|A_k^{(n)}\right|\]
\end{proof}

\begin{example}
\quad Fourier 级数的发散性:
\[X=C_{2\pi}=\{x \in C[0,2\pi]|x(0)=x(2\pi)\}=\{x \in C(\mathbb{R})|x(t+2\pi)=x(t) \ , \ \forall t \in \mathbb{R}\}\]
$\forall x \in X$,考察 Fourier 级数:
\[\lim_{n \to \infty}F_n[x(t)]=x(t) \sim \frac{a_0}{2}+\sum_{i=1}^{+\infty}(a_k \cos kt+b_k \sin kt) \ , \ a_k=\frac{1}{\pi}\int_{-\pi}^{\pi}x(s) \cos ks \dd s \ , \ b_k=\frac{1}{\pi}\int_{-\pi}^{\pi}x(s) \sin ks \dd s\]
\end{example}
这里回顾一下 Dirichlet 条件,他只是判断 Fourier 级数收敛的一个充分条件。
\begin{theorem}[Dirichlet 条件]
    若$x \in C_{2\pi}$在$[0,2\pi]$上满足:1、之多有限个第一类间断点;2、至多有限个极值点,则
    \[\lim_{n \to \infty}F_n[x(t)]=\left\{
        \begin{array}{rl}
            x(t) & ,x\text{在}t\text{处连续} \\
            \frac{1}{2}(x(t^-)+x(t^+)) & ,t\text{是}x\text{的第一类间断点}
        \end{array}
    \right.\]
\end{theorem} 
\begin{proof}
下面我们将利用共鸣定理说明不加条件时,Fourier 级数可能不收敛。定义
\[f_n:X \to \mathbb{R} \ , \ x(t) \mapsto f_n[x(0)]=\int_{-\pi}^{\pi}x(s)k_n(s,0) \dd s \quad \text{其中 }k_n(s,t)=\frac{1}{\pi}\left(\frac{1}{2}+\sum_{k=1}^n\cos k(s-t)\right)\]
可以证明:
\[||f_n||=\int_{-\pi}^{\pi}|k_n(s,0)| \dd s \qquad\]
且$||f_n|| \to 0$。然后由共鸣定理可知$\exists \, x \in X$使得$|f_n[x(0)]| \to +\infty$。
\end{proof}

\section{开映射定理与闭图像定理}\label{kb}
\begin{definition}[算子复合(乘法)]
设$X,X_1,X_2$是赋范线性空间,$T_1:X \to X_1$,$T_2:X_1 \to X_2$,$R(T_1) \subset D(T_2)$,定义算子复合$T=T_2 \circ T_1=T_2T_1$,若其满足$\forall x \in X \ , \ Tx=(T_2 \circ T_1)x=T_2(T_1x)$。
\end{definition}
\begin{remark} \quad 
若$T_1 \in \mathscr{B}(X,X_1) \ , \ T_2 \in \mathscr{B}(X_1,X_2)$,则$T \in \mathscr{B}(X,X_2)$。同时,连续算子的复合得到的还是连续算子:
\[||Tx||=||T_2(T_1x)|| \leq ||T_2||\cdot||T_1||\cdot||x|| \quad \Rightarrow \quad ||T|| \leq ||T_1||\cdot||T_2||\]
\end{remark}

\begin{definition}[逆算子]
设$X,Y$是赋范线性空间,$T_1:X \to Y$,若$\exists \, T_1:Y \to X \quad \text{s.t.} \quad T_1 \circ T=I_X \ , \ T \circ T_1=I_Y$,则称$T$可逆,$T_1$是$T$的逆算子,记$T_1=T^{-1}$。
\end{definition}
\begin{remark} \quad 
若$T$是线性的则$T^{-1}$也是线性的;算子$T$可逆当且仅当$T$是一一映射;若$T^{-1}$存在则唯一。
\end{remark}

那么我们会思考如果$T$是有界线性算子,$T^{-1}$是否也是有界,或者说$T^{-1}$是否连续呢?这件事开映射定理及逆算子定理会告诉我们。但是在正式介绍这两个定理之前我们可以从一些比较具体的情况开始逐步推广。
\begin{theorem}
设$T \in \mathscr{B}(X)$是满射,且$\forall x \in X \ , \ \exists \, m>0 \quad \text{s.t.} \quad ||Tx|| \geq m||x||$,则$T$可逆且$T^{-1} \in \mathscr{B}(x)$。
\end{theorem} 
\begin{proof}
$T$是单射:
\[\forall x_1,x_2 \in X \ , \ ||Tx_1-Tx_2|| \geq m||x_1-x_2|| \ , \ ||Tx_1-Tx_2||=0 \quad \Leftrightarrow \quad ||x_1-x_2||=0\]
$T^{-1}$有界:
\[\forall y \in X \ , \ ||T^{-1}y||=||x|| \leq m||Tx||=m||y||\]
\end{proof}

\begin{theorem}
设$X$是 Banach 空间,$T \in \mathscr{B}(X) \ , \ ||T||<1$,则算子$I-T$可逆,且:
\[||(I-T)^{-1}|| \leq \frac{1}{1-||T||}\]
\end{theorem} 
\begin{proof}规定$T^0=I$,则算子$T$的部分级数和$S_n$如下定义:
\[S_n=\sum_{k=0}^{n-1}T^k\]
因此:
\[\forall m>n \ , \ ||S_m-S_n||=\left\|\sum_{k=n}^{m-1}T^k\right\| \leq \sum_{k=n}^{m-1}||T^k|| \leq \sum_{k=n}^{m-1}||T||^k=||T||^n \cdot \frac{1-||T||^{m-n}}{1-||T||}<\frac{||T||^n}{1-||T||} \to 0\]
可知$\{S_n\}$是柯西列,由于$X$是 Banach 空间,故$\{S_n\}$收敛,记其收敛极限为:
\[S=\lim_{n\to\infty}S_n=\lim_{n\to\infty}\sum_{k=0}^nT^k=\sum_{k=0}^{+\infty}T^k\]
为了证明$S$是$I-T$的逆,我们考虑:
\[S_n \circ (I-T)=(I-T) \circ S_n=I-T^n \quad \Rightarrow \quad (I-T) \circ S=S \circ (I-T)=\lim_{n \to +\infty}S_n \circ (I-T)=\lim_{n \to +\infty}(I-T^n)\]
注意到:
\[\lim_{n \to +\infty}||T^n|| \leq \lim_{n \to +\infty}||T||^n=0 \quad \Rightarrow \quad \lim_{n \to +\infty}T^n=\mathbf{0} \ (\text{零算子}) \quad \Rightarrow \quad S=(I-T)^{-1}\]
\[||(I-T)^{-1}||=||S||=\left\|\lim_{n \to +\infty}S_n\right\|=\left\|\lim_{n \to +\infty}\sum_{k=0}^{n-1}T^k\right\| \leq \lim_{n \to +\infty}\sum_{k=0}^{n-1}||T||^k=\lim_{n \to +\infty}\frac{1-||T||^n}{1-||T||}=\frac{1}{1-||T||}\]
\end{proof}
上述定理展示了恒同算子加上一个小的扰动($||T||<1$)时依然可逆,那这件事是否对任意可逆算子成立呢?
\begin{theorem}
设$X$是 Banach 空间,$T, \ T^{-1} \in \mathscr{B}(X)$可逆,若:
\[\exists \, \Delta T \in \mathscr{B}(X) \ , \ ||\Delta T||<\frac{1}{||T^{-1}||} \ , \ S:=T+\Delta T \in \mathscr{B}(X) \ \text{可逆且} \ S^{-1}=\sum_{k=0}^{+\infty}(-1)^k(T^{-1}\Delta T)^kT^{-1}\]
\end{theorem} 
\begin{proof}$S=T+\Delta T=T \circ (I+T^{-1}\Delta T) \ , \ ||T^{-1}\Delta T|| \leq ||T^{-1}|| \cdot ||\Delta T||<1$,由上一定理知$I+T^{-1}\Delta T$可逆,且:
\[(I+T^{-1}\Delta T)^{-1}=\sum_{k=0}^{+\infty}(-T^{-1}\Delta T)^k \quad \Rightarrow \quad S^{-1}=\sum_{k=0}^{+\infty}(-1)^k(T^{-1}\Delta T)^kT^{-1}\]
\end{proof}
\begin{remark} \quad 
$T,T^{-1} \in \mathscr{B}(X)$的这类算子称为正则算子,Banach 空间到 Banach 空间的正则算子构成的集合为开集。
\end{remark}
\begin{definition}[开映射]
定义映射$T: \ X \to Y$,对任意开集$U \subset X$,其像$TU \subset Y$也是开集,则称$T$为开映射。
\end{definition}
\begin{theorem}[开映射定理]\label{theorem:kys}
设$X,X_1$是 Banach 空间,$T \in \mathscr{B}(X,X_1) \ , \ TX=X_1$,则$T$是开映射。
\end{theorem} 
\begin{proof}
由线性可知只需要对零点证明该定理成立:对$\forall y=Tx \ , \ \exists \, \varepsilon>0$和开球$B_{\varepsilon}^X(x) \subset U$,在$TU$中$\exists \, \delta>0$使得开球$B_{\delta}^Y(y)$包含像$TB_{\varepsilon}^X(x)=Tx+\varepsilon TB_1^X(0) \subset B_{\delta}^Y(y)=y+\delta B_1^Y(0)$。分两步证明$T(0)$是$TB(0,1)$的内点:
\begin{proposition}
$T\overline{B^X_1(0)}$在某个$B^Y_{\delta}(0)$中稠密。
\end{proposition}
\begin{proof}
由于$T$是满射,即$TX=X_1$,且$X_1$是全空间:
\[X=\bigcup_{k=1}^{\infty}\overline{B^X_k(0)} \quad \Rightarrow \quad X_1 \subseteq \bigcup_{k=1}^{\infty}T\overline{B^X_k(0)} \quad \Rightarrow \quad X_1=\bigcup_{k=1}^{\infty}T\overline{B^X_k(0)}\]
由于$X_1$是 Banach 空间,为第二纲集[定理\ref{the:Baire}],故存在某个开球$B_{r_0}^Y(y_0) \subset T\overline{B_{k_0}^X(0)}$。对$\forall y \in B_{r_0}^Y(0)$,可以构造$y_1=y_0+y, \ y_2=y_0-y \in B_{r_0}^Y(y_0)$,显然$y=(y_1-y_2)/2$。同时在原像中,对$\forall \varepsilon>0 \ , \ \exists \, x_1,x_2 \in \overline{B^X_k(0)}$使得$||Tx_1-y_1||<\varepsilon \ , \ ||Tx_2-y_2||<\varepsilon$。所以$\exists \, x=(x_1-x_2)/2 \in \overline{B^X_k(0)}$满足:
\[||Tx-y||=\left\|T\left(\frac{x_1-x_2}{2}\right)-\frac{y_1-y_2}{2}\right\|<\varepsilon\]
即$T\overline{B_{k_0}^X(0)}$在$B_{r_0}^Y(0)$中稠密,从而由线性对$\forall \varepsilon>0 \ , \ T\overline{B^X_{\varepsilon}(0)}$在$B_{\delta\varepsilon}^Y(0)$中稠密,其中$\delta=r_0/k_0$,可取$\varepsilon=1$。
\end{proof}
\begin{proposition}\label{proposition:Step2}
$T\overline{B_1^X(0)} \supset B_{\delta/2}^Y(0)$
\end{proposition}
\begin{proof}
由上一小节可知:$T\overline{B^X_{1/2}(0)}$在$B^Y_{\delta/2}(0)$中稠密,则:
\[\forall y \in B^Y_{\delta/2}(0) \ , \ \exists \, x_1 \in B^X_{1/2}(0) \quad \text{s.t.} \quad ||y-Tx_1||<\frac{\delta}{2^2} \quad \Rightarrow \quad \exists \, y_1=y-Tx_1 \in B^Y_{\delta/2^2}(0)\]
归纳可得:
\[\exists \, \{x_n\} \subset B_1^X(0) \ , \ ||x_n|| \leq \frac{1}{x^n} \ , \ \left\|y-T\left(\sum_{k=1}^nx_k\right)\right\|<\frac{\delta}{2^{n+1}}\]
定义部分和$S_n$:
\[S_n=\sum_{k=1}^nx_k \ , \ m>n>N \ , \ ||S_m-S_n||=\left\|\sum_{k=n+1}^mx_k\right\| \leq \sum_{k=n+1}^m||x_k||<\frac{1}{2^{n+1}} \to 0\]
可知$\{S_n\}$是 Cauchy 列,在完备空间$X$中收敛,记为$S$,则:
\[||y-TS||=\left\|y-T\lim_{n \to +\infty}\sum_{k=1}^nx_k\right\| \leq \lim_{n \to +\infty}\left\|y-T\sum_{k=1}^nx_k\right\|=0\]

综上所述,$TB_1^{X}(0) \supset T\overline{B_{1/2}^X(0)} \supset B^Y_{\delta/4}(0)$,即对$x=0 \in B_1^{X}(0)$我们能找到$y=Tx=0 \in TB_1^{X}(0)$是$TB_1^{X}(0)$内点。结合之前的分析,定理得证。
\end{proof}
\begin{figure}[H]
    \center
    \includegraphics[scale=0.3]{./fig/4.4-1.png}
\end{figure}
\end{proof}

完成上面这么一个大工程,我们现在终于能回答一开提及的关于逆算子性质的问题了。
\begin{theorem}[逆算子定理]
设$X,X_1$是 Banach 空间,$T \in \mathscr{B}(X,X_1)$是一一映射(保证开映射),$TX=X_1$,则$T^{-1} \in \mathscr{B}(X_1,X)$。
\end{theorem} 
\begin{proof}
由开映射定理[定理\ref{theorem:kys}]中的命题\ref{proposition:Step2}可知:
\[T\overline{B_1^{X}(0)} \supset B_{\delta/2}^Y(0) \quad \Rightarrow \quad \forall z \in X_1 \ , \  \frac{z\delta}{4||z||} \in B_{\delta/2}^Y(0) \ , \ \exists ! \, T^{-1}\left(\frac{z\delta}{4||z||}\right) \in \overline{B_1^{X}(0)}\]
\[\Rightarrow \quad \left\|T^{-1}\left(\frac{z\delta}{4||z||}\right)\right\| \leq ||T^{-1}|| \cdot \left\|\frac{z\delta}{4||z||}\right\| \leq 1 \quad \Rightarrow \quad \frac{||T^{-1}z||}{||z||} \leq \frac{4}{\delta} \quad \Rightarrow \quad ||T^{-1}|| \leq \frac{4}{\delta}\]
\end{proof}

开映射定理和逆算子定理最重要的一点是得到了$T^{-1}$的连续的条件,该性质可以保证方程$Tx=y$解的稳定性。由逆算子定理我们还有个推论。
\begin{proposition}
    若线性空间$X$上有两个范数$||\cdot||_1,||\cdot||_2$,使得$(X,||\cdot||_1),(X,||\cdot||_2)$都是 Banach 空间,且$||\cdot||_1$强于$||\cdot||_2$,则$||\cdot||_1$与$||\cdot||_2$等价。
\end{proposition}
\begin{proof}
定义映射$I:(X,||\cdot||_1) \to (X,||\cdot||_2) \ , \ x \mapsto x$。
显然$I$是可逆算子,且由$||Ix||_2=||x||_2 \leq \alpha ||x||_1$可知$I$有界。
由逆算子定理,$I^{-1}:(X,||\cdot||_2) \to (X,||\cdot||_1) \ , \ x \mapsto x$,有$||Ix||_1=||x||_1 \leq \beta ||x||_2$,原命题得证。
\end{proof}

\begin{definition}[图像,闭图像]
设$T:X \to Y$是映射,称$G(T)=\{(x,Tx)|x \in X \times Y\}:=\{(x,Tx)|x \in X ,y \in Y\}$为$T$的图像。若$G(T)$是$X \times Y$中的闭集,则称$T$为闭算子,即:
\[\forall \{(x_n,y_n)\} \subset G(T) \ , \ x_n \to x \in X \ , \ y_n \to y \in Y \ , \ (x,y) \in G(T)\]
\end{definition}
\begin{theorem}[闭图像定理]
设$X,Y$是 Banach 空间,若$T:X \to Y$是闭线性算子且$D(T)$是闭线性子空间,则$T$是有界的。
\end{theorem} 
\begin{remark}
\quad 闭图像定理告诉我们如果$T$为闭算子,且其定义域$D(T)$是闭的,那么$T$就是连续的。因为证明一个算子为闭算子比证明它为线性有界算子更容易,所以闭图像定理给出了一个证明连续(有界)线性算子的方法。
\end{remark}
\begin{proof}
由于$T$是闭算子,故$G(T)=\{(x,Tx)|x \in D(T)\}$是$X \times Y$的闭线性子空间,从而$G(T)$是 Banach 空间[定理\ref{theorem:zkj}],考虑算子$p:G(T) \to D(T) \ , \ (x,Tx) \mapsto x$,易知$p$良定义且为一一映射,且满足线性,且$p$有界:
\[||p(x,Tx)||=||x||_X \leq ||(x,Tx)||_{X \times Y} \leq ||x||_X + ||Tx||_Y\]
由逆算子定理$\exists \, p^{-1}:D(T) \to G(T) \ , \ x \mapsto (x,Tx)$有界,即:
\[||p^{-1}x||=||(x,Tx)||_{X \times Y}=||x||_X + ||Tx||_Y \leq \alpha ||x||_X \quad \Rightarrow \quad ||Tx||_Y \leq \beta ||x||_X\]
\end{proof}

\begin{proposition}
考虑$X,Y$是 Banach 空间,$T \in \mathscr{B}(X,Y)$,如果$D(T)$是闭的,则$G(T)$是闭的:
\[\forall\{(x_n,y_n)\} \subset G(T) \ , \ x_n \to x \in D(T) \ , \ y_n=Tx_n \to Tx=y \in T(D(T)) \quad \Rightarrow \quad (x,y) \in G(T)\]
若$D(T)$不是闭的,可以考虑如下延拓:
\[x_n \to x \in \overline{D(T)} \ , \ \overline{T}:\overline{D(T)} \to Y \ , \ x \mapsto T\overline{x}=\lim_{n \to \infty}Tx_n\]
\end{proposition}
\begin{example} \quad 一般来说闭算子不一定有界($X,Y$是 Banach 空间),因为$D(T)$不一定是闭子空间。如:
\[X=C[0,1] \ , \ T=\dv{t} \ , \ D(T)=C^1[0,1]\]
这里,$D(T)$不是闭的,$T$是闭算子,设$x_n \to x \ , \ y_n=Tx_n=x_n' \to y$,则:
\[x_n(t)-x_n(0)=\int_0^tx_n'(s) \dd s \quad \Rightarrow \quad x(t)-x(0)=\int_0^ty(s) \dd s\]
即,$D(T)$中任意收敛于自身的点列在$T[D(T)]$中也收敛于自身,但是$T$不为有界算子,因为存在反例如:
\[\lim_{n \to \infty}T(\sin nx)=+\infty\]
\end{example}

\section{Hahn-Banach 定理}
在回答完算子的逆的存在性和其连续性等之后,我们可能会考虑算子空间的数量,更具体点,如果我们给定一个赋范线性空间$X$,其上的有界线性泛函全体$\mathscr{B}(X,\mathbb{K}) \ (\mathbb{K}=\mathbb{R}/\mathbb{C} \ , \ X\text{是}\mathbb{K}\text{上的线性空间})$的数量有多少,结构又如何?当$X$为有限维时,那么但很简单,$\mathscr{B}(X,\mathbb{K})$与$X$同构,那么在无限维的时候我们又该如何回答这个问题呢?这将是 Hahn-Banach 定理要回答的问题。

Hahn-Banach 定理告诉我们,在线性赋范空间中,我们可以定义“足够多”的不同的线性泛函。
它的思路是:如果我们能把子空间上的线性泛函$M$延拓到全空间$X$上不就可以辣?因为我们很容易找个子空间构建个线性泛函,如果这些子空间的线性泛函能够全都延拓到全空间,那不就说明全空间上可以有“足够多”的线性泛函辣!
\begin{theorem}[Hahn-Banach 定理]\label{HB}
设$M$是$X$的线性子空间,$f \in \mathscr{B}(M,\mathbb{K})=M^*$,则$f$可线性保范延拓至整个空间,即:
\[\exists \, F \in \mathscr{B}(M,\mathbb{K})=X^* \quad \text{s.t.} \quad \forall x \in M \ , \ F|_M=f \ , \ ||F||_X=||f||_M\]
\end{theorem} 
该定理的证明分为两步:1、证明$\mathbb{K}=\mathbb{R}$情形;2、证明$\mathbb{K}=\mathbb{C}$情形。但再此之前,我们需要一个强大的武器 Zorn 引理。有限维任意证明,我们自然而然会想到从有限维借助数学归纳法推广到无穷维,但是如果无穷维是不可数的该怎么办呢,Zorn 引理告诉我们,使用超限归纳法照样推!
\begin{definition}[偏序(半序)关系]
设集合$S$中的一个二元关系$\prec \, :S \times S \to S$,若其满足:
\begin{itemize}
    \item 1. 自反性$x \prec x$;
    \item 2. 传递性$\forall x,y,z \in S \ , \ x \prec y \ , \ y \prec z \quad \Rightarrow \quad x \prec z$;
    \item 3. 反对称性$x \prec y \ , \ y \prec x \quad \Rightarrow \quad x=y$;
\end{itemize}
则称$\prec$为一个偏序(半序)关系,$(S,\prec)$称为一个偏序集。若$\mathscr{A} \subset S$满足$\forall x,y \in \mathscr{A} \ , \ x \prec y$或$y \prec x$,则称$\mathscr{A}$是$S$的全序子集。
\end{definition}
\begin{example}
\quad $(\mathbb{R},\leq)$是全序集,$(2^X,\subseteq)$是偏序集。
\end{example}
\begin{definition}[上界]
对$\mathscr{A} \subset S$,若$\exists \, \beta \in S \quad \text{s.t.} \quad \forall \alpha \in S \ , \ \alpha \prec \beta$则称$\beta$是$\mathscr{A}$的上界。
\end{definition}
\begin{definition}[极大元]
$\exists \, \beta \in S \quad \text{s.t.} \quad \forall \alpha \in S \ , \ \beta \prec \alpha$都有$\alpha=\beta$则称$\beta$是$S$的极大元。
\end{definition}
\begin{theorem}[Zorn 引理]\label{Zorn}
若$S$中的任意全序子集都有上界,则$S$中必然存在(至少一个)极大元。
\end{theorem} 
Zorn 引理与选择公理等价。方便证明起见使用次可加函数重新表述实空间的 Hahn-Banach 定理。
\begin{definition}[次可加函数]
若$p:X \to \mathbb{R}$满足:
\begin{itemize}
\item 1. $\forall x,y \in X \ , \ p(x+y) \leq p(x) + p(y)$;
\item 2. $\forall x \in X \ , \ t \in \mathbb{R}^+ \ , \ p(tx)=tp(x)$;
\end{itemize}
则称$p$为次可加函数。
\end{definition}
\begin{example}
\quad 若$X=\mathbb{R} \ , \ a,b \in \mathbb{R} \ , \ a>b$,可定义次可加函数$p(x)$:
\[p(x)=\left\{\begin{array}{ll}
    ax & ,x \geq 0 \\
    bx & ,x<0    
\end{array}\right.\]
按照次可加函数的定义可以知道次可加函数都是凸函数,且$p(0)=0$。
\end{example}
\begin{theorem}[Hahn-Banach 定理,$\mathbb{K}=\mathbb{R}$]\label{theorem:ckjhb}
    设$M$是$X$的实线性子空间,$p:X \to \mathbb{R}$是次可加函数,$f \in \mathscr{B}(M,\mathbb{R})=M^*$,且对$\forall x \in M \ , \ f(x) \leq p(x)$。则存在线性延拓$F \in \mathscr{B}(M,\mathbb{R})=X^*$满足$F|_M=f$,且对$\forall x \in X \ , \ F(x) \leq p(x)$。
\end{theorem} 
\begin{remark}
\quad 若 Hahn-Banach 定理在实数域上成立,自然而然地,复数情形下我们会将其考虑分解$f \in \mathscr{B}(M,\mathbb{C}) \ , \ f(x)=\phi(x)+i\psi(x)$和$f(ix)=\phi(ix)+i\psi(ix)$,联立可得$\phi(x)=\psi(ix) \ , \ \psi(x)=-\phi(ix)$,即$f(x)=\phi(x)-i\phi(ix)$,由实空间的情形可知,存在一个$\phi$的实线性延拓$\Phi$使得满足条件的$f$的线性延拓$F(X)=\Phi(x)-i\Phi(ix)$存在。
\end{remark}
\begin{proof}分三步证明该定理:
\begin{proposition}
定理\ref{theorem:ckjhb}可以推出定理\ref{HB}。
\end{proposition}
\begin{proof}
设$||f||_M=\alpha>0$,定义$p:X \to \mathbb{R} \ , \ x \mapsto \alpha||x||$,容易验证$p(x)$为次可加函数且$|f(x)| \leq ||f||_M||x||=p(x)$。
根据定理\ref{theorem:ckjhb},$\exists \, F \in X^*$满足$F|_M=f \ , \ \forall x \in X \ , \ F(x) \leq p(x)=\alpha ||x||$,即$||F||_X \leq \alpha$。
又因为:
\[||F||_X=\sup_{||x||=1 \ , \ x \in X}|Fx| \geq \sup_{||x||=1 \ , \ x \in M}|Fx|=||f||_M=\alpha \quad \Rightarrow \quad ||F||_X=||f||_M\]
\end{proof}
\begin{proposition}\label{proposition:1d}
对$\forall x_1 \in X/M$定义$M_1=\{x+tx_1 \in X|x \in M \ , \ t \in \mathbb{R}\}$,则存在线性延拓:
\[\exists \, F_1 \in \mathscr{B}(M_1,\mathbb{R}) \quad \text{s.t.} \quad F_1|_M=f \ , \ \forall x \in M_1 \ , \ F_1(x) \leq p(x)\]
\end{proposition}
\begin{proof}对$\forall x_1 \in X/M \ , \ y_1,y_2 \in M$有$f(y_1)+f(y_2)=f(y_1+y_2) \leq p(y_1+y_2) \leq p(y_1-x_1) + p(y_2+x_1)$,从而有$f(y_1)-p(y_1-x_1) \leq p(y_2+x_1)-f(y_2)$。依题意有$F_1(x+tx_1)=F_1(x)+tF_1(x_1) \leq p(x+tx_1)$,故$F_1(x_1)$满足:
\[\left\{
\begin{array}{ll}
    F_1(x_1) \leq p(x/t+x_1)-f(x/t) & ,t \geq 0 \\
    F_1(x_1) \geq f(x/t)-p(x/t-x_1) & ,t < 0
\end{array}
\right.\]
故只需取:
\[F_1(x_1) \in \left[\sup_{y_1 \in M}f(y_1)-p(y_1-x_1),\inf_{y_2 \in M}p(y_2+x_1)-f(y_2)\right]\]
即可找到满足要求的$F_1$。
\end{proof}
\begin{proposition}
使用超限归纳法将命题\ref{proposition:1d}推广到任意维度。
\end{proposition}
\begin{proof}
考虑集合$S=\{F|F\text{是}f\text{满足定理要求的线性延拓}\}$,定义偏序$\prec$,若$F_1,F_2 \in S \ , \ F_1 \prec F_2$则$D(F_1) \subseteq D(F_2) \ , \ F_2|_{D(F_1)}=F_1$,容易证明$(S,\prec)$是偏序集。下面验证$(S,\prec)$中任意全序子集$\mathscr{F}$都有上界。

定义$\Phi:D \to \mathbb{R}$满足$F \in \mathscr{F} \ , \ x \in D(F)$有$\Phi(x)=F(x)$,记:
\[D=\bigcup_{F \in \mathscr{F}}D(F)\]
则$\forall F \in \mathscr{F}$有$D(F) \subset D \ , \ \Phi|_{D(F)}=F$,即$F \prec \Phi$,$\Phi$为$\mathscr{F}$的上界。进而由 Zorn 引理可知$S$中有极大元$\Phi$,其定义域为$D(\Phi)=X$。
\end{proof}
\end{proof}
\begin{remark}
\quad 需要注意的是 Hahn-Banach 定理只证明了延拓的存在性,但不保证唯一。
\end{remark}
Hahn-Banach 定理有许多有用的推论。
\begin{proposition}
赋范线性空间$X$上$\forall x_0 \in X \ , \ x_0 \neq 0 \ , \ \exists \, f \in X^* \quad \text{s.t.} \quad ||f||=1 \ , \ f(x_0)=||x_0||$。
\end{proposition}
\begin{proof}
令$M=\{\lambda x_0|\lambda \in \mathbb{K}\} \subset X \ , \ f_0:M \to \mathbb{K} \ , \ \lambda x_0 \mapsto |\lambda|\cdot||x_0||$,显然当$\lambda=1$时$f(x_0)=||x_0||$,则:
\[||f_0||_M=\sup_{\lambda \in \mathbb{K}}\frac{|\lambda|\cdot||x_0||}{||\lambda x_0||}=1\]
再由 Hahn-Banach 定理,$\exists \, f \in X^* \ , \ \eval{f}_{M}=f_0 \ , \ ||f||_X=||f_0||_M=1$,满足题设。
\end{proof}
\begin{proposition}
赋范线性空间$X$上$\forall x_1,x_2 \in X \ , \ x_1 \neq x_2 \ , \ \exists \, f \in X^* \quad \text{s.t.} \quad ||f||=1 \ , \ f(x_1) \neq f(x_2)$。
\end{proposition}
\begin{proof}
令$x_0=x_1-x_2 \neq 0$,则由推论1,$f(x_0)=f(x_1-f_2)=f(x_1)-f(x_2) \neq 0$。
\end{proof}
\begin{remark}
\quad 我们可以用一个有界线性泛函去区分两个点的不同性质。
\end{remark}

\begin{proposition}
$X$是赋范线性空间$M \subset X$是线性子空间,$x_0 \in X$满足:
\[d(x_0,M)=\inf_{x \in M}||x-x_0||>0\]
则$\exists \, f \in X^* \quad \text{s.t.} \quad x \in M \ , \ ||f||=1 \ , \ f(x_0)=d \ , \ f|_M=0$。
\end{proposition}
\begin{proof}
令$M=\{x+tx_0|t \in \mathbb{K} \ , \ x \in M\}$,定义$f:M \to \mathbb{K} \ , \ x+tx_0 \mapsto td$,则一方面:
\[|f(x+tx_0)|=|t|d \leq |t| \cdot \left\|\frac{x}{t}+x_0\right\|=||x+tx_0|| \quad \Rightarrow \quad ||f|| \leq 1\]
另一方面$\exists \{x_n\} \subset M \ , \ ||x_n -x_0|| \to d \ (n \to \infty)$:
\[|f(x_n-x_0)|=|f(x_n)-f(x_0)|=|f(x_0)|=d \leq ||f|| \cdot ||x_n-x_0|| \to ||f||d \quad \Rightarrow \quad ||f|| \geq 1\]
综上所述,$||f||=1$,再由 Hahn-Banach 定理即证。
\end{proof}

\section{对偶空间和共轭算子}
\begin{definition}[对偶空间]
$X^*=\mathscr{B}(X,\mathbb{K})$称为$X$的对偶空间。
\end{definition}
\begin{remark}
\quad 对偶空间一定是 Banach 空间。
\end{remark}
\begin{definition}[共轭算子]
$X,Y$是赋范线性空间,$T \in \mathscr{B}(X,Y)$,定义$T^*:Y^* \to X^*$满足$\forall f \in Y^* \ , \ x \in X \ , \ (T^*f)x=f(Tx)$,则称$T^*$为共轭算子。
\end{definition}
\begin{lemma}
\begin{itemize}
\item 1. $T \in \mathscr{B}(X,Y) \ , \ T^* \in \mathscr{B}(Y^*,X^*)$则$||T^*||=||T||$;
\item 2. $\forall \alpha \in \mathbb{K} \ , \ (\alpha T)^*=\alpha T^*$;
\item 3. $(T_1+T_2)^*=T_1^*+T_2^*$;
\item 4. $(T_1T_2)^*=T_2^*T_1^*$;
\item 5. 若$T$有有界逆$T^{-1}$,则$T^*$也有有界逆且$(T^*)^{-1}=(T^{-1})^*$。
\end{itemize}
或者考虑映射$*:\mathscr{B}(X,Y) \to \mathscr{B}(Y^*,X^*)$,则1:*是有界等距映射,2\&3:线性,4:乘法,5:与逆可交换。
\end{lemma} 
\begin{example}
\quad $X=(\mathbb{R}^n,||\cdot||) \ , \ X^*=X$。
\end{example}
\begin{proof}
分别记$X,X^*$的一组对偶基$\{e_i\}_{i=1}^n \ , \ \{f_i\}_{i=1}^n$,即$f_i(e_j)=\delta_{ij}$,构造$e_i \to f_i$即得$X$与$X^*$之间存在一个等距一一映射,即证。
\end{proof}
\begin{example}
\quad $(L^p[a,b])^*=L^q[a,b]$其中$1/p+1/q=1 \ , \ p \in [1,+\infty)$,反之$L^{\infty}$的对偶空间严格比$L^1$大。
\end{example}
\textbf{Proof}:证明的目标是构造如下等距一一映射
\[\Phi:L^q \to (L^p)^* \ , \ g \mapsto \Phi_g \quad \text{s.t.} \quad \forall f \in L^p \ , \ \Phi_g(f)=\int_a^bf(t)g(t) \dd t\]
线性显然,有界性($h\ddot{o}lder$不等式):
\[||\Phi_g(f)|| \leq ||f||_{L^p} \cdot ||g||_{L^q} \quad \Rightarrow \quad ||\Phi_g|| \leq ||g||_{L^q}\]
下证等距即$||\Phi_g||=||g||_{L^q}$,取
\[f_0=\frac{1}{||g||_{L^q}^{q-1}} \cdot g^{q-1} \cdot \text{sgn}(g) \in L^p \quad (\int_a^b|f_0|^p \dd t=\int_a^b\frac{|g|^{p(q-1)}}{||g||_{L^q}^{p(q-1)}} \dd t=\frac{1}{||g||_{L^q}^q}\int_a^b|g|^q \dd t=1 < +\infty)\]
\[\Phi_g(f_0)=\int_a^bf_0 \cdot g \, \dd t=\frac{1}{||g||_{L^q}^{q-1}}\int_a^bg^{q-1} \cdot \text{sgn}(g) \cdot g \, \dd t=||g||_{L^q} \quad \Rightarrow \quad ||\Phi_g|| \geq \frac{||\Phi_g(f_0)||}{||f_0||_{L^p}}=||g||_{L^q}\]
综上所述即证明等距。下证双射,其中单射$(\Phi_{g_1}=\Phi_{g_2} \ \Leftrightarrow \ g_1=g_2)$是好证的:
\[\forall f_0=(g_1-g_2)^{q-1} \cdot \text{sgn}(g_1-g_2) \in L^p \ , \ \Phi_{g_1}(f_0)-\Phi_{g_2}(f_0)=\int_a^b(g_1-g_2)f_0 \dd t=0\]
则有
\[0=\int_a^b(g_1-g_2)f_0 \dd t=\int_a^b(g_1-g_2) \cdot (g_1-g_2)^{q-1} \cdot \text{sgn}(g_1-g_2) \dd t=||g_1-g_2||_{L^q}^q \quad \Leftrightarrow \quad g_1=g_2\]

要证满射即证
\[\forall F \in (L^p)^* \ , \ \exists \, g \in L^q \quad \text{s.t.} \quad F=\Phi_g \quad \text{i.e.} \quad \forall f \in L^p \ , \ F(f)=\Phi_g(f)=\int_a^bg \cdot f \, \dd t\]
但是直接证明该结论较为困难,我们分三步证明:

\textbf{Step 1}: 先证该结论对简单函数成立,只需证明对特征函数$\chi_s$成立。
\[\chi_s=\chi_{[a,s]}(t)=\left\{
    \begin{array}{ll}
        1 & ,t\in[a,s] \\
        0 & ,t\in(s,b]        
    \end{array}
\right.\]
即证明存在$g$满足
\[F(\chi_s)=\int_a^bg(t)\chi(t) \, \dd t=\int_a^sg(t) \, \dd t \quad \Rightarrow \quad g(s)=F'(\chi_s)\]
那么我们只需要证明$F(\chi_s)$几乎处处可导,又因为几乎处处可导可以由绝对连续推出,故只需证明$F(\chi_s)$绝对连续。
令$G(s)=F(\chi_s) \ \delta_k=(s_k,t_k) \subset [a,b] \ (k=1,2,\cdots,n)$为$[a,b]$互不相交的分割。记$\varepsilon_k=\text{sgn}\left(F(\chi_{t_k}-\chi_{s_k})\right)$,则
\begin{equation*}
    \begin{aligned}
        \sum_{k=1}^n|G(t_k)-G(s_k)| & =\sum_{k=1}^n|F(\chi_{t_k})-F(\chi_{s_k})|=\sum_{k=1}^n|F(\chi_{t_k}-\chi_{s_k})|=\sum_{k=1}^nF(\chi_{t_k}-\chi_{s_k}) \cdot \text{sgn}\left(F(\chi_{t_k}-\chi_{s_k})\right) \\
        & =F\left(\sum_{k=1}^{n}\varepsilon_k(\chi_{t_k}-\chi_{s_k})\right)=F\left(\sum_{k=1}^{n}\varepsilon_k\chi_{[s_k,t_k]}\right) \\
        & \leq ||F|| \cdot \left\|\sum_{k=1}^n\varepsilon_k\chi_{[s_k,t_k]}\right\|_{L^p} \leq ||F|| \cdot \sum_{k=1}^n\left\|\varepsilon_k\chi_{[s_k,t_k]}\right\|_{L^p}=||F|| \cdot \sum_{k=1}^n|t_k-s_k|
    \end{aligned}
\end{equation*}
可知$F(\chi_s)$绝对连续,即
\[\forall \varepsilon>0 \ , \ \exists \, \sum_{k=1}^n|t_k-s_k|<\frac{\varepsilon}{||F||} \quad \text{s.t.} \quad \sum_{k=1}^n|G(t_k)-G(s_k)|<\varepsilon\]
再由
\[G(s)=G(a)+\int_a^sg(t) \, \dd t=\int_a^sg(t) \, \dd t=\int_a^bg(t)\chi_s(t) \, \dd t=\Phi_g(\chi_s)\]
即找到了$g$满足条件,由特征函数情形可知,对简单函数(阶梯函数)该结论也成立,即
\[\forall F \in (L^p)^* \ , \ \exists \, g \in L^q \quad \text{s.t.} \quad \forall \psi(t)=\sum_{k=1}^{n}a_k\chi_{[s_k,t_k]} \in L^p \ , \ F(\psi)=\Phi_g(\psi)=\int_a^bg \cdot \psi \, \dd t\]

\textbf{Step 2}: 再证该结论对有界可测函数成立。\\
若$f$为有界可测函数,则存在简单函数$\{\psi_n\} \quad \text{s.t.} \quad \psi_n \xrightarrow{a.e.} f$,有$Lebesgue$控制收敛定理可知只要$g \in L^1$,即有
\[F(f)=F\left(\lim_{n \to \infty}\psi_n\right)=\lim_{n \to \infty}F\left(\psi_n\right)=\lim_{n \to \infty}\Phi_g(\psi_n)=\lim_{n \to \infty}\int_a^b\psi_n(t)g(t) \, \dd t=\int_a^bf(t)g(t) \, \dd t=\Phi_g(f)\]
下面只需证明找到的$g \in L^q$即可,考虑
\[f=\frac{1}{||g||_{L^q}^{q-1}} \cdot g^{q-1} \cdot \text{sgn}(g) \ , \ f_N=\left\{
    \begin{array}{ll}
        f & ,||f|| \leq N \\ 0 & ,||f||>N
    \end{array}
\right.\]
则
\[|F(f_n)|=|\Phi_g(f_N)|=\frac{1}{||g||_{L^q}^{q-1}}\int_{\{|f| \leq N\}}g^{q-1} \cdot \text{sgn}(g) \cdot g \, \dd t=\frac{1}{||g||_{L^q}^{q-1}}\int_{\{|f| \leq N\}}|g|^q \, \dd t \leq ||F|| \cdot ||f_N||_{L^p}\]
当$n \to +\infty$时,有
\[|F(f_n)|=||g||_{L^q} \leq ||F|| \cdot 1 = ||F||\]
即证$g \in L^q$。

\textbf{Step 3}: 最后证该结论对$\forall f \in L^p$成立。\\
类似Step 2中的构造,考虑$\forall f \in L^p$,存在一列简单可测函数列$\{f_N\} \quad \text{s.t.} \quad F_n \xrightarrow{L^p} f \ , \ ||f_N|| \leq N$,故
\[F(f_N-f)=\int_a^b(f_N-f) \cdot g \, \dd t \leq ||f_n-f||_{L^p} \cdot ||g||_{L^q} \to 0 \quad \Rightarrow \quad F(f_N) \to F(f)\]

综述三步所述,我们通过一步构造两步逼近证明了满射。

\textbf{Q.E.D.}

\begin{example}
\quad $(l^p[a,b])^*=l^q[a,b]$其中$1/p+1/q=1 \ , \ p \in [1,+\infty)$。
\end{example}
\begin{example}
\quad $(C[a,b])^*=V_0[a,b]:=\{v \in V[a,b] \, | \, v(a)=0 \ , \ v(t+0)=v(t) \ , \ \forall t \in [a,b]\}$,其中$V[a,b]$是$[a,b]$上有界变差函数全体。
\end{example}
\begin{example}
\quad 记收敛数列空间为$c$,则$c^*=l^1$。
\end{example}

\section{自反性与弱收敛}
考虑共轭空间的共轭空间,定义为
\begin{definition}[二次共轭空间]
    设$X$为赋范线性空间,$X^{**}=(X^*)^*$是$X$的二次共轭空间。
\end{definition}
联想到负负得正这个我们熟知的在整数上的特性,我们自然会考虑,二次共轭空间是否就是原空间呢?考虑如下映射,称为\textbf{典型映射}:
\[F:X \to X^{**} \ , \ x \mapsto F_x \quad \text{s.t.} \quad \forall f \in X^* \ , \ F_x(f)=f(x)\]
验证$F_x \in X^{**}$即验证其有界线性:
\[F_x(\alpha f+\beta g)=(\alpha f+\beta g)(x)=\alpha f(x)+\beta g(x)=\alpha F_x(f)+\beta F_x(g)\]
\[|F_x(f)|=|f(x)| \leq ||f|| \cdot ||x|| \quad \Rightarrow \quad ||F_x|| \leq ||x||\]
再由$Hahn-Banach$定理:$\exists \, f_0 \quad \text{s.t.} \quad ||f_0||=1 \ , \ f_0(x)=||x||$,则
\[|F_x(f_0)|=|f_0(x)|=||x|| \quad \Rightarrow \quad ||F_x|| \geq ||x||\]
综上所述$F$是一个单的等距映射$||F_x||=||x||$,但不一定满。如果$F$是满的那么二次共轭空间在等距的意义下就是原空间,我们将这类空间称之为\textbf{自反空间}。
\begin{definition}[二次共轭空间]
    设$X$为赋范线性空间,若典型映射$F$是满射则称$X$为自反空间。
\end{definition}
可以来看几个自反空间的例子。
\begin{proposition}
    1、$L^p \ , \ p \in (1,+\infty)$是自反空间;\\
    2、$l^p \ , \ p \in (1,+\infty)$是自反空间;\\
    3、$L^1,L^{\infty},l^1,l^{\infty}$不是自反空间;\\
    4、$C[a,b]$不是自反空间。
\end{proposition}

在引入共轭空间后我们可以新定义一种收敛方式,称为\textbf{弱收敛}。
\begin{definition}[弱收敛]
设$X$为赋范线性空间,$\{x_n\} \subset X$,若$\forall f \in X^* \ , \ \exists \, x_0 \in X, \ , \ f(x_n) \to f(x_0)$则称$\{x_n\}$弱收敛于$x_0$,记为$x_n \xrightarrow{w} x_0$,称$x_0$为$\{x_n\}$的弱极限。
\end{definition}
弱收敛有以下性质:
\begin{lemma}
1、弱极限存在必唯一;\\
2、强收敛$\Rightarrow$弱收敛;\\
3、有限维赋范线性空间是自反空间。
\end{lemma}
\begin{proof}
1、设$x_n \xrightarrow{w} x_0, \ x_1$,则$\forall f \in X^* \ , \ f(x_n)\to f(x_0), \ f(x_1)$,由Hahn-Banach定理可知$x_0=x_1$;\\
\end{proof}