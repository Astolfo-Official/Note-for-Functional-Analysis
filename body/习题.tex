\section*{距离空间与拓扑空间习题}
\begin{example}
\quad 证明:任意开集的并为开集。
\end{example}
\begin{proof}
设$\Lambda$为指标集,$A_{\lambda} \ (\lambda \in \Lambda)$为开集:
\[\forall x \in \bigcup_{\lambda \in \Lambda}A_{\lambda} \ \Rightarrow \ x \in A_{\lambda_i} \ \Rightarrow \ B_{\varepsilon}(x) \subset A_{\lambda_i} \subset \bigcup_{\lambda \in \Lambda}A_{\lambda}\]
\end{proof}

\begin{example}
\quad 证明:在$(X,d)$中,$\overline{A}=A \ \Leftrightarrow \ A^c$是开集。
\end{example}
\begin{proof}充分性:$\overline{A}=A \ \Rightarrow \ A^c$是开集:
\[\forall x \in A^c \ \Rightarrow \ \exists \, B_{\varepsilon}(x) \cap A=\varnothing \ \Leftrightarrow \ B_{\varepsilon}(x) \subset A^c \ \Rightarrow \ A^c\text{是开集}\]
必要性:$A^c$是开集$ \ \Rightarrow \ \overline{A}=A$:
\[\forall x \in A^c \ \Rightarrow \ \exists \, B_{\varepsilon}(x) \subset A^c \ \Leftrightarrow \ B_{\varepsilon}(x) \cap A=\varnothing \ \Rightarrow \ A\text{的所有接触点都在$A$中} \ \Rightarrow \ \overline{A}=A\]
\end{proof}

\begin{example}
\quad 证明$D$是距离空间且$\ x_n(t) \xrightarrow{d} x(t) \ \Leftrightarrow \ x_n(t) \rightrightarrows x(t)\text{且}x_n'(t) \rightrightarrows x'(t)$。
\[D=C^1[0,1]=\{x(t),x'(t) \in C[0,1]\} \quad d(x,y)=\mathop \text{sup}\limits_{t \in [0,1]}|x(t)-y(y)|+\mathop \text{sup}\limits_{t \in [0,1]}|x'(t)-y'(y)|\]
\end{example}

\begin{example}
\quad 设$d$是距离,证明$d/(1+d)$也是距离。
\end{example}

\begin{example}
\quad 证明$d(x,y)=0$时$x=y$在如下空间中成立:
\[X=\{f(z)\text{在}|z|<1\text{上解析},\text{在}|z| \leq 1\text{上连续}\} \ , \ d(x,y)=\mathop \text{sup}\limits_{|z|=1}|x(z)-y(z)|\]
\end{example}
\begin{proof}
当$|z|=1$时这是好证的。当$|z|<1$时,利用极值原理或柯西积分公式(以后者为例):
\[f(z):=x(z)-y(z) \ , \ \forall z_0 \in |z|<1 \ , \ f(z_0)=\int\limits_{|z|=1}\frac{f(z)}{z-z_0}\dd z=\int\limits_{|z|=1}\frac{x(z)-y(z)}{z-z_0}\dd z=0\]
\end{proof}

\begin{example}
\quad 考虑如下距离空间,下面这个例子告诉我们在距离空间中小球可能包含大球,但大球不能太大:
\begin{figure}[htbp]
    \center
    \includegraphics[scale=0.2]{./fig/ex-1.png}
\end{figure}
\[B_4(a)=\{a,b,c,d\} \subset B_3(c)=\{a,b,c,d,e\}\]
证明:若$B_7(a) \subset B_3(b)$则$B_7(a)=B_3(b)$。
\end{example}
\begin{proof}
$\forall x \in B_3(b) \ , \ d(x,a) \leq d(x,b)+d(b,a)<3+3=6<7 \ \Rightarrow \ x \in B_7(a)$
\end{proof}

\begin{example}
\quad $(X,d)$是距离空间,$A \subset X$,令$f(x)=\mathop \text{inf}\limits_{y \in A}d(x,y)$,证明$f(x)$连续。
\end{example}
\begin{proof}
已知三角不等式:$\forall x_1,x_2 \in X \ , \ y \in A \ , \ d(x_1,x_2) \leq d(x_1,y)+d(y,x_2)$,两侧对$y$取下极限:
\[f(x_1) \leq d(x_1,x_2)+f(x_2) \ \Leftrightarrow \ f(x_1)-f(x_2) \leq d(x_1,x_2)\]
同理有:
\[f(x_2)-f(x_1) \leq d(x_1,x_2)\]
因此:
\[\Rightarrow \ |f(x_1)-f(x_2)| \leq d(x_1,x_2) \to 0\]
即:
\[\forall \varepsilon>0 \ , \ d(x_1,x_2)<\delta=\varepsilon \quad \text{s.t.} \quad |f(x_1)-f(x_2)|<\varepsilon\]
\end{proof}

\begin{example}
\quad 证明$C^1[0,1]$是完备距离空间。
\end{example}
\begin{proof}
设$\{x_n(t)\}$是$C^1[0,1]$中的柯西列,$C^1[0,1]$上定义的距离为:
\[d(x_n(t),x_m(t))=\mathop \text{sup}\limits_{t \in [0,1]}|x_n(t)-x_m(t)|+\mathop \text{sup}\limits_{t \in [0,1]}|x_n'(t)-x_m'(t)|\]
由$C[0,1]$的完备性可知,我们只需要证明$\{x_n'(t)\}$是$C[0,1]$中的柯西列,即$x_n'(t) \rightrightarrows h(t) \in C[0,1]$。只需证:
\[x_n(t) \rightrightarrows x(0)+\int_0^th(s)\dd s \ , \ x(0)=\lim_{n \to \infty}x_n(0)\]
验证,利用微积分基本定理:
\[\begin{aligned}
\left|x_n(t)-x(0)-\int_0^th(s)\dd s\right| &\leq \left|x_n(0)+\int_0^tx_n'(s)\dd s-x(0)-\int_0^th(s)\dd s\right|\\
&\leq \left|x_n(0)-x(0)\right|+\int_0^t|x_n'(s)-h(s)|\dd s\\
&\leq \left|x_n(0)-x(0)\right|+\mathop \text{sup}\limits_{t \in [0,1]}|x_n'(t)-h(t)| \to 0
\end{aligned}\]
且上述极限的取值与$t \in [0,1]$无关,故为一致收敛,原命题得证。
\end{proof}

\begin{example}
\quad 证明:闭球套定理成立的距离空间是完备距离空间。
\end{example}
\begin{proof}
原命题等价于设$(X,d)$中任一套闭球套$(r_n \to 0)$都有非空交,那么$(X,d)$完备。设$\{x_n\}$是$X$中的柯西列,由它构造闭球套:
\[\forall k \in \mathbb{Z}_+ \ , \ \exists \, n_k>0 \quad \text{s.t.} \quad m,n \geq n_k \ , \ d(x_m,x_n)<\frac{1}{2^k} \ \Rightarrow \ x_n \in B_{\frac{1}{2^{k+1}}}(x_{n_k})\]
因此存在点列$\{x_{n_k}\}$构造的闭球满足:
\[\overline{B_{\frac{1}{2^{k+1}}}(x_{n_{k+1}})} \subset \overline{B_{\frac{1}{2^k}}(x_{n_k})}\]
当且仅当:
\[\exists \,! \, x \in \bigcap_{k=1}^{\infty}\overline{B}_{\frac{1}{2^k}}(x_{n_k})\]
子列$\{x_{n_k}\}$才能收敛$d(x_{n_k},x) \to 0$。由于柯西利的子列收敛,故原柯西列也收敛。
\end{proof}

\begin{example}
\quad 设$(X,d)$完备,$\tilde{f}$为$X$上的一族连续函数,满足$\tilde{f}=\{F(x)|\forall x \in X \ , \exists \, M_x>0 \quad \text{s.t.} \quad |F(x)| \leq M_x\}$。证明:$\exists \, M>0 \ , \ \exists \, U\text{(开集)} \in X \quad \text{s.t.} \quad \forall x \in U \ , \ F \in \tilde{f} \ , \ |F(x)| \leq M$。
\end{example}
\begin{proof}
Baire 定理应用:设$E_n=\{x|x \in X \ , \ |F(x)| \leq n \ , \ F \in \tilde{f}\}$,因此对:
\[\forall x \in X \ , \ x \in E_{[M_x]+1} \quad \Rightarrow \quad X \subset \bigcup_{n=1}^{\infty}E_n \ \left(or \ X=\bigcup_{n=1}^{\infty}E_n \right)\]
由于$(X,d)$完备,由 Baire 定理,至少存在一个$M$使得$E_M$不是疏集,即存在一个开集$U \subset \overline{E}_M$,对$\forall x \in U \ , \ \exists \, \{x_n\} \subset E_M$使得$x_n \to x$。由连续性可知:
\[|F(x)|=\lim_{n \to \infty}|F(x_n)| \leq M \ , \ \forall F \in \tilde{f}\]
\end{proof}

\begin{example}
在距离空间中证明:连续$\ \Leftrightarrow \ $开集的原像是开集。
\end{example}
\begin{proof}
充分性:设$f:X \to Y$是连续映射,设$G \subset Y$是任一开集,要证$f^{-1}(G)$是$X$中的开集。即证对$\forall x_0 \in f^{-1}(G) \ , \  \exists \, \varepsilon>0$使得$B_{\varepsilon}(f(x_0)) \subset G$。由于$f$是连续映射,即$\forall y \in f^{-1}(G) \ , \  \exists \, \delta>0$使得当$d(y,x_0)<\delta$时有$d_Y(f(y),f(x_0))<\varepsilon$,这也即$f(B_{\delta}(x_0)) \subset B_{\varepsilon}(f(x_0)) \subset G$。因此$B_{\delta}(x_0) \subset f^{-1}(G)$,即$x_0$是$f^{-1}(G)$的内点,由$x_0$的任意性可知$f^{-1}(G)$是开集。
\begin{figure}[htbp]
    \center
    \includegraphics[scale=0.14]{./fig/ex-2.png}
\end{figure}

必要性:设$f:X \to Y$满足开集的原像是开集,则对$\forall x_0 \in X \ , \ \exists \, \varepsilon>0$有开球$B_{\varepsilon}(f(x_0))$是$Y$中的开集,因此$f^{-1}(B_{\varepsilon}(f(x_0)))$是$X$中的开集,且$x_0 \in f^{-1}(B_{\varepsilon}(f(x_0)))$是内点。因此$\exists \, \delta>0$使得$f(B_{\delta}(x_0)) \subset B_{\varepsilon}(f(x_0))$,可知$f$在$x_0$处连续。由$x_0$任意性可得$f$在$X$上连续。
\begin{figure}[htbp]
    \center
    \includegraphics[scale=0.14]{./fig/ex-3.png}
\end{figure}
\end{proof}

\begin{example}
举例说明在压缩映射中
\begin{itemize}
    \item 1. 完备性不可少;
    \item 2. $d(Tx,Ty)<d(x,y) \ \forall x \neq y$是不充分的。
\end{itemize}
\end{example}
\begin{proof}
\begin{itemize}
\item 1. $R-\{0\} \to R-\{0\} \quad x \mapsto x/2$无不动点;
\item 2. 定义映射$T: \ [0,+\infty) \to [0,+\infty), \ T(x)=x+1/(1+x)$,对$\forall x,y \in [0,+\infty)$计算像空间中的距离:
\[d(T(x),T(y))=\left|x+\frac{1}{1+x}-y-\frac{1}{1+y}\right|=\left|1-\frac{1}{(1+x)(1+y)}\right|d(x,y)<d(x,y)\]
当$T(x)=x$时,会得出$1/(1+x)=0$,但该方程无解,无不动点。
\end{itemize}
\end{proof}

\begin{example}
\quad $T:X \to X \ , \ X$完备,$T$在闭球$\overline{B_r(x_0)} \ x_0 \in X$上为压缩映射,即对$\theta \in (0,1)$,满足$d(Tx,Ty) \leq \theta d(x,y)$,且$d(x_0,Tx_0)<(1-\theta)r$,证明:$T$在$\overline{B_r(x_0)}$上有唯一不动点。
\end{example}
\begin{proof}
因为完备闭子空间也是完备的[定理\ref{theorem:zkj}],故只需证$T(\overline{B_r(x_0}) \subset \overline{B_r(x_0)}$,对$\forall x \in \overline{B}_r(x_0)$有:
\[d(Tx,x_0) \leq d(Tx,Tx_0)+d(Tx_0,x_0) \leq \theta d(x,x_0)+(1-\theta)r \leq \theta r+(1-\theta)r=r\]
可知$Tx \in \overline{B_r(x_0)}$,由$x$的任意性可知$T(\overline{B_r(x_0)}) \subset \overline{B_r(x_0)}$。把$T$限制在$\overline{B_r(x_0)}$上:$T:\overline{B_r(x_0)} \to \overline{B_r(x_0)}$,$T$为压缩映射,$\overline{B_r(x_0)}$是完备空间$X$上的闭子空间,所以$\overline{B_r(x_0)}$完备。由压缩映射定理即证。
\end{proof}

\begin{example}
\quad $(t_0,s_0) \in \mathbb{R}^2 \ , \ f(t,s)$在$(t_0,s_0)$的邻域$N$中连续,满足$s_0=f(t_0,s_0)$。同时$\partial_sf(t,s)$在邻域$N$中存在且在$(t_0,s_0)$连续,且$\partial_sf(t_0,s_0)=0$。证明当$\exists \, \delta>0 \ , \ t \in [t_0-\delta,t_0+\delta] \ , \ x(t) \in C[t_0-\delta,t_0+\delta]$时,方程$x(t)=f(t,x(t)) \ , \ x(t_0)=s_0$有唯一解。
\end{example}
\begin{proof}
由$\partial_sf(t,s) \in C(N)$且$\partial_sf(t_0,s_0)=0$,即$\exists \, \delta>0$,当$|t-t_0| \ ,\ |s-s_0| \leq \delta$时,有$|\partial_s(t,s)|=\theta < 1$。令:
\[X=\{x(t) \in C[t_0-\delta,t_0+\delta] \ , \ x(t_0)=s_0 \ , \ \mathop \text{sup}\limits_{t \in [t_0-\delta,t_0+\delta]}|x(t)-s_0| \leq \frac{\delta}{2}\}\]
由$C[t_0-\delta,t_0+\delta]$完备,所以对$\forall \{x_n(t)\} \subset X \ , \ \exists \, x_{\infty}(t) \in C[t_0-\delta,t_0+\delta]$使得$x_n(t) \rightrightarrows x_{\infty}(t)$,且:
\[x_{\infty}(t_0)=\lim_{n\to\infty}x_n(t_0)=\lim_{n\to\infty}s_0=s_0 \ , \ \forall n \in \mathbb{N}_+ \ , \ |x_n(t)-s_0| \leq \frac{\delta}{2} \quad \Rightarrow \quad |x(t)-s_0|=\lim_{n\to\infty}|x_n(t)-s_0|\leq \frac{\delta}{2}\]
上述结论(极限不保严格不等号,所以定义时取闭保持闭性)对$\forall t \in [t_0-\delta,t_0+\delta]$都成立,不等式两侧取上界:
\[\mathop \text{sup}\limits_{t \in [t_0-\delta,t_0+\delta]}|x(t)-s_0| \leq \frac{\delta}{2}\]
即$x \in X$。因此$X$是闭子空间,由完备闭子空间也是完备的[定理\ref{theorem:zkj}]知$X$是完备的。

定义$T: \ X \to X \ , \ T(x(t))=f(t,x(t))$,验证良定义(满足初值条件以及像在原空间中):
\begin{itemize}
\item 1. $T(x(t_0))=f(t_0,x(t_0))=f(t_0,s_0)=s_0=x(t_0)$;
\item 2. $\forall x \in X \ , \ T(x(t)) \in X$:
\[\mathop \text{sup}\limits_{t \in [t_0-\delta,t_0+\delta]}|T(x(t))-s_0|=\mathop \text{sup}\limits_{t \in [t_0-\delta,t_0+\delta]}|f(t,x(t))-s_0|=\mathop \text{sup}\limits_{t \in [t_0-\delta,t_0+\delta]}|f_s(t,\xi(t))||x-s_0| \leq \frac{\delta}{2}\]
\end{itemize}
计算范数$d(Tx,Ty)$:
\[d(Tx,Ty)=\mathop \text{sup}\limits_{t \in [t_0-\delta,t_0+\delta]}|f(t,x(t))-f(t,y(t))|=\mathop \text{sup}\limits_{t \in [t_0-\delta,t_0+\delta]}|f_s(t,\xi(t))||x(t)-y(t)| \leq \theta d(x,y)\]
因此$T$是压缩映射,由压缩映射定理可知$\exists \, ! \, x(t) \in X$满足$x(t)=f(t,x(t))$。
\end{proof}

\begin{example}
\quad 证明:全有界集$A$的有限$\varepsilon$-网可取为$A$的子集。
\end{example}
\begin{proof}
对$\forall \varepsilon>0$,存在$A$的有限$\varepsilon/2$-网$N=\{x_1,x_2,\cdots,x_k\}$,若$B_{\varepsilon/2}(x_i) \cap A =\varnothing$,则可把$x_i$从$N$中去掉。不妨设对每个$x_i \in N$都满足$B_{\varepsilon/2}(x_i) \cap A \neq \varnothing$,取$y_i \in B_{\varepsilon/2}(x_i) \cap A$,显然有$B_{\varepsilon/2}(x_i) \subset B_{\varepsilon}(y_i)$。因此$\{y_1,y_2,\cdots,y_k\} \subset A$为$A$的有限$\varepsilon$-网。
\begin{figure}[htbp]
    \center
    \includegraphics[scale=0.2]{./fig/ex-4.png}
\end{figure}
\end{proof}

\begin{example}
\quad 证明:$C^1[0,1]$中的有界集$A$是$C[0,1]$中的相对紧集。
\end{example}
\begin{definition}[相对紧集]
若集合$M$的闭包$\overline{M}$在空间中是紧集,则称$M$是相对紧集。
\end{definition}
\begin{proof}
$C[0,1]$是距离空间,故$C[0,1]$中的紧集是列紧的闭集,$\overline{A}$是闭集,故只需证明$\overline{A}$是列紧的。分三步证明:
\begin{proposition}
1. $A$在距离空间$C[0,1]$中有界(即一致有界)。
\end{proposition}
\begin{proof}
由$A$在$C^1[0,1]$中有界,可得对$\forall x(t) \in A \ , \ \exists \, K>0$满足$d_C(x(t),0) \leq d_{C^1}(x(t),0) \leq K$,命题得证。
\end{proof}
\begin{proposition}
2. $A$等度连续。
\end{proposition}
\begin{proof}
利用中值定理对$\forall \varepsilon>0 \ , \ x(t) \in A \ , \ t_1,t_2 \in [0,1]$满足$|x(t_1)-x(t_2)|=|x'(\xi)||t_1-t_2| \leq K|t_1-t_2|$,取$|t_1-t_2|<\varepsilon/K$即得$|x(t_1)-x(t_2)|<\varepsilon$,$A$等度连续得证。
\end{proof}
由上述两命题以及 Arzela-Ascoli 定理[定理\ref{the:AA}]可以得$A$是列紧的。
\begin{proposition}
列紧集的闭包还是列紧的。
\end{proposition}
\begin{proof}
任取$\overline{A}$中的柯西列$\{x_n(t)\}$,由闭包的性质,对$\forall x_n(t) \in \overline{A} \ , \ \exists \, y_n(t) \in A$满足$d_C(x_n(t),y_n(t))<1/n$。因为$A$是列紧的,所以$\exists \, n_k$使得$y_{n_k}(t) \to y(t)$,则:
\[d_C(x_{n_k}(t),y(t)) \leq d_C(x_{n_k}(t),y_{n_k}(t))+d_C(y_{n_k}(t),y(t))\leq \frac{1}{n_k}+d_C(y_{n_k}(t),y(t)) \to 0\]
即证$\overline{A}$列紧。
\end{proof}
\end{proof}

\begin{example}
\quad $X$是紧距离空间,$T:X \to X$是连续映射,对$\forall x \neq y$满足$d(Tx,Ty)<d(x,y)$。证明:$T$存在唯一不动点。
\end{example}
\begin{remark}
\quad 需要注意的是这里$T$不一定是压缩映射,如取$X=[0,1]$,定义映射$T:X \to X$:
\[x \mapsto Tx=\frac{x}{1+x}\]
则对$\forall x \neq y \in [0,1]$:
\[|Tx-Ty|=\left|\frac{x}{1+x}-\frac{y}{1+y}\right|=\frac{|x-y|}{|1+x||1+y|}<|x-y|\]
显然上述映射$T$有唯一不动点$T(0)=0$。
\end{remark}
\begin{proof}
定义$f(x)=d(x,Tx)$,易证,$f$在$X$上连续,故由于$X$是紧集可知$f$有最小值,设$x_0$是$f$的极小值点。若$f(x_0)>0$,则$f(x_0) \leq d(Tx_0,T^2X_0)<d(x_0,Tx_0)=f(x_0)$,得出矛盾,故$f(x_0)=0$即$Tx_0=x_0$是不动点。若$\exists \, y_0$使得$Ty_0=y_0$且$y_0 \neq x_0$,则$0<d(x_0,y_0)=d(Tx_0,Ty_0)<d(x_0,y_0)$,得出矛盾,因此不动点唯一。
\end{proof}

\section*{赋范线性空间习题}
\begin{example}
\quad 证明:Banach 空间$c_0=\{x=\{\xi_k\}|\xi_k\to0\} \ , \ ||x||=\mathop \text{sup}\limits_{k}|\xi_k|$可分。
\end{example}
\begin{proof}构造$A=\{r=\{r_k\}|r_k \in \mathbb{Q} \ , \ \exists \, n>0 \ , \ \forall k>n \ , \ r_k=0\}$,显然$A \subset c_0$。依题意,对$\forall \varepsilon>0 \ , \ x=\{\xi_k\} \in c_0 \ , \ \exists \, N>0$当$\forall n \geq N$满足$|\xi_n|<\varepsilon$。显然$\exists \, r=(r_1,r_2,\cdots,r_N,0,0,\cdots) \in A$满足:
\[||x-r||=\mathop \text{sup}\limits_{k \in \mathbb{N}_+}|r_k-\xi_k| \leq \text{max}\left\{\mathop \text{sup}\limits_{k \leq N}|r_k-\xi_k| \ , \ \varepsilon\right\}\leq\varepsilon\]
因为$\text{card}(A)=\text{card}(\mathbb{Q})$,故$c_0$是可分的。
\end{proof}

\begin{example}
\quad 设$x_i,y_i \in \mathbb{K} \ , \ p,q>0$,证明离散形式的 H\"{o}lder 不等式:
\[\frac{1}{p}+\frac{1}{q}=1 \ , \ \sum_{i=1}^n|x_iy_i| \leq \left(\sum_{i=1}^n|x_i|^p\right)^{\frac{1}{p}}\left(\sum_{i=1}^n|y_i|^q\right)^{\frac{1}{q}}\]
和 Minkowski 不等式:
\[p \geq 1 \ , \ \left(\sum_{i=1}^n|x_i+y_i|^p\right)^{\frac{1}{p}} \leq \left(\sum_{i=1}^n|x_i|^p\right)^{\frac{1}{p}}+\left(\sum_{i=1}^n|y_i|^p\right)^{\frac{1}{p}}\]
\end{example}
\begin{proof}
记
\[A=\left(\sum_{i=1}^n|x_i|^p\right)^{\frac{1}{p}} \ , \ B=\left(\sum_{i=1}^n|y_i|^q\right)^{\frac{1}{q}}\]
由 Young 不等式[定理\ref{theorem:young}]
\[\frac{|x_i| \cdot |y_i|}{A \cdot B} \leq \frac{1}{p}\frac{|x_i|^p}{A^p}+\frac{1}{q}\frac{|y_i|^q}{B^q} \quad \Rightarrow \quad \frac{1}{A \cdot B}\sum_{i=1}^n|x_iy_i| \leq \frac{1}{pA^p}\sum_{i=1}^n|x_i|^p+\frac{1}{qB^q}\sum_{i=1}^n|y_i|^q=1\]
\[\Rightarrow \quad \sum_{i=1}^n|x_iy_i| \leq A \cdot B=\left(\sum_{i=1}^n|x_i|^p\right)^{\frac{1}{p}} \cdot \left(\sum_{i=1}^n|y_i|^q\right)^{\frac{1}{q}}\]
离散形式的 H\"{o}lder 不等式即证。下证离散形式的 Minkowski 不等式,易知$1/p+1/q=1 \ \Leftrightarrow \ p=(p-1)q$:
\[\begin{aligned}
\sum_{i=1}^n|x_i+y_i|^p&=\sum_{i=1}^n|x_i+y_i||x_i+y_i|^{p-1} \leq \sum_{i=1}^n|x_i||x_i+y_i|^{p-1}+\sum_{i=1}^n|y_i||x_i+y_i|^{p-1}\\
&\leq \left(\sum_{i=1}^n|x_i|^p\right)^{\frac{1}{p}}\left(\sum_{i=1}^n|x_i+y_i|^{q(p-1)}\right)^{\frac{1}{q}}+\left(\sum_{i=1}^n|y_i|^p\right)^{\frac{1}{p}}\left(\sum_{i=1}^n|x_i+y_i|^{q(p-1)}\right)^{\frac{1}{q}}\\
&\leq \left(\sum_{i=1}^n|x_i|^p\right)^{\frac{1}{p}}\left(\sum_{i=1}^n|x_i+y_i|^{p}\right)^{1-\frac{1}{p}}+\left(\sum_{i=1}^n|y_i|^p\right)^{\frac{1}{p}}\left(\sum_{i=1}^n|x_i+y_i|^{p}\right)^{1-\frac{1}{p}}\\
\end{aligned}\]
\[\Rightarrow \ \left(\sum_{i=1}^n|x_i+y_i|^p\right) \cdot \left(\sum_{i=1}^n|x_i+y_i|^{p}\right)^{\frac{1}{p}-1} \leq \left(\sum_{i=1}^n|x_i|^p\right)^{\frac{1}{p}}+\left(\sum_{i=1}^n|y_i|^p\right)^{\frac{1}{p}}\]
\[\Leftrightarrow \ \left(\sum_{i=1}^n|x_i+y_i|^{p}\right)^{\frac{1}{p}} \leq \left(\sum_{i=1}^n|x_i|^p\right)^{\frac{1}{p}}+\left(\sum_{i=1}^n|y_i|^p\right)^{\frac{1}{p}}\]
离散形式的 Minkowski 不等式即证。
\end{proof}
\newpage
\begin{example}
\quad 设$m(E)<+\infty \ , \ f \in L^{\infty}(E)$,证明:
\[\lim_{p \to \infty}||f||_{L^p}=||f||_{L^{\infty}}\]
\end{example}
\begin{proof}利用 H\"{o}lder 不等式估计$L^p$范数的上界:
\[||f||_{L^p(E)}=\left(\int_E|f|^p\dd t\right)^{\frac{1}{p}}=\left(\int_{E/E_0}|f|^p\dd t\right)^{\frac{1}{p}} \leq \left(m(E) \cdot ||f||^p_{L^{\infty}(E)}\right)^{\frac{1}{p}}=m(E)^{\frac{1}{p}}||f||^p_{L^{\infty}(E)}\]
对上式两侧取极限:
\[\lim_{p \to \infty}||f||_{L^p(E)} \leq \lim_{p \to \infty}m(E)^{\frac{1}{p}}||f||^p_{L^{\infty}(E)}=||f||^p_{L^{\infty}(E)}\]
另一方面,我们令$M=||f||_{L^{\infty}(E)}$,其满足对$\forall \varepsilon>0 \ , \ m(|f| \geq M-\varepsilon)=\delta>0$,都有:
\[||f||_{L^p(E)}=\left(\int_E|f|^p\dd t\right)^{\frac{1}{p}}=\left(\int_{f \geq M-\varepsilon}|f|^p\dd t+\int_{f<M-\varepsilon}|f|^p\dd t\right)^{\frac{1}{p}}\]
\[\geq \left(\int_{|f| \geq M-\varepsilon}|M-\varepsilon|^p\dd t+\int_{|f|<M-\varepsilon}|f|^p\dd t\right)^{\frac{1}{p}} \geq |M-\varepsilon|\delta^{\frac{1}{p}}\]
由$\varepsilon$的任意性:
\[\lim_{p \to \infty}||f||_{L^p(E)} \geq \lim_{p \to \infty} |M-\varepsilon|\delta^{\frac{1}{p}}=||f||_{L^{\infty}(E)}\]
综上所述,有:
\[\lim_{p \to \infty}||f||_{L^p}=||f||_{L^{\infty}}\]
\end{proof}

\section*{有界线性算子习题}