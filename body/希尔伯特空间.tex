\begin{introduction}
    \item 内积空间~\ref{nj}
    \item 正交性与正交集~\ref{zj}
    \item 希尔伯特空间上的算子~\ref{xebtsz}
\end{introduction}
\section{内积空间}\label{nj}
\begin{definition}[内积和内积空间]
令$H$为一个线性空间,定义二元映射$(\cdot,\cdot): \ H \times H \to \mathbb{K}$,若对$\forall x,y \in H \ , \ \alpha,\beta \in \mathbb{K}$都满足:
\begin{itemize}
    \item 1. 非负性:$(x,x) \geq 0$且$(x,x)=0$当且仅当$x=0$;
    \item 2. 对称性:$(x,y)=(y,x)^*$;
    \item 3. 双线性:$(\alpha x + \beta y,z)=\alpha(x,z)+\beta(y,z) \ , \ (z,\alpha x + \beta y)=\alpha^*(z,x)+\beta^*(z,y)$;
\end{itemize}
则称$(\cdot,\cdot)$是$H$上对内积。称$(H,(\cdot,\cdot))$为内积空间。
\end{definition}
\begin{theorem}[Cauchy–Schwarz 不等式]
$\forall x,y \in H \ , \ |(x,y)|^2 \leq (x,x)\cdot(y,y)$
\end{theorem}
\begin{proof}
$\forall \lambda \in \mathbb{K} \ , \ x,y \in H \ , \ 0 \leq (x+\lambda y,x+\lambda y)=(x,x)+\lambda(y,x)+\lambda^*(x,y)+\lambda\lambda^*(y,y)$,对$y=0$,上式显然成立,对$y>0$,可以取$\lambda=-(x,y)/(y,y)$,则上式子可以写成:
\[(x,x)-\frac{(y,x)(x,y)}{(y,y)}-\frac{(y,x)(x,y)}{(y,y)}+\frac{(y,x)(x,y)}{(y,y)}\geq0 \quad \Leftrightarrow \quad |(x,y)|^2 \leq (x,x)\cdot(y,y)\]
\end{proof}

\begin{definition}[Hilbert 空间]
在内积空间$(H,(\cdot,\cdot))$上可以定义范数$||x||=\sqrt{(x,x)}$,称为由内积诱导的范数,易证$(H,||\cdot||)$是赋范线性空间,若$(H,||\cdot||)$完备,则称为 Hilbert 空间。
\end{definition}
\begin{proposition}[内积的连续性]
若$x_n \to x \ , \ y_n \to y$依范数收敛,则$(x_n,y_n) \to (x,y)$:
\[\begin{aligned}
|(x_n,y_n)-(x,y)|&=|(x_n,y_n)-(x,y_n)+(x,y_n)-(x,y)|=|(x_n-x,y_n)+(x,y_n-y)|\\
& \leq |(x_n-x,y_n)|+|(x,y_n-y)| \leq ||x_n-x||\cdot||y_n|| + ||x||\cdot||y_n-y|| \to 0
\end{aligned}\]
\end{proposition}

\begin{example}
\quad 在$l^2$上定义内积$(x,y)$:
\[l^2=\left\{x=\{\xi_i\} \ \Bigg{|} \ \xi_i \in \mathbb{C} \ , \ \sum_{i=1}^{\infty}|\xi_i|^2<+\infty\right\} \ , \ (x,y)=\sum_{i=1}^{\infty}\xi_i\eta_i^* \ , \ (\forall x=\{\xi_i\}, \ y=\{\eta_i\} \in H)\]
\end{example}
\begin{proof}
只需验证$(x,y)<\infty \ , \ \forall x,y \in l^2$:
\[(x,y)=\sum_{i=1}^{\infty}\xi_i\eta_i \leq \sum_{i=1}^{\infty}|\xi_i\eta_i| \leq \sqrt{\sum_{i=1}^{\infty}|\xi_i|^2}\cdot\sqrt{\sum_{i=1}^{\infty}|\eta_i|^2}=||x||_{l^2}\cdot||y||_{l^2}<\infty\]
\end{proof}

\begin{example}
\quad $L^2[a,b]$。
\end{example}
\begin{proof}
\[\forall x,y \in L^2 \ , \ (x,y)=\int_a^bx(t)y^*(t)\dd{t} \leq \left(\int_a^b|x(t)|^2\dd{t}\right)^{\frac{1}{2}}\cdot\left(\int_a^b|y(t)|^2\dd{t}\right)^{\frac{1}{2}}<\infty\]
\end{proof}

范数总能通过内积诱导出,那内积是否都能从范数诱导出呢?
\begin{theorem}
设$H$是$\mathbb{K}$上的内积空间,若范数满足平行四边形性质:$\forall x,y \in H \ , \ ||x+y||^2+||x-y||^2=2||x||^2+2||y||^2$,则可以通过极化恒等式通过范数定义内积,对$\forall x,y \in H$:
\begin{itemize}
\item 1. $\mathbb{K}=\mathbb{R} \ , \ (x,y)_{\mathbb{R}}=(||x+y||^2-||x-y||^2)/4$;
\item 2. $\mathbb{K}=\mathbb{C} \ , \ (x,y)_{\mathbb{C}}=(||x+y||^2-||x-y||^2+i||x+iy||^2-i||x-iy||^2)/4=(x,y)_{\mathbb{R}}+i(x,iy)_{\mathbb{R}}$。
\end{itemize}
\end{theorem}

\begin{example}
\quad $C^k[a,b]$都不是 Hilbert 空间。
\end{example}

\begin{example}
\quad $L^p[a,b]$是 Hilbert 空间当且仅当$p=2$。
\end{example}
\begin{proof}
取:
\[x(t)=\left\{\begin{array}{cl}
(\frac{2}{b-a})^{\frac{1}{p}} & , \ t\in[a,\frac{a+b}{2}]\\
0 & , \ t\in(\frac{a+b}{2},b]
\end{array}\right. \ , \ y(t)=\left\{\begin{array}{cl}
0 & , \ t\in[a,\frac{a+b}{2}]\\
(\frac{2}{b-a})^{\frac{1}{p}} & , \ t\in(\frac{a+b}{2},b]
\end{array}\right.\]
显然$||x||_{L^p}=||y||_{L^p}=1$,而$||x+y||_{L^p}=||x+y||_{L^p}=2^{\frac{1}{p}}$,当且仅当$p=2$满足平行四边形性质。
\end{proof}

\section{正交性与正交集}\label{zj}
\begin{definition}[正交]
设$(H,(\cdot,\cdot))$是内积空间,则对$\forall x,y \in H-\{0\}$可以定义夹角$\theta\in[0,\pi]$满足:
\[\cos\theta=\frac{(x,y)}{||x||\cdot||y||}\in[-1,1]\]
若$(x,y)=0$,即$\theta=\pi/2$,则称$x$与$y$正交,记为$x \perp y$。
\end{definition}
\begin{itemize}
\item 若$M \subset H$,$\exists \, x \in H-M$对$\forall y \in M$都有$(x,y)=0$,则称$x$与$M$正交,记为$x \perp M$;
\item 若$M,N \subset H$,对$\forall x \in N \ , \ y \in M$都有$(x,y)=0$,则称$N$与$M$正交,记为$N \perp M$;
\item 若$M \subset H$,可以定义$M$的正交补$M^{\perp}=\{x \in H|x \perp M\}$,正交补$M^{\perp}$是闭线性子空间。
\end{itemize}

\begin{proposition}
设$M \subset H \ , \ \overline{M}=H$,则$M^{\perp}=\{0\}$。
\end{proposition}
\begin{proof}
设$x \in H \ , \ x \perp M$,由$\overline{M}=H$,可知$\exists \, \{x_n\} \subset M$使得$x_n \to x$,因此:
\[(x,x)=\lim_{n \to \infty}(x_n,x)=0 \quad \Rightarrow \quad x=0\]
\end{proof}

\begin{theorem}
设$H$是 Hilbert 空间,$M \subset H$是闭的凸子空间,则对$\forall x \in H \ , \ \exists \, x_0 \in M$有:
\[||x-x_0||=d(x,M)=\inf_{y \in M}||x-y||\]
\end{theorem}
\begin{proof}
若$x \in M$,取$x_0=x$即证。若$x \notin M$,则有$d(x,M)=d>0$。由下确界的性质,$\exists \, \{x_n\} \subset M$使得$||x_n-x|| \to d$且$||x_n-x||>d$,由$M$是完备闭子空间,只需证明$\{x_n\}$是柯西列,考虑足够大的$m,n>0$:
\[\begin{aligned}
||x_n-x_m||^2&=||(x_n-x)-(x_m-x)||^2=2||x_n-x||^2+2||x_m||^2-||x_n+x_m-2x||^2\\
&=2||x_n-x||^2+2||x_m-x||^2-4\left\|\frac{x_n+x_m}{2}-x\right\|^2
\leq 2||x_n-x||^2+2||x_m-x||^2-4d^2 \to 0
\end{aligned}\]
上述利用了凸性$x_n,x_m \in M \ \Rightarrow \ (x_n+x_m)/2 \in M$,因此$\{x_n\}$是柯西列,且$x_n \to x_0 \in M \ , \ ||x_n-x|| \to d$。
\end{proof}

\begin{theorem}\label{theorem:zjfj}
设$H$是 Hilbert 空间,$M \subset H$是闭凸子空间,则对$\forall x \in H \ , \ \exists \, ! \, x_0 \in M$满足$(x-x_0) \perp M$。
\end{theorem}
\begin{proof}
由上一定理可知$\exists \, x_0 \in M$满足$||x-x_0||=d(x,M)=d>0$,采用反证法证明存在性,假设$\exists z \in M$满足$(x-x_0,z)=\alpha \neq 0$,令$\lambda \in \mathbb{R}$,则有$||x-x_0+\lambda z||^2=||x-x_0||^2+\lambda^2||z||^2+2\lambda(x-x_0,z)=d^2+\lambda^2||z||^2+2\lambda\alpha$,易知存在$\lambda$使得$||x-x_0+\lambda z||^2 \leq d^2$,矛盾。下证唯一性,假设$\exists x_1 \in M$满足$x_1 \neq x_0 \ , \ ||x-x_1||=d$,则有:
\[||x_0-x_1||^2=||(x_0-x)-(x_1-x)||^2=2||x_0-x||^2+2||x_1-x||^2-4\left\|\frac{x_1+x_2}{2}-x\right\|^2\leq 2d^2+2d^2-4d^2=0\]
\end{proof}

\begin{definition}[正交投影,正交分解]
称通过定理\ref{theorem:zjfj}找到的$x_0$为$x$在$M$上的正交投影;令$y=x-x_0$,称$x=x_0+y \in M \oplus M^{\perp}$为正交分解。
\end{definition}

\begin{definition}
设$X$是赋范线性空间,$M \subset X$是闭子空间,若$\exists \, N \subset X$是闭子空间,满足$X=M \oplus N$,则称$M$是可补的。特别的如果$N=M^{\perp}$,则称$X=M \oplus N=M \oplus M^{\perp}$为集合的正交分解。
\end{definition}

\begin{theorem}
设$X$是赋范线性空间,则集合的正交分解成立当且仅当$X$与一个 Hilbert 空间同构。
\end{theorem}

\begin{definition}[正交集/系]
设$X$是内积空间,$\mathscr{E}=\{e_{\alpha} \in X\ | \ \forall \alpha,\beta \in  \Lambda \ , \ (e_{\alpha},e_{\beta})=\delta_{\alpha\beta}\} \subset X$,则称$\mathscr{E}$是$X$中的正交集;若$||e_{\alpha}||=1$,则称$\mathscr{E}$是$X$中的标准正交集。若$\mathscr{E}^{\perp}=\{0\}$,则称$\mathscr{E}$是完备的。
\end{definition}
在内积空间上找到正交集是容易的,那么能找到所有的正交集吗?以及确认正交集的元素有多少?

\begin{theorem}[Gram-Schmidt 正交化]
设$X$是内积空间,$\{x_k\}_{k=1}^n \subset X$,则$\exists \, \{e_k\}_{k=1}^n \subset \mathscr{E}$是标准正交集,且$\text{span}\{x_k\}=\text{span}\{e_k\}$:
\[e_1=\frac{x_1}{||x_1||} \quad , \quad e_k=\left(x_k-\sum_{i=1}^{k-1}(x_k,e_i)e_i\right)\Bigg{/}\left\|x_k-\sum_{i=1}^{k-1}(x_k,e_i)e_i\right\| \quad (k \geq 2)\]
\end{theorem}
\begin{remark}
\quad Gram-Schmidt 正交化可以拓展到可数维。
\end{remark}

\begin{theorem}\label{theorem:wbzjj}
任意内积空间都存在完备正交集。
\end{theorem}
\begin{proof}
设$X$为内积空间的正交集组成的集合,则$\subseteq$构成$X$中的一个偏序关系,$X$中的任意全序子集都有上界,即所有元素的并。由 Zorn 引理[定理\ref{Zorn}],$X$存在极大元$\mathscr{E}_0$。采用反证法证明$\mathscr{E}_0$的完备性,如若不然,则$\exists \, x_0 \in \mathscr{E}_0^{\perp}$且$x_0 \neq 0$,那么$\mathscr{E}_1=\mathscr{E}_0\cap\{0\} \supseteq \mathscr{E}_0$,与$\mathscr{E}_0$是极大元矛盾。
\end{proof}
\begin{remark}
\quad 做一个区分,标准正交集和标准正交基是不一样的,标准正交基指的是$\{e_{\alpha}\}$可以张成全空间$X=\text{span}\{e_{\alpha}\}$。
\end{remark}

\begin{example}
\quad $\mathbb{R}^n \ , \ \mathscr{E}=\{e_k=(0,\cdots,0,1,0,\cdots,0)\}$(第$k$位为1)
\end{example}
\begin{example}
\quad $l^2 \ , \ \mathscr{E}=\{f_k=(0,\cdots,0,1,0,\cdots)\}$(第$k$位为1)
\end{example}
\begin{example}
\quad $L^2_{\mathbb{C}[0,2\pi]}$:
\[\mathscr{E}=\left\{e_n(t)=\frac{1}{\sqrt{2\pi}}e^{int} \ , \ n \in \mathbb{Z}\right\} \quad , \quad (e_n,e_m)=\frac{1}{2\pi}\int_0^{2\pi}e^{int}e^{-imt}\dd{t}=\delta_{mn}\]
\end{example}
以上这三个例子中标准正交集同样也是标准正交基。不可分的 Hilbert 空间不存在完备的可数正交集。
\begin{example}
\quad 扩充的$l^2$:$H=\{f: \ \mathbb{R} \to \mathbb{R} | \ f\text{只在至多可数个点不为0,且}\sum_{x \in \mathbb{R}}|f(x)|^2<\infty\}$(就一数列求和)
\end{example}

\begin{definition}[标准正交基]
称$\{e_{\alpha} \ | \ \alpha \in \Lambda\} \subset H$是 Hilbert 空间$H$中的标准正交基,若对$\forall x \in H$,满足:
\[x=\sum_{\alpha \in \Lambda}(x,e_{\alpha})e_{\alpha}\]
\end{definition}

\begin{theorem}[Bessel 不等式]\label{theorem:Bessel}
设$\mathscr{E}=\{e_{\alpha} \ | \ \alpha \in \Lambda\}$是 Hilbert 空间$H$中的标准正交基,则对$\forall x \in H$,有:
\[\sum_{\alpha \in \Lambda}|(x,e_{\alpha})|^2 \leq ||x||^2 \quad \text{且} \quad \left\|x-\sum_{\alpha \in \Lambda}(x,e_{\alpha})e_{\alpha}\right\|^2=||x||^2-\sum_{\alpha \in \Lambda}|(x,e_{\alpha})|^2\]
\end{theorem}
\begin{proof}
仅证明实数情况,复数类似。取$\mathscr{E}_n=\{e_1,\cdots,e_n\} \subset \mathscr{E}$是有限集,则:
\[\left\|x-\sum_{k=1}^n(x,e_k)e_k\right\|^2=||x||^2-2\sum_{k=1}^n|(x,e_k)|^2+\left\|\sum_{k=1}^n(x,e_k)e_k\right\|^2=||x||^2-\sum_{k=1}^n|(x,e_k)|^2 \geq 0\]
对$\forall m \in \mathbb{N}_+$考虑满足$(x,e_k)>1/m$的项,带入上式子可知至多只有$[m^2||x||]+1$个$k$使得$(x,e_k)>1/m$。取$m \to \infty$,可知只有至多可数个$e_k$使得$(x,e_k)>0$。因此对上式子取极限$n \to \infty$,即证原不等式成立:
\[\left\|x-\sum_{k=1}^{\infty}(x,e_k)e_k\right\|^2=||x||^2-\sum_{k=1}^{\infty}|(x,e_k)|^2 \geq 0 \quad \Leftrightarrow \quad \left\|x-\sum_{\alpha \in \Lambda}(x,e_{\alpha})e_{\alpha}\right\|^2=||x||^2-\sum_{\alpha \in \Lambda}|(x,e_{\alpha})|^2 \geq 0\]
\end{proof}

\begin{theorem}\label{theorem:3eq}
设$\mathscr{E}=\{e_{\alpha} \ | \ \alpha \in \Lambda\}$是 Hilbert 空间$H$中的标准正交集,则以下三个命题等价:
\begin{itemize}
    \item 1. $\mathscr{E}$是$H$的标准正交基;
    \item 2. $\mathscr{E}$是完备的;
    \item 3. Parseval 等式成立:
\[\forall x \in H \ , \ ||x||^2=\sum_{\alpha \in \Lambda}|(x,e_{\alpha})|^2\]
\end{itemize}
\end{theorem}
\begin{proof}
命题1和3显然等价,只需证命题1和2等价。\\
必要性:设$\exists \, x_0 \in \mathscr{E}^{\perp}$,在标准正交基下显然有$(x_0,e_{\alpha})=0$,因此在标准正交基下展开$x_0$:
\[x_0=\sum_{\alpha \in \Lambda}(x_0,e_{\alpha})e_{\alpha}=\sum_{\alpha \in \Lambda}0\cdot e_{\alpha}=0\]
充分性:对$\forall x \in H$,令:
\[y=x-\sum_{\alpha \in \Lambda}(x,e_{\alpha})e_{\alpha} \quad \Rightarrow \quad (y,e_{\beta})=(x,e_{\beta})-\sum_{\alpha \in \Lambda}(x,e_{\alpha})(e_{\alpha},e_{\beta})=(x,e_{\beta})-(x,e_{\beta})=0 \quad \Rightarrow \quad x=\sum_{\alpha \in \Lambda}(x,e_{\alpha})e_{\alpha}\]
且$y \in \mathscr{E}^{\perp}$,因此命题1和2等价得证。
\end{proof}

\begin{theorem}
Hilbert 空间$H$可分当且仅当存在至多可数的标准正交基。
\end{theorem}
\begin{proof}
必要性:由于$H$可分,所以$\exists \, \{y_n\}_{n=1}^{\infty} \subset H$满足$\overline{\text{span}\{y_n\}} \supseteq H$,对$\{y_n\}$做 Gram-Schmidt 正交化,即构造了$H$的至多可数标准正交集,由\ref{theorem:3eq}可知$H$有完备的至多可数的标准正交基。

充分性:设$\mathscr{E}=\{e_k \ | \ k \in F \subseteq \mathbb{N}_+\} \subset H$是至多可数的标准正交基。定义$\text{span}\{e_k\}_{k \in F} \subset H$上的一列有理柯西列$\{q_n=\sum_{k \in F}q_{n,k}e_k\}$,对$\forall x \in H \ , \ \varepsilon>0 \ , \ k \in F \ , \ \exists \, N>0 \ , \ n>N$有$q_{n,k}$满足$|(x,e_k)-q_{n,k}|<\varepsilon/|F|$:
\[||x-q_n||=\left\|x-\sum_{k \in F}q_{n,k}e_k\right\|=\left\|\sum_{k \in F}(x,e_k)e_k-\sum_{k \in F}q_{n,k}e_k\right\| \leq \sum_{k \in F}|(x,e_k)-q_{n,k}|<\varepsilon\]
即对$\forall x \in H$都存在可数集合$\text{span}\{e_k\}_{k \in F}$中都一个列柯西列$q_n \to x$,即 Hilbert 空间$H$可分。
\end{proof}

\begin{theorem}
无穷维可分 Hilbert 空间都与$l^2$同构。
\end{theorem}
\begin{remark}
\quad 如果是有限维则同构于$\mathbb{K}^n$,类似定理\ref{the:B}的证明。
\end{remark}
\begin{proof}
由上一定理,无穷维可分 Hilbert 空间存在一个可数的标准正交基$\mathscr{E}=\{e_k \ | \ k \in \mathbb{N}_+\} \subset H$,记$l^2$的标准正交基为$\{f_k \ | \ k \in \mathbb{N}_+\}$,定义映射$\Phi: \ H \to l^2$,把$H$中的元素映射为$l^2$中的元素:
\[x=\sum_{k=1}^{\infty}(x,e_k)e_k \quad \mapsto \quad \{(x,e_k)\}_{k=1}^{\infty}\]
易证$\Phi$线性且单射,且把$H$中的标准正交基一一映射到$l^2$的标准正交基$e_k \to \{0,\cdots,0,1,0,\cdots\}$(第$k$为1)。考虑 Parseval 等式,对$\forall x \in H$都有:
\[||x||^2_H=\sum_{k=1}^{\infty}|(x,e_k)|^2=||\Phi(x)||^2_{l^2}\]
即对任意$l^2$的元素都对应$H$中的元素$x$,最后验证保内积,$\forall x,y \in H$:
\[(x,y)_H=\left(\sum_{k=1}^{\infty}(x,e_k)e_k,\sum_{k=1}^{\infty}(y,e_k)e_k\right)=\sum_{k=1}^{\infty}(x,e_k)(e_k,y)=(\Phi(x),\Phi(y))_{l^2}\]
\end{proof}

\begin{proposition}[Riemann-Lebesgue 引理]
若$\{e_n\}_{n=1}^{\infty}$是 Hilbert 空间$H$的标准正交集,则$(x,e_n) \to 0$。
\end{proposition}
\begin{remark}
\quad 这是 Bessel 不等式[定理\ref{theorem:Bessel}]的推论,在证明 Bessel 不等式时我们已经证明了。例如,$H=L^2[0,2\pi] \ , \ \forall f \in H$:
\[\lim_{n \to \infty}\int_0^{2\pi}f(t)e^{-int}\dd{t}=0\]
\end{remark}

\section{希尔伯特空间上的算子}\label{xebtsz}
有了内积结构后对赋范线性空间中的有界算子有什么影响?这里只考虑到达空间为$\mathbb{R}$的情况。

设$H=\mathbb{R}^n \ , \ f \in H^*$则对$\forall x \in \mathbb{R}^n, \ \exists \, ! \, a \in \mathbb{R}^n$满足$f(x)=(x,a)=a_1x_1+a_2x_2+\cdots+a_nx_n$。定义零空间(核)$\ker f=\{x \in H|f(x)=0\}$,由于$f$是$H^*$中的有界线性泛函,即把向量映射称数,因此$\dim \operatorname{Im}{f}=\dim \mathbb{R}=1$,因此$\dim\ker f=n-1$。因此$f$对应的向量$a$可以看做核的法向量。同时这也表明有限维情况下,线性泛函的作用实际上相当于内积。

\begin{theorem}[Riesz 表示定理]\label{theorem:rieszbs}
设$H$是 Hilbert 空间,则对$\forall f \in H^* \ , \ \exists \, ! \, x_f \in H$满足对$\forall x \in H$都有$f(x)=(x,x_f)$。
\end{theorem}
\begin{proof}
取核$\ker f=\{x \in H|f(x)=0\} \subset H$,显然$\ker f$是闭子空间。$f=0$时定理显然成立,考虑$f \neq 0$,则$(\ker f)^{\perp} \neq \{0\}$,取$x_0,y \in (\ker f)^{\perp}$,显然$f(x_0),f(y) \neq 0$,记$\lambda=f(y)/f(x_0) \in \mathbb{R}$,由线性$y-\lambda x_0 \in (\ker f)^{\perp}$。而$f(y-\lambda x_0)=f(y)-\lambda f(x_0)=0$,即$y-\lambda x \in \ker f \cap (\ker f)^{\perp}={0}$,因此$\dim (\ker f)^{\perp}=1$。由于$y-\lambda x \in \ker f$:
\[(y-\lambda x_0,x_0)=\left(y-\frac{f(y)}{f(x_0)}x_0,x_0\right)=(y,x_0)-\frac{f(y)}{f(x_0)}(x_0,x_0)=0 \quad \Rightarrow \quad f(y)=\frac{f(x_0)}{||x_0||^2}(y,x_0)=\left(y,\frac{\overline{f(x_0)}}{||x_0||^2}x_0\right)\]
对$\forall x \in H$,可做正交分解$x=x_1+x_2 \ , \ x_1 \in \ker f \ , \ x_2 \in (\ker f)^{\perp}$,设$\exists \, x_0 \in (\ker f)^{\perp}$,取:
\[x_f=\frac{\overline{f(x_0)}}{||x_0||^2}x_0 \quad \Rightarrow \quad f(x)=f(x_2)=\left(x_2,\frac{\overline{f(x_0)}}{||x_0||^2}x_0\right)=(x,x_f)\]
即证存在性。对唯一性,假设$(x,x_f)=(x,x_f')$,则$(x,x_f-x_f')=0$,由$x$的任意性,可知$x_f=x_f'$。
\end{proof}
\begin{remark}
\quad 上述证明表明,存在一一映射$\Phi: \ H^* \to H  \ , \ f \mapsto x_f$,因此有$H \cong H^* \cong H^{**}$。
\end{remark}

\begin{example}
\quad Laplace 方程的弱解。设$\Omega \subset \mathbb{R}^n$是有界开区域(连通开集),对$\forall f \in L^2(\Omega)$,考虑方程:
\begin{equation}\label{eq:Laplace}
\left\{\begin{array}{rl}
-\Delta u = f, & x \in \Omega \\
u = 0, & x \in \partial\Omega \\
\end{array}\right.
\end{equation}
证明方程\eqref{eq:Laplace}存在唯一弱解。
\end{example}

\begin{definition}[局部可积函数]
设$\Omega \subseteq \mathbb{R}^n$是一个非空开集。称可测函数 $f: \ \Omega \to \mathbb{K}$是局部可积的,如果对于任意紧子集 $K \subset \Omega$,有:
\[\int_K |f(x)| \, \dd{x} < \infty\]
上述积分是 Lebesgue 积分,所有在 $\Omega$ 上局部可积的函数构成的集合记作 $L^1_{\text{loc}}(\Omega)$。显然$L^1(\Omega) \subset L^1_{\text{loc}}(\Omega)$。
\end{definition}

\begin{definition}[具有紧支集的光滑函数]
设$\Omega \subseteq \mathbb{R}^n$是一个非空开集。函数$f \in C^\infty(\Omega)$称为具有紧支集的光滑函数,若其支集,定义为:
\[\operatorname{supp} f := \overline{\{ x \in \Omega : f(x) \neq 0 \}}\]
是$\Omega$内的紧子集。所有这样的函数构成的集合记作$C_{\rm{c}}^{\infty}(\Omega)$或$\mathcal{D}(\Omega)$。
\end{definition}

\begin{definition}[弱导数]
对$\alpha = (\alpha_1,\dots,\alpha_n) \in \mathbb{N}^n$,定义长度 $|\alpha| = \alpha_1+\cdots+\alpha_n$,以及偏导数算子:
\[D^\alpha = \frac{\partial^{|\alpha|}}{\partial x_1^{\alpha_1} \cdots \partial x_n^{\alpha_n}}\]
设$u, v \in L^1_{\text{loc}}(\Omega)$,若对任意测试函数$\varphi \in C_c^\infty(\Omega)$都有:
\[\int_{\Omega} u \, D^\alpha \varphi \, \dd{x} = (-1)^{|\alpha|} \int_{\Omega} v \, \varphi \, \dd{x}\]
则称 $v$ 是 $u$ 的一个 $\alpha$ 阶弱导数(或分布导数),记作 $v = D^\alpha u$。
\end{definition}

\begin{definition}[Sobolev 空间 $W^{k,p}(\Omega)$]
定义 Sobolev 空间$W^{k,p}(\Omega)= \left\{ u \in L^p(\Omega) \ | \  
\forall\, |\alpha| \le k, \ p \in [1,+\infty), \ D^\alpha u \in L^p(\Omega)\right\}$,其中 $D^\alpha u$ 是 $u$ 的 $\alpha$ 阶弱导数。对于 $\forall u \in W^{k,p}(\Omega)$,定义其范数如下:
\begin{itemize}
\item 若 $1 \le p < \infty$:
\[\|u\|_{W^{k,p}(\Omega)}=\left( \sum_{|\alpha| \le k} \|D^\alpha u\|_{L^p(\Omega)}^p \right)^{1/p}\]
\item 若 $p = \infty$:
\[\|u\|_{W^{k,\infty}(\Omega)}=\max_{|\alpha| \le k} \|D^\alpha u\|_{L^\infty(\Omega)}\]
\end{itemize}
$(W^{k,p}(\Omega),\|\cdot\|_{W^{k,p}(\Omega)})$ 是 Banach 空间。常记$W^{k,2}(\Omega)$为$H^k(\Omega)$,定义内积:
\[(u,v)_{H^k(\Omega)} = \sum_{|\alpha| \le k} \int_{\Omega} D^\alpha u \, \overline{D^\alpha v} \, \dd{x}\]
$(H^k(\Omega),(\cdot,\cdot)_{H^k(\Omega)})$是 Hilbert 空间,显然$C_{\rm{c}}^{\infty}(\Omega) \subset H^k(\Omega)$,记$H_0^k(\Omega)$为$C_{\rm{c}}^{\infty}(\Omega)$在$H^k(\Omega)$内的闭包。
\end{definition}
\begin{remark}
\quad 验证$(\cdot,\cdot)_{H^k(\Omega)}$是内积,线性和对称性显然,正定性需要用到 Poincar\'{e} 不等式(取$p=2$,记$D^1=D$)。
\end{remark}

\begin{theorem}[Poincar\'{e} 不等式]\label{theorem:Poincare}
设$1\leq p<\infty$且$\Omega\subset\mathbb{R}^n$为有界区域,则对$\forall u\in W_0^{1,p}(U)$存在正常数$C=C(p,\Omega)$使得:
\[\|u\|_{p,U}\leq C\|Du\|_{p,U}\]
\end{theorem}

\begin{definition}[弱解]
称$u \in H_0^1(\Omega)$是方程\eqref{eq:Laplace}的弱解,若其对$\forall v  \in H_0^1(\Omega)$都满足:
\[\int_{\Omega}(-\Delta u) \cdot \nabla v \dd{x}=\int_{\Omega}\nabla u \cdot \nabla v \dd{x}=\int_{\Omega} f \cdot v \dd{x}\]
\end{definition}

\begin{proof}
考虑泛函$F: \ H_0^1(\Omega) \to \mathbb{R} \ , \ v \mapsto \int_{\Omega}f \cdot v \dd{x}$,显然线性,利用Poincar\'{e} 不等式[定理\ref{theorem:Poincare}]验证有界性:
\[|F(v)|=\left|\int_{\Omega}f \cdot v \dd{x}\right| \leq ||f||_{L^2(\Omega)} \cdot ||v||_{L^2(\Omega)} \leq C||f||_{L^2(\Omega)} \cdot ||Dv||_{H_0^1(\Omega)}<\infty\]
由 Riesz 表示定理[定理\ref{theorem:rieszbs}],$\exists \, ! \, u \in H_0^1(\Omega)$满足$F(v)=(u,v)_{H_0^1(\Omega)}$,即对$\forall v \in H_0^1(\Omega)$都满足:
\[\int_{\Omega} f \cdot v \dd{x}=\int_{\Omega}\nabla u \cdot \nabla v \dd{x}=\int_{\Omega}(-\Delta u) \cdot \nabla v \dd{x}\]
\end{proof}

\begin{example}
\quad 散度型椭圆方程的解。设$\Omega \subset \mathbb{R}^n$是有界开区域(连通开集),对$\forall f \in L^2(\Omega)$,考虑方程:
\begin{equation}\label{eq:sdxtyfc}
\left\{\begin{array}{rl}
Lu =-\sum_{i,j=1}^{n}\partial_i(a_{ij}(x)\partial_ju)=f, & x \in \Omega \\
u = 0, & x \in \partial\Omega \\
\end{array}\right.
\end{equation}
证明方程\eqref{eq:sdxtyfc}存在唯一弱解。
\end{example}
