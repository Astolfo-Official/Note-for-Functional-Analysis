\begin{introduction}
    \item 内积空间~\ref{nj}
    \item 正交性与正交集~\ref{zj}
\end{introduction}
\section{内积空间}\label{nj}
\begin{definition}[内积和内积空间]
令$H$为一个线性空间,定义二元映射$(\cdot,\cdot): \ H \times H \to \mathbb{K}$,若对$\forall x,y \in H \ , \ \alpha,\beta \in \mathbb{K}$都满足:
\begin{itemize}
    \item 1. 非负性:$(x,x) \geq 0$且$(x,x)=0$当且仅当$x=0$;
    \item 2. 对称性:$(x,y)=(y,x)^*$;
    \item 3. 双线性:$(\alpha x + \beta y,z)=\alpha(x,z)+\beta(y,z) \ , \ (z,\alpha x + \beta y)=\alpha^*(z,x)+\beta^*(z,y)$;
\end{itemize}
则称$(\cdot,\cdot)$是$H$上对内积。称$(H,(\cdot,\cdot))$为内积空间。
\end{definition}
\begin{theorem}[Cauchy–Schwarz 不等式]
$\forall x,y \in H \ , \ |(x,y)|^2 \leq (x,x)\cdot(y,y)$
\end{theorem}
\begin{proof}
$\forall \lambda \in \mathbb{K} \ , \ x,y \in H \ , \ 0 \leq (x+\lambda y,x+\lambda y)=(x,x)+\lambda(y,x)+\lambda^*(x,y)+\lambda\lambda^*(y,y)$,对$y=0$,上式显然成立,对$y>0$,可以取$\lambda=-(x,y)/(y,y)$,则上式子可以写成:
\[(x,x)-\frac{(y,x)(x,y)}{(y,y)}-\frac{(y,x)(x,y)}{(y,y)}+\frac{(y,x)(x,y)}{(y,y)}\geq0 \quad \Leftrightarrow \quad |(x,y)|^2 \leq (x,x)\cdot(y,y)\]
\end{proof}

\begin{definition}[Hilbert 空间]
在内积空间$(H,(\cdot,\cdot))$上可以定义范数$||x||=\sqrt{(x,x)}$,称为由内积诱导的范数,易证$(H,||\cdot||)$是赋范线性空间,若$(H,||\cdot||)$完备,则称为 Hilbert 空间。
\end{definition}
\begin{proposition}[内积的连续性]
若$x_n \to x \ , \ y_n \to y$依范数收敛,则$(x_n,y_n) \to (x,y)$。
\end{proposition}
\begin{proof}
\[\begin{aligned}
|(x_n,y_n)-(x,y)|&=|(x_n,y_n)-(x,y_n)+(x,y_n)-(x,y)|=|(x_n-x,y_n)+(x,y_n-y)|\\
& \leq |(x_n-x,y_n)|+|(x,y_n-y)| \leq ||x_n-x||\cdot||y_n|| + ||x||\cdot||y_n-y|| \to 0
\end{aligned}\]
\end{proof}

\begin{example}
\quad 在$l^2$上定义内积$(x,y)$:
\[l^2=\left\{x=\{\xi_i\} \ \Bigg{|} \ \xi_i \in \mathbb{C} \ , \ \sum_{i=1}^{\infty}|\xi_i|^2<+\infty\right\} \ , \ (x,y)=\sum_{i=1}^{\infty}\xi_i\eta_i^* \ , \ (\forall x=\{\xi_i\}, \ y=\{\eta_i\} \in H)\]
\end{example}
\begin{proof}
只需验证$(x,y)<\infty \ , \ \forall x,y \in l^2$:
\[(x,y)=\sum_{i=1}^{\infty}\xi_i\eta_i \leq \sum_{i=1}^{\infty}|\xi_i\eta_i| \leq \sqrt{\sum_{i=1}^{\infty}|\xi_i|^2}\cdot\sqrt{\sum_{i=1}^{\infty}|\eta_i|^2}=||x||_{l^2}\cdot||y||_{l^2}<\infty\]
\end{proof}

\begin{example}
\quad $L^2[a,b]$。
\end{example}
\begin{proof}
\[\forall x,y \in L^2 \ , \ (x,y)=\int_a^bx(t)y^*(t)\dd{t} \leq \left(\int_a^b|x(t)|^2\dd{t}\right)^{\frac{1}{2}}\cdot\left(\int_a^b|y(t)|^2\dd{t}\right)^{\frac{1}{2}}<\infty\]
\end{proof}

范数总能通过内积诱导出,那内积是否都能从范数诱导出呢?
\begin{theorem}
设$H$是$\mathbb{K}$上的内积空间,若范数满足平行四边形性质:$\forall x,y \in H \ , \ ||x+y||^2+||x-y||^2=2||x||^2+2||y||^2$,则可以通过极化恒等式通过范数定义内积,对$\forall x,y \in H$:
\begin{itemize}
\item 1. $\mathbb{K}=\mathbb{R} \ , \ (x,y)_{\mathbb{R}}=(||x+y||^2-||x-y||^2)/4$;
\item 2. $\mathbb{K}=\mathbb{C} \ , \ (x,y)_{\mathbb{C}}=(||x+y||^2-||x-y||^2+i||x+iy||^2-i||x-iy||^2)/4=(x,y)_{\mathbb{R}}+i(x,iy)_{\mathbb{R}}$。
\end{itemize}
\end{theorem}

\begin{example}
\quad $C^k[a,b]$都不是 Hilbert 空间。
\end{example}

\begin{example}
\quad $L^p[a,b]$是 Hilbert 空间当且仅当$p=2$。
\end{example}
\begin{proof}
取:
\[x(t)=\left\{\begin{array}{cl}
(\frac{2}{b-a})^{\frac{1}{p}} & , \ t\in[a,\frac{a+b}{2}]\\
0 & , \ t\in(\frac{a+b}{2},b]
\end{array}\right. \ , \ y(t)=\left\{\begin{array}{cl}
0 & , \ t\in[a,\frac{a+b}{2}]\\
(\frac{2}{b-a})^{\frac{1}{p}} & , \ t\in(\frac{a+b}{2},b]
\end{array}\right.\]
显然$||x||_{L^p}=||y||_{L^p}=1$,而$||x+y||_{L^p}=||x+y||_{L^p}=2^{\frac{1}{p}}$,当且仅当$p=2$满足平行四边形性质。
\end{proof}

\section{正交性与正交集}\label{zj}
\begin{definition}[正交]
设$(H,(\cdot,\cdot))$是内积空间,则对$\forall x,y \in H-\{0\}$可以定义夹角$\theta\in[0,\pi]$满足:
\[\cos\theta=\frac{(x,y)}{||x||\cdot||y||}\in[-1,1]\]
若$(x,y)=0$,即$\theta=\pi/2$,则称$x$与$y$正交,记为$x \perp y$。
\end{definition}
\begin{itemize}
\item 若$M \subset H$,$\exists \, x \in H-M$对$\forall y \in M$都有$(x,y)=0$,则称$x$与$M$正交,记为$x \perp M$;
\item 若$M,N \subset H$,对$\forall x \in N \ , \ y \in M$都有$(x,y)=0$,则称$N$与$M$正交,记为$N \perp M$;
\item 若$M \subset H$,可以定义$M$的正交补$M^{\perp}=\{x \in H|x \perp M\}$,正交补$M^{\perp}$是闭线性子空间。
\end{itemize}

\begin{proposition}
设$M \subset H \ , \ \overline{M}=H$,则$M^{\perp}=\{0\}$。
\end{proposition}
\begin{proof}
设$x \in H \ , \ x \perp M$,由$\overline{M}=H$,可知$\exists \, \{x_n\} \subset M$使得$x_n \to x$,因此:
\[(x,x)=\lim_{n \to \infty}(x_n,x)=0 \quad \Rightarrow \quad x=0\]
\end{proof}

\begin{theorem}
设$H$是 Hilbert 空间,$M \subset H$是闭的凸子集,则对$\forall x \in H \ , \ \exists \, x_0 \in M$有:
\[||x-x_0||=d(x,M)=\inf_{y \in M}||x-y||\]
\end{theorem}

