\begin{introduction}
    \item 线性算子谱~\ref{xxszp}
    \item Gelfand 公式~\ref{theorem:Gelfand}
    \item 紧算子谱~\ref{jsjp}
    \item 希尔伯特空间的自共轭紧算子谱~\ref{zgejszp}
    \item 自反性,弱收敛与弱*收敛~\ref{selfback}
    \item 紧算子~\ref{jsz}
\end{introduction}

\section{线性算子谱}\label{xxszp}
回顾有限维的情况,设$X$为$n$维的线性空间,$A: \ X \to X$是线性算子,对$\forall \lambda \in \mathbb{C}$考虑方程$(\lambda \mathbbm{I}-A)x=0$,也即$Ax=\lambda x$,有两种情况:
\begin{itemize}
\item 1. 只有零解,算子$\lambda \mathbbm{I}-A$为单射($\ker(\lambda \mathbbm{I}-A)=\{0\}$且线性),因此$\lambda \mathbbm{I}-A$可逆[定理\ref{theorem:djm}],也可称为正则,其行列式$\det(\lambda \mathbbm{I}-A)\neq0$,此时称$\lambda$为$A$的正则值;
\item 2. 有非零解,算子$\lambda \mathbbm{I}-A$不单进而不可逆,行列式$\det(\lambda \mathbbm{I}-A)=0$,此时称$\lambda$为$A$的特征值,数量为$n$。
\end{itemize}
\begin{definition}[正则算子]
设$X,Y$是赋范线性空间,定义算子$A \in \mathscr{B}(X,Y)$,若$R(A)=Y$且$A^{-1}\in \mathscr{B}(X,Y)$,则称$A$是正则算子。
\end{definition}
现在考虑无穷维时,设$X$是 Banach 空间,定义算子$A \in \mathscr{B}(X)$,对$\forall \lambda \in \mathbb{C}$考虑方程$(\lambda \mathbbm{I}-A)x=0$:
\begin{itemize}
\item 1. 如果该方程只有零解,那么$\lambda \mathbbm{I}-A$是单射,因此$\lambda \mathbbm{I}-A$“\href{https://math.fandom.com/zh/wiki/%E8%B0%B1}{可逆}”\footnote{很多地方都将可逆的扩大化了,严格意义上满射要求算子A的值域$R(A)=Y$,但如果仅在$R(A)$上那么可以宽松地说单射的算子可逆。},分三种情况:
\begin{itemize}
\item 1. $R(\lambda \mathbbm{I}-A)=X$,因此$(\lambda \mathbbm{I}-A)^{-1} \in \mathscr{B}(X)$,记$\lambda\in\rho(A)$为正则值,$\rho(A)$称为预解集;
\item 2. $R(\lambda \mathbbm{I}-A) \neq X$但$\overline{R(\lambda \mathbbm{I}-A)}=X$,$(\lambda \mathbbm{I}-A)^{-1}$存在但不有界,记$\lambda\in\sigma_c(A)$为连续谱;
\item 3. $\overline{R(\lambda \mathbbm{I}-A)} \neq X$,$(\lambda \mathbbm{I}-A)^{-1}$存在但无法判断是否有界,记$\lambda\in\sigma_r(A)$为连续谱。
\end{itemize}
\item 2. 如果方程有非零解,那么$\lambda \mathbbm{I}-A$不是单射,因此$\lambda \mathbbm{I}-A$不可逆,记$\lambda\in\sigma_p(A)$为点谱(特征值)。
\end{itemize}
\begin{definition}[算子的谱]
定义算子$A$的谱为$\sigma(A)=\mathbb{C}-\rho(A)=\sigma_p(A)\cup\sigma_c(A)\cup\sigma_r(A)$。
\end{definition}

\begin{example}
\quad 在$X=L^2[0,1]$,上利用 \href{https://math.fandom.com/zh/wiki/Fourier_%E7%BA%A7%E6%95%B0}{Fourier 分解}定义闭算子[定义\ref{definition:bsz}] $A: \ X \to X$:
\[u(t) \quad \mapsto \quad -\dv[2]{u}{t}=-\dv[2]{t}\sum_{n=-\infty}^{\infty}e^{2i\pi nt}\int_0^1u(t)e^{2i\pi nt}\dd{t}=\sum_{n=-\infty}^{\infty}(2n\pi)^2e^{2i\pi nt}\int_0^1u(t)e^{2i\pi nt}\dd{t}\]
显然$\sigma(A)=\sigma_p(A)=\{(2n\pi)^2 \ | \ n \in \mathbb{Z}\}$,对应特征空间为$\{e^{2i\pi nt} \ | \ n\in\mathbb{Z}\}$。
\end{example}

\begin{example}\label{example:6.2}
\quad 在$X=C[0,1]$上定义算子 $A: \ X \to X \ , \ u(t) \mapsto tu(t)$,证明$\sigma(A)=\sigma_r(A)=[0,1]$。
\end{example}
\begin{proof}
对$\forall \lambda \in \mathbb{C}$都有$(\lambda\mathbbm{I}-A)u(t)=(\lambda-t)u(t)=0$可知方程只有零解$u(t)=0$,因此$(\lambda\mathbbm{I}-A)^{-1}$存在,若$\lambda \notin [0,1]$,显然有$(\lambda\mathbbm{I}-A)^{-1}u=u(t)/(\lambda-t)$,因此:
\[||(\lambda\mathbbm{I}-A)^{-1}u|| \leq \sup_{t \in [0,1]}\frac{||u||}{\lambda-t}\]
进而可知$(\lambda\mathbbm{I}-A)^{-1} \in \mathscr{B}(X)$,即$\lambda$正则,可知$\sigma(A) \subset [0,1]$。若$\lambda \in [0,1]$,则$\overline{R(\lambda\mathbbm{I}-A)}=\overline{\{(\lambda-t)u(t) \ | \ u \in X\}}$,因为$\lambda-t \in X$,可知$\overline{R(\lambda\mathbbm{I}-A)}=\overline{\{u(t) \ | \ u \in X \ , \ u(\lambda)=0\}} \subset C[0,1]$,因此$\sigma(A)=\sigma_r(A)=[0,1]$。
\end{proof}

\begin{example}
\quad 在$X=L^2[0,1]$上类似上一个例子\ref{example:6.2}定义算子 $A: \ X \to X \ , \ u(t) \mapsto tu(t)$,则$\sigma(A)=\sigma_c(A)=[0,1]$。
\end{example}

\begin{theorem}\label{theorem:fskz}
设$X$是 Banach 空间,$A \in \mathscr{B}(X)$,则:
\begin{itemize}
\item 1. $\rho(A)$是开集,也即$\sigma(A)$是闭集;
\item 2. $\sigma(A) \subset \{\lambda \in \mathbb{C} \ | \ |\lambda| \leq ||A||\}$或记为$r_{\sigma}(A) \leq ||A||$。
\end{itemize}
\end{theorem}
\begin{proof}
由定理\ref{theorem:deltazz}对$\forall \lambda \in \rho(A)$,都$\exists \, \delta=||\lambda\mathbbm{I}-A||^{-1}$,使得对$\forall a<\delta$都有$(\lambda+a)\mathbbm{I}-A=(\lambda\mathbbm{I}-A)+a\mathbbm{I}$正则,即对$\forall \lambda \in \rho(A)$其开球$\{\lambda+a \ | \ |a|<\delta\} \subset \rho(A)$,因此$\rho(A)$是开集,第一部分得证。

对$\forall \lambda \in \mathbb{C}$满足$|\lambda|>||A||$,由定理\ref{theorem:nszyjxx}知$\lambda\mathbbm{I}-A=\lambda(\mathbbm{I}-A/\lambda)$正则,因此$\sigma(A) \subset \{\lambda \in \mathbb{C} \ | \ |\lambda| \leq ||A||\}$。
\end{proof}
\begin{figure}[htbp]
    \center
    \includegraphics[scale=0.6]{./fig/6.1-1.png}
\end{figure}
上述定理描述了算子的谱$\sigma(A)$实际上是个闭集且其谱半径$r_{\sigma}(A)$被算子范数$||A||$所控制,如上图所示。

\begin{definition}[谱半径]
谱半径$r_{\sigma}(A)$被定义为$r_{\sigma}(A)=\sup\{|\lambda| \ | \ \lambda \in \sigma(A)\}$。
\end{definition}

\begin{lemma}[谱映射定理]\label{lemma:pysdl}
设 $X$是 Banach 空间,$T\in B(X)$,$p$为多项式,则$\sigma(p(T))=p(\sigma(T))$。
\end{lemma}

\begin{theorem}[Gelfand 公式]\label{theorem:Gelfand}
设 $X$ 是 Banach 空间,$T\in\mathscr{B}(X)$ 是有界线性算子。谱半径$r_{\sigma}(T)$可以如下给出:
\[r_{\sigma}(T) = \lim_{n \to \infty} \|T^n\|^{1/n}\]
且等式右端的极限总是存在。
\end{theorem}
\begin{proof}
首先证明极限存在,令 $a_n = \ln \|T^n\|$。由算子范数的次可乘性 $\|T^{m+n}\| \le \|T^m\|\|T^n\|$ 取对数得$a_{m+n} \le a_m + a_n$,即$\{a_n\}$是次可加数列。对于次可加数列,有\href{https://math.fandom.com/zh/wiki/Fekete_%E5%BC%95%E7%90%86}{经典结论}:
\[\lim_{n \to \infty} \frac{a_n}{n} = \inf_{n \ge 1} \frac{a_n}{n}\]
由于$a_n/n=\ln\bigl(\|T^n\|^{1/n}\bigr)$ 且指数函数连续,故 $\lim\limits_{n \to \infty} \|T^n\|^{1/n}$ 存在,记该极限为 $r(T)$。对$\forall \lambda \in \sigma(T)$,由谱映射定理[引理\ref{lemma:pysdl}]知$\lambda^n \in \sigma(T^n)$。由定理\ref{theorem:fskz}知$|\lambda|^n \le \|T^n\|$。取 $n$ 次根得$|\lambda| \le \|T^n\|^{1/n}$。这对所有 $n$ 成立,故:
\[|\lambda| \le \inf_{n \ge 1} \|T^n\|^{1/n} = \lim_{n \to \infty} \|T^n\|^{1/n} = r(T)\]
对$\lambda \in \sigma(T)$ 取上确界得 $r_{\sigma}(T) \le r(T)$。

设$r > \rho(T)$,则$\{\lambda: |\lambda| = r\} \subset \rho(T)$,因此$(\lambda I - T)^{-1}$在$|\lambda| = r$上解析且有界。记:
\[M(r) = \max_{|\lambda| = r} \|(\lambda I - T)^{-1}\| < \infty\]
对 $|\lambda| > \|T\|$,定理\ref{theorem:nszyjxx}知$(\lambda I - T)^{-1}$有 \href{https://math.fandom.com/zh/wiki/%E8%B0%B1}{Neumann 级数展开}:
\[(\lambda I - T)^{-1} = \sum_{n=0}^{\infty} \frac{T^n}{\lambda^{n+1}}\]
该级数在算子范数下绝对收敛。由复分析的 Cauchy 积分公式,对$n \ge 0$,上述展开式的系数可由积分表示:
\[T^n = \frac{1}{2\pi i} \int_{|\lambda| = r} \lambda^n (\lambda I - T)^{-1} \, \dd\lambda \quad \Rightarrow \quad \|T^n\|\le \frac{1}{2\pi} \cdot \max_{|\lambda| = r} \|(\lambda I - T)^{-1}\| \cdot 2\pi r \cdot r^n = r^{n+1} M(r)\]
因此 $\|T^n\|^{1/n} \le r \cdot \bigl(r M(r)\bigr)^{1/n}$。令 $n \to \infty$,注意到 $\bigl(r M(r)\bigr)^{1/n} \to 1$,得$r(T) \le r$。由于$r > \rho(T)$是任意的,令$r \to \rho(T)^+$即得 $r(T) \le r_{\sigma}(T)$。综合下界与上界,得到 $r(T) = r_{\sigma}(T)$,即
\[r_{\sigma}(T) = \lim_{n \to \infty} \|T^n\|^{1/n}\]
\end{proof}

\begin{theorem}
设 $X$ 是 Banach 空间,$T\in\mathscr{B}(X)$是有界线性算子,则$\sigma(T) \neq \varnothing$。
\end{theorem}
\begin{proof}
假设$\sigma(T) = \varnothing$,对$\forall \lambda \in \mathbb{C}$,由于$\lambda \notin \sigma(T)$,算子 $\lambda I - T$ 可逆,因此$(\lambda I - T)^{-1} \in \mathscr{B}(X)$。对$\forall \lambda_0 \in \mathbb{C}$,当 $|\lambda - \lambda_0| < \|(\lambda_0 I - T)^{-1}\|^{-1}$ 时,利用 Neumann 级数展开有:
\[(\lambda I - T)^{-1}=\sum_{n=0}^{\infty} (\lambda_0 - \lambda)^n (\lambda_0 I - T)^{-n-1}\]
该级数在算子范数下收敛,故$(\lambda I - T)^{-1}$在$\lambda_0$处解析。由$\lambda_0$的任意性,$(\lambda I - T)^{-1}$在整个复平面解析,即为\href{https://math.fandom.com/zh/wiki/%E6%95%B4%E5%87%BD%E6%95%B0}{整函数}。当$|\lambda| > \|T\|$且$|\lambda| \to +\infty$时,
\[(\lambda I - T)^{-1} = \frac{1}{\lambda}\left(I - \frac{T}{\lambda}\right)^{-1}= \frac{1}{\lambda}\sum_{n=0}^{\infty}\left(\frac{T}{\lambda}\right)^n \quad \Rightarrow \quad \|(\lambda I - T)^{-1}\| \le \frac{1}{|\lambda|} \cdot \frac{1}{1 - \|T\|/|\lambda|}= \frac{1}{|\lambda| - \|T\|}\to0\]
因此$(\lambda I - T)^{-1}$在无穷远处趋于零,进而$(\lambda I - T)^{-1}$在$\mathbb{C}$上有界。由 \href{https://math.fandom.com/zh/wiki/Liouville_%E5%AE%9A%E7%90%86}{Liouville 定理},有界整函数必为常值函数。故存在常数 $C \in \mathscr{B}(X)$ 使得 $(\lambda I - T)^{-1} \equiv C$。但前面已证$\|(\lambda I - T)^{-1}\|\to0$,从而$C=0$,即$(\lambda I - T)^{-1}\equiv 0$。然而零算子不可逆,因此假设不成立,$\sigma(T)$非空。
\end{proof}

\begin{example}
\quad 设$X=l^2$,定义算子$A \in \mathscr{B}(X): \ x=(\xi_1,\xi_2,\cdots) \mapsto Ax=(0,\xi_1,\xi_2,\cdots)$,显然$||A||=1$,有$\sigma_p(A)=\varnothing \ , \ \sigma_c(A)=\{\lambda\in\mathbb{C} \ | \ ||\lambda||=1\} \ , \ \sigma_r(A)=\{\lambda\in\mathbb{C} \ | \ ||\lambda||<1\}$。
\end{example}
\begin{proof}
首先由$||A||=1$可知$\sigma(A) \subset \{\lambda \ | \ |\lambda|\leq1\}$,其次考虑方程$(\lambda\mathbbm{I}-A)x=(\lambda\xi_1,\lambda\xi_2-\xi_1,\cdots)=0$,当$\lambda\neq0$时,显然$x=0$;当$\lambda=0$时,显然$x=0$。因此$\sigma_p(A)=\varnothing$。

注意到$A$的共轭算子$A^*: \ X \to X \ , \ x=(\xi_1,\xi_2,\cdots) \mapsto A^*x=(\xi_2,\xi_3,\cdots)$,考虑算子$(\lambda\mathbbm{I}-A)=\lambda^*\mathbbm{I}-A^*$,由共轭算子的定义可知对$\forall x \in X$都$\exists \, y \in X$由关系式$((\lambda\mathbbm{I}-A)x,y)=(x,(\lambda^*\mathbbm{I}-A^*)y)$一一对应。对$\forall y \in \overline{R(\lambda\mathbbm{I}-A)}^{\perp}=R(\lambda\mathbbm{I}-A)^{\perp}=\ker(\lambda^*\mathbbm{I}-A^*)$(定理\ref{theorem:zgeker}和性质\ref{itemize:zjbb}第三点),都有$0=(\lambda^*\mathbbm{I}-A^*)y=(\lambda^*\xi_1-\xi_2,\lambda^*\xi_2-\xi_3,\cdots)$,因此$y=\xi_1(1,\lambda^*,(\lambda^*)^2,\cdots)$。只有当$|\lambda|<1$时$y \in l^2$。

当$|\lambda|<1$时,$R(\lambda\mathbbm{I}-A)^{\perp}=\ker(\lambda^*\mathbbm{I}-A^*)\neq\{0\}$,通过一一映射后按照谱的定义可知$\{|\lambda|<1\} \subset \sigma_r(A)\subset\sigma(A)$。由于$\sigma(A)$是闭集[定理\ref{theorem:fskz}],因此$\sigma(A) \supset \overline{\{|\lambda|<1\}}=\{|\lambda|\leq1\}$,结合最开头的结论可知$\sigma(A)=\{|\lambda|\leq1\}$。

当$|\lambda|=1$时,$\ker(\lambda^*\mathbbm{I}-A^*)=R(\lambda\mathbbm{I}-A)^{\perp}=\overline{R(\lambda\mathbbm{I}-A)}^{\perp}=\{0\}$,因此$\overline{R(\lambda\mathbbm{I}-A)}=X$,由$\sigma(A)=\{|\lambda|\leq1\}$可知$R(\lambda\mathbbm{I}-A)\neq X$,因此$\sigma_c(A)=\{|\lambda|=1\}$,进而可知$\sigma_r(A)=\{|\lambda|<1\}$。

或者可以直接证明当$|\lambda|=1$时$R(\lambda\mathbbm{I}-A)$不是闭的,取点列$\{x_n\}$:
\[x_n=\frac{1}{\sqrt{n}}\left(\lambda^*,(\lambda^*)^2,\cdots,(\lambda^*)^n,0,\cdots\right)\]
计算:
\[(\lambda\mathbbm{I}-A)x_n=\left(\frac{1}{\sqrt{n}},\frac{\lambda^*}{\sqrt{n}}-\frac{\lambda^*}{\sqrt{n}},\cdots,\frac{(\lambda^*)^{n-1}}{\sqrt{n}}-\frac{(\lambda^*)^{n-1}}{\sqrt{n}},-\frac{(\lambda^*)^n}{\sqrt{n}},0,\cdots\right)==\left(\frac{1}{\sqrt{n}},0,\cdots,0,-\frac{(\lambda^*)^n}{\sqrt{n}},0,\cdots\right)\]
因此其范数$||(\lambda\mathbbm{I}-A)x_n||=\sqrt{2/n} \to 0$,说明$\lambda\mathbbm{I}-A$没有正的下界,因此其值域不闭。
\end{proof}

\newpage
\section{紧算子谱}\label{jsjp}
设$X$是 Banach 空间,$A \in \mathscr{C}(X)$,如果想了解$A$的谱,我们可以考虑算子$T=\mathbbm(I)-A$,先考察有限维的情况,如取$X=\mathbb{R}^n$,考虑方程$Tx=y$。如果$T$是单射[定理\ref{theorem:djm}]那对任意给定的$y$显然方程只有唯一解。考虑$T$不是单射,我们令$\{y_1,y_2,\cdots,y_p\}$是$R(T)$的一组基,可以扩充为$\{y_1,y_2,\cdots,y_p,z_1,z_2,\cdots,z_q\}$为$X$的一组基。












\section{希尔伯特空间的自共轭紧算子谱}\label{zgejszp}