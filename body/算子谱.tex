\section{线性算子谱}
回顾有限维的情况,设$X$为$n$维的线性空间,$A: \ X \to X$是线性算子,对$\forall \lambda \in \mathbb{C}$考虑方程$(\lambda \mathbbm{I}-A)x=0$,也即$Ax=\lambda x$,有两种情况:
\begin{itemize}
\item 1. 只有零解,算子$\lambda \mathbbm{I}-A$为单射($\ker(\lambda \mathbbm{I}-A)=\{0\}$且线性),因此$\lambda \mathbbm{I}-A$可逆[定理\ref{theorem:djm}],也可称为正则,其行列式$\det(\lambda \mathbbm{I}-A)\neq0$,此时称$\lambda$为$A$的正则值;
\item 2. 有非零解,算子$\lambda \mathbbm{I}-A$不单进而不可逆,行列式$\det(\lambda \mathbbm{I}-A)=0$,此时称$\lambda$为$A$的特征值,数量为$n$。
\end{itemize}
\begin{definition}[正则算子]
设$X,Y$是赋范线性空间,定义算子$A \in \mathscr{B}(X,Y)$,若$R(A)=Y$且$A^{-1}\in \mathscr{B}(X,Y)$,则称$A$是正则算子。
\end{definition}
现在考虑无穷维时,设$X$是 Banach 空间,定义算子$A \in \mathscr{B}(X)$,对$\forall \lambda \in \mathbb{C}$考虑方程$(\lambda \mathbbm{I}-A)x=0$:
\begin{itemize}
\item 1. 如果该方程只有零解,那么$\lambda \mathbbm{I}-A$是单射,因此$\lambda \mathbbm{I}-A$“\href{https://math.fandom.com/zh/wiki/%E8%B0%B1}{可逆}”\footnote{很多地方都将可逆的扩大化了,严格意义上满射要求算子A的值域$R(A)=Y$,但如果仅在$R(A)$上那么可以宽松地说单射的算子可逆。},分三种情况:
\begin{itemize}
\item 1. $R(\lambda \mathbbm{I}-A)=X$,因此$(\lambda \mathbbm{I}-A)^{-1} \in \mathscr{B}(X)$,记$\lambda\in\rho(A)$为正则值;
\item 2. $R(\lambda \mathbbm{I}-A) \neq X$但$\overline{R(\lambda \mathbbm{I}-A)}=X$,$(\lambda \mathbbm{I}-A)^{-1}$存在但不有界,记$\lambda\in\sigma_c(A)$为连续谱;
\item 3. $\overline{R(\lambda \mathbbm{I}-A)} \neq X$,$(\lambda \mathbbm{I}-A)^{-1}$存在但无法判断是否有界,记$\lambda\in\sigma_r(A)$为连续谱。
\end{itemize}
\item 2. 如果方程有非零解,那么$\lambda \mathbbm{I}-A$不是单射,因此$\lambda \mathbbm{I}-A$不可逆,记$\lambda\in\sigma_p(A)$为点谱(特征值)。
\end{itemize}
\begin{definition}[算子的谱]
定义算子$A$的谱为$\sigma(A)=\mathbb{C}-\rho(A)=\sigma_p(A)\cup\sigma_c(A)\cup\sigma_r(A)$。
\end{definition}

\begin{example}
\quad 在$X=L^2[0,1]$,上利用 \href{https://math.fandom.com/zh/wiki/Fourier_%E7%BA%A7%E6%95%B0}{Fourier 分解}定义闭算子[定义\ref{definition:bsz}] $A: \ X \to X$:
\[u(t) \quad \mapsto \quad -\dv[2]{u}{t}=-\dv[2]{t}\sum_{n=-\infty}^{\infty}e^{2i\pi nt}\int_0^1u(t)e^{2i\pi nt}\dd{t}=\sum_{n=-\infty}^{\infty}(2n\pi)^2e^{2i\pi nt}\int_0^1u(t)e^{2i\pi nt}\dd{t}\]
显然$\sigma(A)=\sigma_p(A)=\{(2n\pi)^2 \ | \ n \in \mathbb{Z}\}$,对应特征空间为$\{e^{2i\pi nt} \ | \ n\in\mathbb{Z}\}$。
\end{example}

\begin{example}\label{example:6.2}
\quad 在$X=C[0,1]$上定义算子 $A: \ X \to X \ , \ u(t) \mapsto tu(t)$,证明$\sigma(A)=\sigma_r(A)=[0,1]$。
\end{example}
\begin{proof}
对$\forall \lambda \in \mathbb{C}$都有$(\lambda\mathbbm{I}-A)u(t)=(\lambda-t)u(t)=0$可知方程只有零解$u(t)=0$,因此$(\lambda\mathbbm{I}-A)^{-1}$存在,若$\lambda \notin [0,1]$,显然有$(\lambda\mathbbm{I}-A)^{-1}u=u(t)/(\lambda-t)$,因此:
\[||(\lambda\mathbbm{I}-A)^{-1}u|| \leq \sup_{t \in [0,1]}\frac{||u||}{\lambda-t}\]
进而可知$(\lambda\mathbbm{I}-A)^{-1} \in \mathscr{B}(X)$,即$\lambda$正则,可知$\sigma(A) \subset [0,1]$。若$\lambda \in [0,1]$,则$\overline{R(\lambda\mathbbm{I}-A)}=\overline{\{(\lambda-t)u(t) \ | \ u \in X\}}$,因为$\lambda-t \in X$,可知$\overline{R(\lambda\mathbbm{I}-A)}=\overline{\{u(t) \ | \ u \in X \ , \ u(\lambda)=0\}} \subset C[0,1]$,因此$\sigma(A)=\sigma_r(A)=[0,1]$。
\end{proof}

\begin{example}
\quad 在$X=L^2[0,1]$上类似上一个例子\ref{example:6.2}定义算子 $A: \ X \to X \ , \ u(t) \mapsto tu(t)$,则$\sigma(A)=\sigma_c(A)=[0,1]$。
\end{example}

\begin{theorem}
设$X$是 Banach 空间,$A \in \mathscr{B}(X)$,则:
\begin{itemize}
\item 1. $\rho(A)$是开集,也即$\sigma(A)$是闭集;
\item 2. $\sigma(A) \subset \{\lambda \in \mathbb{C} \ | \ |\lambda| \leq ||A||\}$或记为$r_{\sigma}(A) \leq ||A||$。
\end{itemize}
\end{theorem}
\begin{proof}
由定理\ref{theorem:deltazz}对$\forall \lambda \in \rho(A)$,都$\exists \, \delta=||\lambda\mathbbm{I}-A||^{-1}$,使得对$\forall a<\delta$都有$(\lambda+a)\mathbbm{I}-A=(\lambda\mathbbm{I}-A)+a\mathbbm{I}$正则,即对$\forall \lambda \in \rho(A)$其开球$\{\lambda+a \ | \ |a|<\delta\} \subset \rho(A)$,因此$\rho(A)$是开集,第一部分得证。

对$\forall \lambda \in \mathbb{C}$满足$|\lambda|>||A||$,由定理\ref{theorem:nszyjxx}知$\lambda\mathbbm{I}-A=\lambda(\mathbbm{I}-A/\lambda)$正则,因此$\sigma(A) \subset \{\lambda \in \mathbb{C} \ | \ |\lambda| \leq ||A||\}$。
\end{proof}

\section{紧算子谱}
\section{希尔伯特空间的自共轭紧算子谱}