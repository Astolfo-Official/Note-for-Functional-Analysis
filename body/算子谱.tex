\section{线性算子谱}
回顾有限维的情况,设$X$为$n$维的线性空间,$A: \ X \to X$是线性算子,对$\forall \lambda \in \mathbb{C}$考虑方程$(\lambda \mathbbm{I}-A)x=0$,也即$Ax=\lambda x$,有两种情况:
\begin{itemize}
\item 1. 只有零解,算子$\lambda \mathbbm{I}-A$为单射($\ker(\lambda \mathbbm{I}-A)=\{0\}$且线性),因此$\lambda \mathbbm{I}-A$可逆[定理\ref{theorem:djm}],也可称为正则,其行列式$\det(\lambda \mathbbm{I}-A)\neq0$,此时称$\lambda$为$A$的正则值;
\item 2. 有非零解,算子$\lambda \mathbbm{I}-A$不单进而不可逆,行列式$\det(\lambda \mathbbm{I}-A)=0$,此时称$\lambda$为$A$的特征值,数量为$n$。
\end{itemize}
\begin{definition}[正则算子]
设$X,Y$是赋范线性空间,定义算子$A \in \mathscr{B}(X,Y)$,若$R(A)=Y$且$A^{-1}\in \mathscr{B}(X,Y)$,则称$A$是正则算子。
\end{definition}
现在考虑无穷维时,设$X$是 Banach 空间,定义算子$A \in \mathscr{B}(X)$,对$\forall \lambda \in \mathbb{C}$考虑方程$(\lambda \mathbbm{I}-A)x=0$:
\begin{itemize}
\item 1. 如果该方程只有零解,那么$\lambda \mathbbm{I}-A$是单射,因此$\lambda \mathbbm{I}-A$“可逆”\footnote{很多地方都将可逆的扩大化了,严格意义上满射要求算子A的值域$R(A)=Y$,但如果仅在$R(A)$上那么可以宽松地说单射的算子可逆。},分三种情况:
\begin{itemize}
\item 1. $R(\lambda \mathbbm{I}-A)=X$,因此$(\lambda \mathbbm{I}-A)^{-1} \in \mathscr{B}(X)$,记$\lambda\in\rho(A)$为正则值;
\item 2. $R(\lambda \mathbbm{I}-A) \neq X$但$\overline{R(\lambda \mathbbm{I}-A)}=X$,$(\lambda \mathbbm{I}-A)^{-1}$存在但不有界,记$\lambda\in\sigma_c(A)$为连续谱;
\item 3. $\overline{R(\lambda \mathbbm{I}-A)} \neq X$,$(\lambda \mathbbm{I}-A)^{-1}$存在但无法判断是否有界,记$\lambda\in\sigma_r(A)$为连续谱。
\end{itemize}
\item 2. 如果方程有非零解,那么$\lambda \mathbbm{I}-A$不是单射,因此$\lambda \mathbbm{I}-A$不可逆,记$\lambda\in\sigma_p(A)$为点谱(特征值)。
\end{itemize}
\begin{definition}[算子的谱]
定义算子$A$的谱为$\sigma(A)=\mathbb{C}-\rho(A)=\sigma_p(A)\cup\sigma_c(A)\cup\sigma_r(A)$。
\end{definition}
\section{紧算子谱}
\section{希尔伯特空间的自共轭紧算子谱}