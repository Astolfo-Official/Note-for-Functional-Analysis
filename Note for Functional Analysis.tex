\documentclass[lang=cn,10pt]{elegantbook}
\usepackage{physics}
\usepackage{unicode-math}
\usepackage{bbm}
\usepackage{float}
\title{Note for Functional Analysis}

\author{Astolfo}
\version{1.0}

\extrainfo{Too young too simple, sometimes naive.}

\setcounter{tocdepth}{3}

\cover{cover.jpg}

% 本文档命令
\usepackage{array}
\newcommand{\ccr}[1]{\makecell{{\color{#1}\rule{1cm}{1cm}}}}

% 修改标题页的橙色带
% \definecolor{customcolor}{RGB}{32,178,170}
% \colorlet{coverlinecolor}{customcolor}

\begin{document}

\maketitle
\frontmatter


\mainmatter

\chapter*{来自\textit{Astolfo}}
\setcounter{page}{1}
\pagenumbering{roman}
这份讲义是\textit{Astolfo}在大二下旁听\&大三下重修(悲)泛函分析时顺手整理的笔记,\textit{Astolfo}试图通过这种方法让自己好好学习,虽然是是为了打好数理基础,但是似乎打好数理基础还不如多学学编程对工作有用,更多是作为智力运动防止大脑退化罢了(

这份讲义也将作为本人随笔的一部分加入《\textit{Astolfo}的量子化学》系列笔记中,作为对\href{https://github.com/Astolfo-Official/Strange-handouts/blob/main/%E6%80%9D%E6%98%8E%E6%B5%B7%E6%B4%8B%E8%81%8C%E4%B8%9A%E6%8A%80%E6%9C%AF%E5%AD%A6%E9%99%A2%E6%B1%82%E7%94%9F%E6%8C%87%E5%8D%97.pdf}{SMU花梨I讲义}第一小节的数学结构补充。

但是在重修完之后忙于学业暂时搁置了该项目,同时由于大三到大四因为本人不爱学英语导致升学压力巨大,差点没书读。然后就到了博一,又因为神秘学校 PhD 要上两年课和老板的任务(大嘘)繁忙,导致这次大鸽一共鸽了两年半。

在第三学期之后趁着放假回了趟家,当时没啥事干开始修改此前写的花梨I的讲义,才想起来还有这个烂尾工程没完成,遂在博二的第二学期开始动工,上班时间带薪摸鱼写自己的稿子,属于是薅资本主义羊毛了(喜

在最后冲刺本讲义的过程中正值 2026 年春节,遂把下面这张大草春联贴予大家赏玩。最后祝各位学有所成,新年多发新 paper 多拿新 offer。

\begin{figure}[htbp]
    \center
    \includegraphics[scale=0.3]{./figure/zhenghuo.jpg}
\end{figure}
\begin{flushright}
    \textit{Astolfo} \\ 2022年2月——2026年3月
\end{flushright}
\tableofcontents
\thispagestyle{empty}
\newpage
\setcounter{page}{1}
\pagenumbering{arabic}
\chapter{正式内容前的瞎扯}
回顾常微分方程中的提到的$Picard$定理:

\paragraph*{$Picard$定理} \quad 对初值问题
\[\dv{}{t}x(t)=f(x(t),t) \qquad x(t_0)=x_0 \tag{1-1}\]
如果$f(x(t),t)$在区域$D$上满足$Lipschitz$条件($|f(x_1,t)-f(x_2,t)| \leq L|x_1-x_2|,L>0$对$\forall t \in D$都成立,其实只要在$t_0$的邻域成立即可),
则$\exists \, \varepsilon > 0$,初值问题在$t \in (t_0-\varepsilon,t_0+\varepsilon)$有唯一解$x(t)$。

回顾初等的证法,原初值问题等价于证明积分方程:
\[x(t)=x_0+\int_{t_0}^tf(x(s),s)\dd s \tag{1-2}\]

为证明上述积分方程的存在唯一性,我们可以构造如下函数列$\{x_n(t)\}$(又称$Picard$序列):
\[\begin{aligned}
x_1(t)&=x_0+\int_{t_0}^tf(x_0(s),s)\dd s\\
x_2(t)&=x_0+\int_{t_0}^tf(x_1(s),s)\dd s\\
&\vdots \\
x_{n+1}(t)&=x_0+\int_{t_0}^tf(x_n(s),s)\dd s
\end{aligned}\]
若证得函数列$\{x_n(t)\}$收敛,则其一定会收敛于$x(t)$,即证明初值问题(1)。证明如下:
\begin{equation*}
    \begin{aligned}
        |x_{n+1}-x_n| & =\left|\int_{t_0}^tf(x_n(s),s)\dd s-\int_{t_0}^tf(x_{n-1}(s),s)\dd s\right| \leq \int_{t_0}^t \left| f(x_n(s),s)-f(x_{n-1}(s),s) \right| \dd s \\
        & \leq L\int_{t_0}^t \left| x_n(s)-x_{n-1}(s) \right| \dd s
    \end{aligned}
\end{equation*}
在$t_0$的邻域$(t_0-\varepsilon,t_0+\varepsilon)$上,我们容易得到(取上下确界不改变不等号):
\[|x_{n+1}-x_n| \leq {\mathop {\text{sup}}\limits_{t \in (t_0-\varepsilon,t_0+\varepsilon)}} (L\varepsilon)|x_{n}-x_{n-1}| \leq \cdots \leq {\mathop {\text{sup}}\limits_{t \in (t_0-\varepsilon,t_0+\varepsilon)}} (L\varepsilon)^n|x_1-x_0|\]
由于在$(t_0-\varepsilon,t_0+\varepsilon)$上$f(x(t),t)$满足$Lipschitz$条件,容易得到$f(x(t),t)$在该区间一致连续,故下式有限:
\[|x_1-x_0|=|\int_{t_0}^tf(x_0(s),s)\dd s|<+\infty\]
则:
\[|x_{n+1}-x_n| \leq (L\varepsilon)^nM \ , \ M:={\mathop {\text{sup}}\limits_{t \in (t_0-\varepsilon,t_0+\varepsilon)}} |x_1-x_0| <+\infty\]
进而我们可以得到:
\[|x_m(t)-x_n(t)| \leq M\sum_{i=n}^{m-1}(L\varepsilon)^i\]
取$\varepsilon \leq 1/2L$:
\[|x_m(t)-x_n(t)| \leq M\sum_{i=n}^{m-1}(L\varepsilon)^i \leq M\sum_{i=n}^{m-1}\frac{1}{2^i} < \frac{M}{2^{n-1}}\]
上式两边取极限即证明函数列$\{x_n(t)\}$为一致柯西列,且
\[x_n(t) \rightrightarrows x(t)\]
这时我们回到积分方程
\[x_{n+1}(t)=x_0+\int_{t_0}^tf(x_n(s),s)\dd s\]
对两边取极限,由于$f(x(t),t)$在$(t_0-\varepsilon,t_0+\varepsilon)$上一致连续,积分于极限可交换,(2)得证,进而原初值问题得证。
\[x(t)=\lim_{n \rightarrow +\infty}\left(x_0+\int_{t_0}^tf(x_n(s),s)\dd s\right)=x_0+\int_{t_0}^tf \left (\lim_{n \rightarrow +\infty}x_n(s),s \right )\dd s=x_0+\int_{t_0}^tf(x(s),s)\dd s\]

如果从泛函分析的观点出发,我们可做出如下不严格的说明:

原命题可以写成,取一个连续函数$x(t)$(这里连续意味着可导)满足:
\[x(t)=x_0+\int_{t_0}^tf(x(s),s)\dd s\]

我们记$t_0$的邻域$(t_0-\varepsilon,t_0+\varepsilon)$上全体连续函数的集合为$C(t_0-\varepsilon,t_0+\varepsilon)$,定义一个映射$T$:
\[T \ : \ C(t_0-\varepsilon,t_0+\varepsilon) \rightarrow C(t_0-\varepsilon,t_0+\varepsilon)\]
\[T[x(t)]=x_0+\int_{t_0}^tf(x(s),s)\dd s\]
这时我们的目标就变成了找一个$x(t)$满足$T[x(t)]=x(t)$,即不动点。

我们继续定义$C(t_0-\varepsilon,t_0+\varepsilon)$上的距离(函数空间上的距离):
\[d(x(t),y(t))={\mathop {\text{sup}}\limits_{t \in (t_0-\varepsilon,t_0+\varepsilon)}} |x(t)-y(t)|\]
则:
\[d(T[x(t)],T[y(t)])={\mathop {\text{sup}}\limits_{t \in (t_0-\varepsilon,t_0+\varepsilon)}} \left|\int_{t_0}^tf(x(s),s)\dd s-\int_{t_0}^tf(y(s),s)\dd s \right| \leq L\varepsilon d(x(t),y(t))\]
当取$L\varepsilon < 1$时,我们得到一个压缩映射,由$Banach$不动点定理即证明完毕:
\[d(T[x(t)],T[y(t)]) < d(x(t),y(t))\]

很显然上述只能叫说明,我们并没有定义一个$Banach$不动点定理适用的距离空间出来。不过也足以说明在泛函分析中我们比较感兴趣的两个对象:函数空间和函数空间上的映射。
在上述的说明中,我们通过距离这个几何概念诱导出了这个函数空间的拓扑结构,当然除此之外还有许多其他的拓扑结构也值得讨论,我们在泛函分析中一般默认讨论的函数空间的代数结构是无穷维线性空间。

还有一个值得思考的问题,我们在初等的证明中使用了柯西收敛定理,那么,在函数空间中柯西列一定收敛吗?
答案是否定的,函数空间中只有完备函数空间的柯西列才是收敛的,这个在后续的学习中会涉及。
\chapter{距离空间与拓扑空间}
\begin{introduction}
    \item 距离空间~\ref{def:Metric}
    \item 距离空间的点集~\ref{pointset}
    \item 距离空间的完备性~\ref{complete}
    \item Baire定理~\ref{the:Baire}
    \item 压缩映射定理~\ref{zip}
    \item 拓扑空间~\ref{topo}
    \item 距离空间上的紧集~\ref{compact}
    \item Arzela-Ascoli定理~\ref{the:AA}
  \end{introduction}
\section{距离空间 Metric Space}
\subsection{定义及举例}
事实上,我们可以通过定义距离这个几何上的概念很好地定义收敛和连续这两个我们关心的分析上的概念。
例如,在欧氏空间$\mathbb{R}^n=\{\overrightarrow{x}=(x_1,x_2,\cdots,x_n)|x_i \in \mathbb{R} \ (i=1,2,\cdots,n)\}$上,我们定义距离:
\[d(\overrightarrow{x},\overrightarrow{y})=\sqrt{\sum_{i=1}^n\left|x_i-y_i\right|^2}\]
通过距离$d$,我们可以定义收敛和连续:
\begin{proposition}
\begin{itemize}
\item 1. \textbf{收敛性}:$\overrightarrow{x}_m \rightarrow \overrightarrow{x} \ \Leftrightarrow \ d(\overrightarrow{x},\overrightarrow{y}) \rightarrow 0 \ \Leftrightarrow \ x_{m,i} \rightarrow x_i, \ (i=1,2,\cdots,n)$
\item 2. \textbf{连续性}:定义函数$f:\mathbb{R}^n \rightarrow \mathbb{R}^m$,如果$f$满足
\[\overrightarrow{x}_0 \in \mathbb{R}^n\ , \ \forall \varepsilon >0 \ , \ \exists \, \delta >0 \ , \quad \text{s.t.} \quad d_{\mathbb{R}^n}(\overrightarrow{x},\overrightarrow{x}_0)<\delta \ , \ d_{\mathbb{R}^m}(f(\overrightarrow{x}),f(\overrightarrow{x}_0))<\varepsilon\]
则称$f$在点$\overrightarrow{x}_0$连续,进而可以推广到处处连续。
\end{itemize}
\end{proposition}

从欧氏空间出发,我们可以定义抽象距离空间:
\begin{definition}[抽象距离空间] \label{def:Metric}
$X$是一个非空集合,$d$是一个定义在$X \times X$上的函数,满足:
\begin{itemize}
    \item 1. 非负性:$d(x,y) \geq 0$且$d(x,y)=0$当且仅当$x=y$;
    \item 2. 对称性:$d(x,y)=d(y,x)$;
    \item 3. 三角不等式:$d(x,y) \leq d(x,z)+d(z,y)$。
\end{itemize}
\end{definition}
\begin{example} \quad 连续函数空间$C[a,b]$,定义距离
\[d(x(t),y(t))=\mathop{\text{sup}}\limits_{t \in [a,b]}\left|x(t)-y(t)\right|\]
\end{example}
\begin{proof}
前两个性质显然成立,主要证第三条,由三角不等式恒成立:
\[\left|x(t)-y(t)\right| \leq \left|x(t)-z(t)\right|+\left|z(t)-y(t)\right|\]
两边取上确界即证毕:
\[\mathop{\text{sup}}\limits_{t \in [a,b]}\left|x(t)-y(t)\right| \leq \mathop{\text{sup}}\limits_{t \in [a,b]}\left|x(t)-z(t)\right|+\mathop{\text{sup}}\limits_{t \in [a,b]}\left|z(t)-y(t)\right| \quad \Leftrightarrow \quad d(x,y) \leq d(x,z)+d(z,y)\]
\end{proof}

\begin{example} \quad 空间$s=\{x=\{\xi_i\}_{i=1}^{+\infty} \ , \ \xi_i \in \mathbb{R}\}$($x,y,z$对应的分量符号$\xi,\eta,\psi$),定义距离
\[d(x,y)=\sum_{i=1}^{+\infty}\frac{1}{2^i}\frac{|\xi_i-\eta_i|}{1+|\xi_i-\eta_i|}\]
\end{example}
\begin{proof}
前两条性质显然,下证第三条,由三角不等式:
\begin{equation*}
\begin{aligned}
\sum_{i=1}^{+\infty}\frac{1}{2^i}\frac{|\xi_i-\eta_i|}{1+|\xi_i-\eta_i|} & \leq \sum_{i=1}^{+\infty}\frac{1}{2^i}\frac{\left|\xi_i-\psi_i\right|+\left|\psi_i-\eta_i\right|}{1+\left|\xi_i-\psi_i\right|+\left|\psi_i-\eta_i\right|} \\
& =\sum_{i=1}^{+\infty}\frac{1}{2^i}\frac{\left|\xi_i-\psi_i\right|}{1+\left|\xi_i-\psi_i\right|+\left|\psi_i-\eta_i\right|}+\sum_{i=1}^{+\infty}\frac{1}{2^i}\frac{\left|\psi_i-\eta_i\right|}{1+\left|\xi_i-\psi_i\right|+\left|\psi_i-\eta_i\right|} \\
& \leq \sum_{i=1}^{+\infty}\frac{1}{2^i}\frac{\left|\xi_i-\psi_i\right|}{1+\left|\xi_i-\psi_i\right|}+\sum_{i=1}^{+\infty}\frac{1}{2^i}\frac{\left|\psi_i-\eta_i\right|}{1+\left|\psi_i-\eta_i\right|}=d(x,z)+d(z,y)
\end{aligned}
\end{equation*}
\end{proof}

\begin{example} \quad 空间$S(E) \ E \subseteq \mathbb{R}$是$E$上$Lebesgue$可积函数全体,定义距离
\[d(x,y)=\int_E\frac{\left|x(t)-y(t)\right|}{1+\left|x(t)-y(t)\right|}\dd t\]
\end{example}
\begin{proof}
证明类似例2。顺便一说,在这个距离定义下的收敛等价于依测度收敛,下一小节会证明。
\end{proof}

\begin{example} \quad 空间$L^2(E)=\{f\text{在$E$上可测} \ , \ \int_E|f|^2<+\infty\}$上,我们定义距离
\[d(x,y)=\sqrt{\int_E|x(t)-y(t)|^2\dd t}\]
\end{example}
\begin{proof}
同样,前两条性质显然,下证第三条:
\begin{equation*}
    \begin{aligned}
        d^2(x,y) & =\int_E|x(t)-y(t)|^2\dd t=\int_E[x^2(t)+y^2(t)-2x(t)y(t)]\dd t \\
        & \leq \int_Ex^2(t)\dd t+\int_Ey^2(t)\dd t+2\int_E|x(t)||y(t)|\dd t \leq \int_Ex^2(t)\dd t+\int_Ey^2(t)\dd t+2\sqrt{\int_Ex^2(t)\dd t}\sqrt{\int_Ey^2(t)\dd t} \\
        & \leq \left(\sqrt{\int_Ex^2(t)\dd t}+\sqrt{\int_Ey^2(t)\dd t}\right)^2
    \end{aligned}
\end{equation*}
此时,如果我们令$x=f-h \ , \ y=g-h$,则上式可改写为
\[d^2(f,g) \leq (d(f,h)+d(g,h))^2\]
原不等式即证。欧氏空间是距离空间也可类似证明。
\end{proof}

\subsection{距离空间的性质}
在上一小节我们介绍了距离空间以及距离的概念,我们可以很自然地引入一般度量意义下的收敛。
\begin{proposition}
\begin{itemize}
\item 1. 距离空间$(X,d)$上的收敛:我们称$x_n$收敛于$x$,当其满足
\[\lim_{n \rightarrow \infty}d(x_n,x)=0\]
\item 2. 逐点收敛:$f_n(x) \rightarrow f(x) \ , \ x \in [a,b]$
\[\forall x \in [a,b] \ , \ \forall \varepsilon>0 \ , \ \exists \, N(\varepsilon,x)>0 \ , \quad \text{s.t.} \quad n \geq N \ , \ |f_n(x)-f(x)|<\varepsilon\]
\item 3. 一致收敛:$f_n(x) \rightrightarrows f(x) \ , \ (n \rightarrow \infty)$
\[\forall x \in [a,b] \ , \ \forall \varepsilon>0 \ , \ \exists \, N(\varepsilon)>0 \ , \quad \text{s.t.} \quad n \geq N \ , \ |f_n(x)-f(x)|<\varepsilon\]
\item 4. 依测度收敛:$f_n(x) \rightarrow f(x)\textbf{ in measure} \ , \ (n \rightarrow \infty)$
\[\forall \sigma>0 \ , \ \lim_{n \rightarrow \infty}m\{|f_n(x)-f(x)| \geq \sigma\}=0\]
\end{itemize}
\end{proposition}
逐点收敛和一致收敛的区别仅在$N$是否与$x$有关,现在我们来考察一下这些收敛跟距离的关系。我们知道距离这个概念能诱导出收敛,但是收敛并不意味着一定有距离意义。很显然,逐点收敛没有对应的距离意义,因为$N$与$x$有关。其他两种收敛可以对应的距离空间上的收敛。
\begin{example} \quad 连续函数空间$C[a,b]$上距离的收敛等价于一致收敛
\[d(x(t),y(t))=\mathop {\text{sup}}\limits_{t \in [a,b]}|x(t)-y(t)|\]
\end{example}
\begin{proof}
在连续函数空间$C[a,b]$上的收敛$x_n(t) \rightarrow x(t), \ (n \rightarrow \infty)$:
\[\lim_{n \rightarrow \infty}d(x_n(t),x)=\lim_{n \rightarrow \infty}\mathop {\text{sup}}\limits_{t \in [a,b]}|x_n(t)-x(t)|=0 \quad \Leftrightarrow \quad \forall \varepsilon>0 \ , \ \exists \, N(\varepsilon)>0 \quad \text{s.t.} \quad n \geq N \ , \ \mathop {\text{sup}}\limits_{t \in [a,b]}|x_n(t)-x(t)|<\varepsilon\]
由于取 sup,$N$与自变量$t$无关,即
\[\forall t \in [a,b] \ , \ \forall \varepsilon>0 \ , \ \exists \, N(\varepsilon)>0 \ , \quad \text{s.t.} \quad n \geq N \ , \ |x_n(t)-x(t)|<\varepsilon\]
\end{proof}

\begin{example} \quad 
空间$S(E) \ , \ E \subseteq \mathbb{R}$是$E$上$Lebesgue$可积函数全体上的收敛等价于依测度收敛
\[d(x,y)=\int_E\frac{\left|x(t)-y(t)\right|}{1+\left|x(t)-y(t)\right|}\dd t\]
\end{example}
\begin{proof} 距离收敛 $\Rightarrow$ 依测度收敛
\[\lim_{n \rightarrow \infty}d(x_n(t),x(t))=\lim_{n \rightarrow \infty}\int_E\frac{\left|x_n(t)-x(t)\right|}{1+\left|x_n(t)-x(t)\right|}\dd t=0\]
记$M=\{|x_n(t)-x(t)| \geq \sigma\}\cap E$:
\[\int_E\frac{\left|x_n(t)-x(t)\right|}{1+\left|x_n(t)-x(t)\right|}\dd t \geq \int_M\frac{\left|x_n(t)-x(t)\right|}{1+\left|x_n(t)-x(t)\right|}\dd t \geq \int_M\frac{\sigma}{1+\sigma}\dd t=\frac{\sigma}{1+\sigma}m(M)\]
则有
\[0 \leq m(M) \leq \frac{1+\sigma}{\sigma}d(x_n(t),x(t))\rightarrow 0, \ (n \rightarrow \infty) \quad \Leftrightarrow \quad f_n(x) \rightarrow f(x)\textbf{ in measure} \ , \ (n \rightarrow \infty)\]
依测度收敛 $\Rightarrow$ 距离收敛
\begin{equation*}
\begin{aligned}
\int_E\frac{\left|x_n(t)-x(t)\right|}{1+\left|x_n(t)-x(t)\right|}\dd t & =\int_M\frac{\left|x_n(t)-x(t)\right|}{1+\left|x_n(t)-x(t)\right|}\dd t+\int_{E-M}\frac{\left|x_n(t)-x(t)\right|}{1+\left|x_n(t)-x(t)\right|}\dd t \\
& \leq m(M)+\frac{\sigma}{1+\sigma}m(E-M) \leq m(M)+\frac{\sigma}{1+\sigma}m(E)
\end{aligned}
\end{equation*}
\[\forall \varepsilon>0 \ , \ let \ \sigma=\frac{\varepsilon}{m(E)} \ (\text{假设} \ m(E)>1) \ , \ \exists \, N>0 \quad \text{s.t.} \quad n \geq N \ , \ m(M)<\varepsilon \ , \ d(x_n(t),x(t)) \leq \varepsilon+\frac{\varepsilon}{1+\sigma}<2\varepsilon\]
\[\Leftrightarrow \quad \lim_{n \rightarrow \infty}d(x_n(t),x(t))=0\]
\end{proof}

距离空间还有其他的性质如下:
\begin{lemma}
\begin{itemize}
\item 1. 极限存在必定唯一;
\item 2. $x_n \rightarrow x$,则$\{x_{n,i}\} \subset \{x_n\} \quad \text{s.t.} \quad x_{n,i} \rightarrow x$ (距离意义下的收敛);
\item 3. $x_n \rightarrow x$,$y_n \rightarrow y$,则有$d(x_n,y_n) \rightarrow d(x,y)$。
\end{itemize}
\end{lemma}
\begin{proof}
前两个性质容易得到,我们下证第三条,由三角不等式
\[d(x_n,y_n) \leq d(x_n,x)+d(x,y)+d(y,y_n) \Leftrightarrow d(x_n,y_n)-d(x,y) \leq d(x_n,x)+d(y_n,y)\]
\[d(x,y) \leq d(x,x_n)+d(x_n,y_n)+d(y_n,y) \Leftrightarrow d(x_n,y_n)-d(x,y) \geq d(x_n,x)+d(y_n,y)\]
\[\Rightarrow \ |d(x_n,y_n)-d(x,y)|=d(x_n,x)+d(y_n,y) \ \rightarrow 0 \ (n \rightarrow \infty)\]
\end{proof}

我们也可以定义距离空间上的连续。
\begin{definition}[距离空间$(X,d)$上的连续]
设$f:(X,d_X) \rightarrow (Y,d_Y)$是距离空间中的映射。我们称$f$在$x_0$处连续,如果其满足
\[\forall x_0 \in X \ , \ \forall \varepsilon>0 \ , \ \exists \, \delta>0 \ , \quad \text{s.t.} \quad d_Y(f(x),f(x_0))<\varepsilon \ , \ if \ d_X(x,x_0)<\delta\]
\end{definition}
定义了单点连续后我们容易得到如果$f$在任意$x_0 \in X$上都连续我们就称$f$在$X$上连续。

还有一个比较值得探讨的东西是等距映射。
\begin{definition}[等距映射]
设$f:(X,d_X) \rightarrow (Y,d_Y)$是距离空间中的映射,若满足
\[\forall x,y \in X \ , \ d_Y(f(x),f(y))=d_X(x,y)\]
则称$f$为等距映射,称$(X,d_X)$与$(Y,d_Y)$等距(isometry)。
\end{definition}
显然,等距映射是连续且单射的,且$x \rightarrow f(x)$是双射。

举个例子,$\mathbb{R}^n \rightarrow \mathbb{R}^n$的等距映射有且仅有合同变换一种$f(\overrightarrow{x})=A\overrightarrow{x}+\overrightarrow{x}_0, \ A \in O(n) \ (\text{正交矩阵})$。

\subsection{距离空间的点集} \label{pointset}
这一小节主要是想说明一般度量空间内的如下几个定义:开集、内部、闭集、闭包、稠密性、可分性。这些概念在拓扑学中有更广泛的定义。首先我们想在度量空间内定义开集,我们首先需要定义开球:
\begin{definition}[开球]
我们称$B_r(x)=S(x,r)=\{y \in X \ , \ d(x,y)<r\}$为以$x$为球心$r$为半径的开球。
\end{definition}
\begin{definition}[开集]
设$A \subset X$,如果$A$满足$\forall x \in A \ , \ \exists \, \varepsilon>0 \quad \text{s.t.} \quad B_{\varepsilon}(x) \subset A$,则称$A$为$X$中的开集。
\end{definition}
进而,我们还能定义内点和内部。
\begin{definition}[内点]
设$A \subset X$是开集,若$x \in X \ , \ \exists \, \varepsilon>0 \quad \text{s.t.} \quad B_{\varepsilon}(x) \subset A$,则称$x$是$A$的内点。
\end{definition}
\begin{definition}[内部]
$A$的全部内点构成的集合称为$A$的内部,记作$\mathring{A}$。
\end{definition}
容易看出$A$是开集$ \ \Leftrightarrow \ A=\mathring{A}$。下面我们来看看开集的基本性质。
\begin{lemma}\label{theorem:2.2}
设$(X,d)$是距离空间,则有:
\begin{itemize}
    \item 1. $X$,$\varnothing$是开集;
    \item 2. 任意多个开集的并是开集;
    \item 3. 有限多个开集的交是开集。
\end{itemize}
\end{lemma}
\begin{proof}
前两个性质是显然的,下面我们来看看第三个性质的证明。假设$G_i \ (i=1,2,\cdots,k)$都是开集,
\[\bigcap_{i=1}^kG_i \neq \varnothing \ \Leftrightarrow \ x \in \bigcap_{i=1}^kG_i  \quad \text{且} \quad \exists \, \varepsilon_i >0 \ , \ B_{\varepsilon_i}(x) \subset G_i \quad (i=1,2,\cdots,k)\]
则,我们可以取$\varepsilon=\text{min}\{\varepsilon_1,\varepsilon_2,\cdots,\varepsilon_k\}$使得$B_{\varepsilon}(x) \subset B_{\varepsilon_i}(x) \subset G_i, \ (i=1,2,\cdots,k)$,即
\[B_{\varepsilon}(x) \subset \bigcap_{i=1}^kG_i\]
\end{proof}

为什么必须是有限个开集的交?从上面取$\varepsilon$我们应该能看出来,如果是取无限个开集的交,$\varepsilon$中的min将无法取到,而应取inf,而取inf的后果将有可能使得$\varepsilon=0$,让后续证明失效。
\begin{definition}[接触点]
设$A \subset X$是开集,若$x \in X \ , \ \forall \varepsilon>0 \quad \text{s.t.} \quad B_{\varepsilon}(x) \cap A \neq \varnothing$,则称$x$是$A$的接触点。
\end{definition}

当$x \in A$时,$x$一定是$A$的接触点,$A$的接触点不一定属于$A$。

\begin{definition}[闭包]
$A$的全部接触点构成的集合称为$A$的闭包,记作$\overline{A}$。
\end{definition}
\begin{definition}[闭集]
\begin{itemize}
\item 定义1:若$A=\overline{A}$则称$A$为闭集;
\item 定义2:若$A^c$为开集,则$A$为闭集。
\end{itemize}
\end{definition}

下面我们来看看闭包的性质。
\begin{theorem}
设$(X,d)$是距离空间,$A,B \subset X$,则有$1. \ \overline{\varnothing}=\varnothing \, ; \quad 2. \ A \subset \overline{A} \, ; \quad 3. \ \overline{\overline{A}}=\overline{A} \, ; \quad 4. \ \overline{A \cup B}=\overline{A} \cup \overline{B}$。
\end{theorem}
\begin{proof}前两点是显然的,我们来证明一下第三第四点。
\begin{itemize}
\item 3. 假设$x \in \overline{\overline{A}}$,即$\forall \varepsilon>0 \ , \ B_{\varepsilon}(x) \cap \overline{A} \neq \varnothing$
\[\Rightarrow \ \exists \, y \in B_{\varepsilon}(x) \cap \overline{A} \ \Rightarrow \ B_{\varepsilon}(y) \cap A \neq \varnothing \ \Rightarrow \ B_{\varepsilon}(y) \subset B_{2\varepsilon}(x) \ \Rightarrow \ B_{2\varepsilon}(x) \cap A \neq \varnothing \ \Rightarrow \ x \in \overline{A}\]
\item 4. 设$x \in \overline{A \cup B}$,即$\forall \varepsilon>0 \ , \ B_{\varepsilon}(x) \cap (A \cup B) \neq \varnothing \ \Rightarrow \ B_{\varepsilon}(x) \cap A \neq \varnothing \ \text{or} \ B_{\varepsilon}(x) \cap B \neq \varnothing$,若$x \notin \overline{A} \cup \overline{B}$则
\[\exists \, \varepsilon_1,\varepsilon_2>0 \quad \text{s.t.} \quad B_{\varepsilon_1}(x) \cap A=\varnothing \ , \ B_{\varepsilon_2}(x) \cap B=\varnothing\]
进而我们可以取
\[\varepsilon=\text{min}\{\varepsilon_1,\varepsilon_2\} \quad \text{s.t.} \quad B_{\varepsilon}(x) \cap A=\varnothing \ , \ B_{\varepsilon}(x) \cap B=\varnothing\]
这与条件矛盾,故$x \in \overline{A} \cup \overline{B}$。

反之,设$x \in \overline{A} \cup \overline{B}$,不妨设$x \in A$,则$\forall \varepsilon>0 \ , \ B_{\varepsilon}(x) \cap A \neq \varnothing \ \Rightarrow \ \forall \varepsilon>0 \ , \ B_{\varepsilon}(x) \cap (A \cup B) \neq \varnothing \ \Rightarrow  \ x \in \overline{A \cup B}$。
\end{itemize}
\end{proof}

下面我们来看闭集的一些性质。
\begin{theorem}
设$(X,d)$是距离空间,则有:
\begin{itemize}
\item 1. $X$,$\varnothing$是闭集;
\item 2. 任意多个闭集的交是闭集;
\item 3. 有限多个闭集的并是闭集。
\end{itemize}
\end{theorem}

这些性质的证明可以通过 De Morgan's Law 证明类似[定理\ref{theorem:2.2}]。
\begin{theorem}[De Morgan's Law]
\[\left(\bigcup\limits_{\lambda \in \Lambda}A_{\lambda}\right)^c=\bigcap\limits_{\lambda \in \Lambda}A_{\lambda}^c \qquad \left(\bigcap\limits_{\lambda \in \Lambda}A_{\lambda}\right)^c=\bigcup\limits_{\lambda \in \Lambda}A_{\lambda}^c\]
\end{theorem}

闭集还有如下性质,同时它也是闭集的另一种定义方式:
\begin{theorem}
    设$(X,d)$是距离空间,$A$为闭集,$\{x_n\} \in A$,若$x_n \rightarrow x$,则$x \in A$。
\end{theorem}

下面我们来看稠密性和可分性。
\begin{definition}[稠密性]
    若$A \subset X$且$\overline{A}=X$则称$A$在$X$中是稠密的(原本应该是$X \subseteq \overline{A}$,当$X$取全空间的时候只能是$\overline{A}=X$)。
\end{definition}
\begin{definition}[可分性]
    若$X$有可数的稠密子集则称$X$是可分的。
\end{definition}
\begin{example}  \quad $\mathbb{R}$中,$\mathbb{Q}$是稠密的($\mathbb{R}$上任意小的区间都有$x \in \mathbb{Q}$);$\mathbb{Q}$是可数的。
\end{example}
\begin{example}
 \quad $C[a,b]$是可分的
\end{example}
\begin{proof}利用 Weierstrass 定理
\begin{theorem}[Weierstrass 定理]\label{theorem:Weierstrass}
    设$f(x)$是$[a,b]$上的连续函数,那么存在一列多项式$\{P_n(x)\}$,使得
    \[P_n(x) \rightrightarrows f(x), \ (n \rightarrow \infty)\]
\end{theorem}

$\forall P_n(x)$,存在有理多项式列$\{q_{n,i}(x)\}_{i=1}^{\infty}$使得$q_{n,i}(x) \rightrightarrows P_n(x) \ (i \rightarrow \infty)$,则
\[\forall k \in \mathbb{N}_+ \ , \ \exists \, n_k,i_k \ , \ q_{n_k,i_k}(x) \rightrightarrows f(x)\]

只需证明
\[d(q_{n_k,i_k}(t),f(t)) \rightarrow 0 \ (n \rightarrow \infty) \ \Leftrightarrow \ q_{n_k,i_k}(x) \rightrightarrows f(x)\]

利用三角不等式容易得到
\[\forall k \in \mathbb{Z}_+ \ , \ \exists \, n_k>0 \quad \text{s.t.} \quad n \geq n_k \ , \ |P_n(t)-f(t)|<\frac{1}{k} \ , \ \forall t \in [a,b]\]
\[\exists \, i_k>0 \ , \quad \text{s.t.} \quad i \geq i_k \ , \ |q_{n_k,i_k}(t)-P_{n_k}(t)|<\frac{1}{k} \ , \ \forall t \in [a,b]\]
\[|q_{n_k,i_k}(t)-f(t)| \leq |q_{n_k,i_k}(t)-P_{n_k}(t)|+|P_n(t)-f(t)|<\frac{1}{k}+\frac{1}{k}=\frac{2}{k} \ , \ \forall t \in [a,b]\]

对上述证明做一下形象化证明,上述的证明实质上是相当于取一个对角线子列$\{a_{i,j}\}$:
\[
\begin{array}{cccccc}
    a_{1,1} & a_{1,2} & a_{1,3} & \cdots & \rightarrow & a_1 \\
    a_{2,1} & a_{2,2} & a_{2,3} & \cdots & \rightarrow & a_2 \\
    \vdots & \vdots & \vdots & \vdots & \rightarrow & \vdots \\
      &   &   &   &   & \downarrow \\
    a_{\infty,1} & a_{\infty,2} & a_{\infty,3} & \cdots & \rightarrow & a_{\infty}
\end{array}    
\]

\end{proof}
\begin{example}
 \quad $l^{\infty}=\{x|\left\{\xi_i\right\}_{i=1}^{\infty} \ (\xi_i \in \mathbb{R}) \ , \ d(x,0)<\infty\}$,$x$为实数列且$l^{\infty} \in \{\text{所有实数列}\}$,定义$l^{\infty}$上的距离:
\[d(x,y)=\mathop {\text{sup}}\limits_{i=1,2,\cdots}|x_i-y_i|, \ x=\left\{x_i\right\}_{i=1}^{\infty} \ , \ y=\left\{y_i\right\}_{i=1}^{\infty}\]
是不可分的。
\end{example}
\begin{proof}
取集合$D=\{\left\{\xi_i\right\}_{i=1}^{\infty} \ , \ \xi_i=0,1\}$显然$\forall x,y \in D \ , \ d(x,y)=0,1$。如果我们把$D$中的每一位放到$0.$后面,可以看出就是一个$[0,1]$区间上的二进制表示,故$D$的基数与$\mathbb{R}$相同$(\text{card}(D)=\text{card}([0,1])=\text{card}(\mathbb{R}))$。我们利用反证法,假设$l^{\infty}$是可分的,则其可以表示为一个可数稠密子集的闭包$l^{\infty}=\overline{A}$,由$A$稠密且可数:
\[\forall x \in l^{\infty} \ , \ l^{\infty}=\overline{A} \ \Rightarrow \ B_{\frac{1}{4}}(x)\cap A \neq \varnothing \ \Rightarrow \ y \in B_{\frac{1}{4}}(x)\cap A \ \Rightarrow \ x \in B_{\frac{1}{3}}(y) \ \Rightarrow \ x \in \bigcup_{a \in A}B_{\frac{1}{3}}(a) \ \text{即} \ l^{\infty} \subset \bigcup_{a \in A}B_{\frac{1}{3}}(a)\]
$A$可数,$D$不可数,且有$D \subset l^{\infty} \subset \bigcup_{a \in A}B_{\frac{1}{3}}(a)$,则
\[\exists \, x \neq y \ , \ x,y \in D \quad \text{s.t.} \quad x,y \in \bigcup_{a \in A}B_{\frac{1}{3}}(a) \ \Rightarrow \ d(x,y)=1 \leq d(x,a)+d(y,a) \leq \frac{2}{3}\]
矛盾,故$l^{\infty}$不可分。
\begin{figure}[htbp]
    \center
    \includegraphics[scale=0.5]{./fig/2.1.3_1.png}
\end{figure}
\end{proof}

\subsection{距离空间的完备性}\label{complete}
\begin{definition}[柯西列]
    设$(X,d)$是距离空间,若$X$中的点列$\{x_n\}_{n=1}^{\infty}$满足:
    \[\{x_n\}_{n=1}^{\infty} \in X \ , \ \forall \varepsilon>0 \ , \ \exists \, N>0 \quad \text{s.t.} \quad n,m \geq N \ , \ d(x_n,x_m)<\varepsilon\]
    则称$\{x_n\}_{n=1}^{\infty}$为柯西列。
\end{definition}
\begin{example}
 \quad 收敛点列一定是柯西列,柯西列是否一定收敛?$\mathbb{R}^n$中是,一般距离空间不成立。
\end{example}
\begin{definition}[完备距离空间]
    设$(X,d)$是距离空间。若该距离空间中所有柯西列都收敛,则称距离空间$(X,d)$是完备的。
\end{definition}

\begin{example}
 \quad $S=(C[0,1],d_1), \ d_1(x,y)=\int_0^1|x(t)-y(t)|\dd t$这个距离空间是不完备的。
\end{example}
\begin{proof}
找反例,令
\[x_n(t)=\left \{
\begin{array}{cl}
0 & , \ t \in [0,\frac{1}{2}-\frac{1}{n}) \\
\frac{n}{2}x-\frac{n}{4}+\frac{1}{2} & , \ t \in [\frac{1}{2}-\frac{1}{n},\frac{1}{2}+\frac{1}{n}] \\
1 & , \ t \in (\frac{1}{2}+\frac{1}{n},1]
\end{array}
\right .\]
\begin{figure}[htbp]
    \center
    \includegraphics[scale=0.4]{./fig/2.1.4_1.png}
\end{figure}
\[\forall \, m>n \ , \ d_1(x_n,x_m)=\int_0^1|x_n(t)-x_m(t)|\dd t \leq \int_{\frac{1}{2}-\frac{1}{n}}^{\frac{1}{2}+\frac{1}{n}}|x_n(t)-x_m(t)|\dd t \leq \frac{2}{n} \cdot 1=\frac{2}{n}\]
\[\Rightarrow \quad \forall \varepsilon>0 \ , \exists \, N>\frac{2}{\varepsilon} \quad \text{s.t.} \quad m>n \geq N \ , \ d_1(x_m,x_n) \leq \frac{2}{n} < \varepsilon\]
因此$\{x_n(t)\}$是柯西列。且可以看出$x_n(t)$是逐点收敛(先固定t)于下列函数$x_{\infty}(t)$。
\[x_n(t) \rightarrow x_{\infty}(t)=\left\{
\begin{array}{ll}
    0 & ,t \in [0,\frac{1}{2}) \\
    \frac{1}{2} & ,t=\frac{1}{2} \\
    1 & ,t \in (\frac{1}{2},1]
\end{array}\right.\]

采用反证法,假设
\[y \in S \quad \text{s.t.} \quad d_1(x_n,y) \rightarrow 0 \ (n \rightarrow \infty) \ \Leftrightarrow \ \int_0^1|x_n(t)-y(t)|\dd t \rightarrow 0 \ (n \rightarrow \infty)\]

则有
\[d_1(x_{\infty},y) \leq d_1(x_{\infty},x_n)+d_1(x_n,y) \rightarrow 0 \ \Rightarrow \ x_{\infty}(t)=y(t)\]

由$y \in S \ \Rightarrow \ y \in C[0,1]$可得矛盾。故上述函数列$\{x_n(t)\}$并不在距离空间$(S,d_1)$中收敛。
\end{proof}
\begin{example} \quad $L^2[0,1]$完备
\[L^2[0,1]=\{f\text{在$[0,1]$上可测} \ , \ \int_0^1|f|^2<+\infty\}, \ d(x,y)=\sqrt{\int_0^1(x(t)-y(t))^2\dd t}\]
\end{example}
\begin{proof}
假设$\{x_n(t)\} \in L^2[0,1]$是柯西列,要证其在距离空间收敛$L^2[0,1]$只需证其有收敛子列,取子列$\{x_{n_k}(t)\}$:
\[\forall k \in \mathbb{Z}_+ \ , \ \exists \, n_k>0 \quad \text{s.t.} \quad n,m>n_k \ , \ d(x_n,x_m)<\frac{1}{2^k}, \ (\text{取 }n_{k+1}>n_{k}, \ k\in\mathbb{Z}_+)\]
可以证明子列$\{x_{n_k}(t)\}$有界,考虑距离的平方:
\[\int_0^1|x_{n_k}|^2\dd t=\int_0^1|x_{n_1}+\sum_{i=2}^k(x_{n_i}-x_{n_{i-1}})|^2\dd t \leq \int_0^1|x_{n_1}|^2\dd t+\sum_{i=2}^k\int_0^1|x_{n_i}-x_{n_{i-1}}|^2\dd t=\int_0^1|x_{n_1}|^2\dd t+\sum_{i=2}^k\frac{1}{4^i}<+\infty\]
\begin{theorem}[Fatou 引理]
    设$(S,\Sigma,\mu)$是测度空间,$\{f_n(x)\}$是测度空间上的实值非负可测函数列,则
    \[\int_E\varliminf_{n \to \infty}f_n(x)\dd x \leq \varliminf_{n \to \infty}\int_Ef_n(x)\dd x\]
    其中函数极限是在逐点收敛的意义上的极限,函数取值和积分可以是无穷大。
\end{theorem}
记子列$\{x_{n_k}(t)\}$的极限点为$x_{n_{\infty}}$,由 Fatou 引理可证明极限点也在$L^2$中:
\[\int_0^1|x_{n_{\infty}}|^2\dd t=\int_0^1\varliminf_{k \to \infty}|x_{n_1}+\sum_{i=2}^k(x_{n_i}-x_{n_{i-1}})|^2\dd t \leq \varliminf_{k \to \infty}\int_0^1|x_{n_1}+\sum_{i=2}^k(x_{n_i}-x_{n_{i-1}})|^2\dd t=\varliminf_{k \to \infty}\int_0^1|x_{n_k}|^2\dd t<+\infty\]
存在收敛子列$\{x_{n_k}(t)\}$可知柯西列$\{x_n(t)\}$收敛,记其收敛于$x_{\infty}(t)$。类似地,柯西列$\{x_n(t)\}$在$L^1$中也收敛:
\[\int_0^1|x_{n_{\infty}}|\dd t=\int_0^1\varliminf_{k \to \infty}|x_{n_1}+\sum_{i=2}^k(x_{n_i}-x_{n_{i-1}})|\dd t \leq \varliminf_{k \to \infty}\int_0^1|x_{n_1}+\sum_{i=2}^k(x_{n_i}-x_{n_{i-1}})|\dd t=\varliminf_{k \to \infty}\int_0^1|x_{n_k}|\dd t<+\infty\]
\begin{theorem}[控制收敛定理]\label{theorem:kzsl}
设$(X,\mathscr{F},\mu)$是测度空间,$\{f_n\}$是测度空间上的可测函数列,在$X$中逐点收敛或几乎处处(逐点)收敛于可测函数$f$,且存在$L^1$函数$g$使得对任意$n$都有$|f_n|\leq g$,则有$f \in L^1$,且
\[\lim_{n \to \infty}\int_X f_n\dd\mu = \int_X \lim_{n \to \infty}f_n\dd\mu = \int_X f\dd\mu\]
且$\{f_n\}$在$L^1$范数意义下收敛于$f$,即
\[\lim_{n \to \infty}\int_X|f_n-f|\dd\mu=0\]
\end{theorem}
考虑如下柯西列$\{|x_n|^2\}$,容易看出其至少被一个$L^1$常函数控制(有界),由控制收敛定理得:
\[\lim_{n \to \infty}\int_0^1|x_n(t)|^2\dd t=\int_0^1|x_{\infty}(t)|^2\dd t \quad \Rightarrow \quad \lim_{k \to \infty}\int_0^1|x_n(t)-x_{\infty}(t)|^2\dd t=0\]
\end{proof}
\begin{example}
 \quad $L^1[0,1], \ d_1(x,y)=\int_0^1|x(t)-y(t)|\dd t$完备
\end{example}
\begin{proof}
在$(C^1[0,1],d_1)$的例子中我们证明了连续函数可能收敛到不连续的函数,但他们都是可测函数,因此将这部分不连续的可测函数纳入即可得到完备的空间。在例\ref{example:l1complete}中有详细证明。
\[S=\left(C^1[0,1],d_1\right) \xrightarrow{\text{完备化}} L^1[0,1]\]
\end{proof}

\begin{theorem}
设$(X,d)$是完备距离空间,则
\begin{itemize}
    \item 1. 闭球套定理;
    \item 2. Baire 纲定理。
\end{itemize}
\end{theorem}
\begin{theorem}[闭球套定理]
设$(X,d)$是完备距离空间,假设存在一列闭球$\{\overline{B}_{r_i}(x_i)\}$满足$\overline{B}_{r_1}(x_1) \supset \overline{B}_{r_2}(x_2) \supset \cdots$且$r_i \to 0$,则
\[\exists \, ! \ x \in \bigcap_{i=1}^{\infty}\overline{B}_{r_i}(x_i)\]
\end{theorem}
\begin{proof}
任意两个球的球心之间的距离都小于等于较大的球的半径:
\[\overline{B}_{r_j}(x_j) \subset \overline{B}_{r_i}(x_i) \ \Rightarrow \ \forall \varepsilon>0 \ , \ \exists \, N>0 \quad \text{s.t.} \quad j>i>N \ , \ d(x_i,x_j)<r_i<\varepsilon\]
所以球心列$\{x_i\}$是柯西列,完备的距离空间内所有柯西列都收敛,即存在$x \in X$使得$d(x_i,x) \to 0$。假设存在另外一个极限点$y \in X$使得$d(x_i,y) \to 0$,那么由三角不等式$d(x,y) \leq d(x,x_i)+d(x_i,y) \rightarrow 0$可知$x=y$。
\end{proof}

在介绍 Baire 定理之前,我们需要介绍一些定义:
\begin{definition}[疏集]\label{definition:sj}
    设$E \subset X$,若$\overline{E}$不包含任何开球,则称$E$为疏集。
\end{definition}
\begin{example}
 \quad $\mathbb{Z} \subset \mathbb{R} \ , \ \overline{\mathbb{Z}}=\mathbb{Z}$无开区间,为疏集。
\end{example}

\begin{definition}[第一纲集,第二纲集]
    距离空间$X$可以分成可数个疏集的并,则称$X$为第一纲集,否则称为第二纲集。
\end{definition}
\begin{example}
 \quad $\mathbb{Q}$是第一纲集,$\mathbb{R}$是第二纲集。
\end{example}
\begin{theorem}[Baire 定理]\label{the:Baire}
    完备距离空间一定是第二纲集。
\end{theorem}
\begin{proof}
反证法:假设完备距离空间$X$是第一纲集,即
\[X=\bigcup_{n=1}^{\infty}E_n \ , \ E_n\text{是疏集}\]
由疏集的定义可知$\overline{E}_1$中无开集,即$\forall x_0 \in X \ , \ r_0>0 \ , \ B_{r_0}(x_0) \nsubseteq \overline{E}_1$,取$x_1 \in B_{r_0}(x_0)-\overline{E}_1=B_{r_0}(x_0)\cap\overline{E}_1^c$,注意到$B_{r_0}(x_0)\cap\overline{E}_1^c$是开集,则$\exists \, r_1>0$满足$B_{2r_1}(x_1) \subset B_{r_0}(x_0)-\overline{E}_1$,进而可以得到一个闭球$\overline{B}_{r_1}(x_1) \subset B_{2r_1}(x_1) \subset B_{r_0}(x_0)-\overline{E}_1$。由于$\overline{E}_2$也是疏集,故$B_{r_1}(x_1) \nsubseteq \overline{E}_2$,则$\exists \, r_2>0$满足$B_{2r_2}(x_2) \subset B_{r_1}(x_1)-\overline{E}_2$,进而可以得到一个新的闭球$\overline{B}_{r_2}(x_2) \subset B_{2r_2}(x_2)\cap(\overline{E}_1\cup\overline{E}_2)^c$。
\begin{figure}[htbp]
    \center
    \includegraphics[scale=0.4]{./fig/2.1.4_2.png}
\end{figure}

由归纳得存在一系列闭球$\{\overline{B}_{r_n}(x_n)\}_{n=1}^{\infty}$,满足$\overline{B}_{r_1}(x_1) \supset \overline{B}_{r_2}(x_2) \supset \overline{B}_{r_3}(x_3) \supset \cdots$且
\[\overline{B}_{r_n}(x_n)\subset\bigcap_{i=1}^n\overline{E}_i^c \quad \Rightarrow \quad \bigcap_{i=1}^{\infty}\overline{B}_{r_i}(x_i) \subset \bigcap_{i=1}^{\infty}\overline{E}_i^c=\left(\bigcup_{i=1}^{\infty}\overline{E}_i\right)^c=X^c=\varnothing\]
得出矛盾,故完备距离空间一定是第二纲集。
\end{proof}

\subsection{距离空间的完备化与压缩映射} \label{zip}
\begin{example}\label{example:l1complete}
\quad $S=(C[0,1],d_1) \subset (L^1[0,1],d_1) \ $,$ \ C[0,1]$在$L^1[0,1]$中稠密,称$L^1[0,1]$是$C[0,1]$完备化。
\end{example}
\begin{proof}
设$x \in L^1[0,1]$,则
\[\int_0^1|x(t)|\dd t=\sum_{n=0}^{+\infty} \ \int\limits_{[0,1]\cap\{n \leq |x| \leq n+1\}}|x(t)|\dd t<+\infty \quad \Rightarrow \quad \exists \, N>0 \quad \text{s.t.} \quad n \geq N \ , \ \int\limits_{[0,1]\cap\{|x|>N\}}|x(t)|\dd t<\frac{1}{n}\]
令
\[\tilde{x}(t)=\left \{
\begin{array}{ll}
    x(t) & , x \leq N \\ 0 & , x>N
\end{array}    
\right .\]
\begin{theorem}[Lusin 定理]\label{theorem:lusin}
设$A \in \mathscr{L} \ , \ m(A)<+\infty \ , \ f:A \to \mathbb{R}$是处处有限的$Lebesgue$可测函数,则
\[\forall \varepsilon>0 \ , \ \exists \, g \in C(A):A\to \mathbb{R} \quad \text{s.t.} \quad m(\{g \neq f\}) \leq \varepsilon\]
这表明连续函数和可测函数没太大的差别。
\end{theorem}
由 Lusin 定理,存在连续函数$g_n(t)$,仅仅在测度小于$\frac{1}{nN}$的集合上与$\tilde{x}(t)$不同,即
\[m([0,1]/\{\tilde{x}(t)=g_n(t)\})<\frac{1}{nN}\]
此时有
\begin{equation*}
    \begin{aligned}
        d_1(g_n(t),x(t)) & \leq d_1(g_n(t),\tilde{x}(t))+d_1(\tilde{x}(t),x(t))=\int_0^1|g_n(t)-\tilde{x}(t)|\dd t+\int_0^1|\tilde{x}(t)-x(t)|\dd t \\
        & \leq \frac{1}{nN} \cdot N+\int\limits_{[0,1]\cap\{|x(t)|>N\}}|x(t)|\dd t=\frac{1}{n}+\frac{1}{n}=\frac{2}{n} \to 0
    \end{aligned}
\end{equation*}
\end{proof}

\begin{theorem}[完备化定理]
    设$(X,d)$是距离空间,则存在$(\tilde{X},\tilde{d})$是完备距离空间,以及等距映射$F:(X,d) \rightarrow (\tilde{X},\tilde{d})$,且$F(x) \subset (\tilde{X},\tilde{d})$是稠密的。
    称$(\tilde{X},\tilde{d})$是$(X,d)$的完备化。若$(\tilde{Y},\tilde{d}_Y)$是另一个完备化,那么$(\tilde{X},\tilde{d})$与$(\tilde{Y},\tilde{d}_Y)$等距(等距意义下的唯一)。
\end{theorem}
\begin{proof}证明思路与从$\mathbb{Q}$到$\mathbb{R}$的完备化类似,补全$\mathbb{Q}$中所有柯西列的极限。
\begin{definition}[等价类]
考虑X中所有柯西列的集合,我们称$[\{x_n\}]$和$[\{y_n\}]$等价,如果其满足:
\[\lim_{n \to \infty}d(x_n,y_n)=0\]
\end{definition}
称$X$中柯西列等价类全体为$\tilde{X}=\{\tilde{x}=[\{x_n\}]\}$,其中$\{x_n\}$是$X$中的柯西列,$[\cdot]$表示等价类。定义距离
\[\tilde{d}(\tilde{x},\tilde{y})=\lim_{n \to \infty}d(x_n,y_n)\]
下面我们将证明$(\tilde{X},\tilde{d})$是良定义的距离空间。
\begin{proposition}[$(\tilde{X},\tilde{d})$是良定义的距离空间]
\begin{itemize}
\item 1. $\tilde{d}$存在:考虑三角不等式展开$d(x_n,y_n)$:
\[|d(x_n,y_n)-d(x_m,y_m)| \leq d(x_n,x_m)+d(x_m,y_m)+d(y_n,y_m)-d(x_m,y_m)=d(x_n,x_m)+d(y_n,y_m)\]
由于$\{x_n\},\{y_n\}$是柯西列,可知$\{d(x_n,y_n)\}$是$\mathbb{R}$上的柯西列,故其收敛,$\tilde{d}$存在。
\item 2. $\tilde{d}$的取值与代表元素$x_n,y_n$选取无关:
\[|d(\tilde{x}_n,\tilde{y}_n)-d(x_n,y_n)| \leq d(\tilde{x}_n,x_n)+d(\tilde{y}_n,y_n)\to 0\]
\item 3. $\tilde{d}$为距离:非负性和对称性显然,只需证明三角不等式:
\[\tilde{d}(\tilde{x},\tilde{y})=\lim_{n \to \infty}d(x_n,y_n) \leq \lim_{n \to \infty}d(x_n,z_n)+\lim_{n \to \infty}d(z_n,y_n)=\tilde{d}(\tilde{x},\tilde{z})+\tilde{d}(\tilde{z},\tilde{y})\]
\end{itemize}
\end{proposition}
更近一步,可以证明$(\tilde{X},\tilde{d})$是完备的。
\begin{proposition}[$(\tilde{X},\tilde{d})$是完备的]
设$\{\tilde{x}_n\}$是$(\tilde{X},\tilde{d})$中的柯西列,对$\forall \tilde{x}_n \in \tilde{X}$总可以取代表元$\{x_n\}$的子列$\{x_{n,k}\}$,当$i,j \geq k$,时满足$d(x_{n,i},x_{n,j})\leq 2^{-k}$,由等价类的性质,我们用新的子列代替原柯西列,则记该柯西列的等价类为$\tilde{x}_n=[\{x_{n,p}\}]$,前一个角标表示$\tilde{X}$中柯西列的第$n$位,后一个角标表示$\tilde{x}_n$对应的$X$中的柯西列第$p$个元素。考虑对角列$\{x_{n,n}\}$,取$m>n$,
\[d(x_{n,n},x_{m,m}) \leq d(x_{n,n},x_{n,m})+d(x_{n,m},x_{m,m}) \leq \frac{1}{2^n}+d(x_{n,m},x_{m,m})\]
对$d(x_{n,m},x_{m,m})$使用$\{\tilde{x}_n\}$的柯西性,即可以选取子列代表原列,当$\forall k \ , \ \exists \, n_k>k$满足当$n,m \geq n_k$时
\[\tilde{d}(\tilde{x}_n,\tilde{x}_m)=\lim_{i \to \infty}d(x_{n,i},x_{m,i})<\frac{1}{2^k} \quad \Leftrightarrow \quad \exists \, I_k>0 \quad \text{s.t.} \quad i \geq I_k \ , \ d(x_{n,i},x_{m,i})<\frac{1}{2^k}\]
进而再利用三角不等式
\[d(x_{n,m},x_{m,m}) \leq d(x_{n,m},x_{n,{I_k}})+d(x_{n,{I_k}},x_{m,{I_k}})+d(x_{m,{I_k}},x_{m,m}) \leq \frac{1}{2^m}+\frac{1}{2^k}+\frac{1}{2^m} \leq \frac{3}{2^k}\]
因此对角列$\{x_{n,n}\}$是$X$中的柯西列:
\[\forall k>0 \ , \ \exists \, m,n \geq \max\{n_k,I_k\} \ , \ d(x_{n_n},x_{m_m}) \leq \frac{1}{2^n}+d(x_{n,m},x_{m,m}) < \frac{1}{2^n}+\frac{3}{2^k} < \frac{4}{2^k}\]
记上述对角列在$\tilde{X}$中的等价类为$\tilde{x}=[\{x_{n,n}\}]$:
\[\tilde{d}(\tilde{x}_n,\tilde{x})=\lim_{i \to \infty}d(x_{n,i},x_{i,i}) \leq \lim_{i \to \infty}d(x_{n,i},x_{n,n})+\lim_{i \to \infty}d(x_{n,n},x_{i,i}) \leq \frac{1}{2^n}+\frac{1}{2^n}=\frac{2}{2^n}\to 0\]
即$(\tilde{X},\tilde{d})$中的柯西列$\{\tilde{x}_n\}$是收敛的且收敛到$\tilde{x}\in\tilde{X}$。
\end{proposition}
然后我们可以构建从原距离空间$(X,d)$到完备距离空间$(\tilde{X},\tilde{d})$的映射$F$。
\begin{proposition}[映射$F:(X,d) \rightarrow (\tilde{X},\tilde{d})$是等距映射,且$F(x) \subset (\tilde{X},\tilde{d})$是稠密的]
我们这里定义映射$F:(X,d) \to (\tilde{X},\tilde{d}), \ x \mapsto \tilde{x}=\{x,x,x,\cdots\}$。$F$是等距映射
\[\tilde{d}(F(x),F(y))=\lim_{n \to \infty}d(\tilde{x},\tilde{y})=d(x,y)\]
记$\tilde{x}=[\{x_n\}] \in \tilde{X} \ , \ \tilde{k}_n=\{x_n,x_n,\cdots\} \in F(X)$。对$\forall \tilde{x} \in \tilde{X} \ , \ \exists \, \tilde{k}_n \in F(X)$:
\[\tilde{d}(\tilde{k}_n,\tilde{x})=\lim_{i \to \infty}d(k_{n_i},x_i)=\lim_{i \to \infty}d(x_n,x_i) \leq \frac{1}{2^n} \ \Rightarrow \ \tilde{d}(\tilde{k}_n,\tilde{x}) \to 0\]
即每一个$\tilde{X}$中到元素$\tilde{x}$都可以在$F(X)$中找到一个序列使得其距离趋近于0,即$F(X)$在$\tilde{X}$中是稠密的。
\end{proposition}
至此,我们完成了完备化定理存在性部分的证明,我们构造了一个(类)完备距离空间$(\tilde{X},\tilde{d})$,并指出了从原空间$(X,d)$到完备距离空间$(\tilde{X},\tilde{d})$的映射为等距映射,还顺便证明了在等价类意义下,$(X,d)$到$(\tilde{X},\tilde{d})$的映射$F$是一个单射,也即到$(\tilde{X},\tilde{d})$的一个子空间$F(X)$的映射$F$是一个双射。下证唯一性。
\begin{proposition}[完备化在等距意义下的唯一性]
设$(\tilde{Y},\tilde{d}_Y)$是$X$的另一个完备化,定义$G:\tilde{Y} \to \tilde{X}, \ y \mapsto [\{x_n\}]$其中$\{x_n\}$是$X$中的一个柯西列,满足$\tilde{d}_Y(x_n,y) \to 0$。因为$\tilde{Y}$,$\tilde{X}$是完备的,故任意给定两个$X$中的柯西列在$\tilde{Y}$,$\tilde{X}$中都能找到对应的像,分别记作$y_1,y_2$和$[\{x_n\}_1],[\{x_n\}_2]$,则
\[\tilde{d}_Y(G(y_1),G(y_2))=\tilde{d}_Y([\{x_n\}_1],[\{x_n\}_2])=\lim_{n \to \infty}\tilde{d}_Y(x_{n_1},x_{n_2})=\tilde{d}_Y(y_1,y_2)\]
即$G$为等距映射(是单射)。 
\end{proposition}
通俗的理解就是等距映射复合上等距映射还是等距映射。
\begin{figure}[htbp]
    \center
    \includegraphics[scale=0.15]{./fig/2.1.5.png}
\end{figure}
\end{proof}

在完备空间上我们可以定义压缩映射:
\begin{definition}[压缩映射]
    设$(X,d)$是距离空间。若有一映射$T:X \to X$满足
    \[\exists \, \theta \in (0,1) \quad \text{s.t.} \quad \forall x,y \in X \ , \ d(T(x),T(y)) \leq \theta d(x,y)\]
    则称$T$为压缩映射。
\end{definition}
\begin{definition}[不动点]
    若$T(x)=x$则称$x$为$T$的不动点。
\end{definition}
\begin{theorem}[压缩映射原理]
    设$(X,d)$是完备距离空间,压缩映射$T:X \to X$,那么$T$有唯一不动点。
\end{theorem}
\begin{proof}设$T$是压缩映射,$x \in X$,当$m>n$时:
\begin{equation*}
    \begin{aligned}
        d(T^n(x),T^m(x)) & \leq \sum_{i=0}^{m-n-1}d(T^{n+i}(x),T^{n+i+1}(x)) \leq \sum_{i=0}^{m-n-1}\theta^{n+i}d(x,T(x)) \\
        & =\theta^nd(x,T(x))\sum_{i=0}^{m-n-1}\theta^i=\frac{\theta^n(1-\theta^{m-n})}{1-\theta}d(x,T(x)) < \frac{\theta^n}{1-\theta}d(x,T(x))
    \end{aligned}
\end{equation*}
即$\{T^n(x)\}$是柯西列:
\[\forall \varepsilon>0 \ , \ \exists \, N>0 \quad \text{s.t.} \quad n>N \ , \ \theta^n<\frac{(1-\theta)\varepsilon}{d(x,T(x))} \ \Rightarrow \ \forall m,n \geq N \ , \ d(T^n(x),T^m(x))<\varepsilon\]
由于$\{T^n(x)\}$是柯西列,且$X$完备,故$\exists \, x_{\infty} \in X$满足$d(T^n(x),x_{\infty}) \to 0$。由压缩映射的定义可知$T$是连续的:
\[x_{\infty}=\lim_{n \to \infty}T^n(x)=\lim_{n \to \infty}T(T^{n-1}(x))=T(\lim_{n \to \infty}T^{n-1}(x))=T(x_{\infty})\]
设存在两个不动点$T(x)=x$和$T(y)=y$,则$d(x,y)=d(T(x),T(y)) \leq \theta d(x,y) \ \Rightarrow \ d(x,y)=0$,即不动点唯一。
\end{proof}
\begin{lemma}[推广的压缩映射原理]
设$(X,d)$是完备距离空间,$T:X \to X$满足
\[\exists \, n_0 \in \mathbb{Z} \ , \ \theta \in (0,1) \quad \text{s.t.} \quad d(T^{n_0}(x),T^{n_0}(y))<\theta d(x,y) \ , \ \forall x,y \in X\]
则$T$有唯一不动点。(不要求$T$是压缩映射,但$T$复合$n_0$次后是压缩映射)
\end{lemma}
\begin{proof}由于$T^{n_0}$是压缩映射,故由压缩映射原理可知$T^{n_0}$有唯一不动点,即$\exists \, ! \, x \in X$满足$T^{n_0}(x)=x$,因此在等式两边都复合上一次映射$T$:
\[T \circ T^{n_0}(x)=T(x) \quad \Leftrightarrow \quad T^{n_0}(T(x))=T(x)\]
即$T(x)$是$T^{n_0}$的不动点,由唯一性可知$T(x)=x$。
\end{proof}

下面我们来看几个压缩映射原理应用的例子(证明方程解的唯一性)。
\begin{example} \quad 证明以下方程存在唯一解:
\[\vec{\xi}-A\vec{\xi}=\vec{b} \ (\vec{\xi},\vec{b}\in\mathbb{R}^n \ , \ A=[A]_{n \times n}) \ , \ \text{设}A=\{a_{ij}\}\text{且满足} \ 0<\sum_{i=1}^n\sum_{j=1}^na_{ij}^2<1\]
\end{example}
\begin{proof}设$X=\mathbb{R}^n$(完备的),定义映射$T:X \to X, \ T(\vec{\xi})=A\vec{\xi}+\vec{b}$。取$\vec{\xi},\vec{\eta} \in X$:
\begin{equation*}
\begin{aligned}
d(\vec{\xi},\vec{\eta}) & =|T(\vec{\xi})-T(\vec{\eta})|=|A(\vec{\xi}-\vec{\eta})|=\sqrt{\sum_{i=1}^n\left(\sum_{j=1}^na_{ij}(\xi_j-\eta_j)\right)^2} \\
& \leq \sqrt{\sum_{i=1}^n\left(\sum_{j=1}^na_{ij}\right)^2\left(\sum_{k=1}^n(\xi_k-\eta_k)\right)^2}=\sqrt{\sum_{i=1}^n\sum_{j=1}^na_{ij}^2}|\vec{\xi}-\vec{\eta}|=\theta|\vec{\xi}-\vec{\eta}| \ , \ \theta \in (0,1)
\end{aligned}
\end{equation*}
后续证明略。
\end{proof}

\begin{example} \quad 证明:当$\lambda$足够小时 Fredholm 积分方程有唯一解。
\[x(t)=\psi(t)+\lambda\int_a^bK(t,s)x(s)\dd s \ , \ K(t,s) \in C[a,b] \times [a,b] \ , \ \psi(t) \in C[a,b]\]
\end{example}
\begin{proof}设$X=C[a,b]$(完备的),对$x(t),y(t) \in X$,定义距离
\[d(x,y)=\mathop {\text{sup}}\limits_{t \in [a,b]}|x(t)-y(t)|\]
再定义映射$T:C[a,b] \to C[a,b]$:
\[T(x(t))=\psi(t)+\lambda\int_a^bK(t,s)x(s)\dd s\]
则:
\[d(T(x(t)),T(y(t)))=\mathop {\text{sup}}\limits_{t \in [a,b]}|T(x(t))-T(y(t))|=\lambda\mathop {\text{sup}}\limits_{t \in [a,b]}\left|\int_a^bK(t,s)(x(s)-y(s))\dd s\right|\]
由$K(t,s)$在闭区域上连续可得$K(t,s)$有界:
\[d(T(x(t)),T(y(t))) \leq \lambda\mathop {\text{sup}}\limits_{t \in [a,b]}\int_a^b|K(t,s)||x(s)-y(s)|\dd s \leq \lambda\mathop {\text{sup}}\limits_{t \in [a,b]}|K(t,s)|d(x,y)(b-a)\equiv\Gamma(b-a)\]
当$\Gamma<1$时,$T$为压缩映射:
\[d(T(x(t)),T(y(t))) \leq \theta d(x,y)\]
后续证明略。
\end{proof}

\begin{example} \quad 证明:Volterra 积分方程有唯一解。
\[x(t)=\psi(t)+\lambda\int_a^tK(t,s)x(s)\dd s \ , \ K(t,s) \in C(([a,b] \times [a,b])\cap\{s<=t\}) \ , \ \psi(t) \in C[a,b]\]
\end{example}
\begin{proof}定义映射$T:C[a,b] \to C[a,b]$:
\[T(x(t))=\psi(t)+\lambda\int_a^tK(t,s)x(s)\dd s\]
一次映射距离:
\[d(T(x(t)),T(y(t)))=\lambda\mathop {\text{sup}}\limits_{t \in [a,b]}\left|\int_a^tK(t,s)(x(s)-y(s))\dd s\right| \leq \lambda\mathop {\text{sup}}\limits_{t \in [a,b]}\left|K(t,s)\right|d(x,y)(t-a)\]
二次映射距离:
\begin{equation*}
    \begin{aligned}
        d(T^2(x(t)),T^2(y(t))) & =\lambda\mathop {\text{sup}}\limits_{t \in [a,b]}\left|\int_a^tK(t,s)\lambda\left|\int_a^tK(t,s)(x(s)-y(s))\dd s\right|\dd s\right| \\
        & \leq \lambda^2\left(\mathop {\text{sup}}\limits_{t \in [a,b]}\left|K(t,s)\right|\right)^2d(x,y)\mathop {\text{sup}}\limits_{t \in [a,b]}\left|\int_a^t\int_a^s\dd\tilde{s}\dd s\right| \\
        & =\lambda^2\left(\mathop {\text{sup}}\limits_{t \in [a,b]}\left|K(t,s)\right|\right)^2d(x,y)\mathop {\text{sup}}\limits_{t \in [a,b]}\left(\frac{(t-a)^2}{2}\right)=\lambda^2\left(\mathop {\text{sup}}\limits_{t \in [a,b]}\left|K(t,s)\right|\right)^2d(x,y) \cdot \frac{(b-a)^2}{2}
    \end{aligned}
\end{equation*}
易得
\[d(T^n(x(t)),T^n(y(t))) \leq \left(\mathop {\text{sup}}\limits_{t \in [a,b]}\left|K(t,s)\right|\right)^n \cdot \frac{\lambda^n(b-a)^n}{n!}d(x,y):=\frac{\lambda^nM^n(b-a)^n}{n!}d(x,y)\]
\[\lim_{n \to \infty}\frac{\lambda^nM^n(b-a)^n}{n!}=0 \ \Rightarrow \ \exists \, n_0>0 \quad \text{s.t.} \quad \theta=\frac{\lambda^{n_0}M^{n_0}(b-a)^{n_0}}{{n_0}!}<1\]
$T^{n_0}$是压缩映射,后续证明略。
\end{proof}

\begin{proposition}[$C[a,b]$的完备性]
一般来说$C[a,b]$默认的距离应该是
\[d(x,y)=\mathop {\text{sup}}\limits_{t \in [a,b]}|x(t)-y(t)|\]
我们设$\{x_n(t)\}$是柯西列:
\[d(x_n(t),x_m(t)) \to 0 \quad \Leftrightarrow \quad \forall t \in [a,b] \ , \ \forall \varepsilon>0 \ , \ \exists \, N>0 \quad \text{s.t.} \quad\forall m,n \geq N \ , \ |x_n(t)-x_m(t)|<\varepsilon\]
即$x_n(t)$逐点收敛到$x(t)$,如将上式中$m \to \infty$,则:
\[\forall \varepsilon>0 \ , \ \exists \, N>0 \ , \quad \text{s.t.} \quad \forall n \geq N \ , \ |x_n(t)-x(t)|<\varepsilon\]
即$x_n(t) \rightrightarrows x(t)$(连续函数一致收敛到连续函数上)。
\end{proposition}

\begin{example} \quad 设$\phi(x),G(s)$有界连续,$G(s)$在$\mathbb{R}$上 Lipschitz 连续,即$|G(a)-G(b)| \leq L|a,b| \ \forall a,b \in \mathbb{R}$,
\[\left\{\begin{array}{c}
\pdv{}{t}f(x,t)=\Delta f(x,t)+G(f(x,y)) \quad x \in \mathbb{R}^n \ , \ t \in [0,+\infty) \\
f(x,0)=\phi(x)
\end{array}\right.\]
证明以上方程存在唯一解。
\end{example}
\begin{proof}热方程的解等于热核$H(x,y,t)$卷积上初始条件,令
\[f_0(x,t)=\int_{\mathbb{R}^n}H(x,y,t)\phi(y)\dd y \ , \ H(x,y,t)=\frac{1}{(4 \pi t)^{\frac{n}{2}}}\exp{-\frac{|x-y|^2}{4t}}\]
满足
\[\pdv{}{t}f_0=\Delta f_0, \ \eval{f_0(x,t)}_{t=0}=\phi(x)\]
定义映射$T$:
\[T(f(x,t))=f_0(x,t)+\int_0^t\int_{\mathbb{R}^n}H(x,y,t-s)G(f(y,s))\dd y\dd s\]
计算距离
\[d(T(f_1(x,t)),T(f_2(x,t)))=\mathop {\text{sup}}\limits_{x \in \mathbb{R}^n,t \in [0,+\infty)}|T(f_1(x,t))-T(f_2(x,t))|\]
由于$G$是Lipschitz连续的,我们有:
\[d(T(f_1(x,t)),T(f_2(x,t))) \leq L\int_0^t\int_{\mathbb{R}^n}H(x,y,t-s)d(f_1(y,s),f_2(y,s))\dd y\dd s\]
利用热核存在一个全局最大值,可以估计$d(T(f_1(x,t)),T(f_2(x,t)))$的上界,从而证明该映射是压缩映射。
\end{proof}

\section{拓扑空间 Topological Space} \label{topo}
\subsection{真·拓扑空间不完全简介}
在距离空间中,我们是通过距离这个概念引入了开球、开集等集合概念与距离空间中的收敛这个概念,由开球和开集进而导出连续性、紧性;由收敛性我们给出完备性的概念。
\begin{figure}[htbp]
    \center
    \includegraphics[scale=0.4]{./fig/2.2.1.png}
\end{figure}
但实际上,我们知道(可能不知道)连续性、紧性不一定需要由距离这个概念诱导得到,定义他们只需要开集。

以及在收敛中,我们也发现了弱收敛、逐点收敛并不具有距离概念。

显然距离空间并不是一个足够广泛的空间可以支撑得起这些概念,因此,我们需要考虑一个更广泛的空间,这里我们讲简要介绍拓扑空间。
\begin{definition}[拓扑]
设$X \neq \varnothing$,记$P(X)=\{X\text{的所有子集}\}$,设$\tau \in P(X)$满足:
\begin{itemize}
\item 1. $\varnothing,X \in \tau$;
\item 2. $\tau$中任意个集合的并属于$\tau$;
\item 3. $\tau$中有限个集合的交属于$\tau$。
\end{itemize}
则称$\tau$是$X$上的拓扑,$\tau$中的集合称为开集。
\end{definition}
拓扑就是指定集合上的哪些子集是开集。
\begin{example}
\quad $(X,d)$是距离空间,$\tau_d=\{x \ \text{中的开集}\}$
\end{example}
\begin{example}
\quad  令$X=\{0,1\}$,则:
\begin{itemize}
    \item 1. $\tau=\{\varnothing,X\}$,称为平凡拓扑;
    \item 2. $\tau=\{\varnothing,\{0\},X\}$,是个拓扑;
    \item 3. $\tau=\{\varnothing,\{0\},\{1\},X\}$,称为离散拓扑,由$X$的所有子集组成的拓扑,包含$X$的所有单点集。
\end{itemize}
\end{example}

若在$X$上,$\tau_1 \subset \tau_2$,则称$\tau_2$强于$\tau_1$。
\begin{definition}[闭集]
    若$A^c \in \tau$,则称$A$是闭集。($A$是闭集$\ \Leftrightarrow \ A=\overline{A}$)
\end{definition}
\begin{definition}[邻域]
    设$x \in X \ , \ U_x \subset \tau \ , \ x \in U_x$,则称$U_x$是$x$的一个邻域(包含$x$的开集)
\end{definition}
\begin{definition}[连续]
    定义$f:X \to Y$是拓扑空间到拓扑空间的映射,设$x_0 \in X$,若对任意$f(x_0)$的邻域$V_{f(x_0)}$都存在$X$中$x_0$的邻域$U_{x_0}$使得$f(U_{x_0}) \subset V_{f(x_0)}$,则$f$称在$x_0$处连续。
\end{definition}
\begin{theorem}
    若$X,Y$是拓扑空间,$f:X \to Y$是连续映射当且仅当开集的原像是开集。
\end{theorem}
这里我们规定符号$f^{-1}$仅表示原像$f:X \to Y \ (A \subseteq Y), \ f^{-1}(A)=\{x \in X \ , \ f(x) \in A\}$。最后给个奇怪的例子作为这一小节的总结,在下一小节虽然我们也将介绍一些拓扑空间中可以定义的概念,但我们更愿意将他放到距离空间的背景下来考量,因为这门课是泛函分析(乐)。
\begin{example}
\quad 在$X=\{0,1\}$上定义拓扑$\tau_D=\{\varnothing,\{0\},\{1\},X\}$,单点集在$\mathbb{R}^n$和距离空间中是闭集,但这里是开集。
\end{example}

\subsection{紧集}
在介绍距离空间上的紧性之前,我们需要介绍一下四种分离公理并指认距离空间所属的公理。
\begin{proposition}[分离公理]
\begin{itemize}
    \item $T_1$公理:$\forall x \neq y \ , \ \exists \, U_x,U_y$满足:$x \in U_x \ , \ y \notin U_x \ ; \ y \in U_y \ , \ x \notin U_y$;
    \item $T_2$公理:$\forall x \neq y \ , \ \exists \, U_x \cap U_y=\varnothing$满足:$x \in U_x \ , \ y \notin U_x \ ; \ y \in U_y \ , \ x \notin U_y$;
    \item $T_3$公理:$\forall$ 闭集$A \cap x=\varnothing \ , \ \exists \, U_x \cap U_A=\varnothing$满足:$x \in U_x \ , \ A \notin U_x \ ; \ A \in U_A \ , \ x \notin U_A$;
    \item $T_4$公理:$\forall$ 闭集$A \cap B=\varnothing \ , \ \exists \, U_A \cap U_B=\varnothing$满足:$A \in U_A \ , \ B \notin U_A \ ; \ B \in U_B \ , \ B \notin U_A$。
\end{itemize}
\end{proposition}
很显然,这四个公理的限制逐级加强,下面我们可以来看看这几个公理的形象化描述:
\begin{figure}[htbp]
    \center
    \includegraphics[scale=0.4]{./fig/2.2.2.png}
\end{figure}

$T_2$公理又称为 Hausdorff 性质,距离空间是满足$T_4$公理的,对于这件事我们可以来证明一下。
\begin{theorem}
    距离空间满足$T_4$公理。
\end{theorem}
\begin{proof}
在两不相交的闭集$A,B$上利用距离定义以下两函数:
\[f_A(x):=\mathop {\text{inf}}\limits_{y \in A}d(x,y) \ , \ f_B(x):=\mathop {\text{inf}}\limits_{y \in B}d(x,y)\]
显然有$f_A(x)>0 \ \Rightarrow \ x \notin A$,如若不然,即存在$A$之外的点$x$满足$f_A(x)=0$,则$\exists \, \{y_n\} \subset A $满足$d(x,y_n) \to 0$,即$x$是$A$的接触点,由于$A$是闭集,故$x \in A$,矛盾。同理可得$f_B(x)>0 \ \Rightarrow \ x \notin B$。

定义邻域:
\[U_B=\bigcup_{x \in B}B_{\frac{1}{2}f_A(x)}(x) \ , \ U_A=\bigcup_{x \in A}B_{\frac{1}{2}f_B(x)}(x)\]
显然$U_B,U_A$都是开集,且$B \subset U_B,A \subset U_A$。假设$\exists \, z \in U_A \cap U_B$,则$\exists \, x \in A, \ y \in B$满足
\[z \in B_{\frac{1}{2}f_B(x)}(x) \ , \ z \in B_{\frac{1}{2}f_A(y)}(y) \quad \Rightarrow \quad d(x,z)<\frac{1}{2}f_B(x) \ , \ d(z,y)<\frac{1}{2}f_A(y)\]
\[d(x,y) \leq d(x,z)+d(z,y)<\frac{1}{2}f_B(x)+\frac{1}{2}f_A(y)<\frac{1}{2}d(x,y)+\frac{1}{2}d(x,y)=d(x,y)\]
得出矛盾,因此$U_A \cap U_B=\varnothing$,距离空间满足$T_4$公理。
\end{proof}
\begin{definition}[子集拓扑]
    $(X,\tau)$为拓扑空间,$A \subset X$,令$\tau_A=\{U \cap A \ , \ U \in \tau\}$,则称$\tau_A$为$X$上的子集拓扑(诱导拓扑)。
\end{definition}
\begin{definition}[紧(致)性]
    $(X,\tau)$为拓扑空间,若$X$的任意开覆盖$\{A_{\lambda}\}$都有有限子覆盖,即存在有限个开集$\{A_{\lambda_i}\}$
    \[X \subset \bigcup_{\lambda \in \Lambda}A_{\lambda} \ \Rightarrow \ X \subset \bigcup_{i=1}^kA_{\lambda_i}\]
    可以将$X$完全覆盖,则称$X$是紧(致)的。
\end{definition}
\begin{example}
\quad $\mathbb{R}^n$上的闭区间(有界闭集)是紧的。
\end{example}
\begin{definition}[紧集]
    $(X,\tau)$为拓扑空间,$A \subset (X,\tau)$,若$(A,\tau_A)$是紧的,则称$A$是$X$中的紧集。
\end{definition}
我们知道欧氏空间中紧集就是有界闭集,那么这个性质在一般的拓扑空间中还成立吗?
\begin{theorem}
    紧空间的闭子集一定是紧集。
\end{theorem}
\begin{proof}
设$X$是一紧空间(或者拓扑空间中的紧子集),$A \subset X$,设存在$A$的开覆盖$U_{\lambda}$:
\[A \subset \bigcup_{\lambda \in \Lambda}U_{\lambda} \ \Rightarrow \ X \subset \left(\bigcup_{\lambda \in \Lambda}U_{\lambda}\right) \cup A^c\]
因为$X$是紧的,存在有限个开集$U_{\lambda_1},U_{\lambda_2},\cdots,U_{\lambda_k}$满足:
\[X \subset \left(\bigcup_{i=1}^kU_{\lambda_i}\right) \cup A^c \quad \Rightarrow \quad A \subset X \subset \left(\bigcup_{i=1}^kU_{\lambda_i}\right) \cup A^c\]
故$A$也是紧的。
\end{proof}
\begin{theorem}
    $T_2$空间的紧子集是闭集。
\end{theorem}
\begin{proof}
设$X$是$T_2$空间,$A \subset X$,需证明$A^c$为开集,即需证$\forall x \in A^c \ , \ \exists \, U_x \subset A^c$。假设$y \in A \ , \ x \in A^c$,由$T_2$公理,存在$x$的开邻域$U_{x,y}$,$y$的开邻域$V_{x,y}$($U_{x,y},V_{x,y}$选取与$x,y$都有关)满足$U_{x,y} \cap V_{x,y}=\varnothing$,由紧性:
\[\exists \, y_1,y_2,\cdots,y_k \in A \quad \text{s.t.} \quad A \subset \bigcup_{i=1}^kV_{x,y_i}\]
令
\[U_x=\bigcup_{i=1}^kU_{x,y_i} \quad \Rightarrow \quad U_x \cap \bigcup_{i=1}^kV_{x,y_i}=\varnothing \quad \Rightarrow \quad U_x \cap A=\varnothing \quad \Leftrightarrow \quad U_x \cap A^c\]
\end{proof}

可以看到,在一般拓扑空间下紧和闭是两个等同的概念,紧比闭更强一点。
\begin{theorem}
    紧空间(子集)在连续映射下的像也是紧的。
\end{theorem}
\begin{proof}设$A \subset X$是紧的,从$X$到$Y$的连续映射$f$,需证明$f(A) \subset Y$是紧的。设$f(A)$有开覆盖$\{G_{\alpha}\}$,由$f$为连续映射可得$f^{-1}(G_{\alpha})$是开集且
\[A \subset \bigcup_{\alpha}f^{-1}(G_{\alpha}) \quad \Rightarrow \quad \exists \, \alpha_1,\alpha_2,\cdots,\alpha_k \quad \text{s.t.} \quad A \subset \bigcup_{i=1}^kf^{-1}(G_{\alpha_1}) \ , \ f(A) \subset \bigcup_{i=1}^kG_{\alpha_i}\]
\end{proof}
\begin{figure}[htbp]
    \center
    \includegraphics[scale=0.2]{./fig/2.2.2-1.png}
\end{figure}


\subsection{刻画距离空间上的紧集} \label{compact}
\begin{definition}[距离空间中的紧集]
    $(X,d)$为距离空间,$A \subset X$,任一$A$的开球覆盖$\{B_{r_{\lambda}}(x_{\lambda})\}$,必有一有限子覆盖$\{B_{r_{\lambda_i}}(x_{\lambda_i})\}$,则称$A$是$X$中的紧集。
\end{definition}
\begin{theorem}
距离空间上的紧集是闭集。
\end{theorem}
\begin{proof}
设$A \subset X$是距离空间$X$上的紧集,对$\forall y \in X/A$,构造$A$的开覆盖:$\{B_{r_{\lambda}}(x_{\lambda})|x_{\lambda} \in A, \ r_{\lambda}=d(x_{\lambda},y)/2\}$。由于$A$是紧集,上述开覆盖存在有限子覆盖$\{B_{r_i}(x_i)|x_i \in A, \ r_i=d(x_i,y)/2, \ i=1,2,\cdots,n\}$。因此构造开球$B_{r_{\min}}(y), \ r_{\min}=\min\{r_1,r_2,\cdots,r_n\}$。可以证明$B_{r_{\min}}(y)\cap A=\varnothing$:利用反证法,假设$z \in B_{r_{\min}}(y)\cap A=\varnothing$,则:
\[d(x_i,y) \leq d(x_i,z)+d(z,y)<r_{\min}+r_{\min}=d(x_i,y)\]
得出矛盾,故$B_{r_{\min}}(y)\cap A=\varnothing$,即说明$X/A$是开集,因此$A$是闭集。
\end{proof}
\begin{definition}[距离空间中的有界性]
    $(X,d)$为距离空间,$A \subset X$,$\exists \, R>0 \ , \ x \in X$满足$A \subset B_R(x)$,则称$A$有界。
\end{definition}
\begin{example}
\quad $\mathbb{R}^n$上的有界闭集是紧集,但在一般距离空间中不再成立。
\end{example}
\begin{definition}[(相对)列紧性]
    若集合$A$中的任意点列都有收敛子列,则称$A$是列紧的,不要求极限也在$A$中(当$A$为闭时,则极限必然在$A$中)。
\end{definition}
\begin{example}
\quad $l^2$空间是完备的,其上的闭球$\overline{B}_1(0)$不是紧集。
\[l^2=\left\{\{x_n\}_{n=1}^{\infty} \ , \ x_n \in \mathbb{R} \ , \ \sum_{n=1}^{\infty}x_n^2<\infty\right\}, \quad d(x,y)=\sqrt{\sum_{n=1}^{\infty}(x_n-y_n)^2}\]
\end{example}
\begin{proof}
令$x_n=\{0,0,\cdots,1,0,\cdots\}$(第$n$位为$1$),显然有$x_n \in \overline{B}_1(0)$因为$d(x_n,0)=\sqrt{1^2}=1$。同时$\forall n \neq m \ , \ d(x_n,x_m)=\sqrt{1^2+1^2}=\sqrt{2}$,当$m,n$足够大的时候并不趋于0,因此$\overline{B}_1(0)$不是列紧的。度量空间中列紧的闭集就是紧集(下面会证明),所以$\overline{B}_1(0)$不是紧集。
\end{proof}
可以看出列紧是比紧更弱一点的条件,但在距离空间中两者等价。
\begin{theorem}\label{theorem:jlj}
距离空间中的子集是紧集当且仅当它是列紧的闭集。
\end{theorem}
\begin{proof}
先证充分条件:紧$ \ \Rightarrow \ $列紧闭,上面已经证明了紧集是闭集,采用反证法,假设存在$\{y_n\} \subset A$,它的任意子列不收敛于$A$中,即$\forall x \in A \ , \ \exists \, r_x>0 \ , \ B_{r_x}(x) \cap \{y_n\}=\{y_{n_1},y_{n_2},\cdots,y_{n_k}\} \ (k<\infty)$。显然$\{B_{r_x}(x)\}$是$A$的一个开覆盖,由$A$的紧性,存在一个有限子覆盖$\{B_{r_{x_i}}(x_i)\}\cap \{y_n\}=\{y_{n_1},y_{n_2},\cdots,y_{n_j}\} \ (j<\infty)$,因为$\{y_n\} \subset A$是柯西列,其个数不是有限个,故矛盾。

必要性比较难证,我们需要先引入全有界的概念,然后通过几个引理逐步证明。
\begin{definition}[全有界性]
$(X,d)$是距离空间,$A \subset X$,对$\forall \varepsilon>0 \ , \ \exists \, k(\varepsilon)<\infty \ , \ x_i \in A \ (i=1,2,\cdots,k)$,$\{B_{\varepsilon}(x_i)\}$是$A$的一个开覆盖,则称$A$是全有界的,开球的球心构成的集合$\{x_i\}_{i=1}^k$称为$A$的有限$\varepsilon$-网。全有界集一定是有界集。
\end{definition}
\begin{figure}[htbp]
    \center
    \includegraphics[scale=0.25]{./fig/2.2.3.png}
\end{figure}
\begin{example}
\quad $\mathbb{R}^n$中的有界集$\ \Leftrightarrow \ $全有界集。
\end{example}

\begin{proposition}[全有界集是可分的]\label{lemma:qyjkf}
设$A$为全有界集,对任意$n \in \mathbb{Z}_+$,存在$A$的有限$\varepsilon$-网$B_n$,不妨取$B_n \subset A$,考虑所有$B_n$并成的集合$B$:
\[\forall x \in A \ , \ \varepsilon>0 \ , \ \exists \, n_{\varepsilon}=\left[\frac{1}{\varepsilon}\right]+1 \quad \text{s.t.} \quad \exists \, y \in B_{n_{\varepsilon}} \ , \ d(x,y)<\frac{1}{n_{\varepsilon}}<\varepsilon \quad \Rightarrow \quad B_{\varepsilon}(x) \cap B_{n_{\varepsilon}} \neq \varnothing\]
进而$B_{\varepsilon}(x) \cap B \neq \varnothing$,因此$x \in \overline{B}$,即$A \subset \overline{B}$,$A$可分得证。
\end{proposition}

\begin{proposition}[距离空间中的列紧集是全有界的]\label{lemma:ljqyj}
反证:若列紧的$A$不是全有界的,则存在$\varepsilon_n>0$,$A$没有有限$\varepsilon_n$-网,因此对$\forall x_0 \in A \ , \ A-B_{\varepsilon_n}(x_0) \neq \varnothing$,取$x_1 \in A-B_{\varepsilon_n}(x_0)$,进而由归纳得存在点列$\{x_k\} \subset A$:
\[x_k \in A-\bigcup_{i=0}^{k-1}B_{\varepsilon_n}(x_i) \neq \varnothing\]
该点列不收敛,因为$\forall i \neq j \ , \ d(x_i,x_j)>\varepsilon_n$,即$A$中点列不全有收敛子列,与题设矛盾。
\end{proposition}

\begin{proposition}[距离空间中的子集是全有界的,那子集的任一开覆盖必有可数子覆盖]\label{lemma:kskfg}
设$A$为全有界集,由引理\ref{lemma:qyjkf}可得$A$是可分的,设$A$的稠密可数子集为$S$,假设有开覆盖$\{G_{\alpha}\}$,$\forall x \in A \ , \ \exists \, \alpha $满足$x \in G_{\alpha}$由于$G_{\alpha}$是开集,因此$\exists \,r>0$满足$B_r(x) \subset G_{\alpha}$。由稠密性,$B_{r/4}(x) \cap S \neq \varnothing$,可以取$y \in B_{r/4}(x) \cap S$。再取$r' \in \left(r/4,r/2\right) \cap \mathbb{Q}$,则有$x \in B_{r'}(y)$,且$B_{r'}(y) \subset B_{r}(x) \subset G_{\alpha}$。因此$A$存在如下可数开覆盖:
\[A \subset \bigcup_{r' \in \mathbb{Q}}\bigcup_{y \in S}B_{r'(y)}(y)\]
由于$B_{r'}(y)\subset G_{\alpha}$,因此$A$被可数个$G_{\alpha}$覆盖。
\end{proposition}

现在证必要条件:列紧闭$ \ \Rightarrow \ $紧。设$A$是是列紧闭集,可知$A$是全有界的[引理\ref{lemma:ljqyj}],且其任意开覆盖$\{G_{\alpha}\}$必有可数子覆盖$\{G_{\alpha_i}|i=1,2,\cdots\}$[引理\ref{lemma:kskfg}]。采用反证法,设$A$不是紧的,则存在一个$\{G_{\alpha}\}$的无有限子覆盖:
\[\exists \, \{G_{\alpha_i}\}_{i=1}^{\infty} \ , \ \forall n \in \mathbb{Z}_+ \ , \ A \nsubset \bigcup_{i=1}^nG_{\alpha_i} \quad \Rightarrow \quad x_n \in A-\bigcup_{i=1}^nG_{\alpha_i}\neq\varnothing\]
易知,$\forall n \in \mathbb{Z}_+$,$G_{\alpha_n}$中只包含$\{x_n\}$中的有限个点。

由$A$是列紧的闭集可知,$\exists \, \{x_{n_k}\} \subset \{x_n\}$满足$x_{n_k} \to x \in A$。可取$x \in G_{\alpha_{n_0}}$,由于$G_{\alpha_{n_0}}$是开集,$\exists \, r>0$满足$B_r(x) \subset G_{\alpha_{n_0}}$,列紧性保证$\exists \, N>0$满足$k \geq N \ , \ x_{n_k} \in B_r(x)$,与上述无限子覆盖的要求矛盾。
\end{proof}
至此我们发现在距离空间中上述定义的三个关键概念强弱如下:紧>列紧>全有界。
\begin{theorem}
完备距离空间中的子集是紧集当且仅当它是全有界闭集。
\end{theorem}
\begin{proof}
由定理\ref{theorem:jlj}和引理\ref{lemma:qyjkf}可知必要性显然,下证充分性,只需证明完备距离空间中全有界集是列紧的。设$A$全有界,$\{x_n\} \subset A$,存在$A$的有限$1$-网$N_1$,满足$\exists \, x \in N_1$,$B_1(x) \cap \{x_n\}$包含无数多个点,取其中一个记为$x_{n_1}$,显然$B_1(x) \cap A$也是全有界的,存在$B_1(x) \cap A$的有限$1/2$-网$N_2$,同上可得新的点$x_{n_2}$,由归纳得,存在点列$\{x_{n_k}\}$,满足$d(x_{n_k},x_{n_{k+m}})<2^{-k}$,进而可知$\{x_{n_k}\}$是柯西列。由完备性,$\exists \, x \in A$使得$x_{n_k} \to x$。
\end{proof}

\begin{figure}[htbp]
    \center
    \includegraphics[scale=0.25]{./fig/2.2.3-1.png}
\end{figure}

\begin{definition}[一致有界]
    $A \subset C[a,b]$且$\exists \, k>0 \ , \ \forall x(t) \in A \ , \ t \in [a,b]$满足$|x(t)| \leq k$,则称$A$是一致有界的。
\end{definition}
上述一致有界与距离空间中的有界是同一回事。
\begin{definition}[等度(一致)连续]
    $A \subset C[a,b] \ , \ \forall \varepsilon>0, \ x(t) \in A \ , \ \exists \, \delta>0 \ , \ t_1,t_2 \in [a,b] \ , \ |t_1-t_2|<\delta$有$|x(t_1)-x(t_2)|<\varepsilon$,则称$A$是等度(一致)连续的。
\end{definition}
可以看出,等度连续是对一族函数的”一致连续“,其$\varepsilon,\delta$的取值与函数无关,容易看出比一致连续强。

\begin{example}
\quad $\exists \, M>0 \ , \ \forall x(t) \in A \ , \ \forall t \in [a,b] \quad \text{s.t.} \quad |x'(t)| \leq M \ $,则$A$是等度连续。
\end{example}
\begin{proof}
$|x(t_1)-x(t_2)|=|x'(\xi)||t_1-t_2| \leq M|t_1-t_2| \quad \xi \in (t_1,t_2) \ : \ |t_1-t_2| \to 0 \ , \ |x(t_1)-x(t_2)| \to 0$
\end{proof}

\begin{theorem}[Arzela-Ascoli 定理] \label{the:AA}
    $C[a,b]$的子集是列紧的当且仅当它是一致有界且等度连续的。
\end{theorem}
\begin{proof}
首先证明必要性:由引理\ref{lemma:qyjkf}知,列紧必定(全)有界,对$\forall \varepsilon>0$,存在$A$的有限$\varepsilon/3$-网$N$满足$\forall x_i \in N \ , \ \exists \,\delta_i \ , \ t_1,t_2 \in [a,b] \ , \ |t_1-t_2|<\delta_i $有$|x(t_1)-x(t_2)|<\varepsilon$,取$\delta=\min\{\delta_i\}$,则$\exists \, i$满足$x(t) \subset B_{\varepsilon/3}(x_i(t))$,当$|t_1-t_2|<\delta \ , \ \forall x(t) \in A$时,有
\[|x(t_1)-x(t_2)| \leq |x(t_1)-x_i(t_1)|+|x_i(t_1)-x_i(t_2)|+|x_i(t_2)-x(t_2)|<\frac{\varepsilon}{3}+\frac{\varepsilon}{3}+\frac{\varepsilon}{3}=\varepsilon\]
等度连续得证。

\begin{figure}[htbp]
    \center
    \includegraphics[scale=0.3]{./fig/2.2.3-2.png}
\end{figure}

下证充分性。由于$C[a,b]$完备,故只需证明$C[a,b]$全有界。由一致有界条件$\exists \, k>0 \ , \ \forall x(t) \in A \ , \ t \in [a,b]$满足$|x(t)|<k$以及等度连续条件$\forall \varepsilon>0 \ , \ \exists \, \delta>0 \ , \ |t_1-t_2|<\delta \ , \ \forall x(t) \in A$有$|x(t_1)-x(t_2)|<\varepsilon$,对函数族的定义域和值域做划分。取$[-k,k]$划分$\{y_0,y_1,\cdots,y_n\}:y_0=-k \ , \ y_n=k \ , \ 0<y_{i+1}-y_i<\varepsilon/5$,取$[a,b]$划分$\{t_0,t_1,\cdots,t_m\}:t_0=a \ , \ t_m=b \ , \ 0<y_{i+1}-y_i<\delta$。按如下规则绘制函数:若$x(t_i) \in [y_k,y_{k+1}]$,则令$f(t_i)=y_k$,其余部分用直线连接,则$\forall t \in [a,b] \ , \ \exists \, t_i$满足当$t \in [t_i,t_{i+1}]$时有:
\begin{equation*}
\begin{aligned}
|x(t)-f(t)| &\leq |x(t)-x(t_i)|+|x(t_i)-f(t_i)|+|f(t_i)-f(t)|\\
&<\frac{\varepsilon}{5}+\frac{\varepsilon}{5}+|f(t_i)-f(t)|\\
&<\frac{2}{5}\varepsilon+|f(t_i)-f(t_{i+1})|\\
&<\frac{2}{5}\varepsilon+|f(t_i)-x(t_i)|+|x(t_i)-x(t_{i+1})|+|x(t_{i+1})-f(t_{i+1})|\\
&<\frac{2}{5}\varepsilon+\frac{1}{5}\varepsilon+\frac{1}{5}\varepsilon+\frac{1}{5}\varepsilon=\varepsilon\\
\end{aligned}
\end{equation*}
因$f$至多有$(n+1)^{m+1}$个,因此这些$f$构成$A$的一个有限$\varepsilon$-网。全有界性得证。
\end{proof}
\chapter{赋范线性空间}
\begin{introduction}
    \item Banach 空间~\ref{Banach}
    \item 有界变差函数~\ref{bounded variation}
    \item $L^p$空间~\ref{lp}
    \item $L^{\infty}$空间~\ref{infty}
    \item 赋范线性空间的基~\ref{baseset}
    \item 有限维赋范线性空间~\ref{the:B}
    \item Riesz 引理~\ref{Riesz}
  \end{introduction}
\section{赋范线性空间概述}
\subsection{范数,赋范线性空间,Banach 空间} \label{Banach}
在泛函分析中我们大多处理的都是线性空间。无穷维线性空间区别于有限维线性空间,一些在有限维线性空间中成立的事情在无穷维线性空间中不一定成立:例如,单位闭球在有限维线性空间中是紧致的但在无穷维线性空间中不一定紧致,这点在上一章的讨论中有证明。

在线性代数中,我们定义过$\mathbb{R}^n$上(有限维线性空间)的模长:
\[\mathbb{R}^n \ : \ \vec{v} \in \mathbb{R}^n \ , \ |\vec{v}|=\sqrt{\sum_{i=1}^nx_i^2}\]

在无限维线性空间我们也同样可以定义类似的概念——范数(norm)可以看作是模长的推广(其实都是距离空间中的距离)。
\begin{definition}[范数]
设$X$是定义在$\mathbb{R}$或者$\mathbb{C}$上的线性空间,函数$|| \cdot ||:X \to \mathbb{R}$如果满足
\begin{itemize}
    \item 1. 非负性:$\forall x \in X \ , \ ||x|| \geq 0 \ \text{and} \ ||x||=0 \ \text{if and only if} \ x=0 \ $
    \item 2. 齐次性:$\forall x \in X \ , \ \forall a \in \mathbb{R} \ or \ \mathbb{C} \ , \ ||ax||=|a|||x|| \ $
    \item 3. 三角不等式:$\forall x,y \in X \ , \ ||x+y|| \leq ||x||+||y|| \ $
\end{itemize}
则称$||\cdot||$是$X$上的一个范数,$(X,||\cdot||)$称为赋范线性空间。
\end{definition}
容易证明,赋范线性空间是距离空间,只需把范数中的元素换成两个元素之差即可。
\begin{definition}[强收敛(依范数收敛)]
若$d(x_n,x) \to 0$即$||x_n-x|| \to 0$,则称$x_n$强收敛于$x$或称依范数收敛。
\end{definition}
我们将完备的赋范线性空间称为 Banach 空间。

同一个集合上可以定义不同的范数:
\begin{figure}[H]
    \centering
    \begin{minipage}[t]{0.3\textwidth}
    \centering
    \includegraphics[width=3.5cm]{./fig/3.1.1-1.png}
    \caption{$(\mathbb{R}^2,|\vec{x}|) \ , \ |\vec{x}|=\sqrt{x_1^2+x_2^2}$}
    \end{minipage}
    \hspace{2cm}
    \begin{minipage}[t]{0.3\textwidth}
    \centering
    \includegraphics[width=3.5cm]{./fig/3.1.1-2.png}
    \caption{$(\mathbb{R}^2,|\vec{x}|) \ , \ |\vec{x}|=|x_1|+|x_2|$}
    \end{minipage}
\end{figure}
\begin{example}
\quad $C[a,b] \ , \ ||x(t)||=\mathop \text{sup}\limits_{t \in [a,b]}|x(t)|$是 Banach 空间,无穷维且可分。
\end{example}
\begin{proof}
\[||x(t)|| \leq 0 \ , \ \text{若}||x(t)||=0 \ \Rightarrow \ \mathop \text{sup}\limits_{t \in [a,b]}|x(t)|=0 \ \Leftrightarrow \ \forall t \in [a,b] \ , \ |x(t)|=0 \ \Rightarrow \ x(t)=0\]
\[||ax(t)||=\mathop \text{sup}\limits_{t \in [a,b]}|ax(t)|=\mathop \text{sup}\limits_{t \in [a,b]}|a||x(t)|=|a|||x(t)||\]
\[||x(t)+y(t)||=\mathop \text{sup}\limits_{t \in [a,b]}|x(t)+y(t)| \leq \mathop \text{sup}\limits_{t \in [a,b]}|x(t)|+\mathop \text{sup}\limits_{t \in [a,b]}|y(t)|=||x(t)||+||y(t)||\]
\end{proof}

\begin{example}
\quad $l^{\infty}$是所有有界数列构成的集合,设$X=(x_1,x_2,\cdots)$,$||X||=\mathop \text{sup}\limits_{i \in \mathbb{N}_+}|x_i|$。容易验证:$||\cdot||$是范数以及$(l^{\infty},||\cdot||)$是 Banach 空间,无穷维且不可分。
\end{example}

\begin{example}
\quad $L^1[a,b]$是$[a,b]$上所有可积函数(等价类)构成的集合,利用 Lebesgue 积分定义范数
\[||x(t)||_{L^1[a,b]}=\int_a^b|x(t)|\dd t\]
\end{example}
\begin{proof}其余两点容易验证,仅验证正定性:
\[||x(t)||_{L^1[a,b]}=\int_a^b|x(t)|\dd t \geq 0 \ \text{若}||x(t)||_{L^1[a,b]}=0 \ \Rightarrow \ |x(t)|=0 \quad a.e. \ \Rightarrow \ x(t)=0 \quad a.e.\]
\end{proof}

\begin{example}
\quad $(C^1[a,b],||\cdot||_{L^1[a,b]})$不完备,故不是 Banach 空间。这件事可以通过范数诱导出的距离加以证明:
\[d(x(t),y(t))=\int_a^b|x(t)-y(t)|\dd t\]
\end{example}

\begin{definition}[全变差,有界变差函数] \label{bounded variation}
定义函数$x(t)$的全变差定义为:
\[\mathop \text{V}\limits_a^b(x(t))=\mathop \text{sup}\limits_p \sum |x(t_{i+1})-x(t_i)|\]
其中$p$为$[a,b]$上的任意有限划分$a=t_0<t_1<t_2<\cdots<t_n=b$。若函数$x(t)$满足:
\[\mathop \text{V}\limits_a^b(x(t))<\infty\]
则称$x(t)$为有界变差函数。有界变差函数全体组成的空间记为$V[a,b]$。
\end{definition}
\begin{proposition}[全变差空间$V[a,b]$]
定义$V[a,b]$上的范数:
\[||x(t)||=|x(a)|+\mathop \text{V}\limits_a^b(x(t))\]
验证其是定义良好的范数:
\begin{itemize}
\item 1. 正定性:$||x(t)|| \geq 0$显然,若$||x(t)||=0$,则有$x(a)=0$且$\mathop \text{V}\limits_a^b(x(t))=0$。取对半划分$\forall t \in [a,b]$:
\[|x(a)-x(t)|+|x(t)-x(b)| \leq \mathop \text{V}\limits_a^b(x(t))=0 \ \Rightarrow \ |x(a)-x(t)|=|x(t)-x(b)|=0 \ \Rightarrow \ x(t)=x(a)=0\]
\item 2. 齐次性:显然
\item 3. 三角不等式:利用不等式$\text{sup}(A+B) \leq \text{sup}A+\text{sup}B$:
\begin{equation*}
    \begin{aligned}
        ||x(t)+y(t)|| & =|x(a)+y(a)|+\mathop \text{V}\limits_a^b(x(t)+y(t))\\
        & =|x(a)+y(a)|+\mathop \text{sup}\limits_p \sum|x_{i+1}(t)-x_i(t)+y_{i+1}(t)-y_i(t)| \\
        & \leq |x(a)|+|y(a)|+\mathop \text{sup}\limits_p \sum|x_{i+1}(t)-x_i(t)|+\mathop \text{sup}\limits_p \sum|y_{i+1}(t)-y_i(t)|\\
        &=||x(t)||+||y(t)||
    \end{aligned}
\end{equation*}
\end{itemize}
\end{proposition}

\begin{example}
\quad $V[a,b]$是$[a,b]$上的有界变差函数构成的集合,是 Banach 空间。
\end{example}
\begin{proof}
设$\{x_n\}$是$V[a,b]$中的柯西列,即$\forall \varepsilon>0 \ , \ \exists \, N>0 \quad \text{s.t.} \quad n,m \geq N \ , \ ||x_m-x_n||<\varepsilon$,对$[a,b]$上的任意有限划分$a=t_0<t_1<t_2<\cdots<t_n=b$都有:
$$|x_m(a)-x_n(a)|+\mathop \text{sup}\limits_p \sum |x_m(t_{i+1})-x_n(t_{i+1})-x_m(t_i)+x_n(t_i)|<\varepsilon$$
由此可知$\{x_n(a)\}$是柯西列,类似的取对半划分:$\forall t \in [a,b]$有:
\[|x_n(a)-x_n(t)|+|x_n(t)-x_n(b)| \leq \mathop \text{V}\limits_a^b(x_n(t))<\varepsilon \quad \Rightarrow \quad |x_n(a)-x_n(t)|<\varepsilon\]
即$\{x_n(t)-x_n(a)\}$是柯西列,有限个柯西列的和还是柯西列,故可知$\{x_n(t)\}$是柯西列。记柯西列极限为$x(t)$,要证明$x(t)$在$V[a,b]$中,利用有限加和与极限可交换:
\[\mathop \text{V}\limits_a^b(x(t))=x(a)+\mathop \text{sup}\limits_p \lim_{n \to \infty}\sum |x_n(t_{i+1})-x_n(t_i)|=\lim_{n \to \infty} \left ( x_n(a)+\mathop \text{sup}\limits_p \sum |x_n(t_{i+1})-x_n(t_i)| \right )\]
因为$\{x_n(t)\}$是$V[a,b]$中的柯西列,故$\{x_n(t)\}$是$V[a,b]$中的有界集:
\[\forall n \in \mathbb{N}_+ \ , \ \exists \, M>0 \quad \text{s.t.} \quad ||x_n(t)|| \leq M \quad \Rightarrow \quad \mathop \text{V}\limits_a^b(x(t)) \leq M \quad \Leftrightarrow \quad \mathop \text{V}\limits_a^b(x(t)) \in V[a,b]\]
综上所述,$\forall t \in [a,b]$,$x_n(t)$逐点收敛到$x(t) \in V[a,b]$。在此基础上,可以证明$x_n(t)$强收敛于$x(t)$:
\[||x_n(t)-x(t)||=|x_n(a)-x(a)|+\mathop \text{sup}\limits_p \sum |x_n(t_{i+1})-x(t_{i+1})-x_n(t_i)+x(t_i)|\]
\[\exists \, N>0 \quad \text{s.t.} \quad n>N \ , \ |x_n(a)-x(a)|=\lim_{m \to \infty}|x_n(a)-x_m(a)|<\varepsilon\]
\[\forall p \ , \ \mathop \text{sup}\limits_p \sum |x_n(t_{i+1})-x(t_{i+1})-x_n(t_i)+x(t_i)|=\lim_{m \to \infty}\mathop \text{sup}\limits_p \sum |x_n(t_{i+1})-x_m(t_{i+1})-x_n(t_i)+x_m(t_i)|\]
\[\leq \lim_{m \to \infty}\mathop \text{sup}\limits_p \sum |x_n(t_{i+1})-x_n(t_i)|+\lim_{m \to \infty}\mathop \text{sup}\limits_p \sum |x_m(t_{i+1})-x_m(t_i)|<\varepsilon+\varepsilon=2\varepsilon\]
即$\forall \varepsilon>0 \ , \ \exists \, N>0$满足$n>N \ , \ ||x_n(t)-x(t)||<3\varepsilon$即是$x_n \to x$。
\end{proof}

\subsection{赋范线性空间的基本性质}
\begin{theorem}
    设$(X,||\cdot||)$是线性赋范空间,则:
    \begin{itemize}
        \item 1. 范数是连续函数,即$x_n \to x \ \Rightarrow \ ||x_n|| \to ||x|| \ (n \to \infty)$
        \item 2. 线性运算是连续映射:
        \begin{itemize}
            \item 1. $x_n \to x \ , \ y_n \to y \ \Rightarrow \ x_n+y_n \to x+y$
            \item 2. $x_n \to x \ , \ a_n \to a \ (\in \mathbb{R} \ or \ \mathbb{C}) \ \Rightarrow \ a_nx_n \to ax$
        \end{itemize}
    \end{itemize}
\end{theorem}
\begin{proof}
\begin{itemize}
\item 1. 
\[||x_n||=||x_n-x+x|| \leq ||x_n-x||+||x|| \ \Rightarrow \ ||x_n||-||x|| \leq ||x_n-x||\]
\[||x||=||x-x_n+x_n|| \leq ||x-x_n||+||x_n|| \ \Rightarrow \ ||x||-||x_n|| \leq ||x_n-x||\]
\[\Rightarrow \ \text{\Large|} ||x||-||x_n|| \text{\Large|} \leq ||x_n-x|| \to 0 \ \Rightarrow \ ||x_n|| \to ||x||\]
\item 2-1
\[||x_n+y_n-x-y|| \leq ||x_n-x||+||y_n-y|| \to 0\]
\item 2-2
\[||a_nx_n-ax|| \leq ||a_nx_n-ax_n+ax_n-ax|| \leq ||a_nx_n-ax_n||+||ax_n-ax||=|a_n-a|||x_n||+|a|||x_n-x|| \to 0\]
\end{itemize}
\end{proof}

在线性空间中可以利用加法定义级数,赋范线性空间中当然也可以定义级数:
\begin{definition}[级数,部分和,级数收敛]
设赋范线性空间中的函数列$\{x_n\}_{n=1}^{\infty}$,定义级数为
\[\sum_{n=1}^{\infty}x_n=x_1+x_2+x_3+\cdots\]
定义部分和$S_n$
\[S_n=\sum_{i=1}^nx_i-x_1+x_2+\cdots+x_n\]
若部分和有极限$S_n \to S$,则称级数存在极限。
\end{definition}
\begin{theorem}
\[(X,||\cdot||) \ \text{是$Banach$空间,若级数} \ \sum_{n=1}^{\infty}||x_n|| \ \text{收敛,则} \ \sum_{n=1}^{\infty}x_n \ \text{收敛且} \ \left\| \sum_{n=1}^{\infty}x_n \right\| \leq \sum_{n=1}^{\infty}||x_n||\]
\end{theorem}
\begin{proof}
部分和的模长
\[||S_j||=\left\| \sum_{n=1}^jx_n \right\| \leq \sum_{n=1}^j||x_n||\]
由于模长的级数$\sum ||x_n||$存在:
\[\sum_{n=1}^{\infty}||x_n||<+\infty \quad \Leftrightarrow \quad \forall \varepsilon>0 \ , \ \exists \, N>0 \quad \text{s.t.} \quad j>i \geq N \ , \ \sum_{n=i}^j||x_n||<\varepsilon \quad \Rightarrow \quad ||S_{j+1}-S_{i-1}||=\sum_{n=i}^j||x_n||<\varepsilon\]
可知部分和序列$\{S_n\}$是柯西列,即$\exists \, S \in X$满足$S_n \to S$,即$\sum x_n$收敛。
进而,由范数$||\cdot||$的连续性,对不等式两边取极限即证毕。
\[\left\| \sum_{n=1}^jx_n \right\| \leq \sum_{n=1}^j||x_n|| \quad \Rightarrow \quad \left\| \sum_{n=1}^{\infty}x_n \right\| \leq \sum_{n=1}^{\infty}||x_n||\]
\end{proof}

\begin{definition}[凸集]
设$X$是线性空间,若集合$A \subset X$满足$\forall x,y \in A \ , \ t \in (0,1)$有$tx+(1-t)y \in A$,则称$A$为凸集。
\end{definition}
\begin{figure}[htbp]
    \center
    \includegraphics[scale=0.22]{./fig/3.1.2.png}
\end{figure}

\begin{theorem}
    设$(X,||\cdot||)$是赋范线性空间,原点的球邻域$B_r(0)$是凸集。
\end{theorem}
\begin{proof}
$B_r(0)=\{x|x \in X \ ,||x||<r\}$,设$x,y \in B_r(0) \ , \ t \in (0,1)$,则有
\[||tx+(1-t)y|| \leq ||tx||+||(1-t)y||=|t|||x||+|1-t|||y||<tr+(1-t)r=r \quad \Rightarrow \quad tx+(1-t)y \in B_r(0)\]
\end{proof}

\begin{proposition}
    $(X,||\cdot||)$中任意球是凸集。
\end{proposition}
凸集还有其他基本性质,如任意个凸集的交集是凸集(赋范线性空间中任意个球的交集是凸集)。

关于凸集的性质也有与不动点定理相关的,下面我们简要叙述一下该定理。
\begin{theorem}[Schauder 不动点定理]
    设$X$是 Banach 空间,$A \subset X$是紧凸集,$T:A \to A$是连续映射,则$T$有不动点(不一定唯一)。
\end{theorem}
\paragraph*{例} \quad $T:[0,1] \to [0,1]$
\begin{figure}[htbp]
    \center
    \includegraphics[scale=0.2]{./fig/3.1.2-1.png}
\end{figure}
不动点在$y=x$上,连续映射保证了任意一个$T$都必然与$y=x$有交点。

\section{$L^p$空间} \label{lp}
\begin{definition}[$L^p$空间]
设$E$是$\mathbb{R}$上的可测集。$L^p$空间定义为
\[L^p(E)=\left\{x(t) \ \text{可测} \ , \ \int_E|x(t)|^p<+\infty\right\}\]
这里$x(t)$是等价类,为方便书写取一代表元,这里$x(t) \sim y(t)$当且仅当$x(t)=y(t) \quad a.e.$,定义范数:
\[||x(t)||_{L^p(E)}=\left(\int_E|x(t)|^p\dd t\right)^{\frac{1}{p}}\]
\end{definition}
\begin{proof}
\begin{itemize}
\item 1. 正定性
\[||x(t)|| \geq 0 \ \text{显然,若}||x(t)||=0 \ \Rightarrow \ \int_E|x(t)|^p\dd t=0 \ \text{,不妨设} \ m(E)>0 \ \Rightarrow \ x(t)=0 \quad a.e.\]
\item 2. 齐次性
\[||kx(t)||=\left(\int_Ek|x(t)|^p\dd t\right)^{\frac{1}{p}}=\left(|k|^p\int_E|x(t)|^p\dd t\right)^{\frac{1}{p}}=|k|\left(\int_E|x(t)|^p\dd t\right)^{\frac{1}{p}}\]
\item 3. 三角不等式:Minkowski 不等式(要证明 Minkowski 不等式,我们需要一些其他的基础不等式)
\end{itemize}
\end{proof}
\begin{theorem}[Young 不等式]
    \[\text{若} \forall p,q>0 \ \text{满足} \ \frac{1}{p}+\frac{1}{q}=1 \ \text{,则对} \ \forall a,b>0 \ \text{都有} \ |ab| \leq \frac{|a|^p}{p}+\frac{|b|^q}{q}\]
\end{theorem}
\begin{proof}
不妨设$a,b>0$,
\[\frac{1}{p}+\frac{1}{q}=1 \quad \Leftrightarrow \quad \frac{q}{p}+1=q \quad \Leftrightarrow \quad 1-q=-\frac{q}{p}\]
\[|ab| \leq \frac{|a|^p}{p}+\frac{|b|^q}{q} \quad \Leftrightarrow \quad ab^{1-q} \leq \frac{1}{p}\left(\frac{a^p}{b^q}\right)+\frac{1}{q} \quad \Leftrightarrow \quad \left(\frac{a^p}{b^q}\right)^{\frac{1}{p}}-1 \leq \frac{1}{p}\left(\frac{a^p}{b^q}-1\right)\]
令$t=a^p/b^q>0$,则只需证:
\[t^{\frac{1}{p}}-1 \leq \frac{1}{p}(t-1) \ , \ (t>0 \ ,p>0) \quad \Rightarrow \quad \text{定义:} f(t)=t^{\frac{1}{p}}-\frac{1}{p}t-1+\frac{1}{p} \ , \ (t>0 \ , \ p>0)\]
\[f'(t)=\frac{1}{p}t^{\frac{1}{p}-1}-\frac{1}{p}=\frac{1}{p}\left(t^{\frac{1}{p}-1}-1\right) \quad \Rightarrow \quad f(x) \leq f(1)=0 \quad \Rightarrow \quad t^{\frac{1}{p}}-1 \leq \frac{1}{p}(t-1) \ , \ (t>0 \ , \ p>0)\]
\end{proof}
在使用 Young 不等式的时候我们可以用到一下小技巧:
\begin{lemma}
\[\text{若} \forall p,q>0 \ \text{满足} \ \frac{1}{p}+\frac{1}{q}=1 \ \text{,则对} \ \forall a,b>0 \ , \ \forall \varepsilon>0 \ \text{都有} \ |ab|=|\varepsilon a \cdot \frac{1}{\varepsilon}b| \leq \varepsilon^p\frac{|a|^p}{p}+\frac{1}{\varepsilon^q}\frac{|b|^q}{q}\]
\end{lemma}
由于$\varepsilon$的任意性,可以用这种小技巧将一些性质不好的项(我不喜欢的项)压住。
\begin{theorem}[H\"{o}lder 不等式]
    设$E$是可测集,$x(t),y(t)$是$E$上的可测函数,则有
    \[\int_E|x(t)y(t)|\dd t \leq \left(\int_E|x(t)|^p\dd t\right)^{\frac{1}{p}}\left(\int_E|y(t)|^q\dd t\right)^{\frac{1}{q}} \ , \ (p,q>0 \ , \ \frac{1}{p}+\frac{1}{q}=1)\]
    若$x(t) \in L^p(E) \ , \ y(t) \in L^q(E)$则
    \[\int_E|x(t)y(t)|\dd t \leq ||x(t)||_{L^p(E)}||y(t)||_{L^q(E)}\]
\end{theorem}
\begin{proof}
设
\[A=||x(t)||_{L^p(E)}=\left(\int_E|x(t)|^p\dd t\right)^{\frac{1}{p}} \ , \ B=||y(t)||_{L^q(E)}=\left(\int_E|y(t)|^q\dd t\right)^{\frac{1}{q}}\]
由 Young 不等式:
\[\frac{|x(t)| \cdot |y(t)|}{A \cdot B} \leq \frac{1}{p}\frac{|x(t)|^p}{A^p}+\frac{1}{q}\frac{|y(t)|^q}{B^q} \ \Rightarrow \ \frac{\int_E|x(t)y(t)|\dd t}{A \cdot B} \leq \frac{1}{p}\frac{\int_E|x(t)|^p\dd t}{A^p}+\frac{1}{q}\frac{\int_E|y(t)|^q\dd t}{B^q}=1\]
\end{proof}

\begin{theorem}[Minkowski 不等式]
设$x(t),y(t) \in L^p(E) \ (p \geq 1)$,则有
\[\left(\int_E|x(t)+y(t)|^p\dd t\right)^{\frac{1}{p}} \leq \left(\int_E|x(t)|^p\dd t\right)^{\frac{1}{p}}+\left(\int_E|y(t)|^p\dd t\right)^{\frac{1}{p}}\]
即
\[||x(t)+y(t)||_{L^p(E)} \leq ||x(t)||_{L^p(E)}+||y(t)||_{L^p(E)}\]
\end{theorem}
\begin{proof}
$p=1$时,显然;$p>1$时,有
\[\frac{1}{p}+\frac{1}{q}=1 \quad \Leftrightarrow \quad p=(p-1)q\]
由 H\"{o}lder 不等式:
\[\int_E|x(t)+y(t)|^p\dd t=\int_E|x(t)+y(t)|\cdot|x(t)+y(t)|^{p-1}\dd t\]
\[\leq \int_E|x(t)|\cdot|x(t)+y(t)|^{p-1}\dd t+\int_E|y(t)|\cdot|x(t)+y(t)|^{p-1}\dd t\]
\[\leq \left(\int_E|x(t)|^p\dd t\right)^{\frac{1}{p}}\left(\int_E|x(t)+y(t)|^{q(p-1)}\dd t\right)^{\frac{1}{q}}+\left(\int_E|y(t)|^p\dd t\right)^{\frac{1}{p}}\left(\int_E|x(t)+y(t)|^{q(p-1)}\dd t\right)^{\frac{1}{q}}\]
\[\leq \left(\int_E|x(t)|^p\dd t\right)^{\frac{1}{p}}\left(\int_E|x(t)+y(t)|^{p}\dd t\right)^{1-\frac{1}{p}}+\left(\int_E|y(t)|^p\dd t\right)^{\frac{1}{p}}\left(\int_E|x(t)+y(t)|^{p}\dd t\right)^{1-\frac{1}{p}}\]
\[\Rightarrow \ \left(\int_E|x(t)+y(t)|^p\dd t\right) \cdot \left(\int_E|x(t)+y(t)|^{p}\dd t\right)^{\frac{1}{p}-1} \leq \left(\int_E|x(t)|^p\dd t\right)^{\frac{1}{p}}+\left(\int_E|y(t)|^p\dd t\right)^{\frac{1}{p}}\]
\[\Leftrightarrow \ \left(\int_E|x(t)+y(t)|^{p}\dd t\right)^{\frac{1}{p}} \leq \left(\int_E|x(t)|^p\dd t\right)^{\frac{1}{p}}+\left(\int_E|y(t)|^p\dd t\right)^{\frac{1}{p}}\]
\end{proof}

\section{$L^p$空间的性质}
\begin{theorem}[Riesz-Fischer 定理]\label{theorem:rf}
    $L^p(E) \ (p \geq 1)$是$Banach$空间。\footnote{这里我们指定$E$是$\mathbb{R}$上的可测集,因为说明这种情况比较简单(懒}
\end{theorem}
\begin{proof}目标是通过$L^p$中的收敛控制$L^1$的收敛(H\"{o}lder 不等式),进而得到柯西列收敛从而证明完备性。设$\{x_n(t)\}$是$L^p(E)$中的柯西列,取子列$\{x_{n_k}(t)\}$满足$||x_{n_{k+l}}-x_{n_k}||<2^{-k} \ (k=1,2,3,\cdots)$。在下面的讨论中我们需要用到$m(E)<+\infty$这个条件,但这并不是总能满足,因此取可测集$E_1 \subset E \ , \ m(E_1)<+\infty$,由 H\"{o}lder 不等式:
\begin{equation*}
\begin{aligned}
\int_{E_1}|x_{n_{k+1}}(t)-x_{n_k}(t)|\dd t & \leq \left(\int_{E_1}1^q\dd t\right)^{\frac{1}{q}}\left(\int_{E_1}|x_{n_{k+1}}(t)-x_{n_k}(t)|^p\right)^{\frac{1}{p}} \\
& \leq \left[m(E_1)\right]^{\frac{1}{q}}||x_{n_{k+1}}(t)-x_{n_k}(t)||_{L^p(E_1)} \leq \left[m(E_1)\right]^{\frac{1}{q}}\cdot 2^{-k}
\end{aligned}
\end{equation*}
由 Fatou 引理:
\[\int_{E_1} \lim_{n \to \infty}\sum_{k=1}^n|x_{n_{k+1}}(t)-x_{n_k}(t)|\dd t \leq \lim_{n \to \infty} \int_{E_1} \sum_{k=1}^n|x_{n_{k+1}}(t)-x_{n_k}(t)|\dd t \leq \left[m(E_1)\right]^{\frac{1}{q}}\sum_{k=1}^{\infty}\frac{1}{2^k}<\infty\]
因此级数$\sum|x_{n_{k+1}}(t)-x_{n_k}(t)|$在$E_1$上$a.e.$(几乎处处)收敛。我们知道以下两个事实:1. $E$可以写成可数个有限测度的子集的并集;2. 由可列可加性保证可数个零测集的并集还是零测集。因此可以论断因此级数$\sum|x_{n_{k+1}}(t)-x_{n_k}(t)|$在$E$上$a.e.$(几乎处处)收敛。进而:
\[x_{n_l}(t)=x_{n_1}(t)+\sum_{k=1}^{l-1}\left(x_{n_{k+1}}(t)-x_{n_k}(t)\right) \leq x_{n_1}(t)+\sum_{k=1}^{l-1}|x_{n_{k+1}}(t)-x_{n_k}(t)| \ \text{在} \ E \ \text{上} \ a.e. \ \text{收敛}\]

记$x_{n_l}(t)$极限为$x(t)$,柯西列得到的是逐点收敛,为了证明完备性,下面要证明的自然是$x(t) \in L^p(E)$且$x_{n_l}(t) \to x(t)$,由 Fatou 引理(也可以用控制收敛定理)及柯西列的有界性:
\[\int_E|x(t)|^p\dd t=\int_E\lim_{l \to \infty}|x_{n_l}(t)|^p\dd t \leq \lim_{l \to \infty}\int_E|x_{n_l}(t)|^p\dd t=\lim_{l \to \infty}||X_{n_l}(t)||^p_{L^p(E)}<+\infty \quad \Leftrightarrow \quad x(t) \in L^p(E)\]
\begin{equation*}
\begin{aligned}
||x_{n_k}(t)-x(t)||^p_{L^p(E)} & =\int_E|x_{n_k}(t)-x(t)|^p\dd t=\int_E\lim_{l \to \infty}|x_{n_k}(t)-x_{n_l}(t)|^p\dd t \\
& \leq \lim_{l \to \infty}\int_E|x_{n_k}(t)-x_{n_l}(t)|^p\dd t=\lim_{l \to \infty}||x_{n_k}(t)-x_{n_l}(t)||^p_{L^p(E)} \leq \left(\frac{1}{2^k}\right)^p \to 0
\end{aligned}
\end{equation*}
即$x_{n_k}(t)$强收敛于$x(t)$:$x_{n_k}(t) \to x(t)$。
\end{proof}

\begin{theorem}
    $L^p(E) \ (p \geq 1)$是可分的。
\end{theorem}
\begin{proof}只证明$E=[a,b]$的情形,其他的不太好证,利用可数的多形式逐步逼近$L^p$函数:有理多项式$\Rightarrow$多项式$\Rightarrow$连续函数$\Rightarrow$有界$L^p$函数$\Rightarrow$$L^p$函数。设$x(t) \in L^p(E)$,令
\[x_n(t)=\left \{
\begin{array}{ll}
    x(t) & , \ |x(t)| \leq n \\ 0 & , \ |x(t)|>n
\end{array}
\right.
\quad \Rightarrow \quad \int_E|x(t)|^p\dd t=\sum_{n=1}^{\infty} \ \int\limits_{E \cap \{n<|x(t)| \leq n+1\}}|x(t)|^p\dd t<\infty\]
\[\Rightarrow \quad \forall \varepsilon>0 \ , \ \exists \, N>0 \quad \text{s.t.} \quad n \geq N \ , \ \int\limits_{E \cap \{|x(t)|>n\}}|x(t)|^p\dd t<\left(\frac{\varepsilon}{3}\right)^p\]
\[\Rightarrow \quad ||x_n(t)-x(t)||_{L^p(E)}=\left(\int_E|x_n(t)-x(t)|^p\dd t\right)^{\frac{1}{p}}=\left( \ \int\limits_{E \cap \{|x(t)|>n\}}|x(t)|^p\dd t\right)^{\frac{1}{p}}<\left(\frac{\varepsilon}{3}\right)^{p \cdot \frac{1}{p}}=\frac{\varepsilon}{3}\]
取定一个$n \geq N$,由 Lusin 定理[定理\ref{theorem:lusin}]:
\[\exists \, y(t) \in C[a,b] \ , \ \mathop \text{sup}\limits_{t \in [a,b]}|y(t)| \leq 2n \quad \text{s.t.} \quad m(\{y(t) \neq x(t)\} \cap E)<\left(\frac{\varepsilon}{3}\right)^p \cdot \frac{1}{(3n)^p}\]
对任意$y(t) \in C[a,b]$,$\min\{|y(t)|,2n\} \in C[a,b]$且因为$|x(t)|<n$,上述取法对计算$m(\{y(t)=x_n(t)\})$没有影响。
\begin{equation*}
\begin{aligned}
||x_n(t)-y(t)||_{L^p(E)}&=\left(\int_E|x_n(t)-y(t)|^p\dd t\right)^{\frac{1}{p}}=\left( \ \int\limits_{E \cap \{x_n(t) \neq y(t)\}}|x_n(t)-y(t)|^p\dd t\right)^{\frac{1}{p}}\\
&\leq \left( \ \int\limits_{E \cap \{x_n(t) \neq y(t)\}}(n+2n)^p\dd t\right)^{\frac{1}{p}} \leq \left(\left(\frac{\varepsilon}{3}\right)^p \cdot \frac{1}{(3n)^p} \cdot (3n)^p\right)^{\frac{1}{p}}=\frac{\varepsilon}{3}
\end{aligned}
\end{equation*}
由 Weierstrass 定理,存在有理系数多项式$P(t)$使得
\[\mathop \text{sup}\limits_{t \in [a,b]}|P(t)-y(t)|<\frac{\varepsilon}{3m(E)^{\frac{1}{p}}} \quad \Rightarrow \quad ||P(t)-y(t)||_{L^p(E)}=\left(\int_E|P(t)-y(t)|^p\dd t\right)^{\frac{1}{p}} \leq \left(m(E) \cdot \frac{\varepsilon^p}{3^pm(E)}\right)^{\frac{1}{p}}=\frac{\varepsilon}{3}\]
由三角不等式
\[||P(t)-x(t)||_{L^p(E)} \leq ||P(t)-y(t)||_{L^p(E)}+||y(t)-x_n(t)||_{L^p(E)}+||x_n(t)-x(t)||_{L^p(E)}<\varepsilon\]
由$\varepsilon$的任意性,得$||P(t)-x(t)||_{L^p(E)} \to 0$。即有理多项式集合在$L^p(E)$中稠密,即$L^p(E)$可分。
\end{proof}

在这 Riesz-Fischer 定理[定理\ref{theorem:rf}]的证明中,我们有提到过不同范数之间的强弱:
\begin{proposition}[范数强弱]
设$m(E)<\infty \ , \ p_1<p_2$,由$H\ddot{o}lder$不等式:
\[\left(\int_E|x(t)|^{p_1}\dd t\right)^{\frac{1}{p_1}} \leq \left[\left(\int_E\left(|x(t)|^{p_1}\right)^{\frac{p_2}{p_1}}\dd t\right)^{\frac{p_1}{p_2}}\left(\int_E1^{\frac{p_2}{p_2-p_1}}\dd t\right)^{1-\frac{p_1}{q_2}}\right]^{\frac{1}{p_1}} \leq \left(\int_E|x(t)|^{p_2}\dd t\right)^{\frac{1}{p_2}} \cdot m(E)^{\frac{1}{p_1}-\frac{1}{p_2}}\]
即$x(t) \in L^{p_2}(E) \ \Rightarrow \ x(t) \in L^{p_1}(E) \ , \ \text{即} \ L^{p_2}(E) \subset L^{p_1}(E)$。
\end{proposition}

我们自然会好奇,上面这个性质当$p \to \infty$时成立吗?
\begin{definition}[本质有界]
若存在零测集$E_0 \ , \ m(E_0)=0$使$x(t)$在$E/E_0$上有界,则称$x(t)$在$E$上本质有界(essentially bounded)。
\end{definition}
\begin{definition}
我们将$L^{\infty}(E)$空间定义为$L^{\infty}(E)=\left\{E\text{上本质有界的可测函数构成的集合}\right\}$,并且定义$L^{\infty}(E)$范数:
\[||x(t)||_{L^p(E)}=\mathop \text{inf}\limits_{\substack{E_0 \subset E \\ m(E_0)=0}}\mathop \text{sup}\limits_{E/E_0}|x(t)|\]
\end{definition}
\begin{proof}
可以验证$||x(t)||_{L^p(E)}$也是范数,同时可以证明$||x(t)||_{L^p(E)}<\infty \ \Leftrightarrow \ $本质有界。

若$x(t)$本质有界,则$\exists \, E_0 \quad \text{s.t.} \quad E/E_0$上$|x(t)|$有界$ \ \Rightarrow \ ||x(t)||_{L^{\infty}(E)}<\infty$。反之,由下确界的性质有:
\[\forall n \in \mathbb{Z}_+ \ , \ \exists \,E_n \subset E \ , \ m(E_n)=0 \quad \text{s.t.} \quad \mathop \text{sup}\limits_{E/E_n}|x(t)|<||x(t)||_{L^{\infty}(E)}+\frac{1}{n}\]
\[E_0=\bigcup_{n=1}^{\infty}E_n \ , \ m(E_n)=0 \quad \Rightarrow \quad m(E_0)=0 \quad \Rightarrow \quad \mathop \text{sup}\limits_{E/E_0}|x(t)|<||x(t)||_{L^{\infty}(E)}<\infty\]
即$x(t)$本质有界。
\end{proof}

\begin{definition}[本质上界] \label{infty}
    我们称$||x(t)||_{L^{\infty}} (E)$为$x(t)$在$E$上的本质上界(确实是对$|x(t)|$而言的)。
\end{definition}

\begin{proposition}[$L^{\infty}(E)$空间的性质]
限定$m(E)<+\infty$,则$\forall x(t) \in L^{\infty}(E) \ , \ x(t) \in L^p(E) \ (p \geq 1)$,反之则不然,这也即:
\[L^{\infty}(E) \subset \bigcap_{p \geq p_0}L^{p}(E) \ (\forall p_0 \geq 1)\]
但需要注意
\[L^{\infty}(E) \neq \bigcap_{p \geq p_0}L^{p}(E) \ (\forall p_0 \geq 1)\]
同时我们也容易看出来$||x||_{L^p(E)} \leq ||x||_{L^{\infty}(E)}$,毕竟正常$L^p$范数可以看作是对函数值的p次几何平均,而$L^{\infty}$范数只取了极大值。
\end{proposition}

\begin{example}
\quad $x(t)=\ln t \ , \ t \in (0,1]$,可以验证$x(t) \in L^p(0,1] \ (p \geq 1)$但$x(t) \notin L^{\infty}(E)(0,1]$。
\end{example}

\begin{theorem}
    $(L^{\infty}(E),||\cdot||_{L^{\infty}(E)})$是 Banach 空间。
\end{theorem}
\begin{proof}
设$\{x_n(t)\}$是柯西列,则$\exists \, \{x_{n_{k}}(t)\} \subset \{x_n(t)\}$满足$\forall k,l \in \mathbb{N} \ , \ ||x_{n_{k+l}}(t)-x_{n_{k}}(t)||<2^{-k}$,因此$\exists \, E_{k,l}$满足$m(E_{k,l})=0$,使得对$\forall k,l \in \mathbb{N}, \ t \in E/E_{k,l}$都有$|x_{n_{k+l}}(t)-x_{n_{k}}(t)|<2^{-k}$成立。由可列可加性:
\[E_0=\bigcup_{k,l\in \mathbb{N}}E_{k,l} \quad \Rightarrow \quad m(E_0)=0 \quad \Rightarrow \quad \forall t \in E/E_0 \quad \text{s.t.} \quad \forall k,l \in \mathbb{N} \ , \ |x_{n_{k+l}}(t)-x_{n_{k}}(t)|<2^{-k}\]
因此,如下级数存在:
\[\forall t \in E/E_0 \quad \text{s.t.} \quad \sum_{k=1}^{\infty}|x_{n_{k+1}}(t)-x_{n_{k}}(t)|<\sum_{k=1}^{\infty}2^{-k}=1<+\infty\]
定义子列$\{x_{n_k}(t)\}$的极限为$x(t)$,则
\[\forall t \in E/E_0 \quad \text{s.t.} \quad |x(t)|=\left|x_{n_1}(t)+\sum_{k=1}^{\infty}(x_{n_{k+1}}(t)-x_{n_{k}}(t))\right|<+\infty\]
因此$x_{n_k}(t) \to x(t)\in L^{\infty}(E)$,证明完逐点收敛后下证依范数收敛。对$\forall t \in E/E_0$,
\[|x_{n_{k+l}}(t)-x_{n_{k}}(t)|<2^{-k} \quad \Rightarrow \quad \forall l \in \mathbb{N}_+ \ , \  \mathop \text{sup}\limits_{E/E_0}|x_{n_{k+l}}(t)-x_{n_{k}}(t)| \leq 2^{-k} \quad \Rightarrow \quad \mathop \text{sup}\limits_{E/E_0}|x(t)-x_{n_{k}}(t)| \leq 2^{-k}\]
因此对以下范数:
\[||x(t)-x_{n_k}(t)||_{L^{\infty}(E)}=\mathop \text{inf}\limits_{\substack{E_0 \subset E \\ m(E_0)=0}} \mathop \text{sup}\limits_{E/E_0}|x(t)-x_{n_{k}}(t)| \leq 2^{-k} \ (\forall k=1,2,\cdots)\]
令$k \to \infty \ , \ x_{n_k}(t) \to x(t)$即$x_{n_k}(t)$依范数收敛到$x(t)$。
\end{proof}

\begin{definition}[离散化的$L^p$空间:$l^p$]
在$\mathbb{Z}_+$上考虑计数测度$\mu_c$,数列$x=\{\xi_n\}_{n=1}^{\infty}$可视为$\mathbb{Z}_+$的可测函数,积分定义为:
\[\int_{\mathbb{Z}_+}|x|^p\dd\mu_c=\sum_{n=1}^{\infty}|\xi_n|^p\]
我们定义$l^p$空间:
\[l^p=\{\{\xi_n\}_{n=1}^{\infty} \ , \ \sum_{n=1}^{\infty}|\xi_n|^p<\infty\}, \quad ||x||_{l^p}=\left(\sum_{n=1}^{\infty}|\xi_n|^p\right)^{\frac{1}{p}}\]
当$p=\infty$时:
\[l^{\infty}=\{\{\xi_n\}_{n=1}^{\infty} \ , \ \mathop \text{sup}\limits_{n \in \mathbb{Z}_+}|\xi_n|<\infty\}, \quad ||x||_{l^{\infty}}=\mathop \text{sup}\limits_{n \in \mathbb{Z}_+}|\xi_n|\]
可以证明$l^p \ , \ l^{\infty}$是$Banach$空间。
\end{definition}

\section{赋范线性空间的其他性质}
\subsection{赋范线性空间的子空间与完备化}
\begin{definition}[赋范线性空间的子空间]
设$(X,||\cdot||)$是赋范线性空间,$X_1$是$X$的子空间,若$(X_1,||\cdot||)$是赋范线性空间,则称$(X_1,||\cdot||)$是$(X,||\cdot||)$的一个子空间。
\end{definition}
\begin{theorem}
\begin{itemize}
    \item 1. 赋范线性空间的完备子空间是 Banach 空间;
    \item 2. Banach 间的闭子空间是 Banach 空间。
\end{itemize}
\end{theorem}
\begin{example}
\quad $l^{\infty}$是 Banach 空间,$||x||=\mathop \text{sup}\limits_{n \in \mathbb{Z}_+}|\xi_n| \ , \ x=\{\xi_n\}$,记$c=\{l^{\infty}\text{中的收敛数列全体}\}$,$c$是 Banach 空间。
\end{example}
\begin{proof}
只需证明$c$是$l^{\infty}$的闭子空间。设$x_n=\{\xi_{n,i}\} \in c$,$x_n \to x=\{\xi_i\}$,需证$x \in c$,即证$\{\xi_i\}$是收敛数列:
\[\forall \varepsilon>0 \ , \ \exists \, N>0 \quad \text{s.t.} \quad n \geq N \ , \ ||x_n-x||_{l^{\infty}}<\varepsilon \ \Rightarrow \ \exists \, N>0 \quad \text{s.t.} \quad n \geq N \ , \ \mathop \text{sup}\limits_{i \in \mathbb{Z}_+}|\xi_{n,i}-\xi_i|<\varepsilon\]
由于$\{\xi_{n,i}\}$是收敛点列,故它是柯西列,则对$\forall \varepsilon>0 \ , \ \exists \, I_n>0 \ , \ i,j \geq I_n$有$|\xi_{n,i}-\xi_{n,j}|<\varepsilon$。因此:
\[|\xi_{i}-\xi_{j}| \leq |\xi_{i}-\xi_{n,1}|+|\xi_{n,i}-\xi_{n,j}|+|\xi_{n,j}-\xi_{j}|<3\varepsilon\]
即$x=\{\xi_i\}$是柯西列且是实数列,故$x$收敛,从而$x \in c$。
\end{proof}

我们知道,赋范线性空间是一个距离空间,距离空间是可以完备化的,那么我们自然就会去思考,这样完备化得到的距离空间还是不是赋范线性空间呢?为此我们需要验证其线性结构(加法,数乘)以及范数。
\begin{theorem}
    设$(X,||\cdot||)$是不完备的赋范线性空间,它作为距离空间有完备化$\tilde{X}$,可以证明$\tilde{X}$是$Banach$空间。
\end{theorem}
\paragraph*{recall} \quad 设$\tilde{x}=[\{x_n\}] \in \tilde{X}$是$X$中柯西列的等价类。
\paragraph*{定义加法} \quad $\tilde{x}+\tilde{y}=[\{x_n+y_n\}]$\\
容易验证: 1.柯西列相加还是柯西列; 2.加法的定义与等价类无关。
\paragraph*{定义数乘} \quad $k\tilde{x}=[\{kx_n\}]$\\
容易验证: 1.柯西列数乘后还是柯西列; 2.数乘的定义与等价类无关。
\paragraph*{定义范数}
\[||\tilde{x}||_{\tilde{X}}=\lim_{n \to \infty}||x_n||_X\]
我们需要验证:1.$\{\tilde{x}_n\}$是柯西列; 2.$||\tilde{x}||_{\tilde{X}}$的取值与代表元的选取无关。\\
我们知道,$\{x_n\}_{n=1}^{\infty} \subset X$是柯西列时,可以得到$\{||x_n||\}_{n=1}^{\infty}$是收敛的柯西列。
\[\forall \varepsilon>0 \ , \ \exists \, N>0 \quad \text{s.t.} \quad \forall m>n \geq N \ , \ ||x_n-x_m||<\varepsilon\]
\[\forall \varepsilon>0 \ , \ \exists \, N>0 \quad \text{s.t.} \quad \forall m>n \geq N \ , \ \left|||x_n||-||x_m||\right| \leq ||x_n-x_m||<\varepsilon\]
当$m \to \infty$时,记$x:=x_m \ (m \to \infty)$:
\[\forall \varepsilon>0 \ , \ \exists \, N>0 \quad \text{s.t.} \quad \forall n \geq N \ , \ \left|||x_n||-||x||\right|<\varepsilon\]
1.即证。

我们定义等价类为$x \mapsto \tilde{x}=[\{x,x,\cdots\}]$,则$x$可嵌入$\tilde{x}$中,则
\[||\tilde{x}||_{\tilde{X}}=\lim_{n \to \infty}||x_n||_X=||x||_X\]
两个范数的定义是一致的,2.也证毕。\\
\textbf{Q.E.D.}

\subsection{商空间与乘积空间}
在有限维空间中,商空间的定义如下,我们先定义$V$是定义在数域$\mathbf{K}$上的一个向量空间:
\begin{definition}[商空间]
    商空间$V/A=\{\tilde{x}=x+A\}$,其中$\tilde{x}$是一个等价类,等价关系$\sim$定义如下:$x \sim y$当且仅当$x-y \in A$。
\end{definition}
再者,我们定义商映射$\pi:V \to V/A \ , \ x \mapsto \tilde{x}$,显然,这是个线性映射且是满射。
\[\text{加法:}\tilde{x}+\tilde{y}=\tilde{x+y} \qquad \text{数乘:}k\tilde{x}=\tilde{kx} \ (x,y \in V \ , \ k \in \mathbb{K})\]
注:有限维的子空间都是闭子空间。

可以看到,商空间其实是我们所熟知的平行这个概念的推广,现在这个空间中的跟某个$x$之差属于$A$的元素$\tilde{x}$视作一个等价类,可以说说$\tilde{x}$跟x只差了一个$A$。

下面我们来看无限维的情况:假设X是赋范线性空间,$M$是$X$的闭子空间,在商空间$X/M$中定义范数
\[||\tilde{x}||=\mathop \text{inf}\limits_{y \in \tilde{x}}||y||=\mathop \text{inf}\limits_{y-x \in M}||y||=\mathop \text{inf}\limits_{z \in M}||x+z||\]
则课称$X/M$是关于$M$的赋范商空间。我们可以验证$||\tilde{x}||$是范数:
\paragraph*{1、正定性} \quad $||\tilde{x}||>0$是显然的,而当
\[||\tilde{x}||=0 \ , \ \exists \, \{y_n\}_{n=1}^{\infty} \subset \tilde{x} \quad \text{s.t.} \quad \lim_{n \to \infty}||y_n||=0\]
而由于$M$是闭子空间,$x+M$也是闭子空间,故$\exists \, y \in \tilde{x} \quad \text{s.t.} \quad y_n \to y \ (n \to \infty)$
\[||y||=\lim_{n \to \infty}||y_n||=0 \ \Rightarrow \ y=0 \ \Rightarrow \ \tilde{x}=\tilde{y}=\tilde{0}\]
\paragraph*{2、齐次性}
\[||k\tilde{x}||=\mathop \text{inf}\limits_{y \in k\tilde{x}=\tilde{kx}}||y||=\mathop \text{inf}\limits_{z \in \tilde{x}}||kz||=|k|\mathop \text{inf}\limits_{z \in \tilde{x}}||z||=|k|||\tilde{x}||\] 
\paragraph*{3、三角不等式} \quad $\tilde{x},\tilde{y} \in X/M$
\[||\tilde{x}+\tilde{y}||=\mathop \text{inf}\limits_{z \in \tilde{x+y}}||z||=\mathop \text{inf}\limits_{z \in M}||x+y+z|| \leq \mathop \text{inf}\limits_{z \in M}\left(||x+\frac{1}{2}z||+||y+\frac{1}{2}z||\right)\]
\[\leq \mathop \text{inf}\limits_{z \in M}||x+\frac{1}{2}z||+||y+\mathop \text{inf}\limits_{z \in M}\frac{1}{2}z||=||\tilde{x}||+||\tilde{y}||\]
关于商空间,我们有以下定理。
\begin{theorem}
    若$X$是$Banach$空间,$M$是$X$的闭子空间,那么$X/M$是$Banach$空间。
\end{theorem}
\textbf{Proof:} 设$\{\tilde{x}_n\}$是$X/M$中的柯西列,取其子列$\{\tilde{x}_{n_k}\}$使得$||\tilde{x}_{n_{k+1}}-\tilde{x}_{n_k}||<1/2^k \ (k=1,2,\cdots)$,由范数的定义:
\[\mathop \text{inf}\limits_{y_k \in \tilde{x}_{n_{k+1}}-\tilde{x}_{n_k}}||y_k||<\frac{1}{2^k} \quad (\forall k \in \mathbb{Z}_+) \quad \Leftrightarrow \quad \exists \, y_k \in \tilde{x}_{n_{k+1}}-\tilde{x}_{n_k} \quad \text{s.t.} \quad ||y_k||<\frac{1}{2^k}\]
\[\Rightarrow \quad \sum_{k=1}^{\infty}y_k \ \text{在$X$中收敛} \quad \Rightarrow \quad x_{n+1}+\sum_{k=1}^{\infty}y_k \ \text{在$X$中收敛,记极限为} \ S\]
记上述级数的部分和为$S_i$,则易知$S_i$在$X$中的收敛
\[S_i=x_{n_1}+\sum_{k=1}^{i-1}y_k\]
则
\[\tilde{S}_i=\tilde{x}_{n_1}+\sum_{k=1}^{i-1}\tilde{y}_k=\tilde{x}_{n_1}+\sum_{k=1}^{i-1}\left(\tilde{x}_{n_{k+1}}-\tilde{x}_{n_k}\right)=\tilde{x}_{n_i}\]
\[\Rightarrow \quad ||\tilde{x}_{n_i}-\tilde{x}||=||\tilde{S}_i-\tilde{x}||=\mathop \text{inf}\limits_{y \in \tilde{S}_i-\tilde{x}}||y|| \leq ||S_i-x||\rightarrow 0 \quad (i \rightarrow \infty)\]
子列收敛,故原柯西列也收敛。

\textbf{Q.E.D.}
\begin{definition}[乘积空间]
    设$(X_1,||\cdot||_{X_1})$与$(X_2,||\cdot||_{X_2})$是赋范线性空间,则称$(X_1 \times X_2,||\cdot||)$为乘积空间,其中范数定义如下
    \[||(x_1,x_2)||=||x_1||_{X_1}+||x_2||_{X_2} \quad (\forall x_1 \in X_1,x_2 \in X_2)\]
\end{definition}
若$(X_1,||\cdot||_{X_1})$与$(X_2,||\cdot||_{X_2})$是$Banach$空间,则$(X_1 \times X_2,||\cdot||)$也为$Banach$空间。

\subsection{赋范线性空间的基} \label{baseset}
对有限维赋范线性空间$X$,如果我们称$\{e_i\}_{i=1}^{\infty} \subset X$为是$X$的一组基,则其应满足
\[\forall x \in X \ , \ \exists \, ! (\xi_1,\xi_2,\cdots,\xi_n) \in \mathbb{R}^n \quad \text{s.t.} \quad x=\sum_{i=1}^n\xi_ie_i\]

类似的,我们也希望基的概念能拓展到无限维的赋范线性空间上来\footnote{毕竟泛函分析又名无穷维线性代数(乐},类似的我们可能会有以下两种想法(然而并不会):
\paragraph*{$Hamel$基}
如果我们保留有限组合这个性质,我们按照如下方式可以定义$Hamel$基:
\begin{definition}[$Hamel$基]
    设无穷维赋范线性空间$X$的线性无关子集为$\{x_{\alpha}\} \ (\alpha \in \Lambda)$,若其张成的空间(称为线性包)满足$\text{Span}\{x_{\alpha}\}=X$,则称$\{x_{\alpha}\}$为$X$的一个$Hamel$基。
\end{definition}
这样的基是有限维赋范线性空间的基的直接推广,它总是存在的(这点可以利用$Zorn$引理证明,但这个结果其实挺显然而且我不关心这个证明所以我不证),但缺点也很明显,就是基的个数可能会太多,这显然会给运算带来很大的麻烦。

于是便有了另外一种基的出现:
\paragraph*{$Schauder$基}
我们将组合的个数适当拓展到至多可数,我们便可按照如下方式定义$Schauder$基:
\begin{definition}[$Schauder$基]
    设无穷维赋范线性空间$X$中的点列$\{e_n\}_{n=1^{\infty}}$若满足
    \[\forall x \in X \ , \ \exists \, !\{\xi_n\}_{n=1}^{\infty} \ (\xi_n \in \mathbb{R}^1) \quad \text{s.t.} \quad x=\sum_{i=1}^{\infty}\xi_ie_i\]
    则称$\{e_n\}_{n=1^{\infty}}$是$X$的一个$Schauder$基。
\end{definition}
当然,从定义可以看出,不是所有赋范线性空间都有$Schauder$基,$Schauder$基的存在要求其赋范线性空间是可分的。
\paragraph*{例1} \quad $l^p \ (p \geq 1) \ , \ e_i=\{0,\cdots,0,1,0,\cdots\}$ (第$i$位为1),可验证$\{e_n\}_{n=1}^{\infty}$是$Schauder$基。
\paragraph*{例2} \quad $L^2[0,2\pi]$可验证$\{1,\cos nx,\sin nx \ (n=1,2,\cdots)\}$是$Schauder$基。

对于$Schauder$基还有以下小结论:
\[x=\sum_{i=1}^{\infty}\xi_ie_i=\lim_{n \to \infty}\sum_{i=1}^{n}\xi_ie_i \quad \Leftrightarrow \quad ||x-\sum_{i=1}^n\xi_ie_i||=||\sum_{i=n+1}^{\infty}\xi_ie_i|| \rightarrow 0 \quad (n \rightarrow 0) \quad \Rightarrow \quad \xi_i \rightarrow 0 \ (i \to \infty)\]
该结论可以用于快速判断某些点列是不是$Schauder$基,同时,这其实也暗示了可分的赋范线性空间其实“维数”也并不是很多。
\paragraph*{例1} \quad $l^{\infty} \ , \ e_i=\{0,\cdots,0,1,0,\cdots\}$ (第$i$位为1),$\{e_n\}_{n=1}^{\infty}$不是$Schauder$基。

这是显然的,因为该点列末端并不$\to 0$。
\begin{theorem}
    $Banach$空间$X$有$Schauder$基,则$X$可分。
\end{theorem}
\textbf{Proof:}
\[\text{令}A=\left\{\left.x=\sum_{i=1}^nr_ie_i\right|r_i \in \mathbb{Q} \ , \ n \in \mathbb{N}\right\}\text{,则$A$是可数集}\]
下证$A$在$X$中稠密,即:
\[\forall \varepsilon>0 \ , \ x \in X \ , \ x=\sum_{i=1}^{\infty}\xi_ie_i \quad \Leftrightarrow \quad \lim_{n \to \infty}||x-\sum_{i=1}^n\xi_ie_i||=0 \quad \Leftrightarrow \quad \exists \, N \in \mathbb{N} \quad \text{s.t.} \quad ||x-\sum_{i=1}^N\xi_ie_i||<\frac{\varepsilon}{2}\]
容易看出:
\[\exists \, r_1,r_2,\cdots,r_N \in \mathbb{Q} \ , \ y=\sum_{i=1}^Nr_ie_i \in A \quad \text{s.t.} \quad ||y-\sum_{i=1}^N\xi_ie_i||=||\sum_{i=1}^N(r_i-\xi_i)e_i|| \leq \sum_{i=1}^N|r_i-\xi_i|\cdot||e_i||\]
利用$Schwartz$不等式:
\[\sum_{i=1}^N|r_i-\xi_i|\cdot||e_i|| \leq \sqrt{\sum_{i=1}^N|r_i-\xi_i|^2} \cdot \sqrt{\sum_{i=1}^N||e_i||^2} \leq N \cdot \mathop \text{max}\limits_i||e_i|| \cdot \mathop \text{max}\limits_i|r_i-\xi_i|<\frac{\varepsilon}{2}\]
故有
\[||x-y|| \leq ||x-\sum_{i=1}^N\xi_ie_i||+||y-\sum_{i=1}^N\xi_ie_i||<\varepsilon\]

\textbf{Q.E.D.}

但是反之并不一定成立,可分的$Banach$空间并不一定有$Schauder$基。
\vspace{0.5cm}
同一个空间可以定义不同的范数,也即不同的距离函数,那我们自然而然地会想问不同距离函数之间是否有关系,他们的收敛性如何让,拓扑关系如何?
\begin{definition}[等价范数]
    设$||\cdot||_1,||\cdot||_2$是赋范线性空间$X$上的两个范数,若满足
    \[\forall x \in X \ , \ \exists \, c>0 \quad \text{s.t.} \quad ||x||_1 \leq c||x||_2\]
    则称$||\cdot||_2$比$||\cdot||_1$强。\\
    如果$||\cdot||_1$比$||\cdot||_2$强且$||\cdot||_2$比$||\cdot||_1$强,则称$||\cdot||_2,||\cdot||_1$等价,即
    \[\forall x \in X \ , \ \exists \, c_1,c_2>0 \quad \text{s.t.} \quad c_1||x||_1 \leq ||x||_2 \leq c_2||x||_1\]
\end{definition}
一个范数比一个范数更强,这种说法看起来比较模糊,其实强想描述的事情就是一个范数能被另外一个范数控制:
\begin{theorem}
    $||\cdot||_2$比$||\cdot||_1$强$ \quad \Leftrightarrow \quad \forall \{x_n\}_{n=1}^{\infty} \subset X \ , \ \text{if} \ ||x_n||_2 \to 0 \ , \ ||x_n||_1 \to 0$
\end{theorem}
\textbf{Proof:}\\
$"\Rightarrow" \quad \exists \, c>0 \quad \text{s.t.} \quad 0 \leq ||x_n||_1 \leq c||x_n||_2 \to 0 \quad \Rightarrow \quad ||x_n||_1 \to 0$\\
$"\Leftarrow" \quad$ 反证法,如若不然,则有:$\forall k \in \mathbb{N} \ , \ \exists \, x_k \in X \quad \text{s.t.} \quad ||x_k||_1>k||x_n||_2$\\
\[\text{令} \ y=\frac{x_k}{||x_k||_1} \ , \ \text{则} \ ||y_k||_1=\left\|\frac{x_k}{||x_k||_1}\right\|_1=1>k\frac{||x_k||_2}{||x_k||_1}=k\left\|\frac{x_k}{||x_k||_1}\right\|_2\]
\[\Rightarrow \quad ||y_k||_2<\frac{1}{k} \to 0 \quad (k \to \infty) \ \text{,但} \ ||y_k||_1=1\]
推出矛盾,故原命题成立。

\textbf{Q.E.D.}

\section{有限维赋范线性空间}
我们常常戏称希尔伯特空间是大号欧氏空间,是因为两者具有类似的空间结构和内积定义,加之可数和有限其实没多大差别。
但是在更一般的距离空间中,如果我们还想获得和欧氏空间类似的性质时,我们对空间的要求可以预料到会有相应的提高。
但这样的提高具体的条件是什么呢,看看本小结的标题就知道了。
\begin{theorem}\label{the:B}
    \textbf{1.} \quad 有限维赋范线性空间上的任意两个范数等价;\\
    \textbf{2.} \quad 任意$n$维赋范线性空间$X$与$\mathbb{R}^n$代数同构,拓扑同胚\footnote{代数同构指的是作为线性空间两者同构,可以用一个保持线性结构的双射关联;拓扑同胚指的是两者可以用一个双连续映射关联。}。
\end{theorem}
\textbf{Proof:} \quad 设$X$是$n$维赋范线性空间,$\{e_1,e_2,\cdots,e_n\}$是$X$的基,定义如下映射
\[T: \qquad X \to \mathbb{R}^n \qquad x=\sum_{i=1}^n\xi_ie_i \mapsto \xi=(\xi_1,\xi_2,\cdots,\xi_n)\]
易知$T$是良定义的,且是双射(因为基表示是唯一的),则$X$与$\mathbb{R}^n$线性同构,下证同胚。

考虑范数$||\cdot||_X$和$||x||_T:=||TX||_{\mathbb{R}^n}$,任意验证后者确实是范数。

第一步,通过算子的有界即连续(不知道为什么算子范数的内容会出现在这里)验证$T^{-1}$的连续性。
\[\forall x \in X \ , \ ||x||_X\left\|\sum_{i=1}^n\xi_ie_i\right\| \leq \sum_{i=1}^n|\xi_i|||e_i|| \leq \left(\sum_{i=1}^n|\xi_i|^2\right)^{\frac{1}{2}}\left(\sum_{i=1}^n||e_i||^2\right)^{\frac{1}{2}}:=\alpha||\xi||_{\mathbb{R}^n}\]
\[\Leftrightarrow \quad ||TX||_{\mathbb{R}^n} \geq \frac{1}{\alpha}||x|| \quad \Leftrightarrow \quad ||\xi||_{\mathbb{R}^n} \geq \frac{1}{\alpha}||T^{-1}\xi|| \quad \Rightarrow \quad \alpha \geq ||T^{-1}||\]
然后由$T^{-1}$的有界性知其连续,这个性质的证明等下一章再说。

第二步,同样的方法,利用上一步的结论,考虑函数$f$:
\[f: \qquad \mathbb{R}^n \to \mathbb{R} \qquad \xi \mapsto ||T^{-1}\xi||=||x||\]
容易验证$f$连续,$\forall \xi \eta \in \mathbb{R}^n \ , \ x:=T^{-1}\xi \ , \ y:=T^{-1}\eta$
\[|f(\xi)-f(\eta)|=|||x||-||y||| \leq ||x-y|| \leq \alpha ||T(x-y)||_{\mathbb{R}^n}=\alpha ||\xi-\eta||_{\mathbb{R}^n}\]

由于有限维,$f$限制在单位球面$S^{n-1} \subset \mathbb{R}^n$(紧集)上的最值可被达到,这里考虑最小值,即
\[\exists \, \xi_0=\sum_{i=1}^n\xi_0^ie_i \in S^{n-1} \quad \text{s.t.} \quad \beta:=f(\xi_0)=\mathop \text{min}\limits_{x \in S^{n-1}}f(x)=\left\|\sum_{i=1}^n\xi_0^ie_i\right\|>0\]
记$q=||\xi||_{\mathbb{R}^n}$,则
\[||x||=q\left\|\frac{x}{q}\right\|=qf\left(\frac{\xi}{q}\right) \geq q\beta=\beta||Tx||_{\mathbb{R}^n}\]
同上可知然后由$T$连续。

综上所述,$T$的拓扑同胚可证,最后由于范数$||x||$的任意性可知原命题成立。(实际上证明的是两个范数$||\cdot||_X$和$||x||_T:=||TX||_{\mathbb{R}^n}$等价)

\textbf{Q.E.D.}

上述定理告诉我们有限维的赋范线性空间基本上可以直接当成欧氏空间来看,当然是在范数意义下。
从证明中我们还可以知道,任意有限维赋范线性空间都是完备的,其上点列的依范数收敛直接等价于坐标分量收敛,这点可以通过证明中定义的映射$T$看出。

最重要的,有限维赋范线性空间中的有界集列紧,有界闭集是紧集。这个性质还可以用于判定一个赋范线性空间是否为有限维,我们将在后面证明这个定理,但在证明之前我们需要先介绍一个非常有用的工具——$Riesz$引理。
\begin{theorem}[$Riesz$引理]\label{Riesz}
    设$X_0$是赋范线性空间$X$的真闭子空间(不要求有限维),则
    \[\forall \varepsilon>0 \ , \ \exists \, x_0 \in X / X_0 \quad \text{s.t.} \quad ||x_0||=1 \ , \ d(x_0,X_0):=\mathop \text{inf}\limits_{x \in X_0}d(x_0,x) \geq 1-\varepsilon\]
\end{theorem}
\textbf{Proof:} 依照定理描述,我们知道
\[\forall x_1 \in X/X_0 \ , \ d:=d(x_1,X_0)=\mathop \text{inf}\limits_{x \in X_0}d(x_1,x)=\mathop \text{inf}\limits_{x \in X_0}||x_1-x||\]
由于$X_0$闭,可知$d>0$,且由下确界性质可知
\[\forall \varepsilon \in (0,1) \ , \ \exists \, x_2 \in X_0 \quad \text{s.t.} \quad d(x_1,x_2)=||x_1-x_2||<\frac{d}{1-\varepsilon}\]
则此时我们如下定义$x_0$
\[x_0=\frac{x_1-x_2}{||x_1-x_2||} \quad \Rightarrow \quad ||x_0||=1 \ , \ \forall x \in X_0 \ , \ ||x-x_0||=\left\|x-\frac{x_1-x_2}{||x_1-x_2||}\right\|=\frac{1}{||x_1-x_2||}\left\|||x_1-x_2||x+x_2-x_1\right\|\]
式中$||x_1-x_2||x+x_2 \in X_0$,故
\[\forall x \in X_0 \ , \ ||x-x_0|| \geq \frac{1}{||x_1-x_2||} \cdot d >\frac{1-\varepsilon}{d} \cdot d=1-\varepsilon\]

\textbf{Q.E.D.}

关于$Riesz$引理有很多可以细嗦的东西,但是我们先来看一幅图直观感受一下。
\begin{figure}[H]
    \center
    \includegraphics[scale=0.25]{./fig/3.5-1.png}
\end{figure}

对于$d(x_0,X_0)$我们首先应该能想到应该对应于$x_0 \perp X_0$,但是在赋范线性空间我们没办法定义垂直这件事,而$Riesz$引理告诉我们我们能找到一个差不多垂直的元素,当然由于差不多垂直,所以并不能保证$d(x_0,X_0) \geq 1$成立。
但是如果我们在有限维空间中考虑这件事,由完备性可知确实一定存在一个点满足$d(x_0,X_0) \geq 1$,\textbf{所以还是有限维香啊}。
我们可以严格描述并证明上面这句话。

\begin{proposition}
    若$X_0$为有限维真闭子空间,则$\exists \, x_0 \in S \quad \text{s.t.} \quad d(x_0,X_0)=1$。
\end{proposition}
\textbf{Proof:} 
\[\forall y_0 \in X/X_0 \ , \ d:=d(y_0,X_0)>0 \ , \ \exists \, x_0 \in X_0 \quad \text{s.t.} \quad d \leq d(t_0,x_n)<d(y_0,x_0)-\frac{1}{n}\]
通过上述方式构造出的点列$\{x_n\}$满足$||x_n|| \leq ||x_n-y_0||+||y_0||<+\infty$,是有限维闭子空间上的点列,由有限维赋范线性空间的性质可知$\{x_n\}$列紧,不妨设$x_n \to x_0 \in X_0$,其中$x_0$被称为\textbf{最佳逼近元}。
\[d \leq d(y_0,x_n)<d(y_0,x_0)-\frac{1}{n} \ (n \to \+\infty) \quad \Rightarrow \quad d(y_0,x_0)=d\]
取球面上另外一点$x'$满足如下条件可验证$d(x',X_0)=1$。
\[x'=\frac{y_0-x_0}{||y_0-x_0||} \quad \Rightarrow \quad d(x',X_0)=\mathop \text{inf}\limits_{z \in X_0}\left\|\frac{y_0-x_0}{||y_0-x_0||}-z\right\|=\frac{1}{||y_0-x_0||}\mathop \text{inf}\limits_{z \in X_0}\left\|||y_0-x_0||z+x_0-y_0\right\|=\frac{1}{d} \cdot d=1\]

\textbf{Q.E.D.}

顺便严格定义一下最佳逼近元。
\begin{definition}[最佳逼近元]
    设$(X,||\cdot||)$是赋范线性空间,$Y$是$X$的子空间,对$x \in X$,如果
    \[\exists \, y_0 \in Y \quad \text{s.t.} \quad ||x-y_0||=d(x,Y)=\mathop \text{inf}\limits_{y \, \in Y}||x-y||\]
    则称$y_0$为$Y$上关于$X$的最佳逼近元。
\end{definition}

但是在一般的情形下(无限维),最佳逼近元可能不存在,我们可以看下面一个例子。

\paragraph*{例1} \quad 设$X=\left\{f \in C[0,1]|f(0)=0\right\}$,其中$X_0 \in X$为
\[X_0=\left\{f \in X \Big{|} \int_0^1f(x)\dd x=0\right\}\]
则不存在$f \in S \quad \text{s.t.} \quad d(f,X_0)=1$。\\
\textbf{Proof:} \quad 采用反证法,假设$\exists \, f \in S \quad \text{s.t.} \quad d(f,X_0)=1$。
\[f_0 \in X \quad \Rightarrow \quad f_0 \in C[0,1] \ , \ f_0(0)=0 \ , \ ||f_0||=\mathop \text{sup}\limits_{x \in [0,1]}|f(x)|=1 \quad \Rightarrow \quad \left|\int_0^1f_0(x)\dd x\right|<1\]
另一方面
\[\forall f \in X/X_0 \ , \ \exists \, a \in \mathbb{R}^1 \quad \text{s.t.} \quad \int_0^1(f_0-af)\dd x=0 \quad (a=\int_0^1f_0(x)\dd x \Big{/} \int_0^1f(x)\dd x)\]
即$f_0-af \in X_0$,则
\[||f_0-(f_0-af)||=|a|||f|| \geq d(f_0,X_0)=1 \quad \Rightarrow \quad \left|\int_0^1f_0(x)\dd x\right|||f|| \geq \left|\int_0^1f(x)\dd x\right|\]
特别的,我们可以取$f$为一族函数$f_n(x)=x^{1/n}$,则
\[||f_n||=1 \ , \ \left|\int_0^1f_n(x)\dd x\right|=\frac{n}{n+1}\]
进而
\[\left|\int_0^1f_0(x)\dd x\right|||f_n|| \geq \left|\int_0^1f_n(x)\dd x\right|=\frac{n}{n+1} \to 1 \quad (n \to +\infty)\]
得出矛盾。\\
\textbf{Q.E.D.}

同时,即使最佳逼近元存在也可能不唯一,再看一个例子。

\paragraph*{例2} \quad $X=\mathbb{R^2} \ , \ ||\mathbf{x}||=||(x_1,x_2)||=\text{max}\left\{|x_1|,|x_2|\right\}$,取$\mathbf{x}_0=(0,1) \ , \ \mathbf{e}_1=(1,0) \ , \ X_0=\text{span}\{\mathbf{e}_1\}$,
\[\forall \mathbf{y} \in X_0 \ , \ \mathbf{y}=\lambda \mathbf{e}_1=(\lambda,0) \ , \ ||\mathbf{x}_0-\mathbf{y}||\text{max}\left\{|\lambda|,1\right\} \geq 1\]
则可知$\forall ||\mathbf{y}|| \leq 1$都是最佳逼近元。

下面我们就可以开始证明之前提到过的判定赋范线性空间有限维的定理了。

\begin{theorem}
    赋范线性空间$X$为有限维当且仅当$X$中任意有界集都是列紧的。
\end{theorem}
\textbf{Proof:} 一方面,由于$X$与$\mathbb{R}^n$同胚,在$X$上的有界集$E$的像集$T(E)$在$\mathbb{R}^n$有界,而$\mathbb{R}^n$上的有界集$T(E)$列紧,再由同胚可知$X$上的有界集$E$列紧。

另一方面,采用反证法,假设$X$是无穷维,任取$e_1 \in S=\{x \in X | \, ||x||=1\}$,由$Riesz$引理
\[X_1=\text{span}\{e_1\} \ , \ \exists \, e_2 \in S \ , \ d(e_2,X_1)>1-\frac{1}{2}=\frac{1}{2}\]
进而
\[X_2=\text{span}\{e_1,e_2\} \ , \ e_3 \in S \ , \ d(e_3,X_2)>\frac{1}{2} \quad \Rightarrow \quad d(e_3,e_1)>\frac{1}{2} \ , \ d(e_3,e_2)>\frac{1}{2}\]
由反证条件,上述操作可以一直做下去,可以得到序列$\{e_n\} \subset X$满足
\[\forall i,j \in \mathbb{N} \ , \ d(e_i,e_j)>\frac{1}{2} \ , \ ||e_i||=1\]
由于$X$中任意有界集都是列紧的可知$\exists \, \{e_{n_k}\} \subset \{e_i\}$收敛,故矛盾。

\textbf{Q.E.D.}
\chapter{有界线性算子}
\begin{introduction}
    \item 有界线性算子~\ref{BX}
    \item Banach-Steinhaus 定理~\ref{BS}
    \item 开映射定理与闭图像定理~\ref{kb}
    \item Hahn-Banach 定理~\ref{HB}
    \item Zorn 引理~\ref{Zorn}
\end{introduction}
\section{有界线性算子}\label{BX}
在正式介绍赋范线性空间的有界线性算子之前,我们可以回忆一下有限维的情形,设$X,Y$为有限维空间,根据上一小节的描述,我们不妨假设$X=\mathbb{R}^m,Y=\mathbb{R}^n$,考虑线性变换$L:\mathbb{R}^m \to \mathbb{R}^n$,选取基底$\{e_i\}_{i=1}^m \subset X , \ \{f_j\}_{j=1}^n \subset Y$,线性变换$L$在基底上有如下关系:
\[Le_i=\sum_j^na_{ij}f_j \ (a_{ij} \in \mathbb{R})\]
则$L$等价于矩阵$A=(a_{ij})_{1 \leq i \leq m , 1 \leq j \leq n}$,取:
\[x=\sum_{i=1}^m\xi_ie_i \in X \ , \ y=\sum_{j=1}^n\eta_jf_j \quad \Rightarrow \quad Lx=
\begin{pmatrix}
    a_{11} & a_{12} & \cdots & a_{1m} \\
    a_{21} & a_{22} & \cdots & a_{2m} \\
    \vdots & \vdots & \ddots & \vdots \\
    a_{n1} & a_{n2} & \cdots & a_{nm} \\
\end{pmatrix}
\begin{pmatrix}
    \xi_1 \\ \xi_2 \\ \vdots \\ \xi_m \\
\end{pmatrix}
=\begin{pmatrix}
    \eta_1 \\ \eta_2 \\ \vdots \\ \eta_n \\
\end{pmatrix}
=y
\]

\begin{definition}[线性算子]
设$X,Y$是赋范线性空间,$T$是从$X$的线性子空间$D(T) \subset X$到$Y$的映射,若其满足:
\[\forall x,y \in X \ , \ \alpha,\beta \in \mathbb{K} \ , \ T(\alpha x+\beta y)=\alpha Tx+\beta Ty\]
\end{definition}
\begin{remark}
线性算子的定义并不要求其定义域是出发空间的全空间,只要求是出发空间的一个线性子空间即可,当然可以是平凡子空间$(D(T)=X)$,同时无穷维的子空间也可以是稠密的$(C[0,1] \subset L[0,1])$。叫法上的区分:1、$T:D(T)=X \to Y$在$X$上到$Y$上;2、$T:D(T) \subset X \to Y$在$X$中到$Y$上。
\end{remark}
\begin{definition}[有界线性算子]
设$T:X \to Y$是线性算子,若$\forall x \in X \ , \ \exists \, M>0$使得$||Tx||_Y \leq M||x||_X$,则称其为有界线性算子。
\end{definition}
从$X \to Y$的全体有界线性算子记为$\mathscr{B}(X,Y)$。有界线性算子具有一些不错的性质,可以看下面一个定理。
\begin{theorem}\label{theorem:yjlx}
设$T:X \to Y$是线性算子,则:
\begin{itemize}
\item 1. $T$连续等价于$T$在一点处连续;
\item 2. $T$连续等价于$T$有界。
\end{itemize}
\end{theorem}
\begin{proof}
\begin{itemize}
\item 1. 充分性显然,下证必要性:设$T$在$x_0 \in X$连续,即$x_n \to x \ \Rightarrow \ Tx_n \to Tx$,则有:
\[Tx_n-Tx=T(x_n-x+y)-Ty \to 0 \quad (\forall y \in X)\]
\item 2. 充分性:由上一条知只需证明在一点处连续,不妨取为原点,取收敛于原点的点列$\{x_n\}$,由有界性:
\[||Tx_n|| \leq M||x_n|| \to 0 \quad \Rightarrow \quad Tx_n \to 0 \quad \Leftrightarrow \quad Tx_n-T(0) \to 0\]
必要性采用反证法:设$T$连续但是无界,$\forall n \in \mathbb{N} \ , \ \exists \, x_n \in X$满足$||Tx_n||>n||x_n||$,令:
\[y_n=\frac{x_n}{n||x_n||} \ , \ ||y_n||=\frac{||x_n||}{n||x_n||}=\frac{1}{n} \to 0 \quad \Rightarrow \quad y_n \to 0\]
由线性性可知$T(0)=0$,而
\[||Ty_n-T(0)||=||Ty_n||=\left\|\frac{Tx_n}{n||x_n||}\right\|>1\]
与连续性矛盾。
\end{itemize}
\end{proof}

\begin{definition}[算子范数]\label{definition:szfs}
设$T \in \mathscr{B}(X,Y)$,定义算子范数:
\[||T||=\mathop \text{sup}\limits_{x \in X/\{0\}}\frac{||Tx||}{||x||}=\mathop \text{sup}\limits_{||x||=1}||Tx||=\mathop \text{sup}\limits_{||x|| \leq 1}||Tx||\]
\end{definition}
既然叫范数说明$||T||$满足范数要求的条件,同时上述三个定义也是等价的。因此$(\mathscr{B}(X,Y),||\cdot||)$就构成了一个赋范线性空间。
\begin{example}
\quad $X=\mathbb{R}^m,Y=\mathbb{R}^n \ , \ L \in \mathscr{B}(X,Y)$,选定基底后$L$等价于矩阵$A=(a_{ij})_{m \times n}$,分别取$X,Y$空间中的两个元素,$x=(\xi_1,\xi_2,\cdots,\xi_m)^{\text{T}} \ , \ y=(\eta_1,\eta_2,\cdots,\eta_n)^{\text{T}}$满足$Lx=y$,则:
\[||Lx||=||y||=\sqrt{\sum_{i=1}^n|\eta_i|^2}=\sqrt{\sum_{i=1}^n\left|\sum_{j=1}^ma_{ij}\xi_j\right|^2} \leq \sqrt{\sum_{i=1}^n\sum_{j=1}^ma_{ij}^2 \cdot \sum_{j=1}^m\xi_j^2}=\text{tr}(A^{\text{T}}A)^{\frac{1}{2}}||x||:=||A||\cdot||X||\]
由上确界性质可知$||L|| \leq ||A||$,实际上是取等。
\end{example}

\begin{example}
\quad 微分算子,一维情形下,考虑$X=C[0,1] \ , \ T:X \to X \ , \ x(t) \mapsto Tx=x'(t)$,则该算子为无界算子,如取$x_n(t)=\sin nt \ , \ ||x_n||=1$,则$Tx_n=x_n'(t)=n \cos nt \ , \ ||Tx_n||=n \to +\infty$。高维情形下也类似,考虑$\Omega \subset \mathbb{R}^n$,记$C^k(\overline{\Omega})=\{f:\overline{\Omega} \to \mathbb{R}|\partial^{\alpha}f \in C(\overline{\Omega}) \ , \ |\alpha| \leq k\}$,其上范数为:
\[||f||_{C^k}=\mathop \text{max}\limits_{|\alpha| \leq k}||\partial^{\alpha}f||_{C^0}=\mathop \text{max}\limits_{|\alpha| \leq k}\mathop \text{max}\limits_{x \in \overline{\Omega}}|\partial^{\alpha}f(x)|\]
其中:
\[\alpha=(\alpha_1,\alpha_2,\cdots,\alpha_n) \ , \ \partial^{\alpha}f=\partial^{\alpha_1}_{x_1}\partial^{\alpha_2}_{x_2}\cdots\partial^{\alpha_n}_{x_n}f(x_1,x_2,\cdots,x_n) \ , \ \sum_{i=1}^n\alpha_i \leq k\]
因此,在高维版本中上述的一维微分算子可表示为:
\[X=C^{\infty}(\overline{\Omega})=\bigcap_{k=1}^{\infty}C^k(\overline{\Omega}) \ , \ T:X \to X \ , \ x(t) \mapsto Tx=\sum_{|\alpha| \leq k}a_{\alpha}(t)\cdot\partial^{\alpha}x(t)\]
其依然是无界算子,反例类似一维。
\end{example}

\begin{example}
\quad 积分算子,考虑 $L^2$上的 Lebesgue 积分:
\[X=L^2(\overline{\Omega}) \ , \ T:X \to \mathbb{R}^n \ , \ u(x) \mapsto Tu=\int_{\overline{\Omega}}u(x)\psi(x)\dd x\]
由柯西不等式:
\[||Tu||=|Tu| \leq ||u||_{L^2}||\psi||_{L^2} \quad \Rightarrow \quad ||T|| \leq ||\psi||_{L^2}\]
再由:
\[||T\psi||=|T\psi|=\int_{\overline{\Omega}}|\psi|^2=||\psi||_{L^2}^2 \quad \Rightarrow \quad ||T||=||\psi||_{L^2}\]
\end{example}

\section{有界线性算子空间}
上一小节中,我们提及有界线性算子全体构成一个赋范线性空间$(\mathscr{B}(X,Y),||\cdot||)$,自然而然我们也会想知道在这样一个赋范线性空间上,元素的收敛性该如何刻画,这对后面讨论空间完备性时至关重要。
\begin{theorem}
点列$\{T_n\} \subset \mathscr{B}(X,Y)$依范数收敛于$T$当且仅当$\{T_n\}$在单位球面$S=\{x \in X| \, ||x||=1\}$一致收敛于$T$。
\end{theorem} 
\begin{proof}
\[||T_n-T|| \to 0 \quad \Rightarrow \quad \forall x \in S \ , \ ||T_n(x)-T(x)|| \leq \mathop \text{sup}\limits_{x \in S}||(T_n-T)x||=||T_n-T||<\varepsilon\]
\[\forall x \in S \ , \ \exists \, N \in \mathbb{N} \quad \text{s.t.} \quad n>N \ , \ ||T_n(x)-T(x)||<\varepsilon \quad \Rightarrow \quad ||T_n-T||=\mathop \text{sup}\limits_{x \in S}||(T_n-T)x||<\varepsilon\]
\end{proof}

虽然这个定理一眼顶针,但既然上课讲了那我还是放进来了。下面这个定理就没那么显然了,如果我们想知道一个算子空间完备需要什么条件,那么按照定义自然而然地就需要其上的所有柯西列都收敛于其中。我们可能会猜想这需要出发空间和到达空间都完备,但如果考虑算子范数的定义,好像跟出发空间没什么太大关系,于是我们进一步猜测只跟到达空间有关系。那么到底是不是呢,答案是yes,下面我们来叙述这件事并证明它。
\begin{theorem}
如果$Y$是 Banach 空间,则$\mathscr{B}(X,Y)$也是 Banach 空间。
\end{theorem} 
\begin{proof}
设$\{T_n\}$为$\mathscr{B}(X,Y)$中的柯西列,即:
\[\forall \varepsilon>0 \ , \ \exists \, N \in \mathbb{N} \ , \ \forall n>N \ , \ p \in \mathbb{N} \quad \text{s.t.} \quad ||T_{n+p}-T_n||<\varepsilon\]
则:
\[\forall x \in X \ , \ ||T_{n+p}(x)-T_n(x)|| \leq ||T_{n+p}-T_n|| \cdot ||x||<\varepsilon||x||\]
即$\{T_n(x)\}$是$Y$中的柯西列,由完备性知其收敛,设$\forall x \in X \ , \ T_n(x) \to T(x)$,则只需验证$T \in \mathscr{B}(X,Y)$即可:
\begin{itemize}
\item 1. 线性性:
\[T(\alpha x+\beta y)=\lim_{n \to +\infty}T_n(\alpha x+\beta y)=\alpha\lim_{n \to +\infty}T_nx+\beta\lim_{n \to +\infty}T_ny=\alpha Tx+\beta Ty\]
\item 2. 有界性:
\[\forall x \in X \ , \ ||Tx||=\left\|\lim_{n \to +\infty}T_nx\right\|=\lim_{n \to +\infty}||T_nx||<+\infty \quad \Rightarrow \quad ||Tx|| \leq M||x||\]
\[||Tx|| \leq ||Tx-T_nx||+||T_nx|| \leq (\varepsilon+M)||X|| \quad \Rightarrow \quad ||T||<\varepsilon+M\]
\end{itemize}
\end{proof}

\begin{definition}[算子空间中的收敛]
设$T_n,T \in \mathscr{B}(X,Y)$:
\begin{itemize}
\item 1. 依范数收敛:$||T_n-T|| \to 0$;
\item 2. 一致收敛:$\forall \varepsilon>0 \ , \ x \in X \ , \ \exists \, N(\varepsilon) \in \mathbb{N} \quad \text{s.t.} \quad \forall n>N \ , \ ||T_n(x)-T(x)||<\varepsilon$;
\item 3. 逐点收敛:$\forall x \in X \ , \ \varepsilon>0 \ , \ \exists \, N(\varepsilon,x) \in \mathbb{N} \quad \text{s.t.} \quad \forall n>N \ , \ ||T_nx-Tx||<\varepsilon$。
\end{itemize}
\end{definition}
显然依范数收敛可以推出逐点收敛。
\begin{example}
\quad $X=l^p \ , \ \forall x=(\xi_1,\xi_2,\cdots,\xi_n,\cdots)$定义算子$T_n: \ X \to X \ , \ T_nx:=(\xi_{n+1},\xi_{n+2},\cdots)$,则对$\forall x \in X$:
\[||T_nx||=\left(\sum_{k=n}^{+\infty}|\xi_k|^p\right)^{\frac{1}{p}} \to 0\]
即$T_n$逐点收敛于$0$。但是$T_n$不依范数收敛于$0$,考虑$x_k=(0,0,\cdots,0,1,0,\cdots) \in S$ ($1$在第$k$位),则:
\[0 < ||T_n(x_n)||=(1,0,\cdots) \leq ||T_n||\]
\end{example}

\section{Banach-Steinhaus 定理}
\begin{definition}[算子有界性]
考虑算子空间中的一族算子$\{T_{\alpha}\}_{\alpha \in \Lambda} \subset \mathscr{B}(X,Y)$:
\begin{itemize}
\item 1. 一致有界:$\forall x \in X \ , \ \exists \, M>0 \quad \text{s.t.} \quad ||T_{\alpha}x|| \leq M||x||$,可以推出$\exists \, M>0 \quad \text{s.t.} \quad ||T_{\alpha}|| \leq M$;
\item 2. 逐点有界:$\forall x \in X \ , \ \exists \, M(x)>0 \quad \text{s.t.} \quad ||T_{\alpha}x|| \leq M(x)||x||$。
\end{itemize}
\end{definition}

在算子空间中一致有界还是强于逐点有界,那么我们自然会思考在什么条件下逐点有界能反推出一致有界。
\begin{theorem}[一致有界定理]
    $\{T_{\alpha}\}_{\alpha \in \Lambda} \subset \mathscr{B}(X,Y) \ , \ D(T_{\alpha})=X$ ,若$X$是 Banach 空间,则$\{T_{\alpha}\}$一致有界当且仅当$\{T_{\alpha}\}$逐点有界。
\end{theorem} 
\begin{proof}
证明分为两步,先证在某个开集(开球)上成立(Baire 纲定理[定理\ref{the:Baire}]),再证在全空间上成立(利用线性)。从一致有界到逐点有界显然,下证逐点有界到一致有界。定义映射:
\[p:X \to \mathbb{R} \ , \ p(x)=\mathop \text{sup}\limits_{\alpha \in \Lambda}||T_{\alpha}x||\]
显然有$\forall x \in X \ , \ p(x)<+\infty$。定义集合划分:
\[M_k=\{x \in X|p(x) \leq k, \ k \in \mathbb{N}\}=\bigcup_{\alpha \in \Lambda}\{x \in X| \, ||T_{\alpha}x|| \leq k, \ k \in \mathbb{N}\}\]
则显然有:
\[X=\bigcap_{k=1}^{\infty}M_k\]
由 Baire 纲定理,$X$是 Banach 空间,为第二纲集,从而存在$k_0 \in \mathbb{N}$使得$M_{k_0}$不是疏集[定义\ref{definition:sj}],即:
\[\exists \, B_r(x_0)=\{x \in X| \, ||x-x_0||<r\} \quad \text{s.t.} \quad B_r(x_0) \subset \overline{M}_k\]
由$T_{\alpha}$的连续性可知,$[0,k]$的原像$\{x \in X| \, ||T_{\alpha}x|| \leq k\}$是闭集,从而$M_k$也是闭集:
\[B_r(x_0) \subset \overline{M}_k=M_k \quad \Leftrightarrow \quad \forall x \in B_r(x_0) \ , \ p(x)=\mathop \text{sup}\limits_{\alpha \in \Lambda}||T_{\alpha}x|| \leq k\]
第一步即证,利用线性可以将该结论推广至全空间:
\[\forall y \in X \ , \ \alpha \in \Lambda \ , \ \exists \, M=\frac{2k}{r} \quad \text{s.t.} \quad ||T_{\alpha}y||=\left\|T_{\alpha}\left(x_0+\frac{r}{||y||}y\right)-T_{\alpha}x_0\right\| \cdot \frac{||y||}{r} \leq \frac{||y||}{r}(k+k)=M||y||\]
\end{proof}
一致有界定理可以扩展到出发空间是第二纲集,其逆否定理叫共鸣定理。
\begin{theorem}[共鸣定理]
    $\{T_{\alpha}\}_{\alpha \in \Lambda} \subset \mathscr{B}(X,Y) \ , \ D(T_{\alpha})=X$ , $X$是 Banach 空间,若$||T_{\alpha}||=+\infty$则,$\exists \, x \in X \quad \text{s.t.} \quad ||T_{\alpha}x||=+\infty$。
\end{theorem} 
共鸣定理指出了有界线性算子的点点有界等价于一致有界。相比于一般的常见赋范线性空间,算子空间的性质不是那么任意直观看出,如果能利用出发空间和到达空间的性质判断算子空间的性质自然是最好的。
\begin{theorem}[Banach-Steinhaus 定理]\label{BS}
$\{T_n\}_{n=1}^{+\infty} \subset \mathscr{B}(X,Y) \ , \ D(T_{\alpha})=X$ ,若$Y$是 Banach 空间,若:
\begin{itemize}
    \item 1. $\{T_n\}$一致有界;
    \item 2. $\{T_n\}$在$X$的一个稠密子集$G$上逐点收敛($\forall g \in G \ , \ \{T_ng\}$收敛)。
\end{itemize}
则$\exists \, T \in \mathscr{B}(X,Y)$满足$\{T_n\}$逐点收敛于$T$,且:
\[||T|| \leq \varliminf\limits_{n\to\infty}||T_n||\]
\end{theorem} 
\begin{proof}
由于$G$在$X$上稠密,即对$\forall x \in X \ , \ \varepsilon>0\ , \ \exists \, g \in G$满足$||x-g||<\varepsilon$。因此$\exists \, N \in \mathbb{N}$当$\forall m>n>N$时:
\[||T_mx-T_nx|| \leq ||T_mx-T_mg||+||T_mg-T_ng||+||T_ng-T_nx|| \leq \left(||T_m||+||T_n||\right)||x-g||+\varepsilon \leq (2M+1)\varepsilon\]
即$\{T_nx\}$是$Y$空间中的柯西列,由$Y$空间完备:
\[\exists \, y \in Y \quad \text{s.t.} \quad \lim_{n \to \infty}T_nx=y:=Tx\]
下面只需验证$T$为有界线性算子:
\begin{itemize}
\item 1. 线性:
\[T(\alpha x+\beta y)=\lim_{n \to +\infty}T_n(\alpha x+\beta y)=\alpha\lim_{n \to +\infty}T_nx+\beta\lim_{n \to +\infty}T_ny=\alpha Tx+\beta Ty\]
\item 2. 有界性:
\[||Tx||=\left\|\lim_{n \to +\infty}T_nx\right\|=\varliminf\limits_{n\to\infty}||T_nx|| \leq \varliminf\limits_{n\to\infty}||T_n|| \cdot ||x|| \quad \Rightarrow \quad ||T|| \leq \varliminf\limits_{n\to\infty}||T_n||\]
其中第一处等号使用了范数的连续性:由 $T_nx\to Tx$ 可得 $||T_nx||\to||Tx||$,因此:
\[||Tx||=\lim_{n\to\infty}||T_nx||=\varliminf_{n\to\infty}||T_nx||\]
若极限存在则上下极限存在且等于极限$\varliminf=\lim=\varlimsup$。
\end{itemize}
\end{proof}

Banach-Steinhaus 定理告诉我们利用完备到达空间的收敛怎么样推算算子空间的收敛,这无疑是一种极大的方便。同时,该定理也说明了$\mathscr{B}(X,Y)$在强收敛意义下完备。

\begin{example}
\quad 机械求积分公式:令$X=C[a,b] \ , \ \forall x \in X$,存在$[a,b]$的分割$a \leq t_1<t_2<\cdots<t_k\leq b$满足:
\[\int_a^bx(t)\dd t \approx \sum_{k=1}^nA_kx(t_k)\]
则下式(其中$A_k^{(n)}$只与参数$n$有关而与$x(t)$无关):
\[\int_a^bx(t)\dd t=\lim_{n \to \infty}\sum_{k=1}^nA_k^{(n)}x(t_k^{(n)})\]
成立的充要条件为:
\begin{itemize}
\item 1. 
\[\forall n \in \mathbb{N} \ , \ \exists \, M>0 \quad \text{s.t.} \quad \sum_{k=1}^n\left|A_k^{(n)}\right| \leq M\]
\item 2. 上式对$C[a,b]$的一个稠密子集$G$成立。
\end{itemize}
\end{example}
\begin{proof}
构造一列$X=C[a,b]$上的泛函$\{f_n\}$:
\[f_n:X \to \mathbb{R} \ , \ f_n(x)=\sum_{k=1}^nA_k^{(n)}x(t_k^{(n)})\]
为了保证$\{f_n\}$强收敛于原积分,必须保证其线性,至此我们只需要验证:
\[||f_n||=\sum_{k=1}^n\left|A_k^{(n)}\right|\]
即可利用 Banach-Steinhaus 定理证明原命题。一方面:
\[\forall x \in X \ , \ ||f_n(x)||=\left|\sum_{k=1}^nA_k^{(n)}x(t_k^{(n)})\right| \leq ||x||\sum_{k=1}^n\left|A_k^{(n)}\right| \quad \Rightarrow \quad ||f_n|| \leq \sum_{k=1}^n\left|A_k^{(n)}\right|\]
另一方面:
\[\exists \, x_0 \in X \ , \ s_0(t_k^{(n)})=\text{sgn}(A_k^{(n)}) \quad \Rightarrow \quad ||x_0||=1 \ , \ ||f_n(x_0)||=\left|\sum_{k=1}^nA_k^{(n)}\text{sgn}(A_k^{(n)})\right|=\sum_{k=1}^n\left|A_k^{(n)}\right| \leq ||f_n||\]
综上所述:
\[||f_n||=\sum_{k=1}^n\left|A_k^{(n)}\right|\]
\end{proof}

\begin{example}
\quad Fourier 级数的发散性:
\[X=C_{2\pi}=\{x \in C[0,2\pi]|x(0)=x(2\pi)\}=\{x \in C(\mathbb{R})|x(t+2\pi)=x(t) \ , \ \forall t \in \mathbb{R}\}\]
$\forall x \in X$,考察 Fourier 级数:
\[\lim_{n \to \infty}F_n[x(t)]=x(t) \sim \frac{a_0}{2}+\sum_{i=1}^{+\infty}(a_k \cos kt+b_k \sin kt) \ , \ a_k=\frac{1}{\pi}\int_{-\pi}^{\pi}x(s) \cos ks \dd s \ , \ b_k=\frac{1}{\pi}\int_{-\pi}^{\pi}x(s) \sin ks \dd s\]
\end{example}
这里回顾一下 Dirichlet 条件,他只是判断 Fourier 级数收敛的一个充分条件。
\begin{theorem}[Dirichlet 条件]
    若$x \in C_{2\pi}$在$[0,2\pi]$上满足:1、之多有限个第一类间断点;2、至多有限个极值点,则
    \[\lim_{n \to \infty}F_n[x(t)]=\left\{
        \begin{array}{rl}
            x(t) & ,x\text{在}t\text{处连续} \\
            \frac{1}{2}(x(t^-)+x(t^+)) & ,t\text{是}x\text{的第一类间断点}
        \end{array}
    \right.\]
\end{theorem} 
\begin{proof}
下面我们将利用共鸣定理说明不加条件时,Fourier 级数可能不收敛。定义
\[f_n:X \to \mathbb{R} \ , \ x(t) \mapsto f_n[x(0)]=\int_{-\pi}^{\pi}x(s)k_n(s,0) \dd s \quad \text{其中 }k_n(s,t)=\frac{1}{\pi}\left(\frac{1}{2}+\sum_{k=1}^n\cos k(s-t)\right)\]
可以证明:
\[||f_n||=\int_{-\pi}^{\pi}|k_n(s,0)| \dd s \qquad\]
且$||f_n|| \to 0$。然后由共鸣定理可知$\exists \, x \in X$使得$|f_n[x(0)]| \to +\infty$。
\end{proof}

\section{开映射定理与闭图像定理}\label{kb}
\begin{definition}[算子复合(乘法)]
设$X,X_1,X_2$是赋范线性空间,$T_1:X \to X_1$,$T_2:X_1 \to X_2$,$R(T_1) \subset D(T_2)$,定义算子复合$T=T_2 \circ T_1=T_2T_1$,若其满足$\forall x \in X \ , \ Tx=(T_2 \circ T_1)x=T_2(T_1x)$。
\end{definition}
\begin{remark} \quad 
若$T_1 \in \mathscr{B}(X,X_1) \ , \ T_2 \in \mathscr{B}(X_1,X_2)$,则$T \in \mathscr{B}(X,X_2)$。同时,连续算子的复合得到的还是连续算子:
\[||Tx||=||T_2(T_1x)|| \leq ||T_2||\cdot||T_1||\cdot||x|| \quad \Rightarrow \quad ||T|| \leq ||T_1||\cdot||T_2||\]
\end{remark}

\begin{definition}[逆算子]
设$X,Y$是赋范线性空间,$T_1:X \to Y$,若$\exists \, T_1:Y \to X \quad \text{s.t.} \quad T_1 \circ T=I_X \ , \ T \circ T_1=I_Y$,则称$T$可逆,$T_1$是$T$的逆算子,记$T_1=T^{-1}$。
\end{definition}
\begin{remark} \quad 
若$T$是线性的则$T^{-1}$也是线性的;算子$T$可逆当且仅当$T$是一一映射;若$T^{-1}$存在则唯一。
\end{remark}

那么我们会思考如果$T$是有界线性算子,$T^{-1}$是否也是有界,或者说$T^{-1}$是否连续呢?这件事开映射定理及逆算子定理会告诉我们。但是在正式介绍这两个定理之前我们可以从一些比较具体的情况开始逐步推广。
\begin{theorem}
设$T \in \mathscr{B}(X)$是满射,且$\forall x \in X \ , \ \exists \, m>0 \quad \text{s.t.} \quad ||Tx|| \geq m||x||$,则$T$可逆且$T^{-1} \in \mathscr{B}(x)$。
\end{theorem} 
\begin{proof}
$T$是单射:
\[\forall x_1,x_2 \in X \ , \ ||Tx_1-Tx_2|| \geq m||x_1-x_2|| \ , \ ||Tx_1-Tx_2||=0 \quad \Leftrightarrow \quad ||x_1-x_2||=0\]
$T^{-1}$有界:
\[\forall y \in X \ , \ ||T^{-1}y||=||x|| \leq m||Tx||=m||y||\]
\end{proof}

\begin{theorem}
设$X$是 Banach 空间,$T \in \mathscr{B}(X) \ , \ ||T||<1$,则算子$I-T$可逆,且:
\[||(I-T)^{-1}|| \leq \frac{1}{1-||T||}\]
\end{theorem} 
\begin{proof}规定$T^0=I$,则算子$T$的部分级数和$S_n$如下定义:
\[S_n=\sum_{k=0}^{n-1}T^k\]
因此:
\[\forall m>n \ , \ ||S_m-S_n||=\left\|\sum_{k=n}^{m-1}T^k\right\| \leq \sum_{k=n}^{m-1}||T^k|| \leq \sum_{k=n}^{m-1}||T||^k=||T||^n \cdot \frac{1-||T||^{m-n}}{1-||T||}<\frac{||T||^n}{1-||T||} \to 0\]
可知$\{S_n\}$是柯西列,由于$X$是 Banach 空间,故$\{S_n\}$收敛,记其收敛极限为:
\[S=\lim_{n\to\infty}S_n=\lim_{n\to\infty}\sum_{k=0}^nT^k=\sum_{k=0}^{+\infty}T^k\]
为了证明$S$是$I-T$的逆,我们考虑:
\[S_n \circ (I-T)=(I-T) \circ S_n=I-T^n \quad \Rightarrow \quad (I-T) \circ S=S \circ (I-T)=\lim_{n \to +\infty}S_n \circ (I-T)=\lim_{n \to +\infty}(I-T^n)\]
注意到:
\[\lim_{n \to +\infty}||T^n|| \leq \lim_{n \to +\infty}||T||^n=0 \quad \Rightarrow \quad \lim_{n \to +\infty}T^n=\mathbf{0} \ (\text{零算子}) \quad \Rightarrow \quad S=(I-T)^{-1}\]
\[||(I-T)^{-1}||=||S||=\left\|\lim_{n \to +\infty}S_n\right\|=\left\|\lim_{n \to +\infty}\sum_{k=0}^{n-1}T^k\right\| \leq \lim_{n \to +\infty}\sum_{k=0}^{n-1}||T||^k=\lim_{n \to +\infty}\frac{1-||T||^n}{1-||T||}=\frac{1}{1-||T||}\]
\end{proof}
上述定理展示了恒同算子加上一个小的扰动($||T||<1$)时依然可逆,那这件事是否对任意可逆算子成立呢?
\begin{theorem}
设$X$是 Banach 空间,$T, \ T^{-1} \in \mathscr{B}(X)$可逆,若:
\[\exists \, \Delta T \in \mathscr{B}(X) \ , \ ||\Delta T||<\frac{1}{||T^{-1}||} \ , \ S:=T+\Delta T \in \mathscr{B}(X) \ \text{可逆且} \ S^{-1}=\sum_{k=0}^{+\infty}(-1)^k(T^{-1}\Delta T)^kT^{-1}\]
\end{theorem} 
\begin{proof}$S=T+\Delta T=T \circ (I+T^{-1}\Delta T) \ , \ ||T^{-1}\Delta T|| \leq ||T^{-1}|| \cdot ||\Delta T||<1$,由上一定理知$I+T^{-1}\Delta T$可逆,且:
\[(I+T^{-1}\Delta T)^{-1}=\sum_{k=0}^{+\infty}(-T^{-1}\Delta T)^k \quad \Rightarrow \quad S^{-1}=\sum_{k=0}^{+\infty}(-1)^k(T^{-1}\Delta T)^kT^{-1}\]
\end{proof}
\begin{remark} \quad 
$T,T^{-1} \in \mathscr{B}(X)$的这类算子称为正则算子,Banach 空间到 Banach 空间的正则算子构成的集合为开集。
\end{remark}
\begin{definition}[开映射]
定义映射$T: \ X \to Y$,对任意开集$U \subset X$,其像$TU \subset Y$也是开集,则称$T$为开映射。
\end{definition}
\begin{theorem}[开映射定理]\label{theorem:kys}
设$X,X_1$是 Banach 空间,$T \in \mathscr{B}(X,X_1) \ , \ TX=X_1$,则$T$是开映射。
\end{theorem} 
\begin{proof}
由线性可知只需要对零点证明该定理成立:对$\forall y=Tx \ , \ \exists \, \varepsilon>0$和开球$B_{\varepsilon}^X(x) \subset U$,在$TU$中$\exists \, \delta>0$使得开球$B_{\delta}^Y(y)$包含像$TB_{\varepsilon}^X(x)=Tx+\varepsilon TB_1^X(0) \subset B_{\delta}^Y(y)=y+\delta B_1^Y(0)$。分两步证明$T(0)$是$TB(0,1)$的内点:
\begin{proposition}
$T\overline{B^X_1(0)}$在某个$B^Y_{\delta}(0)$中稠密。
\end{proposition}
\begin{proof}
由于$T$是满射,即$TX=X_1$,且$X_1$是全空间:
\[X=\bigcup_{k=1}^{\infty}\overline{B^X_k(0)} \quad \Rightarrow \quad X_1 \subseteq \bigcup_{k=1}^{\infty}T\overline{B^X_k(0)} \quad \Rightarrow \quad X_1=\bigcup_{k=1}^{\infty}T\overline{B^X_k(0)}\]
由于$X_1$是 Banach 空间,为第二纲集[定理\ref{the:Baire}],故存在某个开球$B_{r_0}^Y(y_0) \subset T\overline{B_{k_0}^X(0)}$。对$\forall y \in B_{r_0}^Y(0)$,可以构造$y_1=y_0+y, \ y_2=y_0-y \in B_{r_0}^Y(y_0)$,显然$y=(y_1-y_2)/2$。同时在原像中,对$\forall \varepsilon>0 \ , \ \exists \, x_1,x_2 \in \overline{B^X_k(0)}$使得$||Tx_1-y_1||<\varepsilon \ , \ ||Tx_2-y_2||<\varepsilon$。所以$\exists \, x=(x_1-x_2)/2 \in \overline{B^X_k(0)}$满足:
\[||Tx-y||=\left\|T\left(\frac{x_1-x_2}{2}\right)-\frac{y_1-y_2}{2}\right\|<\varepsilon\]
即$T\overline{B_{k_0}^X(0)}$在$B_{r_0}^Y(0)$中稠密,从而由线性对$\forall \varepsilon>0 \ , \ T\overline{B^X_{\varepsilon}(0)}$在$B_{\delta\varepsilon}^Y(0)$中稠密,其中$\delta=r_0/k_0$,可取$\varepsilon=1$。
\end{proof}
\begin{proposition}\label{proposition:Step2}
$T\overline{B_1^X(0)} \supset B_{\delta/2}^Y(0)$
\end{proposition}
\begin{proof}
由上一小节可知:$T\overline{B^X_{1/2}(0)}$在$B^Y_{\delta/2}(0)$中稠密,则:
\[\forall y \in B^Y_{\delta/2}(0) \ , \ \exists \, x_1 \in B^X_{1/2}(0) \quad \text{s.t.} \quad ||y-Tx_1||<\frac{\delta}{2^2} \quad \Rightarrow \quad \exists \, y_1=y-Tx_1 \in B^Y_{\delta/2^2}(0)\]
归纳可得:
\[\exists \, \{x_n\} \subset B_1^X(0) \ , \ ||x_n|| \leq \frac{1}{x^n} \ , \ \left\|y-T\left(\sum_{k=1}^nx_k\right)\right\|<\frac{\delta}{2^{n+1}}\]
定义部分和$S_n$:
\[S_n=\sum_{k=1}^nx_k \ , \ m>n>N \ , \ ||S_m-S_n||=\left\|\sum_{k=n+1}^mx_k\right\| \leq \sum_{k=n+1}^m||x_k||<\frac{1}{2^{n+1}} \to 0\]
可知$\{S_n\}$是 Cauchy 列,在完备空间$X$中收敛,记为$S$,则:
\[||y-TS||=\left\|y-T\lim_{n \to +\infty}\sum_{k=1}^nx_k\right\| \leq \lim_{n \to +\infty}\left\|y-T\sum_{k=1}^nx_k\right\|=0\]

综上所述,$TB_1^{X}(0) \supset T\overline{B_{1/2}^X(0)} \supset B^Y_{\delta/4}(0)$,即对$x=0 \in B_1^{X}(0)$我们能找到$y=Tx=0 \in TB_1^{X}(0)$是$TB_1^{X}(0)$内点。结合之前的分析,定理得证。
\end{proof}
\begin{figure}[H]
    \center
    \includegraphics[scale=0.3]{./fig/4.4-1.png}
\end{figure}
\end{proof}

完成上面这么一个大工程,我们现在终于能回答一开提及的关于逆算子性质的问题了。
\begin{theorem}[逆算子定理]
设$X,X_1$是 Banach 空间,$T \in \mathscr{B}(X,X_1)$是一一映射(保证开映射),$TX=X_1$,则$T^{-1} \in \mathscr{B}(X_1,X)$。
\end{theorem} 
\begin{proof}
由开映射定理[定理\ref{theorem:kys}]中的命题\ref{proposition:Step2}可知:
\[T\overline{B_1^{X}(0)} \supset B_{\delta/2}^Y(0) \quad \Rightarrow \quad \forall z \in X_1 \ , \  \frac{z\delta}{4||z||} \in B_{\delta/2}^Y(0) \ , \ \exists ! \, T^{-1}\left(\frac{z\delta}{4||z||}\right) \in \overline{B_1^{X}(0)}\]
\[\Rightarrow \quad \left\|T^{-1}\left(\frac{z\delta}{4||z||}\right)\right\| \leq ||T^{-1}|| \cdot \left\|\frac{z\delta}{4||z||}\right\| \leq 1 \quad \Rightarrow \quad \frac{||T^{-1}z||}{||z||} \leq \frac{4}{\delta} \quad \Rightarrow \quad ||T^{-1}|| \leq \frac{4}{\delta}\]
\end{proof}

开映射定理和逆算子定理最重要的一点是得到了$T^{-1}$的连续的条件,该性质可以保证方程$Tx=y$解的稳定性。由逆算子定理我们还有个推论。
\begin{proposition}
    若线性空间$X$上有两个范数$||\cdot||_1,||\cdot||_2$,使得$(X,||\cdot||_1),(X,||\cdot||_2)$都是 Banach 空间,且$||\cdot||_1$强于$||\cdot||_2$,则$||\cdot||_1$与$||\cdot||_2$等价。
\end{proposition}
\begin{proof}
定义映射$I:(X,||\cdot||_1) \to (X,||\cdot||_2) \ , \ x \mapsto x$。
显然$I$是可逆算子,且由$||Ix||_2=||x||_2 \leq \alpha ||x||_1$可知$I$有界。
由逆算子定理,$I^{-1}:(X,||\cdot||_2) \to (X,||\cdot||_1) \ , \ x \mapsto x$,有$||Ix||_1=||x||_1 \leq \beta ||x||_2$,原命题得证。
\end{proof}

\begin{definition}[图像,闭图像]
设$T:X \to Y$是映射,称$G(T)=\{(x,Tx)|x \in X \times Y\}:=\{(x,Tx)|x \in X ,y \in Y\}$为$T$的图像。若$G(T)$是$X \times Y$中的闭集,则称$T$为闭算子,即:
\[\forall \{(x_n,y_n)\} \subset G(T) \ , \ x_n \to x \in X \ , \ y_n \to y \in Y \ , \ (x,y) \in G(T)\]
\end{definition}
\begin{theorem}[闭图像定理]
设$X,Y$是 Banach 空间,若$T:X \to Y$是闭线性算子且$D(T)$是闭线性子空间,则$T$是有界的。
\end{theorem} 
\begin{remark}
\quad 闭图像定理告诉我们如果$T$为闭算子,且其定义域$D(T)$是闭的,那么$T$就是连续的。因为证明一个算子为闭算子比证明它为线性有界算子更容易,所以闭图像定理给出了一个证明连续(有界)线性算子的方法。
\end{remark}
\begin{proof}
由于$T$是闭算子,故$G(T)=\{(x,Tx)|x \in D(T)\}$是$X \times Y$的闭线性子空间,从而$G(T)$是 Banach 空间[定理\ref{theorem:zkj}],考虑算子$p:G(T) \to D(T) \ , \ (x,Tx) \mapsto x$,易知$p$良定义且为一一映射,且满足线性,且$p$有界:
\[||p(x,Tx)||=||x||_X \leq ||(x,Tx)||_{X \times Y} \leq ||x||_X + ||Tx||_Y\]
由逆算子定理$\exists \, p^{-1}:D(T) \to G(T) \ , \ x \mapsto (x,Tx)$有界,即:
\[||p^{-1}x||=||(x,Tx)||_{X \times Y}=||x||_X + ||Tx||_Y \leq \alpha ||x||_X \quad \Rightarrow \quad ||Tx||_Y \leq \beta ||x||_X\]
\end{proof}

\begin{proposition}
考虑$X,Y$是 Banach 空间,$T \in \mathscr{B}(X,Y)$,如果$D(T)$是闭的,则$G(T)$是闭的:
\[\forall\{(x_n,y_n)\} \subset G(T) \ , \ x_n \to x \in D(T) \ , \ y_n=Tx_n \to Tx=y \in T(D(T)) \quad \Rightarrow \quad (x,y) \in G(T)\]
若$D(T)$不是闭的,可以考虑如下延拓:
\[x_n \to x \in \overline{D(T)} \ , \ \overline{T}:\overline{D(T)} \to Y \ , \ x \mapsto T\overline{x}=\lim_{n \to \infty}Tx_n\]
\end{proposition}
\begin{example} \quad 一般来说闭算子不一定有界($X,Y$是 Banach 空间),因为$D(T)$不一定是闭子空间。如:
\[X=C[0,1] \ , \ T=\dv{t} \ , \ D(T)=C^1[0,1]\]
这里,$D(T)$不是闭的,$T$是闭算子,设$x_n \to x \ , \ y_n=Tx_n=x_n' \to y$,则:
\[x_n(t)-x_n(0)=\int_0^tx_n'(s) \dd s \quad \Rightarrow \quad x(t)-x(0)=\int_0^ty(s) \dd s\]
即,$D(T)$中任意收敛于自身的点列在$T[D(T)]$中也收敛于自身,但是$T$不为有界算子,因为存在反例如:
\[\lim_{n \to \infty}T(\sin nx)=+\infty\]
\end{example}

\section{Hahn-Banach 定理}
在回答完算子的逆的存在性和其连续性等之后,我们可能会考虑算子空间的数量,更具体点,如果我们给定一个赋范线性空间$X$,其上的有界线性泛函全体$\mathscr{B}(X,\mathbb{K}) \ (\mathbb{K}=\mathbb{R}/\mathbb{C} \ , \ X\text{是}\mathbb{K}\text{上的线性空间})$的数量有多少,结构又如何?当$X$为有限维时,那么但很简单,$\mathscr{B}(X,\mathbb{K})$与$X$同构,那么在无限维的时候我们又该如何回答这个问题呢?这将是 Hahn-Banach 定理要回答的问题。

Hahn-Banach 定理告诉我们,在线性赋范空间中,我们可以定义“足够多”的不同的线性泛函。
它的思路是:如果我们能把子空间上的线性泛函$M$延拓到全空间$X$上不就可以辣?因为我们很容易找个子空间构建个线性泛函,如果这些子空间的线性泛函能够全都延拓到全空间,那不就说明全空间上可以有“足够多”的线性泛函辣!
\begin{theorem}[Hahn-Banach 定理]\label{HB}
设$M$是$X$的线性子空间,$f \in \mathscr{B}(M,\mathbb{K})=M^*$,则$f$可线性保范延拓至整个空间,即:
\[\exists \, F \in \mathscr{B}(M,\mathbb{K})=X^* \quad \text{s.t.} \quad \forall x \in M \ , \ F|_M=f \ , \ ||F||_X=||f||_M\]
\end{theorem} 
该定理的证明分为两步:1、证明$\mathbb{K}=\mathbb{R}$情形;2、证明$\mathbb{K}=\mathbb{C}$情形。但再此之前,我们需要一个强大的武器 Zorn 引理。有限维任意证明,我们自然而然会想到从有限维借助数学归纳法推广到无穷维,但是如果无穷维是不可数的该怎么办呢,Zorn 引理告诉我们,使用超限归纳法照样推!
\begin{definition}[偏序(半序)关系]
设集合$S$中的一个二元关系$\prec \, :S \times S \to S$,若其满足:
\begin{itemize}
    \item 1. 自反性$x \prec x$;
    \item 2. 传递性$\forall x,y,z \in S \ , \ x \prec y \ , \ y \prec z \quad \Rightarrow \quad x \prec z$;
    \item 3. 反对称性$x \prec y \ , \ y \prec x \quad \Rightarrow \quad x=y$;
\end{itemize}
则称$\prec$为一个偏序(半序)关系,$(S,\prec)$称为一个偏序集。若$\mathscr{A} \subset S$满足$\forall x,y \in \mathscr{A} \ , \ x \prec y$或$y \prec x$,则称$\mathscr{A}$是$S$的全序子集。
\end{definition}
\begin{example}
\quad $(\mathbb{R},\leq)$是全序集,$(2^X,\subseteq)$是偏序集。
\end{example}
\begin{definition}[上界]
对$\mathscr{A} \subset S$,若$\exists \, \beta \in S \quad \text{s.t.} \quad \forall \alpha \in S \ , \ \alpha \prec \beta$则称$\beta$是$\mathscr{A}$的上界。
\end{definition}
\begin{definition}[极大元]
$\exists \, \beta \in S \quad \text{s.t.} \quad \forall \alpha \in S \ , \ \beta \prec \alpha$都有$\alpha=\beta$则称$\beta$是$S$的极大元。
\end{definition}
\begin{theorem}[Zorn 引理]\label{Zorn}
若$S$中的任意全序子集都有上界,则$S$中必然存在(至少一个)极大元。
\end{theorem} 
Zorn 引理与选择公理等价。方便证明起见使用次可加函数重新表述实空间的 Hahn-Banach 定理。
\begin{definition}[次可加函数]
若$p:X \to \mathbb{R}$满足:
\begin{itemize}
\item 1. $\forall x,y \in X \ , \ p(x+y) \leq p(x) + p(y)$;
\item 2. $\forall x \in X \ , \ t \in \mathbb{R}^+ \ , \ p(tx)=tp(x)$;
\end{itemize}
则称$p$为次可加函数。
\end{definition}
\begin{example}
\quad 若$X=\mathbb{R} \ , \ a,b \in \mathbb{R} \ , \ a>b$,可定义次可加函数$p(x)$:
\[p(x)=\left\{\begin{array}{ll}
    ax & ,x \geq 0 \\
    bx & ,x<0    
\end{array}\right.\]
按照次可加函数的定义可以知道次可加函数都是凸函数,且$p(0)=0$。
\end{example}
\begin{theorem}[Hahn-Banach 定理,$\mathbb{K}=\mathbb{R}$]\label{theorem:ckjhb}
    设$M$是$X$的实线性子空间,$p:X \to \mathbb{R}$是次可加函数,$f \in \mathscr{B}(M,\mathbb{R})=M^*$,且对$\forall x \in M \ , \ f(x) \leq p(x)$。则存在线性延拓$F \in \mathscr{B}(M,\mathbb{R})=X^*$满足$F|_M=f$,且对$\forall x \in X \ , \ F(x) \leq p(x)$。
\end{theorem} 
\begin{remark}
\quad 若 Hahn-Banach 定理在实数域上成立,自然而然地,复数情形下我们会将其考虑分解$f \in \mathscr{B}(M,\mathbb{C}) \ , \ f(x)=\phi(x)+i\psi(x)$和$f(ix)=\phi(ix)+i\psi(ix)$,联立可得$\phi(x)=\psi(ix) \ , \ \psi(x)=-\phi(ix)$,即$f(x)=\phi(x)-i\phi(ix)$,由实空间的情形可知,存在一个$\phi$的实线性延拓$\Phi$使得满足条件的$f$的线性延拓$F(X)=\Phi(x)-i\Phi(ix)$存在。
\end{remark}
\begin{proof}分三步证明该定理:
\begin{proposition}
定理\ref{theorem:ckjhb}可以推出定理\ref{HB}。
\end{proposition}
\begin{proof}
设$||f||_M=\alpha>0$,定义$p:X \to \mathbb{R} \ , \ x \mapsto \alpha||x||$,容易验证$p(x)$为次可加函数且$|f(x)| \leq ||f||_M||x||=p(x)$。
根据定理\ref{theorem:ckjhb},$\exists \, F \in X^*$满足$F|_M=f \ , \ \forall x \in X \ , \ F(x) \leq p(x)=\alpha ||x||$,即$||F||_X \leq \alpha$。
又因为:
\[||F||_X=\sup_{||x||=1 \ , \ x \in X}|Fx| \geq \sup_{||x||=1 \ , \ x \in M}|Fx|=||f||_M=\alpha \quad \Rightarrow \quad ||F||_X=||f||_M\]
\end{proof}
\begin{proposition}\label{proposition:1d}
对$\forall x_1 \in X/M$定义$M_1=\{x+tx_1 \in X|x \in M \ , \ t \in \mathbb{R}\}$,则存在线性延拓:
\[\exists \, F_1 \in \mathscr{B}(M_1,\mathbb{R}) \quad \text{s.t.} \quad F_1|_M=f \ , \ \forall x \in M_1 \ , \ F_1(x) \leq p(x)\]
\end{proposition}
\begin{proof}对$\forall x_1 \in X/M \ , \ y_1,y_2 \in M$有$f(y_1)+f(y_2)=f(y_1+y_2) \leq p(y_1+y_2) \leq p(y_1-x_1) + p(y_2+x_1)$,从而有$f(y_1)-p(y_1-x_1) \leq p(y_2+x_1)-f(y_2)$。依题意有$F_1(x+tx_1)=F_1(x)+tF_1(x_1) \leq p(x+tx_1)$,故$F_1(x_1)$满足:
\[\left\{
\begin{array}{ll}
    F_1(x_1) \leq p(x/t+x_1)-f(x/t) & ,t \geq 0 \\
    F_1(x_1) \geq f(x/t)-p(x/t-x_1) & ,t < 0
\end{array}
\right.\]
故只需取:
\[F_1(x_1) \in \left[\sup_{y_1 \in M}f(y_1)-p(y_1-x_1),\inf_{y_2 \in M}p(y_2+x_1)-f(y_2)\right]\]
即可找到满足要求的$F_1$。
\end{proof}
\begin{proposition}
使用超限归纳法将命题\ref{proposition:1d}推广到任意维度。
\end{proposition}
\begin{proof}
考虑集合$S=\{F|F\text{是}f\text{满足定理要求的线性延拓}\}$,定义偏序$\prec$,若$F_1,F_2 \in S \ , \ F_1 \prec F_2$则$D(F_1) \subseteq D(F_2) \ , \ F_2|_{D(F_1)}=F_1$,容易证明$(S,\prec)$是偏序集。下面验证$(S,\prec)$中任意全序子集$\mathscr{F}$都有上界。

定义$\Phi:D \to \mathbb{R}$满足$F \in \mathscr{F} \ , \ x \in D(F)$有$\Phi(x)=F(x)$,记:
\[D=\bigcup_{F \in \mathscr{F}}D(F)\]
则$\forall F \in \mathscr{F}$有$D(F) \subset D \ , \ \Phi|_{D(F)}=F$,即$F \prec \Phi$,$\Phi$为$\mathscr{F}$的上界。进而由 Zorn 引理可知$S$中有极大元$\Phi$,其定义域为$D(\Phi)=X$。
\end{proof}
\end{proof}
\begin{remark}
\quad 需要注意的是 Hahn-Banach 定理只证明了延拓的存在性,但不保证唯一。
\end{remark}
Hahn-Banach 定理有许多有用的推论。
\begin{proposition}
赋范线性空间$X$上$\forall x_0 \in X \ , \ x_0 \neq 0 \ , \ \exists \, f \in X^* \quad \text{s.t.} \quad ||f||=1 \ , \ f(x_0)=||x_0||$。
\end{proposition}
\begin{proof}
令$M=\{\lambda x_0|\lambda \in \mathbb{K}\} \subset X \ , \ f_0:M \to \mathbb{K} \ , \ \lambda x_0 \mapsto |\lambda|\cdot||x_0||$,显然当$\lambda=1$时$f(x_0)=||x_0||$,则:
\[||f_0||_M=\sup_{\lambda \in \mathbb{K}}\frac{|\lambda|\cdot||x_0||}{||\lambda x_0||}=1\]
再由 Hahn-Banach 定理,$\exists \, f \in X^* \ , \ \eval{f}_{M}=f_0 \ , \ ||f||_X=||f_0||_M=1$,满足题设。
\end{proof}
\begin{proposition}
赋范线性空间$X$上$\forall x_1,x_2 \in X \ , \ x_1 \neq x_2 \ , \ \exists \, f \in X^* \quad \text{s.t.} \quad ||f||=1 \ , \ f(x_1) \neq f(x_2)$。
\end{proposition}
\begin{proof}
令$x_0=x_1-x_2 \neq 0$,则由推论1,$f(x_0)=f(x_1-f_2)=f(x_1)-f(x_2) \neq 0$。
\end{proof}
\begin{remark}
\quad 我们可以用一个有界线性泛函去区分两个点的不同性质。
\end{remark}

\begin{proposition}
$X$是赋范线性空间$M \subset X$是线性子空间,$x_0 \in X$满足:
\[d(x_0,M)=\inf_{x \in M}||x-x_0||>0\]
则$\exists \, f \in X^* \quad \text{s.t.} \quad x \in M \ , \ ||f||=1 \ , \ f(x_0)=d \ , \ f|_M=0$。
\end{proposition}
\begin{proof}
令$M=\{x+tx_0|t \in \mathbb{K} \ , \ x \in M\}$,定义$f:M \to \mathbb{K} \ , \ x+tx_0 \mapsto td$,则一方面:
\[|f(x+tx_0)|=|t|d \leq |t| \cdot \left\|\frac{x}{t}+x_0\right\|=||x+tx_0|| \quad \Rightarrow \quad ||f|| \leq 1\]
另一方面$\exists \{x_n\} \subset M \ , \ ||x_n -x_0|| \to d \ (n \to \infty)$:
\[|f(x_n-x_0)|=|f(x_n)-f(x_0)|=|f(x_0)|=d \leq ||f|| \cdot ||x_n-x_0|| \to ||f||d \quad \Rightarrow \quad ||f|| \geq 1\]
综上所述,$||f||=1$,再由 Hahn-Banach 定理即证。
\end{proof}

\section{对偶空间和共轭算子}
\begin{definition}[对偶空间]
$X^*=\mathscr{B}(X,\mathbb{K})$称为$X$的对偶空间。
\end{definition}
\begin{remark}
\quad 对偶空间一定是 Banach 空间。
\end{remark}
\begin{definition}[共轭算子]
$X,Y$是赋范线性空间,$T \in \mathscr{B}(X,Y)$,定义$T^*:Y^* \to X^*$满足$\forall f \in Y^* \ , \ x \in X \ , \ (T^*f)x=f(Tx)$,则称$T^*$为共轭算子。
\end{definition}
\begin{lemma}
\begin{itemize}
\item 1. $T \in \mathscr{B}(X,Y) \ , \ T^* \in \mathscr{B}(Y^*,X^*)$则$||T^*||=||T||$;
\item 2. $\forall \alpha \in \mathbb{K} \ , \ (\alpha T)^*=\alpha T^*$;
\item 3. $(T_1+T_2)^*=T_1^*+T_2^*$;
\item 4. $(T_1T_2)^*=T_2^*T_1^*$;
\item 5. 若$T$有有界逆$T^{-1}$,则$T^*$也有有界逆且$(T^*)^{-1}=(T^{-1})^*$。
\end{itemize}
或者考虑映射$*:\mathscr{B}(X,Y) \to \mathscr{B}(Y^*,X^*)$,则1:*是有界等距映射,2\&3:线性,4:乘法,5:与逆可交换。
\end{lemma} 
\begin{example}
\quad $X=(\mathbb{R}^n,||\cdot||) \ , \ X^*=X$。
\end{example}
\begin{proof}
分别记$X,X^*$的一组对偶基$\{e_i\}_{i=1}^n \ , \ \{f_i\}_{i=1}^n$,即$f_i(e_j)=\delta_{ij}$,构造$e_i \to f_i$即得$X$与$X^*$之间存在一个等距一一映射,即证。
\end{proof}
\begin{example}
\quad $(L^p[a,b])^*=L^q[a,b]$其中$1/p+1/q=1 \ , \ p \in [1,+\infty)$,反之$L^{\infty}$的对偶空间严格比$L^1$大。
\end{example}
\textbf{Proof}:证明的目标是构造如下等距一一映射
\[\Phi:L^q \to (L^p)^* \ , \ g \mapsto \Phi_g \quad \text{s.t.} \quad \forall f \in L^p \ , \ \Phi_g(f)=\int_a^bf(t)g(t) \dd t\]
线性显然,有界性($h\ddot{o}lder$不等式):
\[||\Phi_g(f)|| \leq ||f||_{L^p} \cdot ||g||_{L^q} \quad \Rightarrow \quad ||\Phi_g|| \leq ||g||_{L^q}\]
下证等距即$||\Phi_g||=||g||_{L^q}$,取
\[f_0=\frac{1}{||g||_{L^q}^{q-1}} \cdot g^{q-1} \cdot \text{sgn}(g) \in L^p \quad (\int_a^b|f_0|^p \dd t=\int_a^b\frac{|g|^{p(q-1)}}{||g||_{L^q}^{p(q-1)}} \dd t=\frac{1}{||g||_{L^q}^q}\int_a^b|g|^q \dd t=1 < +\infty)\]
\[\Phi_g(f_0)=\int_a^bf_0 \cdot g \, \dd t=\frac{1}{||g||_{L^q}^{q-1}}\int_a^bg^{q-1} \cdot \text{sgn}(g) \cdot g \, \dd t=||g||_{L^q} \quad \Rightarrow \quad ||\Phi_g|| \geq \frac{||\Phi_g(f_0)||}{||f_0||_{L^p}}=||g||_{L^q}\]
综上所述即证明等距。下证双射,其中单射$(\Phi_{g_1}=\Phi_{g_2} \ \Leftrightarrow \ g_1=g_2)$是好证的:
\[\forall f_0=(g_1-g_2)^{q-1} \cdot \text{sgn}(g_1-g_2) \in L^p \ , \ \Phi_{g_1}(f_0)-\Phi_{g_2}(f_0)=\int_a^b(g_1-g_2)f_0 \dd t=0\]
则有
\[0=\int_a^b(g_1-g_2)f_0 \dd t=\int_a^b(g_1-g_2) \cdot (g_1-g_2)^{q-1} \cdot \text{sgn}(g_1-g_2) \dd t=||g_1-g_2||_{L^q}^q \quad \Leftrightarrow \quad g_1=g_2\]

要证满射即证
\[\forall F \in (L^p)^* \ , \ \exists \, g \in L^q \quad \text{s.t.} \quad F=\Phi_g \quad \text{i.e.} \quad \forall f \in L^p \ , \ F(f)=\Phi_g(f)=\int_a^bg \cdot f \, \dd t\]
但是直接证明该结论较为困难,我们分三步证明:

\textbf{Step 1}: 先证该结论对简单函数成立,只需证明对特征函数$\chi_s$成立。
\[\chi_s=\chi_{[a,s]}(t)=\left\{
    \begin{array}{ll}
        1 & ,t\in[a,s] \\
        0 & ,t\in(s,b]        
    \end{array}
\right.\]
即证明存在$g$满足
\[F(\chi_s)=\int_a^bg(t)\chi(t) \, \dd t=\int_a^sg(t) \, \dd t \quad \Rightarrow \quad g(s)=F'(\chi_s)\]
那么我们只需要证明$F(\chi_s)$几乎处处可导,又因为几乎处处可导可以由绝对连续推出,故只需证明$F(\chi_s)$绝对连续。
令$G(s)=F(\chi_s) \ \delta_k=(s_k,t_k) \subset [a,b] \ (k=1,2,\cdots,n)$为$[a,b]$互不相交的分割。记$\varepsilon_k=\text{sgn}\left(F(\chi_{t_k}-\chi_{s_k})\right)$,则
\begin{equation*}
    \begin{aligned}
        \sum_{k=1}^n|G(t_k)-G(s_k)| & =\sum_{k=1}^n|F(\chi_{t_k})-F(\chi_{s_k})|=\sum_{k=1}^n|F(\chi_{t_k}-\chi_{s_k})|=\sum_{k=1}^nF(\chi_{t_k}-\chi_{s_k}) \cdot \text{sgn}\left(F(\chi_{t_k}-\chi_{s_k})\right) \\
        & =F\left(\sum_{k=1}^{n}\varepsilon_k(\chi_{t_k}-\chi_{s_k})\right)=F\left(\sum_{k=1}^{n}\varepsilon_k\chi_{[s_k,t_k]}\right) \\
        & \leq ||F|| \cdot \left\|\sum_{k=1}^n\varepsilon_k\chi_{[s_k,t_k]}\right\|_{L^p} \leq ||F|| \cdot \sum_{k=1}^n\left\|\varepsilon_k\chi_{[s_k,t_k]}\right\|_{L^p}=||F|| \cdot \sum_{k=1}^n|t_k-s_k|
    \end{aligned}
\end{equation*}
可知$F(\chi_s)$绝对连续,即
\[\forall \varepsilon>0 \ , \ \exists \, \sum_{k=1}^n|t_k-s_k|<\frac{\varepsilon}{||F||} \quad \text{s.t.} \quad \sum_{k=1}^n|G(t_k)-G(s_k)|<\varepsilon\]
再由
\[G(s)=G(a)+\int_a^sg(t) \, \dd t=\int_a^sg(t) \, \dd t=\int_a^bg(t)\chi_s(t) \, \dd t=\Phi_g(\chi_s)\]
即找到了$g$满足条件,由特征函数情形可知,对简单函数(阶梯函数)该结论也成立,即
\[\forall F \in (L^p)^* \ , \ \exists \, g \in L^q \quad \text{s.t.} \quad \forall \psi(t)=\sum_{k=1}^{n}a_k\chi_{[s_k,t_k]} \in L^p \ , \ F(\psi)=\Phi_g(\psi)=\int_a^bg \cdot \psi \, \dd t\]

\textbf{Step 2}: 再证该结论对有界可测函数成立。\\
若$f$为有界可测函数,则存在简单函数$\{\psi_n\} \quad \text{s.t.} \quad \psi_n \xrightarrow{a.e.} f$,有$Lebesgue$控制收敛定理可知只要$g \in L^1$,即有
\[F(f)=F\left(\lim_{n \to \infty}\psi_n\right)=\lim_{n \to \infty}F\left(\psi_n\right)=\lim_{n \to \infty}\Phi_g(\psi_n)=\lim_{n \to \infty}\int_a^b\psi_n(t)g(t) \, \dd t=\int_a^bf(t)g(t) \, \dd t=\Phi_g(f)\]
下面只需证明找到的$g \in L^q$即可,考虑
\[f=\frac{1}{||g||_{L^q}^{q-1}} \cdot g^{q-1} \cdot \text{sgn}(g) \ , \ f_N=\left\{
    \begin{array}{ll}
        f & ,||f|| \leq N \\ 0 & ,||f||>N
    \end{array}
\right.\]
则
\[|F(f_n)|=|\Phi_g(f_N)|=\frac{1}{||g||_{L^q}^{q-1}}\int_{\{|f| \leq N\}}g^{q-1} \cdot \text{sgn}(g) \cdot g \, \dd t=\frac{1}{||g||_{L^q}^{q-1}}\int_{\{|f| \leq N\}}|g|^q \, \dd t \leq ||F|| \cdot ||f_N||_{L^p}\]
当$n \to +\infty$时,有
\[|F(f_n)|=||g||_{L^q} \leq ||F|| \cdot 1 = ||F||\]
即证$g \in L^q$。

\textbf{Step 3}: 最后证该结论对$\forall f \in L^p$成立。\\
类似Step 2中的构造,考虑$\forall f \in L^p$,存在一列简单可测函数列$\{f_N\} \quad \text{s.t.} \quad F_n \xrightarrow{L^p} f \ , \ ||f_N|| \leq N$,故
\[F(f_N-f)=\int_a^b(f_N-f) \cdot g \, \dd t \leq ||f_n-f||_{L^p} \cdot ||g||_{L^q} \to 0 \quad \Rightarrow \quad F(f_N) \to F(f)\]

综述三步所述,我们通过一步构造两步逼近证明了满射。

\textbf{Q.E.D.}

\begin{example}
\quad $(l^p[a,b])^*=l^q[a,b]$其中$1/p+1/q=1 \ , \ p \in [1,+\infty)$。
\end{example}
\begin{example}
\quad $(C[a,b])^*=V_0[a,b]:=\{v \in V[a,b] \, | \, v(a)=0 \ , \ v(t+0)=v(t) \ , \ \forall t \in [a,b]\}$,其中$V[a,b]$是$[a,b]$上有界变差函数全体。
\end{example}
\begin{example}
\quad 记收敛数列空间为$c$,则$c^*=l^1$。
\end{example}

\section{自反性与弱收敛}
考虑共轭空间的共轭空间,定义为
\begin{definition}[二次共轭空间]
    设$X$为赋范线性空间,$X^{**}=(X^*)^*$是$X$的二次共轭空间。
\end{definition}
联想到负负得正这个我们熟知的在整数上的特性,我们自然会考虑,二次共轭空间是否就是原空间呢?考虑如下映射,称为\textbf{典型映射}:
\[F:X \to X^{**} \ , \ x \mapsto F_x \quad \text{s.t.} \quad \forall f \in X^* \ , \ F_x(f)=f(x)\]
验证$F_x \in X^{**}$即验证其有界线性:
\[F_x(\alpha f+\beta g)=(\alpha f+\beta g)(x)=\alpha f(x)+\beta g(x)=\alpha F_x(f)+\beta F_x(g)\]
\[|F_x(f)|=|f(x)| \leq ||f|| \cdot ||x|| \quad \Rightarrow \quad ||F_x|| \leq ||x||\]
再由$Hahn-Banach$定理:$\exists \, f_0 \quad \text{s.t.} \quad ||f_0||=1 \ , \ f_0(x)=||x||$,则
\[|F_x(f_0)|=|f_0(x)|=||x|| \quad \Rightarrow \quad ||F_x|| \geq ||x||\]
综上所述$F$是一个单的等距映射$||F_x||=||x||$,但不一定满。如果$F$是满的那么二次共轭空间在等距的意义下就是原空间,我们将这类空间称之为\textbf{自反空间}。
\begin{definition}[二次共轭空间]
    设$X$为赋范线性空间,若典型映射$F$是满射则称$X$为自反空间。
\end{definition}
可以来看几个自反空间的例子。
\begin{proposition}
    1、$L^p \ , \ p \in (1,+\infty)$是自反空间;\\
    2、$l^p \ , \ p \in (1,+\infty)$是自反空间;\\
    3、$L^1,L^{\infty},l^1,l^{\infty}$不是自反空间;\\
    4、$C[a,b]$不是自反空间。
\end{proposition}

在引入共轭空间后我们可以新定义一种收敛方式,称为\textbf{弱收敛}。
\begin{definition}[弱收敛]
设$X$为赋范线性空间,$\{x_n\} \subset X$,若$\forall f \in X^* \ , \ \exists \, x_0 \in X, \ , \ f(x_n) \to f(x_0)$则称$\{x_n\}$弱收敛于$x_0$,记为$x_n \xrightarrow{w} x_0$,称$x_0$为$\{x_n\}$的弱极限。
\end{definition}
弱收敛有以下性质:
\begin{lemma}
1、弱极限存在必唯一;\\
2、强收敛$\Rightarrow$弱收敛;\\
3、有限维赋范线性空间是自反空间。
\end{lemma}
\begin{proof}
1、设$x_n \xrightarrow{w} x_0, \ x_1$,则$\forall f \in X^* \ , \ f(x_n)\to f(x_0), \ f(x_1)$,由Hahn-Banach定理可知$x_0=x_1$;\\
\end{proof}
\chapter{希尔伯特空间}
\begin{introduction}
    \item 内积空间~\ref{nj}
    \item 正交性与正交集~\ref{zj}
\end{introduction}
\section{内积空间}\label{nj}
\begin{definition}[内积和内积空间]
令$H$为一个线性空间,定义二元映射$(\cdot,\cdot): \ H \times H \to \mathbb{K}$,若对$\forall x,y \in H \ , \ \alpha,\beta \in \mathbb{K}$都满足:
\begin{itemize}
    \item 1. 非负性:$(x,x) \geq 0$且$(x,x)=0$当且仅当$x=0$;
    \item 2. 对称性:$(x,y)=(y,x)^*$;
    \item 3. 双线性:$(\alpha x + \beta y,z)=\alpha(x,z)+\beta(y,z) \ , \ (z,\alpha x + \beta y)=\alpha^*(z,x)+\beta^*(z,y)$;
\end{itemize}
则称$(\cdot,\cdot)$是$H$上对内积。称$(H,(\cdot,\cdot))$为内积空间。
\end{definition}
\begin{theorem}[Cauchy–Schwarz 不等式]
$\forall x,y \in H \ , \ |(x,y)|^2 \leq (x,x)\cdot(y,y)$
\end{theorem}
\begin{proof}
$\forall \lambda \in \mathbb{K} \ , \ x,y \in H \ , \ 0 \leq (x+\lambda y,x+\lambda y)=(x,x)+\lambda(y,x)+\lambda^*(x,y)+\lambda\lambda^*(y,y)$,对$y=0$,上式显然成立,对$y>0$,可以取$\lambda=-(x,y)/(y,y)$,则上式子可以写成:
\[(x,x)-\frac{(y,x)(x,y)}{(y,y)}-\frac{(y,x)(x,y)}{(y,y)}+\frac{(y,x)(x,y)}{(y,y)}\geq0 \quad \Leftrightarrow \quad |(x,y)|^2 \leq (x,x)\cdot(y,y)\]
\end{proof}

\begin{definition}[Hilbert 空间]
在内积空间$(H,(\cdot,\cdot))$上可以定义范数$||x||=\sqrt{(x,x)}$,称为由内积诱导的范数,易证$(H,||\cdot||)$是赋范线性空间,若$(H,||\cdot||)$完备,则称为 Hilbert 空间。
\end{definition}
\begin{proposition}[内积的连续性]
若$x_n \to x \ , \ y_n \to y$依范数收敛,则$(x_n,y_n) \to (x,y)$。
\end{proposition}
\begin{proof}
\[\begin{aligned}
|(x_n,y_n)-(x,y)|&=|(x_n,y_n)-(x,y_n)+(x,y_n)-(x,y)|=|(x_n-x,y_n)+(x,y_n-y)|\\
& \leq |(x_n-x,y_n)|+|(x,y_n-y)| \leq ||x_n-x||\cdot||y_n|| + ||x||\cdot||y_n-y|| \to 0
\end{aligned}\]
\end{proof}

\begin{example}
\quad 在$l^2$上定义内积$(x,y)$:
\[l^2=\left\{x=\{\xi_i\} \ \Bigg{|} \ \xi_i \in \mathbb{C} \ , \ \sum_{i=1}^{\infty}|\xi_i|^2<+\infty\right\} \ , \ (x,y)=\sum_{i=1}^{\infty}\xi_i\eta_i^* \ , \ (\forall x=\{\xi_i\}, \ y=\{\eta_i\} \in H)\]
\end{example}
\begin{proof}
只需验证$(x,y)<\infty \ , \ \forall x,y \in l^2$:
\[(x,y)=\sum_{i=1}^{\infty}\xi_i\eta_i \leq \sum_{i=1}^{\infty}|\xi_i\eta_i| \leq \sqrt{\sum_{i=1}^{\infty}|\xi_i|^2}\cdot\sqrt{\sum_{i=1}^{\infty}|\eta_i|^2}=||x||_{l^2}\cdot||y||_{l^2}<\infty\]
\end{proof}

\begin{example}
\quad $L^2[a,b]$。
\end{example}
\begin{proof}
\[\forall x,y \in L^2 \ , \ (x,y)=\int_a^bx(t)y^*(t)\dd{t} \leq \left(\int_a^b|x(t)|^2\dd{t}\right)^{\frac{1}{2}}\cdot\left(\int_a^b|y(t)|^2\dd{t}\right)^{\frac{1}{2}}<\infty\]
\end{proof}

范数总能通过内积诱导出,那内积是否都能从范数诱导出呢?
\begin{theorem}
设$H$是$\mathbb{K}$上的内积空间,若范数满足平行四边形性质:$\forall x,y \in H \ , \ ||x+y||^2+||x-y||^2=2||x||^2+2||y||^2$,则可以通过极化恒等式通过范数定义内积,对$\forall x,y \in H$:
\begin{itemize}
\item 1. $\mathbb{K}=\mathbb{R} \ , \ (x,y)_{\mathbb{R}}=(||x+y||^2-||x-y||^2)/4$;
\item 2. $\mathbb{K}=\mathbb{C} \ , \ (x,y)_{\mathbb{C}}=(||x+y||^2-||x-y||^2+i||x+iy||^2-i||x-iy||^2)/4=(x,y)_{\mathbb{R}}+i(x,iy)_{\mathbb{R}}$。
\end{itemize}
\end{theorem}

\begin{example}
\quad $C^k[a,b]$都不是 Hilbert 空间。
\end{example}

\begin{example}
\quad $L^p[a,b]$是 Hilbert 空间当且仅当$p=2$。
\end{example}
\begin{proof}
取:
\[x(t)=\left\{\begin{array}{cl}
(\frac{2}{b-a})^{\frac{1}{p}} & , \ t\in[a,\frac{a+b}{2}]\\
0 & , \ t\in(\frac{a+b}{2},b]
\end{array}\right. \ , \ y(t)=\left\{\begin{array}{cl}
0 & , \ t\in[a,\frac{a+b}{2}]\\
(\frac{2}{b-a})^{\frac{1}{p}} & , \ t\in(\frac{a+b}{2},b]
\end{array}\right.\]
显然$||x||_{L^p}=||y||_{L^p}=1$,而$||x+y||_{L^p}=||x+y||_{L^p}=2^{\frac{1}{p}}$,当且仅当$p=2$满足平行四边形性质。
\end{proof}

\section{正交性与正交集}\label{zj}
\begin{definition}[正交]
设$(H,(\cdot,\cdot))$是内积空间,则对$\forall x,y \in H-\{0\}$可以定义夹角$\theta\in[0,\pi]$满足:
\[\cos\theta=\frac{(x,y)}{||x||\cdot||y||}\in[-1,1]\]
若$(x,y)=0$,即$\theta=\pi/2$,则称$x$与$y$正交,记为$x \perp y$。
\end{definition}
\begin{itemize}
\item 若$M \subset H$,$\exists \, x \in H-M$对$\forall y \in M$都有$(x,y)=0$,则称$x$与$M$正交,记为$x \perp M$;
\item 若$M,N \subset H$,对$\forall x \in N \ , \ y \in M$都有$(x,y)=0$,则称$N$与$M$正交,记为$N \perp M$;
\item 若$M \subset H$,可以定义$M$的正交补$M^{\perp}=\{x \in H|x \perp M\}$,正交补$M^{\perp}$是闭线性子空间。
\end{itemize}

\begin{proposition}
设$M \subset H \ , \ \overline{M}=H$,则$M^{\perp}=\{0\}$。
\end{proposition}
\begin{proof}
设$x \in H \ , \ x \perp M$,由$\overline{M}=H$,可知$\exists \, \{x_n\} \subset M$使得$x_n \to x$,因此:
\[(x,x)=\lim_{n \to \infty}(x_n,x)=0 \quad \Rightarrow \quad x=0\]
\end{proof}

\begin{theorem}
设$H$是 Hilbert 空间,$M \subset H$是闭的凸子集,则对$\forall x \in H \ , \ \exists \, x_0 \in M$有:
\[||x-x_0||=d(x,M)=\inf_{y \in M}||x-y||\]
\end{theorem}


\chapter{算子谱}
\section{线性算子谱}
回顾有限维的情况,设$X$为$n$维的线性空间,$A: \ X \to X$是线性算子,对$\forall \lambda \in \mathbb{C}$考虑方程$(\lambda \mathbbm{I}-A)x=0$,也即$Ax=\lambda x$,有两种情况:
\begin{itemize}
\item 1. 只有零解,算子$\lambda \mathbbm{I}-A$为单射($\ker(\lambda \mathbbm{I}-A)=\{0\}$且线性),因此$\lambda \mathbbm{I}-A$可逆[定理\ref{theorem:djm}],也可称为正则,其行列式$\det(\lambda \mathbbm{I}-A)\neq0$,此时称$\lambda$为$A$的正则值;
\item 2. 有非零解,算子$\lambda \mathbbm{I}-A$不单进而不可逆,行列式$\det(\lambda \mathbbm{I}-A)=0$,此时称$\lambda$为$A$的特征值,数量为$n$。
\end{itemize}
\begin{definition}[正则算子]
设$X,Y$是赋范线性空间,定义算子$A \in \mathscr{B}(X,Y)$,若$R(A)=Y$且$A^{-1}\in \mathscr{B}(X,Y)$,则称$A$是正则算子。
\end{definition}
现在考虑无穷维时,设$X$是 Banach 空间,定义算子$A \in \mathscr{B}(X)$,对$\forall \lambda \in \mathbb{C}$考虑方程$(\lambda \mathbbm{I}-A)x=0$:
\begin{itemize}
\item 1. 如果该方程只有零解,那么$\lambda \mathbbm{I}-A$是单射,因此$\lambda \mathbbm{I}-A$“\href{https://math.fandom.com/zh/wiki/%E8%B0%B1}{可逆}”\footnote{很多地方都将可逆的扩大化了,严格意义上满射要求算子A的值域$R(A)=Y$,但如果仅在$R(A)$上那么可以宽松地说单射的算子可逆。},分三种情况:
\begin{itemize}
\item 1. $R(\lambda \mathbbm{I}-A)=X$,因此$(\lambda \mathbbm{I}-A)^{-1} \in \mathscr{B}(X)$,记$\lambda\in\rho(A)$为正则值;
\item 2. $R(\lambda \mathbbm{I}-A) \neq X$但$\overline{R(\lambda \mathbbm{I}-A)}=X$,$(\lambda \mathbbm{I}-A)^{-1}$存在但不有界,记$\lambda\in\sigma_c(A)$为连续谱;
\item 3. $\overline{R(\lambda \mathbbm{I}-A)} \neq X$,$(\lambda \mathbbm{I}-A)^{-1}$存在但无法判断是否有界,记$\lambda\in\sigma_r(A)$为连续谱。
\end{itemize}
\item 2. 如果方程有非零解,那么$\lambda \mathbbm{I}-A$不是单射,因此$\lambda \mathbbm{I}-A$不可逆,记$\lambda\in\sigma_p(A)$为点谱(特征值)。
\end{itemize}
\begin{definition}[算子的谱]
定义算子$A$的谱为$\sigma(A)=\mathbb{C}-\rho(A)=\sigma_p(A)\cup\sigma_c(A)\cup\sigma_r(A)$。
\end{definition}

\begin{example}
\quad 在$X=L^2[0,1]$,上利用 \href{https://math.fandom.com/zh/wiki/Fourier_%E7%BA%A7%E6%95%B0}{Fourier 分解}定义闭算子[定义\ref{definition:bsz}] $A: \ X \to X$:
\[u(t) \quad \mapsto \quad -\dv[2]{u}{t}=-\dv[2]{t}\sum_{n=-\infty}^{\infty}e^{2i\pi nt}\int_0^1u(t)e^{2i\pi nt}\dd{t}=\sum_{n=-\infty}^{\infty}(2n\pi)^2e^{2i\pi nt}\int_0^1u(t)e^{2i\pi nt}\dd{t}\]
显然$\sigma(A)=\sigma_p(A)=\{(2n\pi)^2 \ | \ n \in \mathbb{Z}\}$,对应特征空间为$\{e^{2i\pi nt} \ | \ n\in\mathbb{Z}\}$。
\end{example}

\begin{example}\label{example:6.2}
\quad 在$X=C[0,1]$上定义算子 $A: \ X \to X \ , \ u(t) \mapsto tu(t)$,证明$\sigma(A)=\sigma_r(A)=[0,1]$。
\end{example}
\begin{proof}
对$\forall \lambda \in \mathbb{C}$都有$(\lambda\mathbbm{I}-A)u(t)=(\lambda-t)u(t)=0$可知方程只有零解$u(t)=0$,因此$(\lambda\mathbbm{I}-A)^{-1}$存在,若$\lambda \notin [0,1]$,显然有$(\lambda\mathbbm{I}-A)^{-1}u=u(t)/(\lambda-t)$,因此:
\[||(\lambda\mathbbm{I}-A)^{-1}u|| \leq \sup_{t \in [0,1]}\frac{||u||}{\lambda-t}\]
进而可知$(\lambda\mathbbm{I}-A)^{-1} \in \mathscr{B}(X)$,即$\lambda$正则,可知$\sigma(A) \subset [0,1]$。若$\lambda \in [0,1]$,则$\overline{R(\lambda\mathbbm{I}-A)}=\overline{\{(\lambda-t)u(t) \ | \ u \in X\}}$,因为$\lambda-t \in X$,可知$\overline{R(\lambda\mathbbm{I}-A)}=\overline{\{u(t) \ | \ u \in X \ , \ u(\lambda)=0\}} \subset C[0,1]$,因此$\sigma(A)=\sigma_r(A)=[0,1]$。
\end{proof}

\begin{example}
\quad 在$X=L^2[0,1]$上类似上一个例子\ref{example:6.2}定义算子 $A: \ X \to X \ , \ u(t) \mapsto tu(t)$,则$\sigma(A)=\sigma_c(A)=[0,1]$。
\end{example}

\begin{theorem}
设$X$是 Banach 空间,$A \in \mathscr{B}(X)$,则:
\begin{itemize}
\item 1. $\rho(A)$是开集,也即$\sigma(A)$是闭集;
\item 2. $\sigma(A) \subset \{\lambda \in \mathbb{C} \ | \ |\lambda| \leq ||A||\}$或记为$r_{\sigma}(A) \leq ||A||$。
\end{itemize}
\end{theorem}
\begin{proof}
由定理\ref{theorem:deltazz}对$\forall \lambda \in \rho(A)$,都$\exists \, \delta=||\lambda\mathbbm{I}-A||^{-1}$,使得对$\forall a<\delta$都有$(\lambda+a)\mathbbm{I}-A=(\lambda\mathbbm{I}-A)+a\mathbbm{I}$正则,即对$\forall \lambda \in \rho(A)$其开球$\{\lambda+a \ | \ |a|<\delta\} \subset \rho(A)$,因此$\rho(A)$是开集,第一部分得证。

对$\forall \lambda \in \mathbb{C}$满足$|\lambda|>||A||$,由定理\ref{theorem:nszyjxx}知$\lambda\mathbbm{I}-A=\lambda(\mathbbm{I}-A/\lambda)$正则,因此$\sigma(A) \subset \{\lambda \in \mathbb{C} \ | \ |\lambda| \leq ||A||\}$。
\end{proof}

\section{紧算子谱}
\section{希尔伯特空间的自共轭紧算子谱}
\chapter{习题}
\section*{距离空间与拓扑空间习题}
\begin{example}
\quad 证明:任意开集的并为开集。
\end{example}
\begin{proof}
设$\Lambda$为指标集,$A_{\lambda} \ (\lambda \in \Lambda)$为开集:
\[\forall x \in \bigcup_{\lambda \in \Lambda}A_{\lambda} \ \Rightarrow \ x \in A_{\lambda_i} \ \Rightarrow \ B_{\varepsilon}(x) \subset A_{\lambda_i} \subset \bigcup_{\lambda \in \Lambda}A_{\lambda}\]
\end{proof}

\begin{example}
\quad 证明:在$(X,d)$中,$\overline{A}=A \ \Leftrightarrow \ A^c$是开集。
\end{example}
\begin{proof}必要性:$\overline{A}=A \ \Rightarrow \ A^c$是开集:
\[\forall x \in A^c \ \Rightarrow \ \exists \, B_{\varepsilon}(x) \cap A=\varnothing \ \Leftrightarrow \ B_{\varepsilon}(x) \subset A^c \ \Rightarrow \ A^c\text{是开集}\]
充分性:$A^c$是开集$ \ \Rightarrow \ \overline{A}=A$:
\[\forall x \in A^c \ \Rightarrow \ \exists \, B_{\varepsilon}(x) \subset A^c \ \Leftrightarrow \ B_{\varepsilon}(x) \cap A=\varnothing \ \Rightarrow \ A\text{的所有接触点都在$A$中} \ \Rightarrow \ \overline{A}=A\]
\end{proof}

\begin{example}
\quad 证明$D$是距离空间且$\ x_n(t) \xrightarrow{d} x(t) \ \Leftrightarrow \ x_n(t) \rightrightarrows x(t)\text{且}x_n'(t) \rightrightarrows x'(t)$。
\[D=C^1[0,1]=\{x(t),x'(t) \in C[0,1]\} \quad d(x,y)=\mathop \text{sup}\limits_{t \in [0,1]}|x(t)-y(y)|+\mathop \text{sup}\limits_{t \in [0,1]}|x'(t)-y'(y)|\]
\end{example}

\begin{example}
\quad 设$d$是距离,证明$d/(1+d)$也是距离。
\end{example}

\begin{example}
\quad 证明$d(x,y)=0$时$x=y$在如下空间中成立:
\[X=\{f(z)\text{在}|z|<1\text{上解析},\text{在}|z| \leq 1\text{上连续}\} \ , \ d(x,y)=\mathop \text{sup}\limits_{|z|=1}|x(z)-y(z)|\]
\end{example}
\begin{proof}
当$|z|=1$时这是好证的。当$|z|<1$时,利用极值原理或柯西积分公式(以后者为例):
\[f(z):=x(z)-y(z) \ , \ \forall z_0 \in |z|<1 \ , \ f(z_0)=\int\limits_{|z|=1}\frac{f(z)}{z-z_0}\dd z=\int\limits_{|z|=1}\frac{x(z)-y(z)}{z-z_0}\dd z=0\]
\end{proof}

\begin{example}
\quad 考虑如下距离空间,下面这个例子告诉我们在距离空间中小球可能包含大球,但大球不能太大:
\begin{figure}[htbp]
    \center
    \includegraphics[scale=0.2]{./fig/ex-1.png}
\end{figure}
\[B_4(a)=\{a,b,c,d\} \subset B_3(c)=\{a,b,c,d,e\}\]
证明:若$B_7(a) \subset B_3(b)$则$B_7(a)=B_3(b)$。
\end{example}
\begin{proof}
$\forall x \in B_3(b) \ , \ d(x,a) \leq d(x,b)+d(b,a)<3+3=6<7 \ \Rightarrow \ x \in B_7(a)$
\end{proof}

\begin{example}
\quad $(X,d)$是距离空间,$A \subset X$,令$f(x)=\mathop \text{inf}\limits_{y \in A}d(x,y)$,证明$f(x)$连续。
\end{example}
\begin{proof}
已知三角不等式:$\forall x_1,x_2 \in X \ , \ y \in A \ , \ d(x_1,x_2) \leq d(x_1,y)+d(y,x_2)$,两侧对$y$取下极限:
\[f(x_1) \leq d(x_1,x_2)+f(x_2) \ \Leftrightarrow \ f(x_1)-f(x_2) \leq d(x_1,x_2)\]
同理有:
\[f(x_2)-f(x_1) \leq d(x_1,x_2)\]
因此:
\[\Rightarrow \ |f(x_1)-f(x_2)| \leq d(x_1,x_2) \to 0\]
即:
\[\forall \varepsilon>0 \ , \ d(x_1,x_2)<\delta=\varepsilon \quad \text{s.t.} \quad |f(x_1)-f(x_2)|<\varepsilon\]
\end{proof}

\begin{example}
\quad 证明$C^1[0,1]$是完备距离空间。
\end{example}
\begin{proof}
设$\{x_n(t)\}$是$C^1[0,1]$中的柯西列,$C^1[0,1]$上定义的距离为:
\[d(x_n(t),x_m(t))=\mathop \text{sup}\limits_{t \in [0,1]}|x_n(t)-x_m(t)|+\mathop \text{sup}\limits_{t \in [0,1]}|x_n'(t)-x_m'(t)|\]
由$C[0,1]$的完备性可知,我们只需要证明$\{x_n'(t)\}$是$C[0,1]$中的柯西列,即$x_n'(t) \rightrightarrows h(t) \in C[0,1]$。只需证:
\[x_n(t) \rightrightarrows x(0)+\int_0^th(s)\dd s \ , \ x(0)=\lim_{n \to \infty}x_n(0)\]
验证,利用微积分基本定理:
\[\begin{aligned}
\left|x_n(t)-x(0)-\int_0^th(s)\dd s\right| &\leq \left|x_n(0)+\int_0^tx_n'(s)\dd s-x(0)-\int_0^th(s)\dd s\right|\\
&\leq \left|x_n(0)-x(0)\right|+\int_0^t|x_n'(s)-h(s)|\dd s\\
&\leq \left|x_n(0)-x(0)\right|+\mathop \text{sup}\limits_{t \in [0,1]}|x_n'(t)-h(t)| \to 0
\end{aligned}\]
且上述极限的取值与$t \in [0,1]$无关,故为一致收敛,原命题得证。
\end{proof}

\begin{example}
\quad 证明:闭球套定理成立的距离空间是完备距离空间。
\end{example}
\begin{proof}
原命题等价于设$(X,d)$中任一套闭球套$(r_n \to 0)$都有非空交,那么$(X,d)$完备。设$\{x_n\}$是$X$中的柯西列,由它构造闭球套:
\[\forall k \in \mathbb{Z}_+ \ , \ \exists \, n_k>0 \quad \text{s.t.} \quad m,n \geq n_k \ , \ d(x_m,x_n)<\frac{1}{2^k} \ \Rightarrow \ x_n \in B_{\frac{1}{2^{k+1}}}(x_{n_k})\]
因此存在点列$\{x_{n_k}\}$构造的闭球满足:
\[\overline{B_{\frac{1}{2^{k+1}}}(x_{n_{k+1}})} \subset \overline{B_{\frac{1}{2^k}}(x_{n_k})}\]
当且仅当:
\[\exists \,! \, x \in \bigcap_{k=1}^{\infty}\overline{B}_{\frac{1}{2^k}}(x_{n_k})\]
子列$\{x_{n_k}\}$才能收敛$d(x_{n_k},x) \to 0$。由于柯西利的子列收敛,故原柯西列也收敛。
\end{proof}

\begin{example}
\quad 设$(X,d)$完备,$\tilde{f}$为$X$上的一族连续函数,满足$\tilde{f}=\{F(x)|\forall x \in X \ , \exists \, M_x>0 \quad \text{s.t.} \quad |F(x)| \leq M_x\}$。证明:$\exists \, M>0 \ , \ \exists \, U\text{(开集)} \in X \quad \text{s.t.} \quad \forall x \in U \ , \ F \in \tilde{f} \ , \ |F(x)| \leq M$。
\end{example}
\begin{proof}
Baire 定理应用:设$E_n=\{x|x \in X \ , \ |F(x)| \leq n \ , \ F \in \tilde{f}\}$,因此对:
\[\forall x \in X \ , \ x \in E_{[M_x]+1} \quad \Rightarrow \quad X \subset \bigcup_{n=1}^{\infty}E_n \ \left(or \ X=\bigcup_{n=1}^{\infty}E_n \right)\]
由于$(X,d)$完备,由 Baire 定理,至少存在一个$M$使得$E_M$不是疏集,即存在一个开集$U \subset \overline{E}_M$,对$\forall x \in U \ , \ \exists \, \{x_n\} \subset E_M$使得$x_n \to x$。由连续性可知:
\[|F(x)|=\lim_{n \to \infty}|F(x_n)| \leq M \ , \ \forall F \in \tilde{f}\]
\end{proof}

\begin{example}
在距离空间中证明:连续$\ \Leftrightarrow \ $开集的原像是开集。
\end{example}
\begin{proof}
必要性:设$f:X \to Y$是连续映射,设$G \subset Y$是任一开集,要证$f^{-1}(G)$是$X$中的开集。即证对$\forall x_0 \in f^{-1}(G) \ , \  \exists \, \varepsilon>0$使得$B_{\varepsilon}(f(x_0)) \subset G$。由于$f$是连续映射,即$\forall y \in f^{-1}(G) \ , \  \exists \, \delta>0$使得当$d(y,x_0)<\delta$时有$d_Y(f(y),f(x_0))<\varepsilon$,这也即$f(B_{\delta}(x_0)) \subset B_{\varepsilon}(f(x_0)) \subset G$。因此$B_{\delta}(x_0) \subset f^{-1}(G)$,即$x_0$是$f^{-1}(G)$的内点,由$x_0$的任意性可知$f^{-1}(G)$是开集。
\begin{figure}[htbp]
    \center
    \includegraphics[scale=0.14]{./fig/ex-2.png}
\end{figure}

充分性:设$f:X \to Y$满足开集的原像是开集,则对$\forall x_0 \in X \ , \ \exists \, \varepsilon>0$有开球$B_{\varepsilon}(f(x_0))$是$Y$中的开集,因此$f^{-1}(B_{\varepsilon}(f(x_0)))$是$X$中的开集,且$x_0 \in f^{-1}(B_{\varepsilon}(f(x_0)))$是内点。因此$\exists \, \delta>0$使得$f(B_{\delta}(x_0)) \subset B_{\varepsilon}(f(x_0))$,可知$f$在$x_0$处连续。由$x_0$任意性可得$f$在$X$上连续。
\begin{figure}[htbp]
    \center
    \includegraphics[scale=0.14]{./fig/ex-3.png}
\end{figure}
\end{proof}

\begin{example}
举例说明在压缩映射中
\begin{itemize}
    \item 1. 完备性不可少;
    \item 2. $d(Tx,Ty)<d(x,y) \ \forall x \neq y$是不充分的。
\end{itemize}
\end{example}
\begin{proof}
\begin{itemize}
\item 1. $R-\{0\} \to R-\{0\} \quad x \mapsto x/2$无不动点;
\item 2. 定义映射$T: \ [0,+\infty) \to [0,+\infty), \ T(x)=x+1/(1+x)$,对$\forall x,y \in [0,+\infty)$计算像空间中的距离:
\[d(T(x),T(y))=\left|x+\frac{1}{1+x}-y-\frac{1}{1+y}\right|=\left|1-\frac{1}{(1+x)(1+y)}\right|d(x,y)<d(x,y)\]
当$T(x)=x$时,会得出$1/(1+x)=0$,但该方程无解,无不动点。
\end{itemize}
\end{proof}

\begin{example}
\quad $T:X \to X \ , \ X$完备,$T$在闭球$\overline{B_r(x_0)} \ x_0 \in X$上为压缩映射,即对$\theta \in (0,1)$,满足$d(Tx,Ty) \leq \theta d(x,y)$,且$d(x_0,Tx_0)<(1-\theta)r$,证明:$T$在$\overline{B_r(x_0)}$上有唯一不动点。
\end{example}
\begin{proof}
因为完备闭子空间也是完备的[定理\ref{theorem:zkj}],故只需证$T(\overline{B_r(x_0}) \subset \overline{B_r(x_0)}$,对$\forall x \in \overline{B}_r(x_0)$有:
\[d(Tx,x_0) \leq d(Tx,Tx_0)+d(Tx_0,x_0) \leq \theta d(x,x_0)+(1-\theta)r \leq \theta r+(1-\theta)r=r\]
可知$Tx \in \overline{B_r(x_0)}$,由$x$的任意性可知$T(\overline{B_r(x_0)}) \subset \overline{B_r(x_0)}$。把$T$限制在$\overline{B_r(x_0)}$上:$T:\overline{B_r(x_0)} \to \overline{B_r(x_0)}$,$T$为压缩映射,$\overline{B_r(x_0)}$是完备空间$X$上的闭子空间,所以$\overline{B_r(x_0)}$完备。由压缩映射定理即证。
\end{proof}

\begin{example}
\quad $(t_0,s_0) \in \mathbb{R}^2 \ , \ f(t,s)$在$(t_0,s_0)$的邻域$N$中连续,满足$s_0=f(t_0,s_0)$。同时$\partial_sf(t,s)$在邻域$N$中存在且在$(t_0,s_0)$连续,且$\partial_sf(t_0,s_0)=0$。证明当$\exists \, \delta>0 \ , \ t \in [t_0-\delta,t_0+\delta] \ , \ x(t) \in C[t_0-\delta,t_0+\delta]$时,方程$x(t)=f(t,x(t)) \ , \ x(t_0)=s_0$有唯一解。
\end{example}
\begin{proof}
由$\partial_sf(t,s) \in C(N)$且$\partial_sf(t_0,s_0)=0$,即$\exists \, \delta>0$,当$|t-t_0| \ ,\ |s-s_0| \leq \delta$时,有$|\partial_s(t,s)|=\theta < 1$。令:
\[X=\{x(t) \in C[t_0-\delta,t_0+\delta] \ , \ x(t_0)=s_0 \ , \ \mathop \text{sup}\limits_{t \in [t_0-\delta,t_0+\delta]}|x(t)-s_0| \leq \frac{\delta}{2}\}\]
由$C[t_0-\delta,t_0+\delta]$完备,所以对$\forall \{x_n(t)\} \subset X \ , \ \exists \, x_{\infty}(t) \in C[t_0-\delta,t_0+\delta]$使得$x_n(t) \rightrightarrows x_{\infty}(t)$,且:
\[x_{\infty}(t_0)=\lim_{n\to\infty}x_n(t_0)=\lim_{n\to\infty}s_0=s_0 \ , \ \forall n \in \mathbb{N}_+ \ , \ |x_n(t)-s_0| \leq \frac{\delta}{2} \quad \Rightarrow \quad |x(t)-s_0|=\lim_{n\to\infty}|x_n(t)-s_0|\leq \frac{\delta}{2}\]
上述结论(极限不保严格不等号,所以定义时取闭保持闭性)对$\forall t \in [t_0-\delta,t_0+\delta]$都成立,不等式两侧取上界:
\[\mathop \text{sup}\limits_{t \in [t_0-\delta,t_0+\delta]}|x(t)-s_0| \leq \frac{\delta}{2}\]
即$x \in X$。因此$X$是闭子空间,由完备闭子空间也是完备的[定理\ref{theorem:zkj}]知$X$是完备的。

定义$T: \ X \to X \ , \ T(x(t))=f(t,x(t))$,验证良定义(满足初值条件以及像在原空间中):
\begin{itemize}
\item 1. $T(x(t_0))=f(t_0,x(t_0))=f(t_0,s_0)=s_0=x(t_0)$;
\item 2. $\forall x \in X \ , \ T(x(t)) \in X$:
\[\mathop \text{sup}\limits_{t \in [t_0-\delta,t_0+\delta]}|T(x(t))-s_0|=\mathop \text{sup}\limits_{t \in [t_0-\delta,t_0+\delta]}|f(t,x(t))-s_0|=\mathop \text{sup}\limits_{t \in [t_0-\delta,t_0+\delta]}|f_s(t,\xi(t))||x-s_0| \leq \frac{\delta}{2}\]
\end{itemize}
计算范数$d(Tx,Ty)$:
\[d(Tx,Ty)=\mathop \text{sup}\limits_{t \in [t_0-\delta,t_0+\delta]}|f(t,x(t))-f(t,y(t))|=\mathop \text{sup}\limits_{t \in [t_0-\delta,t_0+\delta]}|f_s(t,\xi(t))||x(t)-y(t)| \leq \theta d(x,y)\]
因此$T$是压缩映射,由压缩映射定理可知$\exists \, ! \, x(t) \in X$满足$x(t)=f(t,x(t))$。
\end{proof}

\begin{example}
\quad 证明:全有界集$A$的有限$\varepsilon$-网可取为$A$的子集。
\end{example}
\begin{proof}
对$\forall \varepsilon>0$,存在$A$的有限$\varepsilon/2$-网$N=\{x_1,x_2,\cdots,x_k\}$,若$B_{\varepsilon/2}(x_i) \cap A =\varnothing$,则可把$x_i$从$N$中去掉。不妨设对每个$x_i \in N$都满足$B_{\varepsilon/2}(x_i) \cap A \neq \varnothing$,取$y_i \in B_{\varepsilon/2}(x_i) \cap A$,显然有$B_{\varepsilon/2}(x_i) \subset B_{\varepsilon}(y_i)$。因此$\{y_1,y_2,\cdots,y_k\} \subset A$为$A$的有限$\varepsilon$-网。
\begin{figure}[htbp]
    \center
    \includegraphics[scale=0.2]{./fig/ex-4.png}
\end{figure}
\end{proof}

\begin{example}
\quad 证明:$C^1[0,1]$中的有界集$A$是$C[0,1]$中的相对紧集。
\end{example}
\begin{definition}[相对紧集]
若集合$M$的闭包$\overline{M}$在空间中是紧集,则称$M$是相对紧集。
\end{definition}
\begin{proof}
$C[0,1]$是距离空间,故$C[0,1]$中的紧集是列紧的闭集,$\overline{A}$是闭集,故只需证明$\overline{A}$是列紧的。分三步证明:
\begin{proposition}
1. $A$在距离空间$C[0,1]$中有界(即一致有界)。
\end{proposition}
\begin{proof}
由$A$在$C^1[0,1]$中有界,可得对$\forall x(t) \in A \ , \ \exists \, K>0$满足$d_C(x(t),0) \leq d_{C^1}(x(t),0) \leq K$,命题得证。
\end{proof}
\begin{proposition}
2. $A$等度连续。
\end{proposition}
\begin{proof}
利用中值定理对$\forall \varepsilon>0 \ , \ x(t) \in A \ , \ t_1,t_2 \in [0,1]$满足$|x(t_1)-x(t_2)|=|x'(\xi)||t_1-t_2| \leq K|t_1-t_2|$,取$|t_1-t_2|<\varepsilon/K$即得$|x(t_1)-x(t_2)|<\varepsilon$,$A$等度连续得证。
\end{proof}
由上述两命题以及 Arzela-Ascoli 定理[定理\ref{the:AA}]可以得$A$是列紧的。
\begin{proposition}
列紧集的闭包还是列紧的。
\end{proposition}
\begin{proof}
任取$\overline{A}$中的柯西列$\{x_n(t)\}$,由闭包的性质,对$\forall x_n(t) \in \overline{A} \ , \ \exists \, y_n(t) \in A$满足$d_C(x_n(t),y_n(t))<1/n$。因为$A$是列紧的,所以$\exists \, n_k$使得$y_{n_k}(t) \to y(t)$,则:
\[d_C(x_{n_k}(t),y(t)) \leq d_C(x_{n_k}(t),y_{n_k}(t))+d_C(y_{n_k}(t),y(t))\leq \frac{1}{n_k}+d_C(y_{n_k}(t),y(t)) \to 0\]
即证$\overline{A}$列紧。
\end{proof}
\end{proof}

\begin{example}
\quad $X$是紧距离空间,$T:X \to X$是连续映射,对$\forall x \neq y$满足$d(Tx,Ty)<d(x,y)$。证明:$T$存在唯一不动点。
\end{example}
\begin{remark}
\quad 需要注意的是这里$T$不一定是压缩映射,如取$X=[0,1]$,定义映射$T:X \to X$:
\[x \mapsto Tx=\frac{x}{1+x}\]
则对$\forall x \neq y \in [0,1]$:
\[|Tx-Ty|=\left|\frac{x}{1+x}-\frac{y}{1+y}\right|=\frac{|x-y|}{|1+x||1+y|}<|x-y|\]
显然上述映射$T$有唯一不动点$T(0)=0$。
\end{remark}
\begin{proof}
定义$f(x)=d(x,Tx)$,易证,$f$在$X$上连续,故由于$X$是紧集可知$f$有最小值,设$x_0$是$f$的极小值点。若$f(x_0)>0$,则$f(x_0) \leq d(Tx_0,T^2X_0)<d(x_0,Tx_0)=f(x_0)$,得出矛盾,故$f(x_0)=0$即$Tx_0=x_0$是不动点。若$\exists \, y_0$使得$Ty_0=y_0$且$y_0 \neq x_0$,则$0<d(x_0,y_0)=d(Tx_0,Ty_0)<d(x_0,y_0)$,得出矛盾,因此不动点唯一。
\end{proof}

\section*{赋范线性空间习题}
\begin{example}
\quad 证明:Banach 空间$c_0=\{x=\{\xi_k\}|\xi_k\to0\} \ , \ ||x||=\mathop \text{sup}\limits_{k}|\xi_k|$可分。
\end{example}
\begin{proof}构造$A=\{r=\{r_k\}|r_k \in \mathbb{Q} \ , \ \exists \, n>0 \ , \ \forall k>n \ , \ r_k=0\}$,显然$A \subset c_0$。依题意,对$\forall \varepsilon>0 \ , \ x=\{\xi_k\} \in c_0 \ , \ \exists \, N>0$当$\forall n \geq N$满足$|\xi_n|<\varepsilon$。显然$\exists \, r=(r_1,r_2,\cdots,r_N,0,0,\cdots) \in A$满足:
\[||x-r||=\mathop \text{sup}\limits_{k \in \mathbb{N}_+}|r_k-\xi_k| \leq \text{max}\left\{\mathop \text{sup}\limits_{k \leq N}|r_k-\xi_k| \ , \ \varepsilon\right\}\leq\varepsilon\]
因为$\text{card}(A)=\text{card}(\mathbb{Q})$,故$c_0$是可分的。
\end{proof}

\begin{example}
\quad 设$x_i,y_i \in \mathbb{K} \ , \ p,q>0$,证明离散形式的 H\"{o}lder 不等式:
\[\frac{1}{p}+\frac{1}{q}=1 \ , \ \sum_{i=1}^n|x_iy_i| \leq \left(\sum_{i=1}^n|x_i|^p\right)^{\frac{1}{p}}\left(\sum_{i=1}^n|y_i|^q\right)^{\frac{1}{q}}\]
和 Minkowski 不等式:
\[p \geq 1 \ , \ \left(\sum_{i=1}^n|x_i+y_i|^p\right)^{\frac{1}{p}} \leq \left(\sum_{i=1}^n|x_i|^p\right)^{\frac{1}{p}}+\left(\sum_{i=1}^n|y_i|^p\right)^{\frac{1}{p}}\]
\end{example}
\begin{proof}
记
\[A=\left(\sum_{i=1}^n|x_i|^p\right)^{\frac{1}{p}} \ , \ B=\left(\sum_{i=1}^n|y_i|^q\right)^{\frac{1}{q}}\]
由 Young 不等式[定理\ref{theorem:young}]
\[\frac{|x_i| \cdot |y_i|}{A \cdot B} \leq \frac{1}{p}\frac{|x_i|^p}{A^p}+\frac{1}{q}\frac{|y_i|^q}{B^q} \quad \Rightarrow \quad \frac{1}{A \cdot B}\sum_{i=1}^n|x_iy_i| \leq \frac{1}{pA^p}\sum_{i=1}^n|x_i|^p+\frac{1}{qB^q}\sum_{i=1}^n|y_i|^q=1\]
\[\Rightarrow \quad \sum_{i=1}^n|x_iy_i| \leq A \cdot B=\left(\sum_{i=1}^n|x_i|^p\right)^{\frac{1}{p}} \cdot \left(\sum_{i=1}^n|y_i|^q\right)^{\frac{1}{q}}\]
离散形式的 H\"{o}lder 不等式即证。下证离散形式的 Minkowski 不等式,易知$1/p+1/q=1 \ \Leftrightarrow \ p=(p-1)q$:
\[\begin{aligned}
\sum_{i=1}^n|x_i+y_i|^p&=\sum_{i=1}^n|x_i+y_i||x_i+y_i|^{p-1} \leq \sum_{i=1}^n|x_i||x_i+y_i|^{p-1}+\sum_{i=1}^n|y_i||x_i+y_i|^{p-1}\\
&\leq \left(\sum_{i=1}^n|x_i|^p\right)^{\frac{1}{p}}\left(\sum_{i=1}^n|x_i+y_i|^{q(p-1)}\right)^{\frac{1}{q}}+\left(\sum_{i=1}^n|y_i|^p\right)^{\frac{1}{p}}\left(\sum_{i=1}^n|x_i+y_i|^{q(p-1)}\right)^{\frac{1}{q}}\\
&\leq \left(\sum_{i=1}^n|x_i|^p\right)^{\frac{1}{p}}\left(\sum_{i=1}^n|x_i+y_i|^{p}\right)^{1-\frac{1}{p}}+\left(\sum_{i=1}^n|y_i|^p\right)^{\frac{1}{p}}\left(\sum_{i=1}^n|x_i+y_i|^{p}\right)^{1-\frac{1}{p}}\\
\end{aligned}\]
\[\Rightarrow \ \left(\sum_{i=1}^n|x_i+y_i|^p\right) \cdot \left(\sum_{i=1}^n|x_i+y_i|^{p}\right)^{\frac{1}{p}-1} \leq \left(\sum_{i=1}^n|x_i|^p\right)^{\frac{1}{p}}+\left(\sum_{i=1}^n|y_i|^p\right)^{\frac{1}{p}}\]
\[\Leftrightarrow \ \left(\sum_{i=1}^n|x_i+y_i|^{p}\right)^{\frac{1}{p}} \leq \left(\sum_{i=1}^n|x_i|^p\right)^{\frac{1}{p}}+\left(\sum_{i=1}^n|y_i|^p\right)^{\frac{1}{p}}\]
离散形式的 Minkowski 不等式即证。
\end{proof}
\newpage
\begin{example}
\quad 设$m(E)<+\infty \ , \ f \in L^{\infty}(E)$,证明:
\[\lim_{p \to \infty}||f||_{L^p}=||f||_{L^{\infty}}\]
\end{example}
\begin{proof}利用 H\"{o}lder 不等式估计$L^p$范数的上界:
\[||f||_{L^p(E)}=\left(\int_E|f|^p\dd t\right)^{\frac{1}{p}}=\left(\int_{E/E_0}|f|^p\dd t\right)^{\frac{1}{p}} \leq \left(m(E) \cdot ||f||^p_{L^{\infty}(E)}\right)^{\frac{1}{p}}=m(E)^{\frac{1}{p}}||f||^p_{L^{\infty}(E)}\]
对上式两侧取极限:
\[\lim_{p \to \infty}||f||_{L^p(E)} \leq \lim_{p \to \infty}m(E)^{\frac{1}{p}}||f||^p_{L^{\infty}(E)}=||f||^p_{L^{\infty}(E)}\]
另一方面,我们令$M=||f||_{L^{\infty}(E)}$,其满足对$\forall \varepsilon>0 \ , \ m(|f| \geq M-\varepsilon)=\delta>0$,都有:
\[||f||_{L^p(E)}=\left(\int_E|f|^p\dd t\right)^{\frac{1}{p}}=\left(\int_{f \geq M-\varepsilon}|f|^p\dd t+\int_{f<M-\varepsilon}|f|^p\dd t\right)^{\frac{1}{p}}\]
\[\geq \left(\int_{|f| \geq M-\varepsilon}|M-\varepsilon|^p\dd t+\int_{|f|<M-\varepsilon}|f|^p\dd t\right)^{\frac{1}{p}} \geq |M-\varepsilon|\delta^{\frac{1}{p}}\]
由$\varepsilon$的任意性:
\[\lim_{p \to \infty}||f||_{L^p(E)} \geq \lim_{p \to \infty} |M-\varepsilon|\delta^{\frac{1}{p}}=||f||_{L^{\infty}(E)}\]
综上所述,有:
\[\lim_{p \to \infty}||f||_{L^p}=||f||_{L^{\infty}}\]
\end{proof}

\section*{有界线性算子习题}
\begin{example}
\quad 设$f$是$[a,b]$上的可测函数,且对$\forall g \in L^p[a,b] \ , \ p\in(1,+\infty)$都有$f \cdot g \in L^1[a,b]$,求证$f \in L^q[a,b]$,其中$1/p+1/q=1$。
\end{example}
\begin{proof}
定义映射$F$,对$\forall g \in L^p$,都有$F: \ L^p \to \mathbb{R}$:
\[F(g)=\int_a^bf \cdot g\dd{x}\]
显然$F$是有界线性泛函。对$\forall N \in \mathbb{N}_+$,定义新函数$f_N(x)$和对应的映射$F_N(x)$:
\[f_N(x)=\left\{\begin{array}{rl}
f(x), & |f(x)| \leq N \\ 0, & |f(x)|>N
\end{array}\right. \quad , \quad F_N(x)=\int_a^bf_N \cdot g \dd{x}\]
显然$\{F_N\}$逐点收敛至$F$。对$\forall g \in L^p[a,b]$,记$|F(g)|$的最大值为$C$,当$N \to \infty$时,有:
\[|F_N(g)|=\left|\int_a^b f_N \cdot g \dd{x}\right|=\left|\int_a^b f \cdot g \dd{x}-\int_{\{|f(x)|>N\}} f \cdot g \dd{x}\right|\leq\left|\int_a^b f \cdot g \dd{x}\right|+\left|\int_{\{|f(x)|>N\}} f \cdot g \dd{x}\right| \leq C+1<+\infty\]
进而可知$||F_N||\leq C+1<+\infty$,即$\{F_N\}$一致有界。由 Banach-Steinhaus 定理[定理\ref{BS}]得$||F_N-F|| \to 0$且:
\[||F||\leq \liminf_{N\to\infty}||F_N||\]
只需验证$||F_N||=||f_N||_{L^q}$,然后取$N \to \infty$即证。一方面:
\[|F_N(g)|=\left|\int_a^b f_N \cdot g \dd{x}\right|\leq||f_N||_{L^q}\cdot||g||_{L^p} \quad \Rightarrow \quad ||F_N|| \leq ||f_N||_{L^q}\]
另一方面,容易验证可取$g=(f_N)^{q-1}\cdot\text{sgn}f_N \in L^p$:
\[|F_N(g)|=\int_a^b |f_N|^q \dd{x}=||f_N||^q_{L^q} \leq ||F_N||\cdot||g||_{L^p}=||F_N||\cdot||f_N||^{q-1}_{L^q} \quad \Rightarrow \quad ||F_N|| \geq ||f_N||_{L^q}\]
综上所述,$||F_N||=||f_N||_{L^q}$,原命题得证。
\end{proof}
\newpage

\begin{example}
\quad 设$H$是 Hilbert 空间,$M \subset X \ , \ f \in M^*$,证明存在$f$的线性延拓$F$使得$||F||=||f||$。
\end{example}
\begin{proof}
做$H$的正交分解$H=M \oplus M^{\perp}$,令$p: \ H \to M$是正交投影算子,即对$\forall x \in H \ , \ \exists \, x_1 \in M \ , \ x_2 \in M^{\perp}$,满足$x=x_1+x_2$,且$p(x)=x_1$,容易验证$||p||=1$。定义泛函$F: \ H \to \mathbb{R} \ , \ x \mapsto f \circ p(x)$,易知$F$是$f$的线性延拓。因此$||F||\geq||f||$。另一方面,$||F||=||f \circ p|| \leq ||f|| \cdot ||p||=||f||$,因此$||F||=||f||$。
\end{proof}

\begin{example}
\quad 设$X$是赋范线性空间,$X_0 \subset X$是闭子空间,证明对$\forall x \in X$都有:
\[d(x,X_0)=\sup\{|f(x) \ | \ \forall f \in X^*, \ ||f||=1, \ f|_{X_0}=0\}\]
\end{example}
\begin{proof}
若$x \in X_0$命题显然成立,现在考虑$\forall x \in X-X_0$。由$X_0 \subset X$是闭子空间,因此$\exists \, \{x_n\} \subset X_0$满足$||x_n-x|| \geq d(x,X_0)$且$||x_n-x|| \to d(x,X_0)$,一方面,令$F^*=\{f \in X^* \ | \ ||f||=1 \ , \ f|_{X_0}=0\}$,对$\forall f \in F^*$都有$|f(x)|=|f(x-x_n)|\leq||f||\cdot||x_n-x||=||x_n-x||$,取极限$n \to \infty$,可得$\sup\limits_{f \in F^*}|f(x)|\leq d(x,X_0)$。

另一方面,考虑闭子空间线性延拓$X_1=\{x_0+\lambda x|x_0 \in X_0 \ , \ \lambda \in \mathbb{K}\}$,定义$f_1: \ X_1 \to \mathbb{R} \ , \ x \mapsto \lambda d(x,X_0)$,显然$f_1|_{X_0}=0$。其范数:
 \[||f_1||=\sup_{t \in X_1}\frac{|f(t)|}{||t||}=\sup_{t \in X_1}\frac{|\lambda|d(x,X_0)}{|\lambda|\cdot||x+\frac{x_0}{\lambda}||}=\sup_{t' \in X_1}\frac{d(x,X_0)}{||t'||} \quad , \quad t':=x+\frac{x_0}{\lambda}\]
 又因为:
 \[||t'||=\left\|x+\frac{x_0}{\lambda}\right\|\geq\inf_{y \in X_0}\left\|x+y\right\|=d(x,X_0)\]
 由上确界的性质可知$||f_1||=1$,即$f_1 \in F^*$,由 Hahn-Banach 定理[定理\ref{HB}]可将$f_1$延拓到全空间$X$上。
\end{proof}

\begin{example}
\quad 设$X$是赋范线性空间,且对$X^*$单位球壳$S=\{f \in X^* \ | \ ||f||=1\}$上的任意柯西列$\{f_n\}$都弱$*$收敛到$f_0 \in S$,证明$S$是弱$*$闭的。
\end{example}
\begin{proof}
即证明对任意$S$上的柯西列$\{f_n\}$都满足$f_n \xrightarrow{w^*} f_0 \in S$。显然$\{f_n\}$一致有界,且对$\forall x \in X$都有$f_n(x) \to f_0(x)$,即逐点收敛,因此由 Banach-Steinhaus 定理[定理\ref{BS}]得$||f_n-f|| \to 0$且:
\[||f||\leq \liminf_{n\to\infty}||f_n||=1\]
即$||f||=1$,因此$f \in S$,原命题得证。
\end{proof}

\section*{希尔伯特空间习题}
\begin{example}
\quad 在 Hilbert 空间$H=L^2[0,1]$中找出$M_1=\{\text{全体多项式}\}$和$M_2=\{\text{常数项为0的多项式}\}$的正交补。
\end{example}
\begin{proof}
因为$\overline{M_1}=H$,因此$M_1^{\perp}=\{0\}$。对$\forall f \in M_2 \ , \ \exists g \in M_1$满足$f=xg$。因此对$\forall u \in M_2^{\perp}$,都有:
\[(u,f)=(u,xg)=\int_0^1u(x) \cdot xg(x)\dd{x}=\int_0^1xu(x) \cdot g(x)\dd{x}=(xu,g)=0\]
即$xu \in M_1^{\perp}$,即$xu=0$,因此$u=0$,即$M_2^{\perp}=\{0\}$。
\end{proof}

\begin{example}
\quad 设$M$是 Hilbert 空间$H$的闭子空间,$\{e_n\}$和$\{f_n\}$分别是$M$和$M^{\perp}$的标准正交基,证明$\{e_n\}\cup\{f_n\}$是$H$的标准正交基。
\end{example}
\begin{proof}
易知$H=M \oplus M^{\perp}$,设对$\forall x \in H$都存在$\exists \, x_1 \in M_1 \ , \ x_2 \in M_2$满足$x=x_1+x_2$即:
\[x_1=\sum_{\alpha}f_{\alpha}e_{\alpha} \ , \ x_2=\sum_{\beta}g_{\beta}e_{\beta} \quad \Rightarrow \quad x=\sum_{\alpha}f_{\alpha}e_{\alpha}+\sum_{\beta}g_{\beta}e_{\beta}\]
故$\{e_n\}\cup\{f_n\}$是标准正交基,且完备,故是标准正交基。
\end{proof}

\begin{example}
\quad 设$H$是 Hilbert 空间,$T\in\mathscr{B}(H)$是$H$上的自共轭有界线性算子,且$T$的像空间$R(T)=TH$是有限维的,证明存在有限个线性无关的元素$\{y_1,y_2,\cdots,y_n\} \subset H$使得对$x \in H$都有:
\[T(x)=\sum_{i=1}^n(x,y_i)y_i\]
\end{example}
\begin{remark}
\quad 若$T$不是自共轭的,取$H=\text{span}\{e_1,e_2\}$且$e_1 \perp e_2$,令$Te_1=e_2 \ , \ Te_2=0$,即$TH$是一维的。则存在$y \in H$使得$Te_1=(e_1,y)y=e_2$,即$y \propto e_2$,因此$Te_1\propto (e_1,e_2)e_2=0$。
\end{remark}
\begin{theorem}\label{theorem:djm}
设 $V$ 是有限维线性空间,$f: V \to V$ 是线性算子。若 $f$ 是单射,则 $f$ 是满射。
\end{theorem}
\begin{proof}
由于 $f$ 是单射,$\ker(f) = \{0\}$。由维数公式$\dim V = \dim \ker(f) + \dim R(f)$,得 $\dim R(f) = \dim V$。又 $R(f) \subseteq V$,故 $R(f) = V$,即 $f$ 是满射,因此 $f$ 是双射。
\end{proof}
\begin{proof}
首先证明一个有限维的特殊情况,然后推广到一般情况,由于有限维的 Hilbert 空间同构与$\mathbb{K}^n$,因此直接取$H=\mathbb{K}^n$,算子$T: \ \mathbb{K}^n \to \mathbb{K}^n$是自共轭的,且是双射。因此存在算子$T$的特征值$\{\lambda_1,\lambda_2,\cdots,\lambda_n\}$与标准正交的特征向量$\{e_1,e_2,\cdots,e_n\}$使得对$i=1,2,\cdots,n$都有$Te_i=\lambda_ie_i$。因此对$\forall x \in H$,取$y_i=\sqrt{\lambda_i}e_i$,都满足:
\[x=\sum_{i=1}^n\alpha_ie_i \quad \Rightarrow \quad Tx=\sum_{i=1}^n\alpha_iTe_i=\sum_{i=1}^n\alpha_i\lambda_ie_i=\sum_{i=1}^n(x,\lambda_ie_i)e_i:=\sum_{i=1}^n(x,y_i)y_i\]
对一般情况,由定理\ref{theorem:zgeker}可知对任意一个自共轭算子$T\in\mathscr{B}(H)$都可做以下正交分解$H=\ker T \oplus R(T)$。定义正交投影算子$p: \ H \to R(T)$和$T_1=T|_{R(T)}: \ R(T) \to R(T)$,显然$T_1$是单射,由定理\ref{theorem:djm}知$T_1$是双射。因此对$\forall x \in H \ , \ \exists \, x_1 \in \ker T \ , \ x_2 \in R(T)$都有$x=x_1+x_2$,即$Tx=T_1 \circ p(x)=T_1(x_2)$,由上面的特殊情况可知:
\[\forall x \in H \ , \ \exists \, \{y_1,y_2,\cdots,y_n\} \subset H \quad \text{s.t.} \quad Tx=T_1(x_2)=\sum_{i=1}^n(x,y_i)y_i\]
\end{proof}

\begin{example}
\quad 设$H$是 Hilbert 空间,证明$\{x_n\} \subset H$强收敛于$x_0 \in H$等价于弱收敛。
\end{example}
\begin{proof}
由 Riesz 表示定理[定理\ref{theorem:rieszbs}]可知, Hilbert 空间的弱收敛等价于对$\forall y \in H$都有$(x_n,y) \to (x_0,y)$。对必要性,对$\forall y \in H$,由 Cauchy–Schwarz 不等式[定理\ref{theorem:Cauchy–Schwarz}]:
\[|(x_n,y)-(x_0,y)|=|(x_n-x_0,y)|\leq||x_n-x_0||\cdot||y|| \to 0\]
即强收敛推出弱收敛。考虑充分性,由弱收敛可知,取$y=x_0$:
\[|(x_n,x_0)-(x_0,x_0)|=|(x_n-x_0,x_0)|=(x_n,x_0)-||x_0||^2 \to 0\]
再取$y=x_n$:
\[|(x_n,x_n)-(x_0,x_n)|=|(x_n-x_0,x_n)|=||x_n||^2-(x_0,x_n) \to 0\]
最后计算范数$||x_n-x_0||^2$:
\[||x_n-x_0||^2=|(x_n-x_0,x_n-x_0)|=|(x_n-x_0,x_n)-(x_n-x_0,x_0)| \leq |(x_n-x_0,x_0)|+|(x_n-x_0,x_n)| \to 0\]
即弱收敛推出强收敛,因此原命题得证。
\end{proof}

\begin{example}
\quad 设$H$是 Hilbert 空间,$f_1,f_2 \in H^*$满足$f_1,f_2 \neq 0$且$\ker f_1 \subset \ker f_2$。证明对$\forall x \in H$都$\exists \, \alpha \in \mathbb{R}$满足$f_1(x)/f_2(x)=\alpha$。
\end{example}
\begin{theorem}\label{theorem:1d}
设$H$是 Hilbert 空间,$f \in H^*$满足$f \neq 0$,则$(\ker f)^{\perp} \cong \mathbb{R}$。
\end{theorem}
\begin{proof}
因为$f \neq 0$,则$(\ker f)^{\perp} \neq \{0\}$,任取$x_0,y \in (\ker f)^{\perp}$,显然$f(x_0),f(y) \neq 0$,记$\lambda=f(y)/f(x_0) \in \mathbb{R}$,由线性$y-\lambda x_0 \in (\ker f)^{\perp}$。而$f(y-\lambda x_0)=f(y)-\lambda f(x_0)=0$,即$y-\lambda x_0 \in \ker f \cap (\ker f)^{\perp}={0}$,因此$y/x_0=\lambda$,即$(\ker f)^{\perp}$中的元素都线性相关,可知$\dim (\ker f)^{\perp}=1$,因此可以构造同构使得$(\ker f)^{\perp} \cong \mathbb{R}$。
\end{proof}
\begin{proof}
由$\ker f_1 \subset \ker f_2$可知$(\ker f_1)^{\perp} \supset (\ker f_2)^{\perp}$,但由定理\ref{theorem:1d}可知$(\ker f_1)^{\perp},(\ker f_2)^{\perp}$都同构与$\mathbb{R}$,因此$(\ker f_1)^{\perp}=(\ker f_2)^{\perp}$。由 Riesz 表示定理[定理\ref{theorem:rieszbs}]可知,$\exists \, x_1,x_2 \in (\ker f_1)^{\perp}=(\ker f_2)^{\perp}$满足$x_1/x_2=\alpha\in\mathbb{R}$满足$f(x_1)=(x,x_1) \ , \ f(x_2)=(x,x_2)$,因此$f(x_1)/f(x_2)=\alpha$。
\end{proof}
\end{document}
